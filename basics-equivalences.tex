\section{Quasi-inverses and (coherent) equivalences}
\label{sec:basics-equivalences}

\begin{defn}
Let $f,g:\prd{x:A}{P(x)}$ be two sections of a dependent type $P:A\to\type$. A
homotopy from $f$ to $g$ is a term of type
\begin{equation*}
f\htpy g\defeq \prd{x:A}f(x)=g(x)
\end{equation*}
\end{defn}

Homotopies are always natural with respect to paths:

\begin{lem}\label{lem:htpy_natural}
Suppose $H:f\htpy g$ is a homotopy between functions $f,g:A\to B$
 and let $p:x=y$. Then there is a term of type
 \begin{equation*}
 H(y)\ct\ap{f}{p}=\ap{g}{p}\ct H(x).
 \end{equation*}
\end{lem}

\begin{defn}
Consider a function $f:A\to B$. A quasi-inverse of $f$ is a triple 
$(g,\eta,\varepsilon)$ consisting of a function $g:B\to A$ and homotopies
$\eta:\idfunc[A]\htpy g\circ f$ and $\varepsilon:f\circ g\htpy \idfunc[B]$.
\end{defn}

\begin{rm}
In other words, $f:A\to B$ has a quasi-inverse if there is a term of type
\begin{equation*}
\sm{g:B\to A}((\idfunc[A]\htpy g\circ f)\times(f\circ g\htpy\idfunc[B])).
\end{equation*}
\end{rm}


% Local Variables:
% TeX-master: "main"
% End:
