% !Tex root = note.tex

\section{Function Extensionality in a Univalent Universe}

\subsection{Function Extensionality}
\be{deff}
Given types $A,B$ we are concerned with the {\bf Function Extensionality} principle for functions $A\ra B$.
\be{center}
{\bf $({\bf FE_{\sf prop}})_{A\ra B}:\;\;$} 
  $\forall_{f_1,f_2:A\ra B}[\forall_{x:A}(f_1x=f_2x)\ra (f_1=f_2)]$
\en{center}
\en{deff}
We aim to prove the following result.
\be{thm}\label{fe:thm}\label{thm:4.2} For types $A,B$, with $B$ in a univalent universe, 
$({\bf FE_{\sf prop}})_{A\ra B}$.
\en{thm}
We will need to assume the following principle as part of our type theory.
\be{center}
{\bf $({\eta_{\sf prop}})_{A\ra B}:\;\;$} 
  $\forall_{f:A\ra B}[\lambda_{x:A}fx = f]$.
\en{center}

In order to carry through the proof of the theorem we use the following rule of our basic type theory concerning definitional equality.  For $f_1,f_2:A\ra B$,
\be{center}
$({\bf FE_{\sf def}})_{A\ra B}$: From $f_1x=_{def}f_2x$ for $x:A$ infer $\lambda_{x:A}f_1x =_{def} \lambda_{x:A}f_2x$.
\en{center}
When the definitional equality, $=_{def}$,  used in this rule is replaced by propositional equality, $=$, then the rule is not a correct rule of the basic type theory, but when combined with $({\eta_{\sf prop}})_{A\ra B}$ is equivalent to $({\bf FE_{\sf prop}})_{A\ra B}$.  I suspect that once Function Extensionality is assumed there will no longer be much need to refer to definitional equality in our informal development.


\subsection{Extensional Properties of Maps}
\be{deff}
If $P[f]$ is a type for $f:A\ra B$ we define $P[f]$ to be {\em extensional in $f$} if, for all $f_1,f_2:A\ra B$
  \[ \forall_{x:A}(f_1x=f_2x)\;\ra\; (P[f_1]\ra P[f_2]).\]
\en{deff} 
\be{deff} For a function $f:A\ra B$ let $EQ(f)$ be the type 
\be{center}
{\em $(f\mbox{ is an equivalence})$}.
\en{center}
\en{deff}
\be{propp}\label{prop:4.5}
For types $A,B$ the type $EQ(f)$ is extensional in $f:A\ra B$.
\en{propp}
\proof  By examining the formal definition of $EQ(f)$ it is evident that $f$ only occurs when applied to an argument.
\qed

\be{cor} \label{cor:4.6}
For types $A,B$ the function $\lambda:(A\ra B)\ra(A\ra B)$ is an equivalence, where 
  \[ \lambda f := \lambda_{x:A} fx\mbox{ for } f:A\ra B.\]
\en{cor}
\proof By $(\eta_{\sf prop})$
  \[ \lambda f = f = 1_{A\ra B}f\mbox{ for } f:A\ra B\]
so that $\forall_{f:A\ra B}\; \lambda f=1_{A\ra B}f$.

As $1_{A\ra B}$ is an equivalence, by Lemma~\ref{lem:2.3}, and $EQ(f)$ is extensional in $f$, by Proposition~\ref{prop:4.5} it follows that $\lambda$ is an equivalence.
\qed

%%%%%%%%%%%%%%%%%
\subsection{The type $Id(B)$}
%%%%%%%%%%%%%%%%%

\be{deff} Let $B$ be a type.
\be{enumerate}
\item Define the type $Id(B):=\Sigma_{y,y':B}(y=y')$.
\item Define the function $\delta_B:B\ra Id(B)$ by
$\delta_B:=\lambda_{y:B}(y,y,\refl{y})$.
\item For $i=1,2$, define  the function $\pi^i_B:Id(B)\ra B$ by
  \[ \pi^i_B(y_1,y_2,z):= y_i\;\;\mbox{ for }y_1,y_2:Id(B), z:(y_1=y_2).\]
\en{enumerate}
\en{deff}
\be{propp}
$\forall_{u:Id(B)}(\pi^1_Bu = \pi^2_Bu)$.
\en{propp}
\proof
For $y_1,y_2:Id(B), z:(y_1=y_2)$
  \[\be{array}{lll}
\pi^1_B(y_1,y_2,z)&=_{def}& y_1\\
                 &=&y_2, \mbox{by } z\\
                 &=_{def}&\pi^2_B(y_1,y_2,z)
  \en{array}\]
So, by Id-induction on $u:Id(B)$, $\pi^1_Bu = \pi^2_Bu$.  
\qed

It follows that if we could assume
the theorem then we would get a proof that 
  \[ (*)\;\;\;\pi^1_B = \pi^2_B\mbox{ for $B$ in a univalent universe.}\]  
Instead in our second proof of the theorem we will use $(*)$ 
and give another proof of (*) in Lemma~\ref{lem:6.7}.

\be{lem}\label{fe:lem-delta-equiv} \label{lem:4.9}
For $i=1,2$,
\be{enumerate}
\item $\pi^i_B\circ\delta_B =_{def} 1_B$ and hence 
$(\pi^i_B\circ\delta_B)y =_{def} y$ for $y:B$.
\item $(\delta_B\circ\pi^i_B)u = u$ for $u:Id(B)$
\item $\delta_B$ is an isomorphism.
\en{enumerate}
\en{lem}
\proof Routine
\qed


\be{lem}\label{lem:4.10}
For $i=1,2$, $\pi^i_B$ is an equivalence.
\en{lem}
\proof For $y:B$ we must show that the type $Dy$ is contractible, where
  \[ Dy := \Sigma_{u:Id(B)}\; \pi^i_Bu=y.\]
Let $dy:=(\delta_By,\refl{y}):Dy$.  We show that $\forall_{v:Dy}\; v=dy$.
By $\Sigma$-folding it suffices to show that
  \[ \forall_{u:Id(B)}\forall_{q:(\pi^i_Bu=y)}\; (u,q)=dy.\]
By another $\Sigma$-folding and using $\pi^i_B(y_1,y_2,p)=_{def}y_i$,
  \[\forall_{y_1,y_2:B}\forall_{p:(y_1=y_2)}\forall_{q:(y_i=y)}
       \;  ((y_1,y_2,p),q)=dy
  \]
By Id-induction it suffices to show that
  \[ \forall_{y':B}\forall_{q:(y'=y)}\; (\delta_By',q)=dy\]
By another Id-induction, this time \`{a} la Christine Pauline, it suffices to observe that
  \[ (\delta_By,\refl{y})=dy\]
by the definition of $dy$.
\qed

\section{First Proof of the theorem}


%%%%%%%%%%%%%%%%%%%%%%%%%%%%%%%%%
\subsection{ The Main Lemma - version 1}
%%%%%%%%%%%%%%%%%%%%%%%%%%%%%%%%%

We now state and  prove version 1 of the main lemma needed to prove the theorem.
\be{lem}[Main Lemma]\label{fe:lem-main1}\label{lem:5.1}
Let $A,A'$ be types in a univalent universe.  For each type $C$ and each
$f:A\ra A'$ define $H_f := f\circ(-): (C\ra A)\ra (C\ra A')$; i.e.
  \[ H_f := \lambda_{g:C\ra A}\;  f\circ g.\]
If $f$ is an equivalence then so is $H_f$.
\en{lem}
\proof Let $\bbU$ be a univalent universe and let $f:A\ra A'$ be an equivalence, where $A,A':\bbU$.  We want to show that $H_f$ is an equivalence.

By Corollary~\ref{cor:3.5} $E_{A,A'}:(A=A')\ra (A\simeq A')$ is surjective.  
As $f$ is an equivalence there is $q:EQ(f)$ so that $(f,q):A\simeq A'$. Let $u:=(f,q):(A\simeq A')$.  So there is $Z_f:(A=A')$ such that $E_{A,A'}Z_f = u$.

\be{description}
\item[Proof that $H_f$ is an equivalence:]
For $X,X':\bbU$, $Z:X=X'$ let 
  \[\tau X X' Z:= \pi_{X,X'}( E_{X,X'}Z)\]
where $\pi_{X,X'}:(X\simeq X')\ra (X\ra X')$ is defined by
  \[ \pi_{X,X'}(f,q):= f\mbox{ for $f:X\ra X',q:EQ(f)$}.\]
 \be{quote}
{\bf Claim:} $H_{\tau A A' Z}$ is an equivalence for $Z:A=A'$.\\
\proof By Id-induction it suffices to show that $H_{\tau X X \refl{X}}$ is an equivalence for $X:\bbU$.  As $\tau X X\refl{X}=1_X$,
  \[ H_{\tau X X\refl{X}} = H_{1_X}=1_{B\ra X}\]
is an equivalence, by Corollary~\ref{cor:4.6}, so that 
$H_{\tau X X \refl{X}}$ is an equivalence.
\qed
\en{quote}
Observe that
  \[ f = \pi_{A,A'}(f,q)=\pi_{A,A'}(E_{A,A'}Z_f)=\tau A A' Z_f\]
so that, by the claim, $H_f$ is an equivalence.
\en{description}
\qed
%\pagebreak

%%%%%%%%%%%%%%%%%%%%%%%%%%%%%%%%%%%%%%%%%%%%%%%%%%%%%
\subsection{Proof of the Theorem - version 1: } 
%%%%%%%%%%%%%%%%%%%%%%%%%%%%%%%%%%%%%%%%%%%%%%%%%%%%%

Let $f_1,f_2:A\ra B$, where $B$ is a type in a univalent universe, and let 
$q:\forall_{x:A}\; f_1x=f_2x$.  We must show that $f_1=f_2$.
Define $f:A\ra Id(B)$ by
  \[ f := \lambda_{x:A}(f_1x,f_2x,qx).\]

Define $d:A\ra Id(B)$ by
  \[ d:=\delta_B\circ f_1\]
Then
  \[\be{array}{ll}
\pi^1_B\circ d &=\lambda_{x:A}\;\pi^1_B(\delta_B(f_1x))\\
              &=\lambda_{x:A}\; f_1x, \mbox{by part 1 of Lemma~\ref{lem:4.9}}\\
              &=\pi^1_B\circ f.
  \en{array}\]
As, by Lemma~\ref{lem:4.10},
 $\pi^1_B$ is an equivalence, by part 2 of the Main Lemma, so is $H_{\pi^1_B}$ so that, by Proposition~\ref{prop:3.4},
$H_{\pi^1_B}$ is injective.  It follows that  $d=f$ and so 
$\pi^2_B\circ d =\pi^2_B\circ f$; i.e. $\lambda_{x:A}f_1x =\lambda_{x:A}f_2x$.
So $f_1=f_2$ by $(\eta_{\sf prop})$.
\qed 



% Local Variables:
% TeX-master: "main"
% End:
