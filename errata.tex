% This is the errata document for the homotopy type theory book.

% This file supports two book sizes:
% - Letter size (8.5" x 11")
% - US Trade size (6" x 9")
%
% To activate one or the other, uncomment the appropriate font size in
% the documentclass below, and then one of the two page geometry incantations
%
% NOTE: The 6" x 9" format is only experimental. It will break the
% title page, for example.

\PassOptionsToPackage{table}{xcolor}

% DOCUMENT CLASS
\documentclass[
%
%10pt % for US Trade 6" x 9" book
%
11pt % for Letter size book
]{article}
\usepackage{etex} % We're running out of registers and dimensions, or some such

\newcounter{chapter}            % So that macros.tex doesn't choke

% PAGE GEOMETRY
%
% Uncomment one of these

% We make the page 40pt taller than the standard LaTeX book.

% OPTION 1: Letter
\usepackage[papersize={8.5in,11in},
            twoside,
            includehead,
            top=1in,
            bottom=1in,
            inner=0.75in,
            outer=1.0in,
            bindingoffset=0.35in]{geometry}

% OPTION 2: US Trade
% \usepackage[papersize={6in,9in},
%             twoside,
%             includehead,
%             top=0.75in,
%             bottom=0.75in,
%             inner=0.5in,
%             outer=0.75in,
%             bindingoffset=0.35in]{geometry}

% HYPERLINKING AND PDF METADATA

\usepackage[pagebackref,
            colorlinks,
            citecolor=darkgreen,
            linkcolor=darkgreen,
            unicode,
            pdfauthor={Univalent Foundations Program},
            pdftitle={Homotopy Type Theory: Univalent Foundations of Mathematics},
            pdfsubject={Mathematics},
            pdfkeywords={type theory, homotopy theory, univalence axiom}]{hyperref}

% OTHER PACKAGES

% Use this package and stick \layout somewhere in the text to see
% page margins, text size and width etc. Useful for debugging page format.
%\usepackage{layout}

%%% Because Germans have umlauts and Slavs have even stranger ways of mangling letters
\usepackage[utf8]{inputenc}

%%% For table {tab:theorems}
\usepackage{pifont}

%%% Multi-Columns for long lists of names
\usepackage{multicol}

%%% Set the fonts
\usepackage{mathpazo}
\usepackage[scaled=0.95]{helvet}
\usepackage{courier}
\linespread{1.05} % Palatino looks better with this

\usepackage{graphicx}
\usepackage{comment}

\usepackage{wallpaper} % For the background image on the cover page

\usepackage{fancyhdr} % To set headers and footers

\usepackage{nextpage} % So we can jump to odd-numbered pages

\usepackage{amssymb,amsmath,amsthm,stmaryrd,mathrsfs,wasysym}
\usepackage{enumitem,mathtools,xspace}
\usepackage{xcolor} % For colored cells in tables we need \cellcolor
\usepackage{booktabs} % For nice tables
\usepackage{array} % For nice tables
\usepackage{supertabular} % For index of symbols
\definecolor{darkgreen}{rgb}{0,0.45,0}
\usepackage{aliascnt}
\usepackage[capitalize]{cleveref}
\usepackage[all,2cell]{xy}
\UseAllTwocells
\usepackage{natbib}
\usepackage{braket} % used for \setof{ ... } macro
\usepackage{tikz}
\usetikzlibrary{decorations.pathmorphing}

\usepackage{etoolbox}           % hacking commands for TOC

\usepackage{mathpartir}         % for formal.tex appendix, section 3

\usepackage[numbered]{bookmark} % add chapter/section numbers to the toc in the pdf metadata

%%%% MACROS FOR NOTATION %%%%
% Use these for any notation where there are multiple options.

%%% Notes and exercise sections
\makeatletter
\newcommand{\sectionNotes}{\phantomsection\section*{Notes}\addcontentsline{toc}{section}{Notes}\markright{\textsc{\@chapapp{} \thechapter{} Notes}}}
\newcommand{\sectionExercises}[1]{\phantomsection\section*{Exercises}\addcontentsline{toc}{section}{Exercises}\markright{\textsc{\@chapapp{} \thechapter{} Exercises}}}
\makeatother

%%% Definitional equality (used infix) %%%
\newcommand{\jdeq}{\equiv}      % An equality judgment
\let\judgeq\jdeq
%\newcommand{\defeq}{\coloneqq}  % An equality currently being defined
\newcommand{\defeq}{\vcentcolon\equiv}  % A judgmental equality currently being defined

%%% Term being defined
\newcommand{\define}[1]{\textbf{#1}}

%%% Vec (for example)

\newcommand{\Vect}{\ensuremath{\mathsf{Vec}}}
\newcommand{\Fin}{\ensuremath{\mathsf{Fin}}}
\newcommand{\fmax}{\ensuremath{\mathsf{fmax}}}
\newcommand{\seq}[1]{\langle #1\rangle}

%%% Dependent products %%%
\def\prdsym{\textstyle\prod}
%% Call the macro like \prd{x,y:A}{p:x=y} with any number of
%% arguments.  Make sure that whatever comes *after* the call doesn't
%% begin with an open-brace, or it will be parsed as another argument.
\makeatletter
% Currently the macro is configured to produce
%     {\textstyle\prod}(x:A) \; {\textstyle\prod}(y:B),\ 
% in display-math mode, and
%     \prod_{(x:A)} \prod_{y:B}
% in text-math mode.
% \def\prd#1{\@ifnextchar\bgroup{\prd@parens{#1}}{%
%     \@ifnextchar\sm{\prd@parens{#1}\@eatsm}{%
%         \prd@noparens{#1}}}}
\def\prd#1{\@ifnextchar\bgroup{\prd@parens{#1}}{%
    \@ifnextchar\sm{\prd@parens{#1}\@eatsm}{%
    \@ifnextchar\prd{\prd@parens{#1}\@eatprd}{%
    \@ifnextchar\;{\prd@parens{#1}\@eatsemicolonspace}{%
    \@ifnextchar\\{\prd@parens{#1}\@eatlinebreak}{%
    \@ifnextchar\narrowbreak{\prd@parens{#1}\@eatnarrowbreak}{%
      \prd@noparens{#1}}}}}}}}
\def\prd@parens#1{\@ifnextchar\bgroup%
  {\mathchoice{\@dprd{#1}}{\@tprd{#1}}{\@tprd{#1}}{\@tprd{#1}}\prd@parens}%
  {\@ifnextchar\sm%
    {\mathchoice{\@dprd{#1}}{\@tprd{#1}}{\@tprd{#1}}{\@tprd{#1}}\@eatsm}%
    {\mathchoice{\@dprd{#1}}{\@tprd{#1}}{\@tprd{#1}}{\@tprd{#1}}}}}
\def\@eatsm\sm{\sm@parens}
\def\prd@noparens#1{\mathchoice{\@dprd@noparens{#1}}{\@tprd{#1}}{\@tprd{#1}}{\@tprd{#1}}}
% Helper macros for three styles
\def\lprd#1{\@ifnextchar\bgroup{\@lprd{#1}\lprd}{\@@lprd{#1}}}
\def\@lprd#1{\mathchoice{{\textstyle\prod}}{\prod}{\prod}{\prod}({\textstyle #1})\;}
\def\@@lprd#1{\mathchoice{{\textstyle\prod}}{\prod}{\prod}{\prod}({\textstyle #1}),\ }
\def\tprd#1{\@tprd{#1}\@ifnextchar\bgroup{\tprd}{}}
\def\@tprd#1{\mathchoice{{\textstyle\prod_{(#1)}}}{\prod_{(#1)}}{\prod_{(#1)}}{\prod_{(#1)}}}
\def\dprd#1{\@dprd{#1}\@ifnextchar\bgroup{\dprd}{}}
\def\@dprd#1{\prod_{(#1)}\,}
\def\@dprd@noparens#1{\prod_{#1}\,}

% Look through spaces and linebreaks
\def\@eatnarrowbreak\narrowbreak{%
  \@ifnextchar\prd{\narrowbreak\@eatprd}{%
    \@ifnextchar\sm{\narrowbreak\@eatsm}{%
      \narrowbreak}}}
\def\@eatlinebreak\\{%
  \@ifnextchar\prd{\\\@eatprd}{%
    \@ifnextchar\sm{\\\@eatsm}{%
      \\}}}
\def\@eatsemicolonspace\;{%
  \@ifnextchar\prd{\;\@eatprd}{%
    \@ifnextchar\sm{\;\@eatsm}{%
      \;}}}

%%% Lambda abstractions.
% Each variable being abstracted over is a separate argument.  If
% there is more than one such argument, they *must* be enclosed in
% braces.  Arguments can be untyped, as in \lam{x}{y}, or typed with a
% colon, as in \lam{x:A}{y:B}. In the latter case, the colons are
% automatically noticed and (with current implementation) the space
% around the colon is reduced.  You can even give more than one variable
% the same type, as in \lam{x,y:A}.
\def\lam#1{{\lambda}\@lamarg#1:\@endlamarg\@ifnextchar\bgroup{.\,\lam}{.\,}}
\def\@lamarg#1:#2\@endlamarg{\if\relax\detokenize{#2}\relax #1\else\@lamvar{\@lameatcolon#2},#1\@endlamvar\fi}
\def\@lamvar#1,#2\@endlamvar{(#2\,{:}\,#1)}
% \def\@lamvar#1,#2{{#2}^{#1}\@ifnextchar,{.\,{\lambda}\@lamvar{#1}}{\let\@endlamvar\relax}}
\def\@lameatcolon#1:{#1}
\let\lamt\lam
% This version silently eats any typing annotation.
\def\lamu#1{{\lambda}\@lamuarg#1:\@endlamuarg\@ifnextchar\bgroup{.\,\lamu}{.\,}}
\def\@lamuarg#1:#2\@endlamuarg{#1}

%%% Dependent products written with \forall, in the same style
\def\fall#1{\forall (#1)\@ifnextchar\bgroup{.\,\fall}{.\,}}

%%% Existential quantifier %%%
\def\exis#1{\exists (#1)\@ifnextchar\bgroup{.\,\exis}{.\,}}

%%% Dependent sums %%%
\def\smsym{\textstyle\sum}
% Use in the same way as \prd
\def\sm#1{\@ifnextchar\bgroup{\sm@parens{#1}}{%
    \@ifnextchar\prd{\sm@parens{#1}\@eatprd}{%
    \@ifnextchar\sm{\sm@parens{#1}\@eatsm}{%
    \@ifnextchar\;{\sm@parens{#1}\@eatsemicolonspace}{%
    \@ifnextchar\\{\sm@parens{#1}\@eatlinebreak}{%
    \@ifnextchar\narrowbreak{\sm@parens{#1}\@eatnarrowbreak}{%
        \sm@noparens{#1}}}}}}}}
\def\sm@parens#1{\@ifnextchar\bgroup%
  {\mathchoice{\@dsm{#1}}{\@tsm{#1}}{\@tsm{#1}}{\@tsm{#1}}\sm@parens}%
  {\@ifnextchar\prd%
    {\mathchoice{\@dsm{#1}}{\@tsm{#1}}{\@tsm{#1}}{\@tsm{#1}}\@eatprd}%
    {\mathchoice{\@dsm{#1}}{\@tsm{#1}}{\@tsm{#1}}{\@tsm{#1}}}}}
\def\@eatprd\prd{\prd@parens}
\def\sm@noparens#1{\mathchoice{\@dsm@noparens{#1}}{\@tsm{#1}}{\@tsm{#1}}{\@tsm{#1}}}
\def\lsm#1{\@ifnextchar\bgroup{\@lsm{#1}\lsm}{\@@lsm{#1}}}
\def\@lsm#1{\mathchoice{{\textstyle\sum}}{\sum}{\sum}{\sum}({\textstyle #1})\;}
\def\@@lsm#1{\mathchoice{{\textstyle\sum}}{\sum}{\sum}{\sum}({\textstyle #1}),\ }
\def\tsm#1{\@tsm{#1}\@ifnextchar\bgroup{\tsm}{}}
\def\@tsm#1{\mathchoice{{\textstyle\sum_{(#1)}}}{\sum_{(#1)}}{\sum_{(#1)}}{\sum_{(#1)}}}
\def\dsm#1{\@dsm{#1}\@ifnextchar\bgroup{\dsm}{}}
\def\@dsm#1{\sum_{(#1)}\,}
\def\@dsm@noparens#1{\sum_{#1}\,}

%%% W-types
\def\wtypesym{{\mathsf{W}}}
\def\wtype#1{\@ifnextchar\bgroup%
  {\mathchoice{\@twtype{#1}}{\@twtype{#1}}{\@twtype{#1}}{\@twtype{#1}}\wtype}%
  {\mathchoice{\@twtype{#1}}{\@twtype{#1}}{\@twtype{#1}}{\@twtype{#1}}}}
\def\lwtype#1{\@ifnextchar\bgroup{\@lwtype{#1}\lwtype}{\@@lwtype{#1}}}
\def\@lwtype#1{\mathchoice{{\textstyle\mathsf{W}}}{\mathsf{W}}{\mathsf{W}}{\mathsf{W}}({\textstyle #1})\;}
\def\@@lwtype#1{\mathchoice{{\textstyle\mathsf{W}}}{\mathsf{W}}{\mathsf{W}}{\mathsf{W}}({\textstyle #1}),\ }
\def\twtype#1{\@twtype{#1}\@ifnextchar\bgroup{\twtype}{}}
\def\@twtype#1{\mathchoice{{\textstyle\mathsf{W}_{(#1)}}}{\mathsf{W}_{(#1)}}{\mathsf{W}_{(#1)}}{\mathsf{W}_{(#1)}}}
\def\dwtype#1{\@dwtype{#1}\@ifnextchar\bgroup{\dwtype}{}}
\def\@dwtype#1{\mathsf{W}_{(#1)}\,}

\newcommand{\suppsym}{{\mathsf{sup}}}
\newcommand{\supp}{\ensuremath\suppsym\xspace}

\def\wtypeh#1{\@ifnextchar\bgroup%
  {\mathchoice{\@lwtypeh{#1}}{\@twtypeh{#1}}{\@twtypeh{#1}}{\@twtypeh{#1}}\wtypeh}%
  {\mathchoice{\@@lwtypeh{#1}}{\@twtypeh{#1}}{\@twtypeh{#1}}{\@twtypeh{#1}}}}
\def\lwtypeh#1{\@ifnextchar\bgroup{\@lwtypeh{#1}\lwtypeh}{\@@lwtypeh{#1}}}
\def\@lwtypeh#1{\mathchoice{{\textstyle\mathsf{W}^h}}{\mathsf{W}^h}{\mathsf{W}^h}{\mathsf{W}^h}({\textstyle #1})\;}
\def\@@lwtypeh#1{\mathchoice{{\textstyle\mathsf{W}^h}}{\mathsf{W}^h}{\mathsf{W}^h}{\mathsf{W}^h}({\textstyle #1}),\ }
\def\twtypeh#1{\@twtypeh{#1}\@ifnextchar\bgroup{\twtypeh}{}}
\def\@twtypeh#1{\mathchoice{{\textstyle\mathsf{W}^h_{(#1)}}}{\mathsf{W}^h_{(#1)}}{\mathsf{W}^h_{(#1)}}{\mathsf{W}^h_{(#1)}}}
\def\dwtypeh#1{\@dwtypeh{#1}\@ifnextchar\bgroup{\dwtypeh}{}}
\def\@dwtypeh#1{\mathsf{W}^h_{(#1)}\,}


\makeatother

% Other notations related to dependent sums
\let\setof\Set    % from package 'braket', write \setof{ x:A | P(x) }.
\newcommand{\pair}{\ensuremath{\mathsf{pair}}\xspace}
\newcommand{\tup}[2]{(#1,#2)}
\newcommand{\proj}[1]{\ensuremath{\mathsf{pr}_{#1}}\xspace}
\newcommand{\fst}{\ensuremath{\proj1}\xspace}
\newcommand{\snd}{\ensuremath{\proj2}\xspace}
\newcommand{\ac}{\ensuremath{\mathsf{ac}}\xspace} % not needed in symbol index
\newcommand{\un}{\ensuremath{\mathsf{upun}}\xspace} % not needed in symbol index, uniqueness principle for unit type

%%% recursor and induction
\newcommand{\rec}[1]{\mathsf{rec}_{#1}}
\newcommand{\ind}[1]{\mathsf{ind}_{#1}}
\newcommand{\indid}[1]{\ind{=_{#1}}} % (Martin-Lof) path induction principle for identity types
\newcommand{\indidb}[1]{\ind{=_{#1}}'} % (Paulin-Mohring) based path induction principle for identity types 

%%% the uniqueness principle for product types, formerly called surjective pairing and named \spr:
\newcommand{\uppt}{\ensuremath{\mathsf{uppt}}\xspace}

% Paths in pairs
\newcommand{\pairpath}{\ensuremath{\mathsf{pair}^{\mathord{=}}}\xspace}
% \newcommand{\projpath}[1]{\proj{#1}^{\mathord{=}}}
\newcommand{\projpath}[1]{\ensuremath{\apfunc{\proj{#1}}}\xspace}

%%% For quotients %%%
%\newcommand{\pairr}[1]{{\langle #1\rangle}}
\newcommand{\pairr}[1]{{\mathopen{}(#1)\mathclose{}}}
\newcommand{\Pairr}[1]{{\mathopen{}\left(#1\right)\mathclose{}}}

% \newcommand{\type}{\ensuremath{\mathsf{Type}}} % this command is overridden below, so it's commented out
\newcommand{\im}{\ensuremath{\mathsf{im}}} % the image

%%% 2D path operations
\newcommand{\leftwhisker}{\mathbin{{\ct}_{\mathsf{l}}}}  % was \ell
\newcommand{\rightwhisker}{\mathbin{{\ct}_{\mathsf{r}}}} % was r
\newcommand{\hct}{\star}

%%% modalities %%%
\newcommand{\modal}{\ensuremath{\ocircle}}
\let\reflect\modal
\newcommand{\modaltype}{\ensuremath{\type_\modal}}
% \newcommand{\ism}[1]{\ensuremath{\mathsf{is}_{#1}}}
% \newcommand{\ismodal}{\ism{\modal}}
% \newcommand{\existsmodal}{\ensuremath{{\exists}_{\modal}}}
% \newcommand{\existsmodalunique}{\ensuremath{{\exists!}_{\modal}}}
% \newcommand{\modalfunc}{\textsf{\modal-fun}}
% \newcommand{\Ecirc}{\ensuremath{\mathsf{E}_\modal}}
% \newcommand{\Mcirc}{\ensuremath{\mathsf{M}_\modal}}
\newcommand{\mreturn}{\ensuremath{\eta}}
\let\project\mreturn
%\newcommand{\mbind}[1]{\ensuremath{\hat{#1}}}
\newcommand{\ext}{\mathsf{ext}}
%\newcommand{\mmap}[1]{\ensuremath{\bar{#1}}}
%\newcommand{\mjoin}{\ensuremath{\mreturn^{-1}}}
% Subuniverse
\renewcommand{\P}{\ensuremath{\type_{P}}\xspace}

%%% Localizations
% \newcommand{\islocal}[1]{\ensuremath{\mathsf{islocal}_{#1}}\xspace}
% \newcommand{\loc}[1]{\ensuremath{\mathcal{L}_{#1}}\xspace}

%%% Identity types %%%
\newcommand{\idsym}{{=}}
\newcommand{\id}[3][]{\ensuremath{#2 =_{#1} #3}\xspace}
\newcommand{\idtype}[3][]{\ensuremath{\mathsf{Id}_{#1}(#2,#3)}\xspace}
\newcommand{\idtypevar}[1]{\ensuremath{\mathsf{Id}_{#1}}\xspace}
% A propositional equality currently being defined
\newcommand{\defid}{\coloneqq}

%%% Dependent paths
\newcommand{\dpath}[4]{#3 =^{#1}_{#2} #4}

%%% singleton
% \newcommand{\sgl}{\ensuremath{\mathsf{sgl}}\xspace}
% \newcommand{\sctr}{\ensuremath{\mathsf{sctr}}\xspace}

%%% Reflexivity terms %%%
% \newcommand{\reflsym}{{\mathsf{refl}}}
\newcommand{\refl}[1]{\ensuremath{\mathsf{refl}_{#1}}\xspace}

%%% Path concatenation (used infix, in diagrammatic order) %%%
\newcommand{\ct}{%
  \mathchoice{\mathbin{\raisebox{0.5ex}{$\displaystyle\centerdot$}}}%
             {\mathbin{\raisebox{0.5ex}{$\centerdot$}}}%
             {\mathbin{\raisebox{0.25ex}{$\scriptstyle\,\centerdot\,$}}}%
             {\mathbin{\raisebox{0.1ex}{$\scriptscriptstyle\,\centerdot\,$}}}
}

%%% Path reversal %%%
\newcommand{\opp}[1]{\mathord{{#1}^{-1}}}
\let\rev\opp

%%% Transport (covariant) %%%
\newcommand{\trans}[2]{\ensuremath{{#1}_{*}\mathopen{}\left({#2}\right)\mathclose{}}\xspace}
\let\Trans\trans
%\newcommand{\Trans}[2]{\ensuremath{{#1}_{*}\left({#2}\right)}\xspace}
\newcommand{\transf}[1]{\ensuremath{{#1}_{*}}\xspace} % Without argument
%\newcommand{\transport}[2]{\ensuremath{\mathsf{transport}_{*} \: {#2}\xspace}}
\newcommand{\transfib}[3]{\ensuremath{\mathsf{transport}^{#1}(#2,#3)\xspace}}
\newcommand{\Transfib}[3]{\ensuremath{\mathsf{transport}^{#1}\Big(#2,\, #3\Big)\xspace}}
\newcommand{\transfibf}[1]{\ensuremath{\mathsf{transport}^{#1}\xspace}}

%%% 2D transport
\newcommand{\transtwo}[2]{\ensuremath{\mathsf{transport}^2\mathopen{}\left({#1},{#2}\right)\mathclose{}}\xspace}

%%% Constant transport
\newcommand{\transconst}[3]{\ensuremath{\mathsf{transportconst}}^{#1}_{#2}(#3)\xspace}
\newcommand{\transconstf}{\ensuremath{\mathsf{transportconst}}\xspace}

%%% Map on paths %%%
\newcommand{\mapfunc}[1]{\ensuremath{\mathsf{ap}_{#1}}\xspace} % Without argument
\newcommand{\map}[2]{\ensuremath{{#1}\mathopen{}\left({#2}\right)\mathclose{}}\xspace}
\let\Ap\map
%\newcommand{\Ap}[2]{\ensuremath{{#1}\left({#2}\right)}\xspace}
\newcommand{\mapdepfunc}[1]{\ensuremath{\mathsf{apd}_{#1}}\xspace} % Without argument
% \newcommand{\mapdep}[2]{\ensuremath{{#1}\llparenthesis{#2}\rrparenthesis}\xspace}
\newcommand{\mapdep}[2]{\ensuremath{\mapdepfunc{#1}\mathopen{}\left(#2\right)\mathclose{}}\xspace}
\let\apfunc\mapfunc
\let\ap\map
\let\apdfunc\mapdepfunc
\let\apd\mapdep

%%% 2D map on paths
\newcommand{\aptwofunc}[1]{\ensuremath{\mathsf{ap}^2_{#1}}\xspace}
\newcommand{\aptwo}[2]{\ensuremath{\aptwofunc{#1}\mathopen{}\left({#2}\right)\mathclose{}}\xspace}
\newcommand{\apdtwofunc}[1]{\ensuremath{\mathsf{apd}^2_{#1}}\xspace}
\newcommand{\apdtwo}[2]{\ensuremath{\apdtwofunc{#1}\mathopen{}\left(#2\right)\mathclose{}}\xspace}

%%% Identity functions %%%
\newcommand{\idfunc}[1][]{\ensuremath{\mathsf{id}_{#1}}\xspace}

%%% Homotopies (written infix) %%%
\newcommand{\htpy}{\sim}

%%% Other meanings of \sim
\newcommand{\bisim}{\sim}       % bisimulation
\newcommand{\eqr}{\sim}         % an equivalence relation

%%% Equivalence types %%%
\newcommand{\eqv}[2]{\ensuremath{#1 \simeq #2}\xspace}
\newcommand{\eqvspaced}[2]{\ensuremath{#1 \;\simeq\; #2}\xspace}
\newcommand{\eqvsym}{\simeq}    % infix symbol
\newcommand{\texteqv}[2]{\ensuremath{\mathsf{Equiv}(#1,#2)}\xspace}
\newcommand{\isequiv}{\ensuremath{\mathsf{isequiv}}}
\newcommand{\qinv}{\ensuremath{\mathsf{qinv}}}
\newcommand{\ishae}{\ensuremath{\mathsf{ishae}}}
\newcommand{\linv}{\ensuremath{\mathsf{linv}}}
\newcommand{\rinv}{\ensuremath{\mathsf{rinv}}}
\newcommand{\biinv}{\ensuremath{\mathsf{biinv}}}
\newcommand{\lcoh}[3]{\mathsf{lcoh}_{#1}(#2,#3)}
\newcommand{\rcoh}[3]{\mathsf{rcoh}_{#1}(#2,#3)}
\newcommand{\hfib}[2]{{\mathsf{fib}}_{#1}(#2)}

%%% Map on total spaces %%%
\newcommand{\total}[1]{\ensuremath{\mathsf{total}(#1)}}

%%% Universe types %%%
%\newcommand{\type}{\ensuremath{\mathsf{Type}}\xspace}
\newcommand{\UU}{\ensuremath{\mathcal{U}}\xspace}
\let\bbU\UU
\let\type\UU
% Universes of truncated types
\newcommand{\typele}[1]{\ensuremath{{#1}\text-\mathsf{Type}}\xspace}
\newcommand{\typeleU}[1]{\ensuremath{{#1}\text-\mathsf{Type}_\UU}\xspace}
\newcommand{\typelep}[1]{\ensuremath{{(#1)}\text-\mathsf{Type}}\xspace}
\newcommand{\typelepU}[1]{\ensuremath{{(#1)}\text-\mathsf{Type}_\UU}\xspace}
\let\ntype\typele
\let\ntypeU\typeleU
\let\ntypep\typelep
\let\ntypepU\typelepU
\renewcommand{\set}{\ensuremath{\mathsf{Set}}\xspace}
\newcommand{\setU}{\ensuremath{\mathsf{Set}_\UU}\xspace}
\newcommand{\prop}{\ensuremath{\mathsf{Prop}}\xspace}
\newcommand{\propU}{\ensuremath{\mathsf{Prop}_\UU}\xspace}
%Pointed types
\newcommand{\pointed}[1]{\ensuremath{#1_\bullet}}

%%% Ordinals and cardinals
\newcommand{\card}{\ensuremath{\mathsf{Card}}\xspace}
\newcommand{\ord}{\ensuremath{\mathsf{Ord}}\xspace}
\newcommand{\ordsl}[2]{{#1}_{/#2}}

%%% Univalence
\newcommand{\ua}{\ensuremath{\mathsf{ua}}\xspace} % the inverse of idtoeqv
\newcommand{\idtoeqv}{\ensuremath{\mathsf{idtoeqv}}\xspace}
\newcommand{\univalence}{\ensuremath{\mathsf{univalence}}\xspace} % the full axiom

%%% Truncation levels
\newcommand{\iscontr}{\ensuremath{\mathsf{isContr}}}
\newcommand{\contr}{\ensuremath{\mathsf{contr}}} % The path to the center of contraction
\newcommand{\isset}{\ensuremath{\mathsf{isSet}}}
\newcommand{\isprop}{\ensuremath{\mathsf{isProp}}}
% h-propositions
% \newcommand{\anhprop}{a mere proposition\xspace}
% \newcommand{\hprops}{mere propositions\xspace}

%%% Homotopy fibers %%%
%\newcommand{\hfiber}[2]{\ensuremath{\mathsf{hFiber}(#1,#2)}\xspace}
\let\hfiber\hfib

%%% Bracket/squash/truncation types %%%
% \newcommand{\brck}[1]{\textsf{mere}(#1)}
% \newcommand{\Brck}[1]{\textsf{mere}\Big(#1\Big)}
% \newcommand{\trunc}[2]{\tau_{#1}(#2)}
% \newcommand{\Trunc}[2]{\tau_{#1}\Big(#2\Big)}
% \newcommand{\truncf}[1]{\tau_{#1}}
%\newcommand{\trunc}[2]{\Vert #2\Vert_{#1}}
\newcommand{\trunc}[2]{\mathopen{}\left\Vert #2\right\Vert_{#1}\mathclose{}}
\newcommand{\ttrunc}[2]{\bigl\Vert #2\bigr\Vert_{#1}}
\newcommand{\Trunc}[2]{\Bigl\Vert #2\Bigr\Vert_{#1}}
\newcommand{\truncf}[1]{\Vert \blank \Vert_{#1}}
\newcommand{\tproj}[3][]{\mathopen{}\left|#3\right|_{#2}^{#1}\mathclose{}}
\newcommand{\tprojf}[2][]{|\blank|_{#2}^{#1}}
\def\pizero{\trunc0}
%\newcommand{\brck}[1]{\trunc{-1}{#1}}
%\newcommand{\Brck}[1]{\Trunc{-1}{#1}}
%\newcommand{\bproj}[1]{\tproj{-1}{#1}}
%\newcommand{\bprojf}{\tprojf{-1}}

\newcommand{\brck}[1]{\trunc{}{#1}}
\newcommand{\bbrck}[1]{\ttrunc{}{#1}}
\newcommand{\Brck}[1]{\Trunc{}{#1}}
\newcommand{\bproj}[1]{\tproj{}{#1}}
\newcommand{\bprojf}{\tprojf{}}

% Big parentheses
\newcommand{\Parens}[1]{\Bigl(#1\Bigr)}

% Projection and extension for truncations
\let\extendsmb\ext
\newcommand{\extend}[1]{\extendsmb(#1)}

%
%%% The empty type
\newcommand{\emptyt}{\ensuremath{\mathbf{0}}\xspace}

%%% The unit type
\newcommand{\unit}{\ensuremath{\mathbf{1}}\xspace}
\newcommand{\ttt}{\ensuremath{\star}\xspace}

%%% The two-element type
\newcommand{\bool}{\ensuremath{\mathbf{2}}\xspace}
\newcommand{\btrue}{{1_{\bool}}}
\newcommand{\bfalse}{{0_{\bool}}}

%%% Injections into binary sums and pushouts
\newcommand{\inlsym}{{\mathsf{inl}}}
\newcommand{\inrsym}{{\mathsf{inr}}}
\newcommand{\inl}{\ensuremath\inlsym\xspace}
\newcommand{\inr}{\ensuremath\inrsym\xspace}

%%% The segment of the interval
\newcommand{\seg}{\ensuremath{\mathsf{seg}}\xspace}

%%% Free groups
\newcommand{\freegroup}[1]{F(#1)}
\newcommand{\freegroupx}[1]{F'(#1)} % the "other" free group

%%% Glue of a pushout
\newcommand{\glue}{\mathsf{glue}}

%%% Circles and spheres
\newcommand{\Sn}{\mathbb{S}}
\newcommand{\base}{\ensuremath{\mathsf{base}}\xspace}
\newcommand{\lloop}{\ensuremath{\mathsf{loop}}\xspace}
\newcommand{\surf}{\ensuremath{\mathsf{surf}}\xspace}

%%% Suspension
\newcommand{\susp}{\Sigma}
\newcommand{\north}{\mathsf{N}}
\newcommand{\south}{\mathsf{S}}
\newcommand{\merid}{\mathsf{merid}}

%%% Blanks (shorthand for lambda abstractions)
\newcommand{\blank}{\mathord{\hspace{1pt}\text{--}\hspace{1pt}}}

%%% Nameless objects
\newcommand{\nameless}{\mathord{\hspace{1pt}\underline{\hspace{1ex}}\hspace{1pt}}}

%%% Some decorations
%\newcommand{\bbU}{\ensuremath{\mathbb{U}}\xspace}
% \newcommand{\bbB}{\ensuremath{\mathbb{B}}\xspace}
\newcommand{\bbP}{\ensuremath{\mathbb{P}}\xspace}

%%% Some categories
\newcommand{\uset}{\ensuremath{\mathcal{S}et}\xspace}
\newcommand{\ucat}{\ensuremath{{\mathcal{C}at}}\xspace}
\newcommand{\urel}{\ensuremath{\mathcal{R}el}\xspace}
\newcommand{\uhilb}{\ensuremath{\mathcal{H}ilb}\xspace}
\newcommand{\utype}{\ensuremath{\mathcal{T}\!ype}\xspace}

% Pullback corner
\newbox\pbbox
\setbox\pbbox=\hbox{\xy \POS(65,0)\ar@{-} (0,0) \ar@{-} (65,65)\endxy}
\def\pb{\save[]+<3.5mm,-3.5mm>*{\copy\pbbox} \restore}

% Macros for the categories chapter
\newcommand{\inv}[1]{{#1}^{-1}}
\newcommand{\idtoiso}{\ensuremath{\mathsf{idtoiso}}\xspace}
\newcommand{\isotoid}{\ensuremath{\mathsf{isotoid}}\xspace}
\newcommand{\op}{^{\mathrm{op}}}
\newcommand{\y}{\ensuremath{\mathbf{y}}\xspace}
\newcommand{\dgr}[1]{{#1}^{\dagger}}
\newcommand{\unitaryiso}{\mathrel{\cong^\dagger}}
\newcommand{\cteqv}[2]{\ensuremath{#1 \simeq #2}\xspace}
\newcommand{\cteqvsym}{\simeq}     % Symbol for equivalence of categories

%%% Natural numbers
\newcommand{\N}{\ensuremath{\mathbb{N}}\xspace}
%\newcommand{\N}{\textbf{N}}
\let\nat\N
\newcommand{\natp}{\ensuremath{\nat'}\xspace} % alternative nat in induction chapter

\newcommand{\zerop}{\ensuremath{0'}\xspace}   % alternative zero in induction chapter
\newcommand{\suc}{\mathsf{succ}}
\newcommand{\sucp}{\ensuremath{\suc'}\xspace} % alternative suc in induction chapter
\newcommand{\add}{\mathsf{add}}
\newcommand{\ack}{\mathsf{ack}}
\newcommand{\ite}{\mathsf{iter}}
\newcommand{\assoc}{\mathsf{assoc}}
\newcommand{\dbl}{\ensuremath{\mathsf{double}}}
\newcommand{\dblp}{\ensuremath{\dbl'}\xspace} % alternative double in induction chapter


%%% Lists
\newcommand{\lst}[1]{\mathsf{List}(#1)}
\newcommand{\nil}{\mathsf{nil}}
\newcommand{\cons}{\mathsf{cons}}
\newcommand{\lost}[1]{\mathsf{Lost}(#1)}

%%% Vectors of given length, used in induction chapter
\newcommand{\vect}[2]{\ensuremath{\mathsf{Vec}_{#1}(#2)}\xspace}

%%% Integers
\newcommand{\Z}{\ensuremath{\mathbb{Z}}\xspace}
\newcommand{\Zsuc}{\mathsf{succ}}
\newcommand{\Zpred}{\mathsf{pred}}

%%% Rationals
\newcommand{\Q}{\ensuremath{\mathbb{Q}}\xspace}

%%% Function extensionality
\newcommand{\funext}{\mathsf{funext}}
\newcommand{\happly}{\mathsf{happly}}

%%% A naturality lemma
\newcommand{\com}[3]{\mathsf{swap}_{#1,#2}(#3)}

%%% Code/encode/decode
\newcommand{\code}{\ensuremath{\mathsf{code}}\xspace}
\newcommand{\encode}{\ensuremath{\mathsf{encode}}\xspace}
\newcommand{\decode}{\ensuremath{\mathsf{decode}}\xspace}

% Function definition with domain and codomain
\newcommand{\function}[4]{\left\{\begin{array}{rcl}#1 &
      \longrightarrow & #2 \\ #3 & \longmapsto & #4 \end{array}\right.}

%%% Cones and cocones
\newcommand{\cone}[2]{\mathsf{cone}_{#1}(#2)}
\newcommand{\cocone}[2]{\mathsf{cocone}_{#1}(#2)}
% Apply a function to a cocone
\newcommand{\composecocone}[2]{#1\circ#2}
\newcommand{\composecone}[2]{#2\circ#1}
%%% Diagrams
\newcommand{\Ddiag}{\mathscr{D}}

%%% (pointed) mapping spaces
\newcommand{\Map}{\mathsf{Map}}

%%% The interval
\newcommand{\interval}{\ensuremath{I}\xspace}
\newcommand{\izero}{\ensuremath{0_{\interval}}\xspace}
\newcommand{\ione}{\ensuremath{1_{\interval}}\xspace}

%%% Arrows
\newcommand{\epi}{\ensuremath{\twoheadrightarrow}}
\newcommand{\mono}{\ensuremath{\rightarrowtail}}

%%% Sets
\newcommand{\bin}{\ensuremath{\mathrel{\widetilde{\in}}}}

%%% Semigroup structure
\newcommand{\semigroupstrsym}{\ensuremath{\mathsf{SemigroupStr}}}
\newcommand{\semigroupstr}[1]{\ensuremath{\mathsf{SemigroupStr}}(#1)}
\newcommand{\semigroup}[0]{\ensuremath{\mathsf{Semigroup}}}

%%% Macros for the formal type theory
\newcommand{\emptyctx}{\ensuremath{\cdot}}
\newcommand{\production}{\vcentcolon\vcentcolon=}
\newcommand{\conv}{\downarrow}
\newcommand{\ctx}{\ensuremath{\mathsf{ctx}}}
\newcommand{\wfctx}[1]{#1\ \ctx}
\newcommand{\oftp}[3]{#1 \vdash #2 : #3}
\newcommand{\jdeqtp}[4]{#1 \vdash #2 \jdeq #3 : #4}
\newcommand{\judg}[2]{#1 \vdash #2}
\newcommand{\tmtp}[2]{#1 \mathord{:} #2}

% rule names
\newcommand{\form}{\textsc{form}}
\newcommand{\intro}{\textsc{intro}}
\newcommand{\elim}{\textsc{elim}}
\newcommand{\comp}{\textsc{comp}}
\newcommand{\uniq}{\textsc{uniq}}
\newcommand{\Weak}{\mathsf{Wkg}}
\newcommand{\Vble}{\mathsf{Vble}}
\newcommand{\Exch}{\mathsf{Exch}}
\newcommand{\Subst}{\mathsf{Subst}}

%%% Macros for HITs
\newcommand{\cc}{\mathsf{c}}
\newcommand{\pp}{\mathsf{p}}
\newcommand{\cct}{\widetilde{\mathsf{c}}}
\newcommand{\ppt}{\widetilde{\mathsf{p}}}
\newcommand{\Wtil}{\ensuremath{\widetilde{W}}\xspace}

%%% Macros for n-types
\newcommand{\istype}[1]{\mathsf{is}\mbox{-}{#1}\mbox{-}\mathsf{type}}
\newcommand{\nplusone}{\ensuremath{(n+1)}}
\newcommand{\nminusone}{\ensuremath{(n-1)}}
\newcommand{\fact}{\mathsf{fact}}

%%% Macros for homotopy
\newcommand{\kbar}{\overline{k}} % Used in van Kampen's theorem

%%% Macros for induction
\newcommand{\natw}{\ensuremath{\mathbf{N^w}}\xspace}
\newcommand{\zerow}{\ensuremath{0^\mathbf{w}}\xspace}
\newcommand{\sucw}{\ensuremath{\mathsf{succ}^{\mathbf{w}}}\xspace}
\newcommand{\nalg}{\nat\mathsf{Alg}}
\newcommand{\nhom}{\nat\mathsf{Hom}}
\newcommand{\ishinitw}{\mathsf{isHinit}_{\mathsf{W}}}
\newcommand{\ishinitn}{\mathsf{isHinit}_\nat}
\newcommand{\w}{\mathsf{W}}
\newcommand{\walg}{\w\mathsf{Alg}}
\newcommand{\whom}{\w\mathsf{Hom}}

%%% Macros for real numbers
\newcommand{\RC}{\ensuremath{\mathbb{R}_\mathsf{c}}\xspace} % Cauchy
\newcommand{\RD}{\ensuremath{\mathbb{R}_\mathsf{d}}\xspace} % Dedekind
\newcommand{\R}{\ensuremath{\mathbb{R}}\xspace}           % Either 
\newcommand{\barRD}{\ensuremath{\bar{\mathbb{R}}_\mathsf{d}}\xspace} % Dedekind completion of Dedekind

\newcommand{\close}[1]{\sim_{#1}} % Relation of closeness
\newcommand{\closesym}{\mathord\sim}
\newcommand{\rclim}{\mathsf{lim}} % HIT constructor for Cauchy reals
\newcommand{\rcrat}{\mathsf{rat}} % Embedding of rationals into Cauchy reals
\newcommand{\rceq}{\mathsf{eq}_{\RC}} % HIT path constructor
\newcommand{\CAP}{\mathcal{C}}    % The type of Cauchy approximations
\newcommand{\Qp}{\Q_{+}}
\newcommand{\apart}{\mathrel{\#}}  % apartness
\newcommand{\dcut}{\mathsf{isCut}}  % Dedekind cut
\newcommand{\cover}{\triangleleft} % inductive cover
\newcommand{\intfam}[3]{(#2, \lam{#1} #3)} % family of rational intervals

% Macros for the Cauchy reals construction
\newcommand{\bsim}{\frown}
\newcommand{\bbsim}{\smile}

\newcommand{\hapx}{\diamondsuit\approx}
\newcommand{\hapname}{\diamondsuit}
\newcommand{\hapxb}{\heartsuit\approx}
\newcommand{\hapbname}{\heartsuit}
\newcommand{\tap}[1]{\bullet\approx_{#1}\triangle}
\newcommand{\tapname}{\triangle}
\newcommand{\tapb}[1]{\bullet\approx_{#1}\square}
\newcommand{\tapbname}{\square}

%%% Macros for surreals
\newcommand{\NO}{\ensuremath{\mathsf{No}}\xspace}
\newcommand{\surr}[2]{\{\,#1\,\big|\,#2\,\}}
\newcommand{\LL}{\mathcal{L}}
\newcommand{\RR}{\mathcal{R}}
\newcommand{\noeq}{\mathsf{eq}_{\NO}} % HIT path constructor

\newcommand{\ble}{\trianglelefteqslant}
\newcommand{\blt}{\vartriangleleft}
\newcommand{\bble}{\sqsubseteq}
\newcommand{\bblt}{\sqsubset}

\newcommand{\hle}{\diamondsuit\preceq}
\newcommand{\hlt}{\diamondsuit\prec}
\newcommand{\hlname}{\diamondsuit}
\newcommand{\hleb}{\heartsuit\preceq}
\newcommand{\hltb}{\heartsuit\prec}
\newcommand{\hlbname}{\heartsuit}
% \newcommand{\tle}{(\bullet\preceq\triangle)}
% \newcommand{\tlt}{(\bullet\prec\triangle)}
\newcommand{\tle}{\triangle\preceq}
\newcommand{\tlt}{\triangle\prec}
\newcommand{\tlname}{\triangle}
% \newcommand{\tleb}{(\bullet\preceq\square)}
% \newcommand{\tltb}{(\bullet\prec\square)}
\newcommand{\tleb}{\square\preceq}
\newcommand{\tltb}{\square\prec}
\newcommand{\tlbname}{\square}

%%% Macros for set theory
\newcommand{\vset}{\mathsf{set}}  % point constructor for cummulative hierarchy V
\def\cd{\tproj0}
\newcommand{\inj}{\ensuremath{\mathsf{inj}}} % type of injections
\newcommand{\acc}{\ensuremath{\mathsf{acc}}} % accessibility

\newcommand{\atMostOne}{\mathsf{atMostOne}}

\newcommand{\power}[1]{\mathcal{P}(#1)} % power set
\newcommand{\powerp}[1]{\mathcal{P}_+(#1)} % inhabited power set

%%%% THEOREM ENVIRONMENTS %%%%

% Hyperref includes the command \autoref{...} which is like \ref{...}
% except that it automatically inserts the type of the thing you're
% referring to, e.g. it produces "Theorem 3.8" instead of just "3.8"
% (and makes the whole thing a hyperlink).  This saves a slight amount
% of typing, but more importantly it means that if you decide later on
% that 3.8 should be a Lemma or a Definition instead of a Theorem, you
% don't have to change the name in all the places you referred to it.

% The following hack improves on this by using the same counter for
% all theorem-type environments, so that after Theorem 1.1 comes
% Corollary 1.2 rather than Corollary 1.1.  This makes it much easier
% for the reader to find a particular theorem when flipping through
% the document.
\makeatletter
\def\defthm#1#2#3{%
  %% Ensure all theorem types are numbered with the same counter
  \newaliascnt{#1}{thm}
  \newtheorem{#1}[#1]{#2}
  \aliascntresetthe{#1}
  %% This command tells cleveref's \cref what to call things
  \crefname{#1}{#2}{#3}}

% Now define a bunch of theorem-type environments.
\newtheorem{thm}{Theorem}[section]
\crefname{thm}{Theorem}{Theorems}
%\defthm{prop}{Proposition}   % Probably we shouldn't use "Proposition" in this way
\defthm{cor}{Corollary}{Corollaries}
\defthm{lem}{Lemma}{Lemmas}
\defthm{axiom}{Axiom}{Axioms}
% Since definitions and theorems in type theory are synonymous, should
% we actually use the same theoremstyle for them?
\theoremstyle{definition}
\defthm{defn}{Definition}{Definitions}
\theoremstyle{remark}
\defthm{rmk}{Remark}{Remarks}
\defthm{eg}{Example}{Examples}
\defthm{egs}{Examples}{Examples}
\defthm{notes}{Notes}{Notes}
% Number exercises within chapters, with their own counter.
\newtheorem{ex}{Exercise}[chapter]
\crefname{ex}{Exercise}{Exercises}

% Display format for sections
\crefformat{section}{\S#2#1#3}
\Crefformat{section}{Section~#2#1#3}
\crefrangeformat{section}{\S\S#3#1#4--#5#2#6}
\Crefrangeformat{section}{Sections~#3#1#4--#5#2#6}
\crefmultiformat{section}{\S\S#2#1#3}{ and~#2#1#3}{, #2#1#3}{ and~#2#1#3}
\Crefmultiformat{section}{Sections~#2#1#3}{ and~#2#1#3}{, #2#1#3}{ and~#2#1#3}
\crefrangemultiformat{section}{\S\S#3#1#4--#5#2#6}{ and~#3#1#4--#5#2#6}{, #3#1#4--#5#2#6}{ and~#3#1#4--#5#2#6}
\Crefrangemultiformat{section}{Sections~#3#1#4--#5#2#6}{ and~#3#1#4--#5#2#6}{, #3#1#4--#5#2#6}{ and~#3#1#4--#5#2#6}

% Display format for appendices
\crefformat{appendix}{Appendix~#2#1#3}
\Crefformat{appendix}{Appendix~#2#1#3}
\crefrangeformat{appendix}{Appendices~#3#1#4--#5#2#6}
\Crefrangeformat{appendix}{Appendices~#3#1#4--#5#2#6}
\crefmultiformat{appendix}{Appendices~#2#1#3}{ and~#2#1#3}{, #2#1#3}{ and~#2#1#3}
\Crefmultiformat{appendix}{Appendices~#2#1#3}{ and~#2#1#3}{, #2#1#3}{ and~#2#1#3}
\crefrangemultiformat{appendix}{Appendices~#3#1#4--#5#2#6}{ and~#3#1#4--#5#2#6}{, #3#1#4--#5#2#6}{ and~#3#1#4--#5#2#6}
\Crefrangemultiformat{appendix}{Appendices~#3#1#4--#5#2#6}{ and~#3#1#4--#5#2#6}{, #3#1#4--#5#2#6}{ and~#3#1#4--#5#2#6}

\crefname{part}{Part}{Parts}

% Number subsubsections
\setcounter{secnumdepth}{5}

% Display format for figures
\crefname{figure}{Figure}{Figures}

% Use cleveref instead of hyperref's \autoref
\let\autoref\cref


%%%% EQUATION NUMBERING %%%%

% The following hack uses the single theorem counter to number
% equations as well, so that we don't have both Theorem 1.1 and
% equation (1.1).
\let\c@equation\c@thm
\numberwithin{equation}{section}


%%%% ENUMERATE NUMBERING %%%%

% Number the first level of enumerates as (i), (ii), ...
\renewcommand{\theenumi}{(\roman{enumi})}
\renewcommand{\labelenumi}{\theenumi}


%%%% MARGINS %%%%

% This is a matter of personal preference, but I think the left
% margins on enumerates and itemizes are too wide.
\setitemize[1]{leftmargin=2em}
\setenumerate[1]{leftmargin=*}

% Likewise that they are too spaced out.
\setitemize[1]{itemsep=-0.2em}
\setenumerate[1]{itemsep=-0.2em}

%%% Notes %%%
\def\noteson{%
\gdef\note##1{\mbox{}\marginpar{\color{blue}\textasteriskcentered\ ##1}}}
\gdef\notesoff{\gdef\note##1{\null}}
\noteson

\newcommand{\Coq}{\textsc{Coq}\xspace}
\newcommand{\Agda}{\textsc{Agda}\xspace}
\newcommand{\NuPRL}{\textsc{NuPRL}\xspace}

%%%% CITATIONS %%%%

% \let \cite \citep

%%%% INDEX %%%%

\newcommand{\footstyle}[1]{{\hyperpage{#1}}n} % If you index something that is in a footnote
\newcommand{\defstyle}[1]{\textbf{\hyperpage{#1}}}  % Style for pageref to a definition

\newcommand{\indexdef}[1]{\index{#1|defstyle}}   % Index a definition
\newcommand{\indexfoot}[1]{\index{#1|footstyle}} % Index a term in a footnote
\newcommand{\indexsee}[2]{\index{#1|see{#2}}}    % Index "see also"


%%%% Standard phrasing or spelling of common phrases %%%%

\newcommand{\ZF}{Zermelo--Fraenkel}
\newcommand{\CZF}{Constructive \ZF{} Set Theory}

\newcommand{\LEM}[1]{\ensuremath{\mathsf{LEM}_{#1}}\xspace}
\newcommand{\choice}[1]{\ensuremath{\mathsf{AC}_{#1}}\xspace}

%%%% MISC %%%%

\newcommand{\mentalpause}{\medskip} % Use for "mental" pause, instead of \smallskip or \medskip

%% Use \symlabel instead of \label to mark a pageref that you need in the index of symbols
\newcounter{symindex}
\newcommand{\symlabel}[1]{\refstepcounter{symindex}\label{#1}}

% Local Variables:
% mode: latex
% TeX-master: "hott-online"
% End:


%%%% Indexing
\usepackage{makeidx}
\makeindex

%%%% Header and footers
\pagestyle{fancyplain}
\setlength{\headheight}{15pt}
\renewcommand{\sectionmark}[1]{\markright{\textsc{\thesection\ #1}}}

\lhead[\fancyplain{}{{\thepage}}]%
      {\fancyplain{}{\nouppercase{\rightmark}}}
\rhead[\fancyplain{}{\nouppercase{\leftmark}}]%
      {\fancyplain{}{\thepage}}
\cfoot[]{}
\lfoot[]{}
\rfoot[]{}

%%%% Chapter & part style
\usepackage{titlesec}
\titleformat{\part}[display]{\fontsize{40}{40}\fontseries{m}\fontshape{sc}\selectfont}{\hfil\partname\ \Roman{part}}{20pt}{\fontsize{60}{60}\fontseries{b}\fontshape{sc}\selectfont\hfil}
\titleformat{\chapter}[display]{\fontsize{23}{25}\fontseries{m}\fontshape{it}\selectfont}{\chaptertitlename\ \thechapter}{20pt}{\fontsize{35}{35}\fontseries{b}\fontshape{n}\selectfont}

%\nonstopmode
\errorstopmode
% THIS IS NOT THE FILE YOU SHOULD PROCESS. IT IS THE "MAIN" FILE,
% BUT IT GETS INCLUDED BY ONE OF THE hott-xxx.tex FILES. THOSE ARE
% THE MAIN ONES.

% DOCUMENT CLASS
\documentclass[epub={split at=section},text,itex,class=book]{internet}

\usepackage{xspace}
\usepackage{hyperref}
\usepackage{aliascnt}
\def\appendix{}
\usepackage[capitalize]{cleveref}
\usepackage{braket}

%%%% MACROS FOR NOTATION %%%%
% Use these for any notation where there are multiple options.

%%% Notes and exercise sections
\makeatletter
\newcommand{\sectionNotes}{\phantomsection\section*{Notes}\addcontentsline{toc}{section}{Notes}\markright{\textsc{\@chapapp{} \thechapter{} Notes}}}
\newcommand{\sectionExercises}[1]{\phantomsection\section*{Exercises}\addcontentsline{toc}{section}{Exercises}\markright{\textsc{\@chapapp{} \thechapter{} Exercises}}}
\makeatother

%%% Definitional equality (used infix) %%%
\newcommand{\jdeq}{\equiv}      % An equality judgment
\let\judgeq\jdeq
%\newcommand{\defeq}{\coloneqq}  % An equality currently being defined
\newcommand{\defeq}{\vcentcolon\equiv}  % A judgmental equality currently being defined

%%% Term being defined
\newcommand{\define}[1]{\textbf{#1}}

%%% Vec (for example)

\newcommand{\Vect}{\ensuremath{\mathsf{Vec}}}
\newcommand{\Fin}{\ensuremath{\mathsf{Fin}}}
\newcommand{\fmax}{\ensuremath{\mathsf{fmax}}}
\newcommand{\seq}[1]{\langle #1\rangle}

%%% Dependent products %%%
\def\prdsym{\textstyle\prod}
%% Call the macro like \prd{x,y:A}{p:x=y} with any number of
%% arguments.  Make sure that whatever comes *after* the call doesn't
%% begin with an open-brace, or it will be parsed as another argument.
\makeatletter
% Currently the macro is configured to produce
%     {\textstyle\prod}(x:A) \; {\textstyle\prod}(y:B),\ 
% in display-math mode, and
%     \prod_{(x:A)} \prod_{y:B}
% in text-math mode.
% \def\prd#1{\@ifnextchar\bgroup{\prd@parens{#1}}{%
%     \@ifnextchar\sm{\prd@parens{#1}\@eatsm}{%
%         \prd@noparens{#1}}}}
\def\prd#1{\@ifnextchar\bgroup{\prd@parens{#1}}{%
    \@ifnextchar\sm{\prd@parens{#1}\@eatsm}{%
    \@ifnextchar\prd{\prd@parens{#1}\@eatprd}{%
    \@ifnextchar\;{\prd@parens{#1}\@eatsemicolonspace}{%
    \@ifnextchar\\{\prd@parens{#1}\@eatlinebreak}{%
    \@ifnextchar\narrowbreak{\prd@parens{#1}\@eatnarrowbreak}{%
      \prd@noparens{#1}}}}}}}}
\def\prd@parens#1{\@ifnextchar\bgroup%
  {\mathchoice{\@dprd{#1}}{\@tprd{#1}}{\@tprd{#1}}{\@tprd{#1}}\prd@parens}%
  {\@ifnextchar\sm%
    {\mathchoice{\@dprd{#1}}{\@tprd{#1}}{\@tprd{#1}}{\@tprd{#1}}\@eatsm}%
    {\mathchoice{\@dprd{#1}}{\@tprd{#1}}{\@tprd{#1}}{\@tprd{#1}}}}}
\def\@eatsm\sm{\sm@parens}
\def\prd@noparens#1{\mathchoice{\@dprd@noparens{#1}}{\@tprd{#1}}{\@tprd{#1}}{\@tprd{#1}}}
% Helper macros for three styles
\def\lprd#1{\@ifnextchar\bgroup{\@lprd{#1}\lprd}{\@@lprd{#1}}}
\def\@lprd#1{\mathchoice{{\textstyle\prod}}{\prod}{\prod}{\prod}({\textstyle #1})\;}
\def\@@lprd#1{\mathchoice{{\textstyle\prod}}{\prod}{\prod}{\prod}({\textstyle #1}),\ }
\def\tprd#1{\@tprd{#1}\@ifnextchar\bgroup{\tprd}{}}
\def\@tprd#1{\mathchoice{{\textstyle\prod_{(#1)}}}{\prod_{(#1)}}{\prod_{(#1)}}{\prod_{(#1)}}}
\def\dprd#1{\@dprd{#1}\@ifnextchar\bgroup{\dprd}{}}
\def\@dprd#1{\prod_{(#1)}\,}
\def\@dprd@noparens#1{\prod_{#1}\,}

% Look through spaces and linebreaks
\def\@eatnarrowbreak\narrowbreak{%
  \@ifnextchar\prd{\narrowbreak\@eatprd}{%
    \@ifnextchar\sm{\narrowbreak\@eatsm}{%
      \narrowbreak}}}
\def\@eatlinebreak\\{%
  \@ifnextchar\prd{\\\@eatprd}{%
    \@ifnextchar\sm{\\\@eatsm}{%
      \\}}}
\def\@eatsemicolonspace\;{%
  \@ifnextchar\prd{\;\@eatprd}{%
    \@ifnextchar\sm{\;\@eatsm}{%
      \;}}}

%%% Lambda abstractions.
% Each variable being abstracted over is a separate argument.  If
% there is more than one such argument, they *must* be enclosed in
% braces.  Arguments can be untyped, as in \lam{x}{y}, or typed with a
% colon, as in \lam{x:A}{y:B}. In the latter case, the colons are
% automatically noticed and (with current implementation) the space
% around the colon is reduced.  You can even give more than one variable
% the same type, as in \lam{x,y:A}.
\def\lam#1{{\lambda}\@lamarg#1:\@endlamarg\@ifnextchar\bgroup{.\,\lam}{.\,}}
\def\@lamarg#1:#2\@endlamarg{\if\relax\detokenize{#2}\relax #1\else\@lamvar{\@lameatcolon#2},#1\@endlamvar\fi}
\def\@lamvar#1,#2\@endlamvar{(#2\,{:}\,#1)}
% \def\@lamvar#1,#2{{#2}^{#1}\@ifnextchar,{.\,{\lambda}\@lamvar{#1}}{\let\@endlamvar\relax}}
\def\@lameatcolon#1:{#1}
\let\lamt\lam
% This version silently eats any typing annotation.
\def\lamu#1{{\lambda}\@lamuarg#1:\@endlamuarg\@ifnextchar\bgroup{.\,\lamu}{.\,}}
\def\@lamuarg#1:#2\@endlamuarg{#1}

%%% Dependent products written with \forall, in the same style
\def\fall#1{\forall (#1)\@ifnextchar\bgroup{.\,\fall}{.\,}}

%%% Existential quantifier %%%
\def\exis#1{\exists (#1)\@ifnextchar\bgroup{.\,\exis}{.\,}}

%%% Dependent sums %%%
\def\smsym{\textstyle\sum}
% Use in the same way as \prd
\def\sm#1{\@ifnextchar\bgroup{\sm@parens{#1}}{%
    \@ifnextchar\prd{\sm@parens{#1}\@eatprd}{%
    \@ifnextchar\sm{\sm@parens{#1}\@eatsm}{%
    \@ifnextchar\;{\sm@parens{#1}\@eatsemicolonspace}{%
    \@ifnextchar\\{\sm@parens{#1}\@eatlinebreak}{%
    \@ifnextchar\narrowbreak{\sm@parens{#1}\@eatnarrowbreak}{%
        \sm@noparens{#1}}}}}}}}
\def\sm@parens#1{\@ifnextchar\bgroup%
  {\mathchoice{\@dsm{#1}}{\@tsm{#1}}{\@tsm{#1}}{\@tsm{#1}}\sm@parens}%
  {\@ifnextchar\prd%
    {\mathchoice{\@dsm{#1}}{\@tsm{#1}}{\@tsm{#1}}{\@tsm{#1}}\@eatprd}%
    {\mathchoice{\@dsm{#1}}{\@tsm{#1}}{\@tsm{#1}}{\@tsm{#1}}}}}
\def\@eatprd\prd{\prd@parens}
\def\sm@noparens#1{\mathchoice{\@dsm@noparens{#1}}{\@tsm{#1}}{\@tsm{#1}}{\@tsm{#1}}}
\def\lsm#1{\@ifnextchar\bgroup{\@lsm{#1}\lsm}{\@@lsm{#1}}}
\def\@lsm#1{\mathchoice{{\textstyle\sum}}{\sum}{\sum}{\sum}({\textstyle #1})\;}
\def\@@lsm#1{\mathchoice{{\textstyle\sum}}{\sum}{\sum}{\sum}({\textstyle #1}),\ }
\def\tsm#1{\@tsm{#1}\@ifnextchar\bgroup{\tsm}{}}
\def\@tsm#1{\mathchoice{{\textstyle\sum_{(#1)}}}{\sum_{(#1)}}{\sum_{(#1)}}{\sum_{(#1)}}}
\def\dsm#1{\@dsm{#1}\@ifnextchar\bgroup{\dsm}{}}
\def\@dsm#1{\sum_{(#1)}\,}
\def\@dsm@noparens#1{\sum_{#1}\,}

%%% W-types
\def\wtypesym{{\mathsf{W}}}
\def\wtype#1{\@ifnextchar\bgroup%
  {\mathchoice{\@twtype{#1}}{\@twtype{#1}}{\@twtype{#1}}{\@twtype{#1}}\wtype}%
  {\mathchoice{\@twtype{#1}}{\@twtype{#1}}{\@twtype{#1}}{\@twtype{#1}}}}
\def\lwtype#1{\@ifnextchar\bgroup{\@lwtype{#1}\lwtype}{\@@lwtype{#1}}}
\def\@lwtype#1{\mathchoice{{\textstyle\mathsf{W}}}{\mathsf{W}}{\mathsf{W}}{\mathsf{W}}({\textstyle #1})\;}
\def\@@lwtype#1{\mathchoice{{\textstyle\mathsf{W}}}{\mathsf{W}}{\mathsf{W}}{\mathsf{W}}({\textstyle #1}),\ }
\def\twtype#1{\@twtype{#1}\@ifnextchar\bgroup{\twtype}{}}
\def\@twtype#1{\mathchoice{{\textstyle\mathsf{W}_{(#1)}}}{\mathsf{W}_{(#1)}}{\mathsf{W}_{(#1)}}{\mathsf{W}_{(#1)}}}
\def\dwtype#1{\@dwtype{#1}\@ifnextchar\bgroup{\dwtype}{}}
\def\@dwtype#1{\mathsf{W}_{(#1)}\,}

\newcommand{\suppsym}{{\mathsf{sup}}}
\newcommand{\supp}{\ensuremath\suppsym\xspace}

\def\wtypeh#1{\@ifnextchar\bgroup%
  {\mathchoice{\@lwtypeh{#1}}{\@twtypeh{#1}}{\@twtypeh{#1}}{\@twtypeh{#1}}\wtypeh}%
  {\mathchoice{\@@lwtypeh{#1}}{\@twtypeh{#1}}{\@twtypeh{#1}}{\@twtypeh{#1}}}}
\def\lwtypeh#1{\@ifnextchar\bgroup{\@lwtypeh{#1}\lwtypeh}{\@@lwtypeh{#1}}}
\def\@lwtypeh#1{\mathchoice{{\textstyle\mathsf{W}^h}}{\mathsf{W}^h}{\mathsf{W}^h}{\mathsf{W}^h}({\textstyle #1})\;}
\def\@@lwtypeh#1{\mathchoice{{\textstyle\mathsf{W}^h}}{\mathsf{W}^h}{\mathsf{W}^h}{\mathsf{W}^h}({\textstyle #1}),\ }
\def\twtypeh#1{\@twtypeh{#1}\@ifnextchar\bgroup{\twtypeh}{}}
\def\@twtypeh#1{\mathchoice{{\textstyle\mathsf{W}^h_{(#1)}}}{\mathsf{W}^h_{(#1)}}{\mathsf{W}^h_{(#1)}}{\mathsf{W}^h_{(#1)}}}
\def\dwtypeh#1{\@dwtypeh{#1}\@ifnextchar\bgroup{\dwtypeh}{}}
\def\@dwtypeh#1{\mathsf{W}^h_{(#1)}\,}


\makeatother

% Other notations related to dependent sums
\let\setof\Set    % from package 'braket', write \setof{ x:A | P(x) }.
\newcommand{\pair}{\ensuremath{\mathsf{pair}}\xspace}
\newcommand{\tup}[2]{(#1,#2)}
\newcommand{\proj}[1]{\ensuremath{\mathsf{pr}_{#1}}\xspace}
\newcommand{\fst}{\ensuremath{\proj1}\xspace}
\newcommand{\snd}{\ensuremath{\proj2}\xspace}
\newcommand{\ac}{\ensuremath{\mathsf{ac}}\xspace} % not needed in symbol index
\newcommand{\un}{\ensuremath{\mathsf{upun}}\xspace} % not needed in symbol index, uniqueness principle for unit type

%%% recursor and induction
\newcommand{\rec}[1]{\mathsf{rec}_{#1}}
\newcommand{\ind}[1]{\mathsf{ind}_{#1}}
\newcommand{\indid}[1]{\ind{=_{#1}}} % (Martin-Lof) path induction principle for identity types
\newcommand{\indidb}[1]{\ind{=_{#1}}'} % (Paulin-Mohring) based path induction principle for identity types 

%%% the uniqueness principle for product types, formerly called surjective pairing and named \spr:
\newcommand{\uppt}{\ensuremath{\mathsf{uppt}}\xspace}

% Paths in pairs
\newcommand{\pairpath}{\ensuremath{\mathsf{pair}^{\mathord{=}}}\xspace}
% \newcommand{\projpath}[1]{\proj{#1}^{\mathord{=}}}
\newcommand{\projpath}[1]{\ensuremath{\apfunc{\proj{#1}}}\xspace}

%%% For quotients %%%
%\newcommand{\pairr}[1]{{\langle #1\rangle}}
\newcommand{\pairr}[1]{{\mathopen{}(#1)\mathclose{}}}
\newcommand{\Pairr}[1]{{\mathopen{}\left(#1\right)\mathclose{}}}

% \newcommand{\type}{\ensuremath{\mathsf{Type}}} % this command is overridden below, so it's commented out
\newcommand{\im}{\ensuremath{\mathsf{im}}} % the image

%%% 2D path operations
\newcommand{\leftwhisker}{\mathbin{{\ct}_{\mathsf{l}}}}  % was \ell
\newcommand{\rightwhisker}{\mathbin{{\ct}_{\mathsf{r}}}} % was r
\newcommand{\hct}{\star}

%%% modalities %%%
\newcommand{\modal}{\ensuremath{\ocircle}}
\let\reflect\modal
\newcommand{\modaltype}{\ensuremath{\type_\modal}}
% \newcommand{\ism}[1]{\ensuremath{\mathsf{is}_{#1}}}
% \newcommand{\ismodal}{\ism{\modal}}
% \newcommand{\existsmodal}{\ensuremath{{\exists}_{\modal}}}
% \newcommand{\existsmodalunique}{\ensuremath{{\exists!}_{\modal}}}
% \newcommand{\modalfunc}{\textsf{\modal-fun}}
% \newcommand{\Ecirc}{\ensuremath{\mathsf{E}_\modal}}
% \newcommand{\Mcirc}{\ensuremath{\mathsf{M}_\modal}}
\newcommand{\mreturn}{\ensuremath{\eta}}
\let\project\mreturn
%\newcommand{\mbind}[1]{\ensuremath{\hat{#1}}}
\newcommand{\ext}{\mathsf{ext}}
%\newcommand{\mmap}[1]{\ensuremath{\bar{#1}}}
%\newcommand{\mjoin}{\ensuremath{\mreturn^{-1}}}
% Subuniverse
\renewcommand{\P}{\ensuremath{\type_{P}}\xspace}

%%% Localizations
% \newcommand{\islocal}[1]{\ensuremath{\mathsf{islocal}_{#1}}\xspace}
% \newcommand{\loc}[1]{\ensuremath{\mathcal{L}_{#1}}\xspace}

%%% Identity types %%%
\newcommand{\idsym}{{=}}
\newcommand{\id}[3][]{\ensuremath{#2 =_{#1} #3}\xspace}
\newcommand{\idtype}[3][]{\ensuremath{\mathsf{Id}_{#1}(#2,#3)}\xspace}
\newcommand{\idtypevar}[1]{\ensuremath{\mathsf{Id}_{#1}}\xspace}
% A propositional equality currently being defined
\newcommand{\defid}{\coloneqq}

%%% Dependent paths
\newcommand{\dpath}[4]{#3 =^{#1}_{#2} #4}

%%% singleton
% \newcommand{\sgl}{\ensuremath{\mathsf{sgl}}\xspace}
% \newcommand{\sctr}{\ensuremath{\mathsf{sctr}}\xspace}

%%% Reflexivity terms %%%
% \newcommand{\reflsym}{{\mathsf{refl}}}
\newcommand{\refl}[1]{\ensuremath{\mathsf{refl}_{#1}}\xspace}

%%% Path concatenation (used infix, in diagrammatic order) %%%
\newcommand{\ct}{%
  \mathchoice{\mathbin{\raisebox{0.5ex}{$\displaystyle\centerdot$}}}%
             {\mathbin{\raisebox{0.5ex}{$\centerdot$}}}%
             {\mathbin{\raisebox{0.25ex}{$\scriptstyle\,\centerdot\,$}}}%
             {\mathbin{\raisebox{0.1ex}{$\scriptscriptstyle\,\centerdot\,$}}}
}

%%% Path reversal %%%
\newcommand{\opp}[1]{\mathord{{#1}^{-1}}}
\let\rev\opp

%%% Transport (covariant) %%%
\newcommand{\trans}[2]{\ensuremath{{#1}_{*}\mathopen{}\left({#2}\right)\mathclose{}}\xspace}
\let\Trans\trans
%\newcommand{\Trans}[2]{\ensuremath{{#1}_{*}\left({#2}\right)}\xspace}
\newcommand{\transf}[1]{\ensuremath{{#1}_{*}}\xspace} % Without argument
%\newcommand{\transport}[2]{\ensuremath{\mathsf{transport}_{*} \: {#2}\xspace}}
\newcommand{\transfib}[3]{\ensuremath{\mathsf{transport}^{#1}(#2,#3)\xspace}}
\newcommand{\Transfib}[3]{\ensuremath{\mathsf{transport}^{#1}\Big(#2,\, #3\Big)\xspace}}
\newcommand{\transfibf}[1]{\ensuremath{\mathsf{transport}^{#1}\xspace}}

%%% 2D transport
\newcommand{\transtwo}[2]{\ensuremath{\mathsf{transport}^2\mathopen{}\left({#1},{#2}\right)\mathclose{}}\xspace}

%%% Constant transport
\newcommand{\transconst}[3]{\ensuremath{\mathsf{transportconst}}^{#1}_{#2}(#3)\xspace}
\newcommand{\transconstf}{\ensuremath{\mathsf{transportconst}}\xspace}

%%% Map on paths %%%
\newcommand{\mapfunc}[1]{\ensuremath{\mathsf{ap}_{#1}}\xspace} % Without argument
\newcommand{\map}[2]{\ensuremath{{#1}\mathopen{}\left({#2}\right)\mathclose{}}\xspace}
\let\Ap\map
%\newcommand{\Ap}[2]{\ensuremath{{#1}\left({#2}\right)}\xspace}
\newcommand{\mapdepfunc}[1]{\ensuremath{\mathsf{apd}_{#1}}\xspace} % Without argument
% \newcommand{\mapdep}[2]{\ensuremath{{#1}\llparenthesis{#2}\rrparenthesis}\xspace}
\newcommand{\mapdep}[2]{\ensuremath{\mapdepfunc{#1}\mathopen{}\left(#2\right)\mathclose{}}\xspace}
\let\apfunc\mapfunc
\let\ap\map
\let\apdfunc\mapdepfunc
\let\apd\mapdep

%%% 2D map on paths
\newcommand{\aptwofunc}[1]{\ensuremath{\mathsf{ap}^2_{#1}}\xspace}
\newcommand{\aptwo}[2]{\ensuremath{\aptwofunc{#1}\mathopen{}\left({#2}\right)\mathclose{}}\xspace}
\newcommand{\apdtwofunc}[1]{\ensuremath{\mathsf{apd}^2_{#1}}\xspace}
\newcommand{\apdtwo}[2]{\ensuremath{\apdtwofunc{#1}\mathopen{}\left(#2\right)\mathclose{}}\xspace}

%%% Identity functions %%%
\newcommand{\idfunc}[1][]{\ensuremath{\mathsf{id}_{#1}}\xspace}

%%% Homotopies (written infix) %%%
\newcommand{\htpy}{\sim}

%%% Other meanings of \sim
\newcommand{\bisim}{\sim}       % bisimulation
\newcommand{\eqr}{\sim}         % an equivalence relation

%%% Equivalence types %%%
\newcommand{\eqv}[2]{\ensuremath{#1 \simeq #2}\xspace}
\newcommand{\eqvspaced}[2]{\ensuremath{#1 \;\simeq\; #2}\xspace}
\newcommand{\eqvsym}{\simeq}    % infix symbol
\newcommand{\texteqv}[2]{\ensuremath{\mathsf{Equiv}(#1,#2)}\xspace}
\newcommand{\isequiv}{\ensuremath{\mathsf{isequiv}}}
\newcommand{\qinv}{\ensuremath{\mathsf{qinv}}}
\newcommand{\ishae}{\ensuremath{\mathsf{ishae}}}
\newcommand{\linv}{\ensuremath{\mathsf{linv}}}
\newcommand{\rinv}{\ensuremath{\mathsf{rinv}}}
\newcommand{\biinv}{\ensuremath{\mathsf{biinv}}}
\newcommand{\lcoh}[3]{\mathsf{lcoh}_{#1}(#2,#3)}
\newcommand{\rcoh}[3]{\mathsf{rcoh}_{#1}(#2,#3)}
\newcommand{\hfib}[2]{{\mathsf{fib}}_{#1}(#2)}

%%% Map on total spaces %%%
\newcommand{\total}[1]{\ensuremath{\mathsf{total}(#1)}}

%%% Universe types %%%
%\newcommand{\type}{\ensuremath{\mathsf{Type}}\xspace}
\newcommand{\UU}{\ensuremath{\mathcal{U}}\xspace}
\let\bbU\UU
\let\type\UU
% Universes of truncated types
\newcommand{\typele}[1]{\ensuremath{{#1}\text-\mathsf{Type}}\xspace}
\newcommand{\typeleU}[1]{\ensuremath{{#1}\text-\mathsf{Type}_\UU}\xspace}
\newcommand{\typelep}[1]{\ensuremath{{(#1)}\text-\mathsf{Type}}\xspace}
\newcommand{\typelepU}[1]{\ensuremath{{(#1)}\text-\mathsf{Type}_\UU}\xspace}
\let\ntype\typele
\let\ntypeU\typeleU
\let\ntypep\typelep
\let\ntypepU\typelepU
\renewcommand{\set}{\ensuremath{\mathsf{Set}}\xspace}
\newcommand{\setU}{\ensuremath{\mathsf{Set}_\UU}\xspace}
\newcommand{\prop}{\ensuremath{\mathsf{Prop}}\xspace}
\newcommand{\propU}{\ensuremath{\mathsf{Prop}_\UU}\xspace}
%Pointed types
\newcommand{\pointed}[1]{\ensuremath{#1_\bullet}}

%%% Ordinals and cardinals
\newcommand{\card}{\ensuremath{\mathsf{Card}}\xspace}
\newcommand{\ord}{\ensuremath{\mathsf{Ord}}\xspace}
\newcommand{\ordsl}[2]{{#1}_{/#2}}

%%% Univalence
\newcommand{\ua}{\ensuremath{\mathsf{ua}}\xspace} % the inverse of idtoeqv
\newcommand{\idtoeqv}{\ensuremath{\mathsf{idtoeqv}}\xspace}
\newcommand{\univalence}{\ensuremath{\mathsf{univalence}}\xspace} % the full axiom

%%% Truncation levels
\newcommand{\iscontr}{\ensuremath{\mathsf{isContr}}}
\newcommand{\contr}{\ensuremath{\mathsf{contr}}} % The path to the center of contraction
\newcommand{\isset}{\ensuremath{\mathsf{isSet}}}
\newcommand{\isprop}{\ensuremath{\mathsf{isProp}}}
% h-propositions
% \newcommand{\anhprop}{a mere proposition\xspace}
% \newcommand{\hprops}{mere propositions\xspace}

%%% Homotopy fibers %%%
%\newcommand{\hfiber}[2]{\ensuremath{\mathsf{hFiber}(#1,#2)}\xspace}
\let\hfiber\hfib

%%% Bracket/squash/truncation types %%%
% \newcommand{\brck}[1]{\textsf{mere}(#1)}
% \newcommand{\Brck}[1]{\textsf{mere}\Big(#1\Big)}
% \newcommand{\trunc}[2]{\tau_{#1}(#2)}
% \newcommand{\Trunc}[2]{\tau_{#1}\Big(#2\Big)}
% \newcommand{\truncf}[1]{\tau_{#1}}
%\newcommand{\trunc}[2]{\Vert #2\Vert_{#1}}
\newcommand{\trunc}[2]{\mathopen{}\left\Vert #2\right\Vert_{#1}\mathclose{}}
\newcommand{\ttrunc}[2]{\bigl\Vert #2\bigr\Vert_{#1}}
\newcommand{\Trunc}[2]{\Bigl\Vert #2\Bigr\Vert_{#1}}
\newcommand{\truncf}[1]{\Vert \blank \Vert_{#1}}
\newcommand{\tproj}[3][]{\mathopen{}\left|#3\right|_{#2}^{#1}\mathclose{}}
\newcommand{\tprojf}[2][]{|\blank|_{#2}^{#1}}
\def\pizero{\trunc0}
%\newcommand{\brck}[1]{\trunc{-1}{#1}}
%\newcommand{\Brck}[1]{\Trunc{-1}{#1}}
%\newcommand{\bproj}[1]{\tproj{-1}{#1}}
%\newcommand{\bprojf}{\tprojf{-1}}

\newcommand{\brck}[1]{\trunc{}{#1}}
\newcommand{\bbrck}[1]{\ttrunc{}{#1}}
\newcommand{\Brck}[1]{\Trunc{}{#1}}
\newcommand{\bproj}[1]{\tproj{}{#1}}
\newcommand{\bprojf}{\tprojf{}}

% Big parentheses
\newcommand{\Parens}[1]{\Bigl(#1\Bigr)}

% Projection and extension for truncations
\let\extendsmb\ext
\newcommand{\extend}[1]{\extendsmb(#1)}

%
%%% The empty type
\newcommand{\emptyt}{\ensuremath{\mathbf{0}}\xspace}

%%% The unit type
\newcommand{\unit}{\ensuremath{\mathbf{1}}\xspace}
\newcommand{\ttt}{\ensuremath{\star}\xspace}

%%% The two-element type
\newcommand{\bool}{\ensuremath{\mathbf{2}}\xspace}
\newcommand{\btrue}{{1_{\bool}}}
\newcommand{\bfalse}{{0_{\bool}}}

%%% Injections into binary sums and pushouts
\newcommand{\inlsym}{{\mathsf{inl}}}
\newcommand{\inrsym}{{\mathsf{inr}}}
\newcommand{\inl}{\ensuremath\inlsym\xspace}
\newcommand{\inr}{\ensuremath\inrsym\xspace}

%%% The segment of the interval
\newcommand{\seg}{\ensuremath{\mathsf{seg}}\xspace}

%%% Free groups
\newcommand{\freegroup}[1]{F(#1)}
\newcommand{\freegroupx}[1]{F'(#1)} % the "other" free group

%%% Glue of a pushout
\newcommand{\glue}{\mathsf{glue}}

%%% Circles and spheres
\newcommand{\Sn}{\mathbb{S}}
\newcommand{\base}{\ensuremath{\mathsf{base}}\xspace}
\newcommand{\lloop}{\ensuremath{\mathsf{loop}}\xspace}
\newcommand{\surf}{\ensuremath{\mathsf{surf}}\xspace}

%%% Suspension
\newcommand{\susp}{\Sigma}
\newcommand{\north}{\mathsf{N}}
\newcommand{\south}{\mathsf{S}}
\newcommand{\merid}{\mathsf{merid}}

%%% Blanks (shorthand for lambda abstractions)
\newcommand{\blank}{\mathord{\hspace{1pt}\text{--}\hspace{1pt}}}

%%% Nameless objects
\newcommand{\nameless}{\mathord{\hspace{1pt}\underline{\hspace{1ex}}\hspace{1pt}}}

%%% Some decorations
%\newcommand{\bbU}{\ensuremath{\mathbb{U}}\xspace}
% \newcommand{\bbB}{\ensuremath{\mathbb{B}}\xspace}
\newcommand{\bbP}{\ensuremath{\mathbb{P}}\xspace}

%%% Some categories
\newcommand{\uset}{\ensuremath{\mathcal{S}et}\xspace}
\newcommand{\ucat}{\ensuremath{{\mathcal{C}at}}\xspace}
\newcommand{\urel}{\ensuremath{\mathcal{R}el}\xspace}
\newcommand{\uhilb}{\ensuremath{\mathcal{H}ilb}\xspace}
\newcommand{\utype}{\ensuremath{\mathcal{T}\!ype}\xspace}

% Pullback corner
\newbox\pbbox
\setbox\pbbox=\hbox{\xy \POS(65,0)\ar@{-} (0,0) \ar@{-} (65,65)\endxy}
\def\pb{\save[]+<3.5mm,-3.5mm>*{\copy\pbbox} \restore}

% Macros for the categories chapter
\newcommand{\inv}[1]{{#1}^{-1}}
\newcommand{\idtoiso}{\ensuremath{\mathsf{idtoiso}}\xspace}
\newcommand{\isotoid}{\ensuremath{\mathsf{isotoid}}\xspace}
\newcommand{\op}{^{\mathrm{op}}}
\newcommand{\y}{\ensuremath{\mathbf{y}}\xspace}
\newcommand{\dgr}[1]{{#1}^{\dagger}}
\newcommand{\unitaryiso}{\mathrel{\cong^\dagger}}
\newcommand{\cteqv}[2]{\ensuremath{#1 \simeq #2}\xspace}
\newcommand{\cteqvsym}{\simeq}     % Symbol for equivalence of categories

%%% Natural numbers
\newcommand{\N}{\ensuremath{\mathbb{N}}\xspace}
%\newcommand{\N}{\textbf{N}}
\let\nat\N
\newcommand{\natp}{\ensuremath{\nat'}\xspace} % alternative nat in induction chapter

\newcommand{\zerop}{\ensuremath{0'}\xspace}   % alternative zero in induction chapter
\newcommand{\suc}{\mathsf{succ}}
\newcommand{\sucp}{\ensuremath{\suc'}\xspace} % alternative suc in induction chapter
\newcommand{\add}{\mathsf{add}}
\newcommand{\ack}{\mathsf{ack}}
\newcommand{\ite}{\mathsf{iter}}
\newcommand{\assoc}{\mathsf{assoc}}
\newcommand{\dbl}{\ensuremath{\mathsf{double}}}
\newcommand{\dblp}{\ensuremath{\dbl'}\xspace} % alternative double in induction chapter


%%% Lists
\newcommand{\lst}[1]{\mathsf{List}(#1)}
\newcommand{\nil}{\mathsf{nil}}
\newcommand{\cons}{\mathsf{cons}}
\newcommand{\lost}[1]{\mathsf{Lost}(#1)}

%%% Vectors of given length, used in induction chapter
\newcommand{\vect}[2]{\ensuremath{\mathsf{Vec}_{#1}(#2)}\xspace}

%%% Integers
\newcommand{\Z}{\ensuremath{\mathbb{Z}}\xspace}
\newcommand{\Zsuc}{\mathsf{succ}}
\newcommand{\Zpred}{\mathsf{pred}}

%%% Rationals
\newcommand{\Q}{\ensuremath{\mathbb{Q}}\xspace}

%%% Function extensionality
\newcommand{\funext}{\mathsf{funext}}
\newcommand{\happly}{\mathsf{happly}}

%%% A naturality lemma
\newcommand{\com}[3]{\mathsf{swap}_{#1,#2}(#3)}

%%% Code/encode/decode
\newcommand{\code}{\ensuremath{\mathsf{code}}\xspace}
\newcommand{\encode}{\ensuremath{\mathsf{encode}}\xspace}
\newcommand{\decode}{\ensuremath{\mathsf{decode}}\xspace}

% Function definition with domain and codomain
\newcommand{\function}[4]{\left\{\begin{array}{rcl}#1 &
      \longrightarrow & #2 \\ #3 & \longmapsto & #4 \end{array}\right.}

%%% Cones and cocones
\newcommand{\cone}[2]{\mathsf{cone}_{#1}(#2)}
\newcommand{\cocone}[2]{\mathsf{cocone}_{#1}(#2)}
% Apply a function to a cocone
\newcommand{\composecocone}[2]{#1\circ#2}
\newcommand{\composecone}[2]{#2\circ#1}
%%% Diagrams
\newcommand{\Ddiag}{\mathscr{D}}

%%% (pointed) mapping spaces
\newcommand{\Map}{\mathsf{Map}}

%%% The interval
\newcommand{\interval}{\ensuremath{I}\xspace}
\newcommand{\izero}{\ensuremath{0_{\interval}}\xspace}
\newcommand{\ione}{\ensuremath{1_{\interval}}\xspace}

%%% Arrows
\newcommand{\epi}{\ensuremath{\twoheadrightarrow}}
\newcommand{\mono}{\ensuremath{\rightarrowtail}}

%%% Sets
\newcommand{\bin}{\ensuremath{\mathrel{\widetilde{\in}}}}

%%% Semigroup structure
\newcommand{\semigroupstrsym}{\ensuremath{\mathsf{SemigroupStr}}}
\newcommand{\semigroupstr}[1]{\ensuremath{\mathsf{SemigroupStr}}(#1)}
\newcommand{\semigroup}[0]{\ensuremath{\mathsf{Semigroup}}}

%%% Macros for the formal type theory
\newcommand{\emptyctx}{\ensuremath{\cdot}}
\newcommand{\production}{\vcentcolon\vcentcolon=}
\newcommand{\conv}{\downarrow}
\newcommand{\ctx}{\ensuremath{\mathsf{ctx}}}
\newcommand{\wfctx}[1]{#1\ \ctx}
\newcommand{\oftp}[3]{#1 \vdash #2 : #3}
\newcommand{\jdeqtp}[4]{#1 \vdash #2 \jdeq #3 : #4}
\newcommand{\judg}[2]{#1 \vdash #2}
\newcommand{\tmtp}[2]{#1 \mathord{:} #2}

% rule names
\newcommand{\form}{\textsc{form}}
\newcommand{\intro}{\textsc{intro}}
\newcommand{\elim}{\textsc{elim}}
\newcommand{\comp}{\textsc{comp}}
\newcommand{\uniq}{\textsc{uniq}}
\newcommand{\Weak}{\mathsf{Wkg}}
\newcommand{\Vble}{\mathsf{Vble}}
\newcommand{\Exch}{\mathsf{Exch}}
\newcommand{\Subst}{\mathsf{Subst}}

%%% Macros for HITs
\newcommand{\cc}{\mathsf{c}}
\newcommand{\pp}{\mathsf{p}}
\newcommand{\cct}{\widetilde{\mathsf{c}}}
\newcommand{\ppt}{\widetilde{\mathsf{p}}}
\newcommand{\Wtil}{\ensuremath{\widetilde{W}}\xspace}

%%% Macros for n-types
\newcommand{\istype}[1]{\mathsf{is}\mbox{-}{#1}\mbox{-}\mathsf{type}}
\newcommand{\nplusone}{\ensuremath{(n+1)}}
\newcommand{\nminusone}{\ensuremath{(n-1)}}
\newcommand{\fact}{\mathsf{fact}}

%%% Macros for homotopy
\newcommand{\kbar}{\overline{k}} % Used in van Kampen's theorem

%%% Macros for induction
\newcommand{\natw}{\ensuremath{\mathbf{N^w}}\xspace}
\newcommand{\zerow}{\ensuremath{0^\mathbf{w}}\xspace}
\newcommand{\sucw}{\ensuremath{\mathsf{succ}^{\mathbf{w}}}\xspace}
\newcommand{\nalg}{\nat\mathsf{Alg}}
\newcommand{\nhom}{\nat\mathsf{Hom}}
\newcommand{\ishinitw}{\mathsf{isHinit}_{\mathsf{W}}}
\newcommand{\ishinitn}{\mathsf{isHinit}_\nat}
\newcommand{\w}{\mathsf{W}}
\newcommand{\walg}{\w\mathsf{Alg}}
\newcommand{\whom}{\w\mathsf{Hom}}

%%% Macros for real numbers
\newcommand{\RC}{\ensuremath{\mathbb{R}_\mathsf{c}}\xspace} % Cauchy
\newcommand{\RD}{\ensuremath{\mathbb{R}_\mathsf{d}}\xspace} % Dedekind
\newcommand{\R}{\ensuremath{\mathbb{R}}\xspace}           % Either 
\newcommand{\barRD}{\ensuremath{\bar{\mathbb{R}}_\mathsf{d}}\xspace} % Dedekind completion of Dedekind

\newcommand{\close}[1]{\sim_{#1}} % Relation of closeness
\newcommand{\closesym}{\mathord\sim}
\newcommand{\rclim}{\mathsf{lim}} % HIT constructor for Cauchy reals
\newcommand{\rcrat}{\mathsf{rat}} % Embedding of rationals into Cauchy reals
\newcommand{\rceq}{\mathsf{eq}_{\RC}} % HIT path constructor
\newcommand{\CAP}{\mathcal{C}}    % The type of Cauchy approximations
\newcommand{\Qp}{\Q_{+}}
\newcommand{\apart}{\mathrel{\#}}  % apartness
\newcommand{\dcut}{\mathsf{isCut}}  % Dedekind cut
\newcommand{\cover}{\triangleleft} % inductive cover
\newcommand{\intfam}[3]{(#2, \lam{#1} #3)} % family of rational intervals

% Macros for the Cauchy reals construction
\newcommand{\bsim}{\frown}
\newcommand{\bbsim}{\smile}

\newcommand{\hapx}{\diamondsuit\approx}
\newcommand{\hapname}{\diamondsuit}
\newcommand{\hapxb}{\heartsuit\approx}
\newcommand{\hapbname}{\heartsuit}
\newcommand{\tap}[1]{\bullet\approx_{#1}\triangle}
\newcommand{\tapname}{\triangle}
\newcommand{\tapb}[1]{\bullet\approx_{#1}\square}
\newcommand{\tapbname}{\square}

%%% Macros for surreals
\newcommand{\NO}{\ensuremath{\mathsf{No}}\xspace}
\newcommand{\surr}[2]{\{\,#1\,\big|\,#2\,\}}
\newcommand{\LL}{\mathcal{L}}
\newcommand{\RR}{\mathcal{R}}
\newcommand{\noeq}{\mathsf{eq}_{\NO}} % HIT path constructor

\newcommand{\ble}{\trianglelefteqslant}
\newcommand{\blt}{\vartriangleleft}
\newcommand{\bble}{\sqsubseteq}
\newcommand{\bblt}{\sqsubset}

\newcommand{\hle}{\diamondsuit\preceq}
\newcommand{\hlt}{\diamondsuit\prec}
\newcommand{\hlname}{\diamondsuit}
\newcommand{\hleb}{\heartsuit\preceq}
\newcommand{\hltb}{\heartsuit\prec}
\newcommand{\hlbname}{\heartsuit}
% \newcommand{\tle}{(\bullet\preceq\triangle)}
% \newcommand{\tlt}{(\bullet\prec\triangle)}
\newcommand{\tle}{\triangle\preceq}
\newcommand{\tlt}{\triangle\prec}
\newcommand{\tlname}{\triangle}
% \newcommand{\tleb}{(\bullet\preceq\square)}
% \newcommand{\tltb}{(\bullet\prec\square)}
\newcommand{\tleb}{\square\preceq}
\newcommand{\tltb}{\square\prec}
\newcommand{\tlbname}{\square}

%%% Macros for set theory
\newcommand{\vset}{\mathsf{set}}  % point constructor for cummulative hierarchy V
\def\cd{\tproj0}
\newcommand{\inj}{\ensuremath{\mathsf{inj}}} % type of injections
\newcommand{\acc}{\ensuremath{\mathsf{acc}}} % accessibility

\newcommand{\atMostOne}{\mathsf{atMostOne}}

\newcommand{\power}[1]{\mathcal{P}(#1)} % power set
\newcommand{\powerp}[1]{\mathcal{P}_+(#1)} % inhabited power set

%%%% THEOREM ENVIRONMENTS %%%%

% Hyperref includes the command \autoref{...} which is like \ref{...}
% except that it automatically inserts the type of the thing you're
% referring to, e.g. it produces "Theorem 3.8" instead of just "3.8"
% (and makes the whole thing a hyperlink).  This saves a slight amount
% of typing, but more importantly it means that if you decide later on
% that 3.8 should be a Lemma or a Definition instead of a Theorem, you
% don't have to change the name in all the places you referred to it.

% The following hack improves on this by using the same counter for
% all theorem-type environments, so that after Theorem 1.1 comes
% Corollary 1.2 rather than Corollary 1.1.  This makes it much easier
% for the reader to find a particular theorem when flipping through
% the document.
\makeatletter
\def\defthm#1#2#3{%
  %% Ensure all theorem types are numbered with the same counter
  \newaliascnt{#1}{thm}
  \newtheorem{#1}[#1]{#2}
  \aliascntresetthe{#1}
  %% This command tells cleveref's \cref what to call things
  \crefname{#1}{#2}{#3}}

% Now define a bunch of theorem-type environments.
\newtheorem{thm}{Theorem}[section]
\crefname{thm}{Theorem}{Theorems}
%\defthm{prop}{Proposition}   % Probably we shouldn't use "Proposition" in this way
\defthm{cor}{Corollary}{Corollaries}
\defthm{lem}{Lemma}{Lemmas}
\defthm{axiom}{Axiom}{Axioms}
% Since definitions and theorems in type theory are synonymous, should
% we actually use the same theoremstyle for them?
\theoremstyle{definition}
\defthm{defn}{Definition}{Definitions}
\theoremstyle{remark}
\defthm{rmk}{Remark}{Remarks}
\defthm{eg}{Example}{Examples}
\defthm{egs}{Examples}{Examples}
\defthm{notes}{Notes}{Notes}
% Number exercises within chapters, with their own counter.
\newtheorem{ex}{Exercise}[chapter]
\crefname{ex}{Exercise}{Exercises}

% Display format for sections
\crefformat{section}{\S#2#1#3}
\Crefformat{section}{Section~#2#1#3}
\crefrangeformat{section}{\S\S#3#1#4--#5#2#6}
\Crefrangeformat{section}{Sections~#3#1#4--#5#2#6}
\crefmultiformat{section}{\S\S#2#1#3}{ and~#2#1#3}{, #2#1#3}{ and~#2#1#3}
\Crefmultiformat{section}{Sections~#2#1#3}{ and~#2#1#3}{, #2#1#3}{ and~#2#1#3}
\crefrangemultiformat{section}{\S\S#3#1#4--#5#2#6}{ and~#3#1#4--#5#2#6}{, #3#1#4--#5#2#6}{ and~#3#1#4--#5#2#6}
\Crefrangemultiformat{section}{Sections~#3#1#4--#5#2#6}{ and~#3#1#4--#5#2#6}{, #3#1#4--#5#2#6}{ and~#3#1#4--#5#2#6}

% Display format for appendices
\crefformat{appendix}{Appendix~#2#1#3}
\Crefformat{appendix}{Appendix~#2#1#3}
\crefrangeformat{appendix}{Appendices~#3#1#4--#5#2#6}
\Crefrangeformat{appendix}{Appendices~#3#1#4--#5#2#6}
\crefmultiformat{appendix}{Appendices~#2#1#3}{ and~#2#1#3}{, #2#1#3}{ and~#2#1#3}
\Crefmultiformat{appendix}{Appendices~#2#1#3}{ and~#2#1#3}{, #2#1#3}{ and~#2#1#3}
\crefrangemultiformat{appendix}{Appendices~#3#1#4--#5#2#6}{ and~#3#1#4--#5#2#6}{, #3#1#4--#5#2#6}{ and~#3#1#4--#5#2#6}
\Crefrangemultiformat{appendix}{Appendices~#3#1#4--#5#2#6}{ and~#3#1#4--#5#2#6}{, #3#1#4--#5#2#6}{ and~#3#1#4--#5#2#6}

\crefname{part}{Part}{Parts}

% Number subsubsections
\setcounter{secnumdepth}{5}

% Display format for figures
\crefname{figure}{Figure}{Figures}

% Use cleveref instead of hyperref's \autoref
\let\autoref\cref


%%%% EQUATION NUMBERING %%%%

% The following hack uses the single theorem counter to number
% equations as well, so that we don't have both Theorem 1.1 and
% equation (1.1).
\let\c@equation\c@thm
\numberwithin{equation}{section}


%%%% ENUMERATE NUMBERING %%%%

% Number the first level of enumerates as (i), (ii), ...
\renewcommand{\theenumi}{(\roman{enumi})}
\renewcommand{\labelenumi}{\theenumi}


%%%% MARGINS %%%%

% This is a matter of personal preference, but I think the left
% margins on enumerates and itemizes are too wide.
\setitemize[1]{leftmargin=2em}
\setenumerate[1]{leftmargin=*}

% Likewise that they are too spaced out.
\setitemize[1]{itemsep=-0.2em}
\setenumerate[1]{itemsep=-0.2em}

%%% Notes %%%
\def\noteson{%
\gdef\note##1{\mbox{}\marginpar{\color{blue}\textasteriskcentered\ ##1}}}
\gdef\notesoff{\gdef\note##1{\null}}
\noteson

\newcommand{\Coq}{\textsc{Coq}\xspace}
\newcommand{\Agda}{\textsc{Agda}\xspace}
\newcommand{\NuPRL}{\textsc{NuPRL}\xspace}

%%%% CITATIONS %%%%

% \let \cite \citep

%%%% INDEX %%%%

\newcommand{\footstyle}[1]{{\hyperpage{#1}}n} % If you index something that is in a footnote
\newcommand{\defstyle}[1]{\textbf{\hyperpage{#1}}}  % Style for pageref to a definition

\newcommand{\indexdef}[1]{\index{#1|defstyle}}   % Index a definition
\newcommand{\indexfoot}[1]{\index{#1|footstyle}} % Index a term in a footnote
\newcommand{\indexsee}[2]{\index{#1|see{#2}}}    % Index "see also"


%%%% Standard phrasing or spelling of common phrases %%%%

\newcommand{\ZF}{Zermelo--Fraenkel}
\newcommand{\CZF}{Constructive \ZF{} Set Theory}

\newcommand{\LEM}[1]{\ensuremath{\mathsf{LEM}_{#1}}\xspace}
\newcommand{\choice}[1]{\ensuremath{\mathsf{AC}_{#1}}\xspace}

%%%% MISC %%%%

\newcommand{\mentalpause}{\medskip} % Use for "mental" pause, instead of \smallskip or \medskip

%% Use \symlabel instead of \label to mark a pageref that you need in the index of symbols
\newcounter{symindex}
\newcommand{\symlabel}[1]{\refstepcounter{symindex}\label{#1}}

% Local Variables:
% mode: latex
% TeX-master: "hott-online"
% End:


\title{Homotopy Type Theory: Univalent Foundations of Mathematics}
\author{The Univalent Foundations Program}

\epubMathInToc
\epubEmbedFont{
  family = STIXGeneral,
  regular = STIXGeneral.otf,
  italic = STIXGeneralItalic.otf,
  bold = STIXGeneralBol.otf,
  bolditalic = STIXGeneralBolIta.otf,
  format = opentype
}

\epubEmbedFont{
  family = STIXIntegral,
  regular = STIXIntDReg.otf,
  bold = STIXIntDBol.otf,
  format = opentype
}


\addcss+math {font-family: serif, STIXGeneral, STIXIntegral;}+
\addcss+.footnote_tgt {vertical-align: super; margin-right: 1em; font-size: 80%;}+
\addcss+.footnote_src {vertical-align: super; margin-right: 1em; font-size: 80%;}+
\addcss+.footnote_tgt:before {content: "\021a9"}+
\addcss+[mathvariant="sans-serif"] {font-family: sans-serif;}+
\begin{document}
\itexSetCatcodes

% NB: This does not actually appear anywhere because we have
% a custom title page.


%\pagestyle{empty}
%
\pagenumbering{roman}

%%%%%%%%%%%%%%%%%%%% Cover page %%%%%%%%%%%%%%%%%%%%
\definecolor{orange}{rgb}{1.0,0.45,0}
\newgeometry{margin=1cm}%
\ThisCenterWallPaper{1.0}{cover}
\vspace*{0.05\textheight}
{\centering
{\color{orange}\fontsize{30}{40}\fontseries{d}\selectfont \ourtitlebroken}
\vfill
{\color{orange}\fontsize{20}{27}\fontshape{it}\selectfont \ourauthorbroken}\par
}
\vspace*{0.02\textheight}

\cleardoublepage
\setcounter{page}{1}


%%%%%%%%%%%%%%%%%%%% Inner title page %%%%%%%%%%%%%%%%%%%%
\vspace*{0.2\textheight}
{\centering
{\fontsize{30}{40}\fontseries{d}\selectfont \ourtitlebroken}\par
\vspace*{0.05\textheight}
{\fontsize{20}{27}\fontshape{it}\selectfont \ourauthorbroken}\par
}
\vfill
\hbox{}

\clearpage

%%%%%%%%%%%%%%%%%%%% Copyright page %%%%%%%%%%%%%%%%%%%%
\newgeometry{margin=1in}%
\hbox{}
\vfill

{\footnotesize
\noindent
\emph{``\ourtitle''}
\copyright\ \ourauthor

\bigskip

\noindent
\emph{``Flames''} cover page photo
\copyright\ Jordan Smith of \href{http://rippah2.deviantart.com/}{rippah2.deviantart.com}.

\bigskip

\noindent
This book is freely available at
{\tt homotopytypetheory.org}.
%

\bigskip

\noindent
This work is licensed under the
\textbf{\emph{Creative Commons Attribution-ShareAlike 3.0 Unported License.}}
%
To view a copy of this license, visit
\href{http://creativecommons.org/licenses/by-sa/3.0/}{http://creativecommons.org/licenses/by-sa/3.0/}.
The following is a human-readable summary of the Legal Code (the full license).

\bigskip

\noindent
\emph{You are free:}
%
\begin{description}
\item[to Share] --- to copy, distribute and transmit the work,
\item[to Remix] --- to adapt the work to make commercial use of the work.
\end{description}
%
\emph{Under the following conditions:}
%
\begin{description}

\item[Attribution] --- You must attribute the work in the manner specified by the author
  or licensor (but not in any way that suggests that they endorse you or your use of the
  work).

\item[Share Alike] --- If you alter, transform, or build upon this work, you may
  distribute the resulting work only under the same or similar license to this one.
\end{description}
%
\emph{With the understanding that:}
\begin{description}

\item[Waiver] --- Any of the above conditions can be waived if you get permission from the
  copyright holder.

\item[Public Domain] --- Where the work or any of its elements is in the public domain
  under applicable law, that status is in no way affected by the license.

\item[Other Rights] --- In no way are any of the following rights affected by the license:
  \begin{itemize}
  \item Your fair dealing or fair use rights, or other applicable copyright exceptions and
    limitations;
  \item The author's moral rights;
  \item Rights other persons may have either in the work itself or in how the work is
    used, such as publicity or privacy rights.
  \end{itemize}
\end{description}
}
\cleardoublepage

%%% Restore page style
\restoregeometry

\pagestyle{fancyplain}

%%% Local Variables: 
%%% mode: latex
%%% TeX-master: "main"
%%% End: 
 %%% Title page and copyright

%\chapter*{Preface}
\label{cha:preface}

\addcontentsline{toc}{chapter}{Preface}

{%%%%%%% macros local to this file, discharged at the end of the file

%\newcommand{\stype}{{\;\sf type}}
%%\newcommand{\nat}{\mathbb{N}}
%\newcommand{\nat}{{\bf N}}
%\newcommand{\rec}{{\sf rec}}
%%\newcommand{\bool}{\bbB}
%\newcommand{\bool}{{\bf B}}
%\newcommand{\app}{{\sf app}}
%\newcommand{\pair}{{\sf pair}}
%\newcommand{\suc}{{\sf succ}}
%\newcommand{\inleft}{{\sf inleft}}
%\newcommand{\inright}{{\sf inright}}
%%\newcommand{\bbzero}{{0\hspace*{-4pt} 0}}
%\newcommand{\emptyt}{{\bf 0}}
%%\newcommand{\bbone}{{1\hspace*{-4pt} 1}}
%\newcommand{\unitt}{{\bf 1}}

%\newcommand{\idtypevar}{{{\sf Id}}}
%\newcommand{\eq}{{{\sf Eq}}}
%

This book was written as a collaborative effort during the 2012-13 Special Year on Univalent Foundations at the Institute for Advanced Study, School of Mathematics, organized by Steve Awodey, Thierry Coquand, and Vladimir Voevodsky.

\bigskip
%\noindent The participants in the special year were:
\centerline{\emph{Participants}}
%
\begin{multicols}{3}{
\begin{itemize}
\item[] Peter Aczel
\item[] Benedikt Ahrens
\item[] Thorsten Altenkirch
\item[] Carlo Angiuli
\item[] Steve Awodey
\item[] Bruno Barras
\item[] Andrej Bauer
\item[] Yves Bertot
\item[] Marc Bezem
\item[] Anthony Bordg
\item[] Guillaume Brunerie
\item[] Thierry Coquand
\item[] Eric Finster
\item[] Daniel Grayson
\item[] Hugo Herbelin
\item[] Andr\'e Joyal
\item[] Chris Kapulkin
\item[] Dan Licata
\item[] Peter Lumsdaine
\item[] Assia Mahboubi
\item[] Per Martin-L\"of
\item[] Sergey Melikhov
\item[] Alvaro Pelayo
\item[] Andrew Polonsky
\item[] Egbert Rijke
\item[] Michael Shulman
\item[] Kristina Sojakova
\item[] Matthieu Sozeau
\item[] Bas Spitters
\item[] Benno van den Berg
\item[] Vladimir Voevodsky
\item[] Michael Warren
\item[] Noam Zeilberger
\end{itemize}
}
\end{multicols}

%%\noindent There were also the following student participants:
%\centerline{\emph{Students}}
%%
%\begin{multicols}{3}{
%\begin{itemize}
%\item[] Carlo Angiuli
%\item[] Anthony Bordg
%\item[] Guillaume Brunerie
%\item[] Chris Kapulkin
%\item[] Egbert Rijke
%\item[] Kristina Sojakova
%\end{itemize}
%}
%\end{multicols}

%\noindent The following people visited at some point:
\centerline{\emph{Visitors}}
%
\begin{multicols}{3}{
\begin{itemize}
\item[] Jeremy Avigad
\item[] Cyril Cohen
\item[] Robert Constable
\item[] Pierre-Louis Curien
\item[] Peter Dybjer
\item[] Mart{\'\i}n Escard{\'o}
\item[] Kuen-Bang Hou (Favonia)
\item[] Nicola Gambino
\item[] Richard Garner
\item[] Georges Gonthier
\item[] Thomas Hales
\item[] Robert Harper
\item[] Martin Hofmann
\item[] Pieter Hofstra
\item[] Joachim Kock
\item[] Zhaohui Luo
\item[] Michael Nahas
\item[] Erik Palmgren
\item[] Dana Scott
\item[] Philip Scott
\item[] Sergei Soloviev
\end{itemize}
}
\end{multicols}

\noindent While each of the above individuals contributed something to this project --- in the form of ideas, words, or deeds --- a few of them contributed much more.  Mike Shulman, in particular, deserves special mention for heroic authorship, and Peter Aczel for originally conceiving of this book and serving as its managing editor.

% To get the list of contributers to github run the command:
% git log --format='%aN' | sort -u

Some other things that should be said ...

Special thanks are due to the Institute for Advanced Study, without which this book would obviously never have come to be --- but which, moreover, provided an ideal environment for this project. ...
\bigskip

\flushright{The Univalent Foundations Program\\
Princeton, April 2013}


}

%%%%%%%%%%%%% end of scope of local macros
% Local Variables:
% TeX-master: "main"
% End:



%\mainmatter % Turn on roman page numbers and numbered chapters

\chapter*{Introduction}
\label{cha:introduction}

\addcontentsline{toc}{chapter}{Introduction}

{%%%%%%% macros local to this file, discharged at the end of the file

%\newcommand{\stype}{{\;\sf type}}
%%\newcommand{\nat}{\mathbb{N}}
%\newcommand{\nat}{{\bf N}}
%\newcommand{\rec}{{\sf rec}}
%%\newcommand{\bool}{\bbB}
%\newcommand{\bool}{{\bf B}}
%\newcommand{\app}{{\sf app}}
%\newcommand{\pair}{{\sf pair}}
%\newcommand{\suc}{{\sf succ}}
%\newcommand{\inleft}{{\sf inleft}}
%\newcommand{\inright}{{\sf inright}}
%%\newcommand{\bbzero}{{0\hspace*{-4pt} 0}}
%\newcommand{\emptyt}{{\bf 0}}
%%\newcommand{\bbone}{{1\hspace*{-4pt} 1}}
%\newcommand{\unitt}{{\bf 1}}

\newcommand{\UU}{{\mathcal U}}
\newcommand{\idtypevar}{{{\sf Id}}}
\newcommand{\eq}{{{\sf Eq}}}


%%% Outline
%%	- The idea of HoTT and the special role of Identity

%
%%	- The idea of informal type theory
%
%%	- Brief history - ok
%
%%	- Constructive math vs. general: no proof by contradiction/ proof relevance/ -- still need to do
%
%%	- How is type theory different from set theory? - ok
%
%%	- How is HoTT different from type theory? -ok
%
%%	- Different readers of these notes: -ok 
%%		- Working mathematicians
%%		- ... CS
%%		- ... Logicians
%%		- homotopy theorists and higher category theorists
%
%%	- weak points and work to do -- still needs to be done
%
%%	- Coq development, new proof assistant, etc. - ok
%%



\subsection*{Type theory}

Type theory was originally invented by Bertrand Russell in 1908 \cite{Russell:1908}, as a device for blocking the paradoxes in the logical foundations of mathematics  that were under investigation at the time. It was later developed as a rigorous formal system  in its own right (under the name ``$\lambda$-calculus") by Alonzo Church \cite{Church:1933cl,Church:1940tu,Church:1941tc}.  Although it is not generally regarded as the foundation for classical mathematics, set theory being more customary, it still has numerous applications, especially in computer science and the theory of programming languages \cite{Pierce:2002tp}.   Per Martin-L\"{o}f \cite{MartinLof:1998tw,MartinLof:1975tb,MartinLof:1982bn,MartinLof:1984tr}, among others,
developed a generalization of Church's system which is now usually called dependent, constructive, or simply {\bf Martin\--L\"of type theory}; this is the basis of the system that we consider here. It was originally intended as a rigorous framework for the formalization of constructive mathematics.  Over the last 40 years it has also become clear that there are close connections between type theory and category theoretic approaches to logic and foundations, particularly topos theory \cite{elephant}

In type theory, unlike set theory, objects are classified using a primitive notion of \emph{type}, similar to the data-types used in programming languages.  These elaborately structured types can be used to express detailed specifications of the objects classified, giving rise to principles of reasoning about these objects.  To take a very simple example, the objects of a product type $A\times B$ are known to be of the form $\langle a, b\rangle$, and so one automatically knows how to construct them and how to decompose them. This is one aspect of type theory that has led to its extensive use in verifying the correctness of computer programs.  The clear reasoning principles associated with the construction of types also form the basis of modern {\bf computer proof assistants}, which are used for formalizing mathematics and verifying the correctness of formalized proofs.  We return to this aspect of type theory below.  (From the point of view of category theory, this is like saying that a product is determined by its universal property.)

%For example, the powerful Coq proof assistant \cite{coq} has recently been used to formalize and verify the correctness of the proof of the celebrated Feit-Thompson Odd-Order theorem \cite{gonthier}.

One problem with understanding type theory from a mathematical point of view, however, has always been that the basic concept of \emph{type} is unlike that of \emph{set} in ways that have been hard to make precise. This difficulty, we believe, has now been solved by the idea of regarding types, not as strange sets (perhaps constructed without using classical logic), but as spaces or, more precisely, as homotopy types of spaces.

In homotopy theory one is concerned with spaces and continuous mappings between them, 
up to homotopy; a \emph{homotopy} between a pair of continuous maps $f \colon X	\to Y$
and  $g \colon X	\to Y$ is 
a continuous map $H \colon X \times [0, 1]	\to Y$ satisfying
$H(x, 0) = f (x)$  and $H(x, 1) = g(x)$. The homotopy $H$ may be thought of as a ``continuous deformation" of $f$ into $g$. The spaces $X$ and $Y$ are said to be \emph{homotopy equivalent}, $X\simeq Y$, if there are continuous maps going back and forth, the composites of which are homotopical to the respective identity mappings, i.e.\ if they are isomorphic ``up to homotopy".  Homotopy equivalent spaces have the same algebraic invariants (e.g.\ homology, or the fundamental group), and are said to have the same \emph{homotopy type}.

\subsection*{Homotopy type theory}

Homotopy type theory (``HoTT") is a new field of mathematics which interprets type theory from a homotopical perspective.
In homotopy type theory, one regards the types as spaces, or homotopy types, and the logical constructions (such as the product $A\times B$) as homotopy-invariant constructions on spaces.   In this way, one is able to manipulate spaces directly without first having to develop point-set topology.
To briefly explain this perspective, consider the basic concept of type theory, namely that
the \emph{term} $a$ is of \emph{type} $A$, which is written:
$$
  a:A.
$$
This expression is traditionally thought of as akin to:
\begin{center}
``$a$ is an element of the set $A$."
\end{center}
However, in HoTT we think of it instead as:
\begin{center}
``$a$ is a point of the space $A$."
\end{center}
Similarly, every term $f : A\to B$ in type theory is regarded as a continuous function from the space $A$ to the space $B$. This perspective clarifies features of type theory which were puzzling from the perspective of types as sets; for instance, that one can have non-trivial types $D$ such that $D\cong D\to D$.  But the key new idea of the homotopy interpretation is that the logical notion of identity $a = b$ of two objects $a, b: A$ of the same type $A$ can be understood as the existence of a \emph{path} $p : a \leadsto b$ from point $a$ to point $b$ in the space $A$.  This also means that two functions $f, g: A\to B$ are identical just in case they are homotopic, since a homotopy is just a family of paths $p_x: f(x) \leadsto g(x)$ in $B$, one for each $x:A$.  In type theory, for every type $A$ there is a (formerly somewhat mysterious) type $\idtypevar_{A}$ of identities between objects of $A$; in HoTT, this is just the \emph{path space} $A^I$ of all continuous maps $I\to A$ from the unit interval.  In this way, a term $p : \idtypevar_{A}(a,b)$ represents a path $p : a \leadsto b$ in $A$. 

The idea of homotopy type theory arose around 2006 in independent work by Awodey and Warren (\cite{AW}) and Voevodsky (\cite{VV}), but it was inspired by 
Hofmann and Streicher's earlier groupoid interpretation~\cite{HofmannM:gromtt}.  Indeed, there is a precise sense in which HoTT relates type theory not only to homotopy theory, but also to higher-dimensional category theory, and in particular to weak $\infty$-groupoids --- corresponding to the close connections between those two subjects now under investigation in algebraic topology, and following the course proposed by Grothendieck.  For instance, the approach of Awodey and Warren uses the machinery of Quillen model categories, but Voevodsky's model of type theory uses Kan simplicial sets, which are one concept of $\infty$-groupoids.  Voevodsky recognized that this simplicial interpretation satisfies a further crucial property, dubbed \emph{univalence},  which is not usually assumed in type theory.  Adding univalence  to type theory in the form of a new axiom has far-reaching consequences, many of which are natural, simplifying and compelling.  The Univalence Axiom also further strengthens the homotopical view of type theory, since it holds in the simplicial model, while failing in the view of types as sets.  

\subsection*{Univalent Foundations}

Very briefly, the basic idea of the Univalence Axiom can be explained as follows.  In type theory, one can have a universe $\UU$, the terms of which are themselves types, $A : \UU$, etc.  Of course, we do not have $\UU:\UU$, so only some types are terms of $\UU$ -- call these the \emph{small} types.  Like any type, $\UU$ has an identity type $\idtypevar_{\UU}$ which expresses the identity relation $A = B$ among small types.  Thinking of  types as spaces, $\UU$ is a space, the points of which are spaces; to understand its identity type, we must ask, what is a path $p : A \leadsto B$ between spaces in $\UU$?  The univalence axiom says that such paths correspond to homotopy equivalences $A\simeq B$, as explained above.  A bit more precisely, given any (small) types $A$ and $B$, in addition to the type $\idtypevar_{\UU}(A,B)$ of identities between $A$ and $B$ there is the type $\eq(A,B)$ of equivalences from $A$ to $B$.  Since the identity map on any object is an equivalence, there is a canonical map,
$$\idtypevar_{\UU}(A,B)\to\eq(A,B).$$
The univalence axiom states that this map is itself an equivalence.  At the risk of oversimplifying, we can state this succinctly as follows:

\begin{description}
\item[Univalence Axiom:]  $(A = B)\ \simeq\ (A\simeq B)$.
\end{description}
%
In other words, identity is equivalent to equivalence. 

From the homotopical point of view, this says that the universe $\UU$ is something like a classifying space for (small) homotopy types, which is a practical and natural assumption.  From the  logical point of view, however, it is revolutionary: it says that isomorphic things are identical!  Mathematicians are of course used to identifying isomorphic structures in practice, but they generally do so with a wink, knowing that the identification is not ``officially" justified by foundations.  But in this new foundational scheme, not only are such structures formally identified, but the different ways in which such identifications may be made themselves form a structure that one can (and should!) take into account.

Voevodsky has christened this new foundational scheme, consisting of the combination of homotopy type theory with the univalence axiom, and possibly some further logical principles, together with an implementation in a computer proof assistant, the {\bf Univalent Foundations of Mathematics}.

\subsection*{Informal type theory}

One difficulty often encountered by the classical mathematician when faced with learning about type theory is that it is usually presented as a fully or partially formalized deductive system.  This style, which is very useful for  proof-theoretic investigations, is not particularly convenient for use in applied, informal reasoning, nor is it even familiar to most working mathematicians, even those who might be interested in foundations of mathematics.  One objective of the present work is to develop an informal style of working \emph{in} HoTT that is at once rigorous and precise, but is also closer to the language and style of presentation of everyday mathematics.    In mathematics, one usually constructs and reasons about mathematical objects in a way that could in principle, one presumes, be formalized in a system of elementary set theory like ZFC -- at least given enough ingenuity and patience.  For the most part, one does not even need to be aware of this possibility, since it largely coincides with the condition that a proof be fully rigorous.  But there are a few aspects of working in ``informal set theory" that one does need to learn to be careful about: the use of collections too large or inchoate to  be sets, the axiom of choice and its equivalents,  the method of proof by contradiction, and so on.  Adopting a new foundational system such as HoTT as the \emph{implicit formal basis} of informal reasoning will require adjusting some of ones instincts and practices, even when reasoning informally.  The present text is intended to serve as an example of this ``new kind of mathematics", which is still informal, but could now in principle be formalized in HoTT, rather than ZFC, again given enough ingenuity and patience.

It is worth emphasizing that in the new system, however, such formalization has a real practical benefit, increasing the likelihood that one will actually want to carry it out; namely, the implementation of the formal system in a computer proof assistant.  In practical terms, this means that it is possible to use currently available proof assistants based on type theory to develop mathematics, to verify the correctness of proofs, to provide some degree of automation, and even to extract numerical algorithms from formal proofs.  We believe that this aspect of the univalent foundations program distinguishes it from other approaches to foundations, by providing a real practical utility for the working mathematician. Indeed, many of the results given here were actually \emph{first} done in a fully formalized form in a proof assistant, and are only now being ``unformalized" for the first time -- a reversal of the usual relation between formal and informal mathematics.   One can imagine a not too distant future when it will be possible for mathematicians  to verify the correctness of their own papers by working within the system of univalent foundations, formalized in a proof assistant, and that doing so will become as natural as typesetting their own papers in Tex. (Whether this proves to be the publishers' dream or their nightmare  remains to be seen.) 

%We refer the reader to \cite{Simpson:2004bt,Hales:2008ud} for two accounts of the use of computer proof assistants in general.   


\subsection*{Constructivity} 

Constructive math vs. general: no proof by contradiction/ proof relevance/
- Proof relevance and classical logic

\subsection*{Weak points and open problems} 

For those interested in contributing to this new branch of mathematics, it may be encouraging to know that there are many interesting open questions.  The most pressing of them is the ``constructivity'' of the Univalence Axiom itself, posed by Voevodsky in \cite{Vo2012}.  


\subsection*{Some references?}

Some other surveys and introductions?

}%%%%%%%%%%%%% end of scope of local macros

% Local Variables:
% TeX-master: "main"
% End:


\part{Foundations}
\label{part:foundations}


\chapter{Introduction to type theory}
\label{sec:typetheory}

% Local Variables:
% TeX-master: "main"
% End:


\chapter{Homotopy type theory}
\label{cha:basics}

\section{Types are higher groupoids}
\label{sec:equality}

Recall that for any type $A$, and any $x,y:A$, we have a identity type $\id[A]{x}{y}$, also written $\idtype[A]{x}{y}$ or just $x=y$.
We can think of inhabitants of $x=y$ as either
\begin{enumerate}
\item evidence that $x$ and $y$ are equal,
\item identifications of $x$ with $y$,
\item paths from $x$ to $y$, or
\item isomorphisms/equivalences from $x$ to $y$.
\end{enumerate}
The first is more traditional in type theory; but in homotopy type theory we often take the latter perspectives as well.
It turns out that the defining rules of identity types, as described in the previous chapter, give them structure which corresponds precisely to that of the continuous paths and (higher) homotopies between them in a space, or a (weak) higher-dimensional groupoid.

This interpretation depends on the ability to \emph{iterate} the identity types.
Recall that type theory (unlike, say, first-order logic) does not distinguish between individuals (e.g.\ $x:A$), which may be the subjects of propositions, and proofs of propositions (e.g.\ evidence for $\id[A]{x}{y}$).
Thus, the identity type can be iterated: we can form the type $\id[{\id[A]{x}{y}}]{p}{q}$ of identifications between identifications $p,q$, and the type $\id[{\id[{\id[A]{x}{y}}]{p}{q}}]{r}{s}$, and so on.
(Obviously, the notation ``$\id[A]{x}{y}$'' has its limitations here.
The style $\idtype[A]{x}{y}$ is only slightly better in iterations: $\idtype[{\idtype[{\idtype[A]{x}{y}}]{p}{q}}]{r}{s}$.)

If the  type $A$ is ``set-like'', such as \nat, these iterated identity types will be uninteresting (see \autoref{sec:basics-sets}).
However, under the interpretation of $p,q : \id[A]{x}{y}$ as paths from $x$ to $y$, an element $r : \id[{\id[A]{x}{y}}]{p}{q}$ can be
thought of as a \emph{path between paths}, or a \emph{homotopy}, and visualized as a 2-dimensional path that fills in the space between $p$ and $q$.
Similarly, $\id[{\id[{\id[A]{x}{y}}]{p}{q}}]{r}{s}$ is the type of 3-dimensional paths between two 2-dimensional paths, and so on.
This ``tower of identity types'' is what we claim has the structure of a higher groupoid.

One of the amazing things about homotopy type theory is that all the higher groupoid structure arises automatically from the induction principle for identity types, as stated in \autoref{sec:identity-types}.
Recall that this says that if
\begin{itemize}
\item for every $x,y:A$ and every $p:\id[A]xy$ we have a type $D(x,y,p)$, and
\item for every $a:A$ we have an element $d(a):D(a,a,\refl a)$, 
\end{itemize}
then
\begin{itemize}
\item there exists an element $J_{D,d}(x,y,p):D(x,y,p)$ for \emph{every} two elements $x,y:A$ and $p:\id[A]xy$, such that $J_{D,d}(a,a,\refl a) \jdeq d(a)$.
\end{itemize}
In other words, given dependent functions
\begin{align*}
D & :\prd{x,y:A}{p:\id{x}{y}} \type\\
d & :\prd{a:A} D(a,a,\refl{a})
\end{align*}
there is a dependent function
\[J_{D,d}:\prd{x,y:A}{p:\id{x}{y}} D(x,y,p)\]
such that 
\begin{equation}\label{eq:Jconv}
J_{D,d}(a,a,\refl{a})\jdeq d(a)
\end{equation}
for every $a:A$.
The notation $J$ is traditional for this function, but we will not use it very much.
Usually, every time we apply this induction rule we will either not care about the specific function being defined, or we will immediately give it a different name.

Informally, the induction principle for identity types says that if we want to construct an object (or prove a statement) which depends on an inhabitant $p:\id[A]xy$ of an identity type, then it suffices to perform the construction (or the proof) in the special case when $x$ and $y$ are the same (judgmentally) and $p$ is the reflexivity element $\refl{x}:x=x$ (judgmentally).
When writing informally, we may express this with a phrase such as ``by induction, it suffices to assume\dots''.
This reduction to the ``reflexivity case'' is analogous to the reduction to the ``base case'' and ``inductive step'' in an ordinary proof by induction on the natural numbers, and also to the ``left case'' and ``right case'' in a proof by case analysis on a disjoint union or disjunction.

The ``conversion rule''~\eqref{eq:Jconv} is less familiar in the context of proof by induction on natural numbers, but there is an analogous notion in the related concept of definition by recursion.
If a sequence $(a_n)_{n\in \mathbb{N}}$ is defined by giving $a_0$ and specifying $a_{n+1}$ in terms of $a_n$, then in fact the $0^{\mathrm{th}}$ term of the resulting sequence \emph{is} the given one, and the given recurrence relation relating $a_{n+1}$ to $a_n$ holds for the resulting sequence.
(This may seem so obvious as to not be worth saying, but if we view a definition by recursion as an algorithm for calculating values of a sequence, then it is precisely the process of executing that algorithm.)
The rule~\eqref{eq:Jconv} is analogous: it says that if we define an object $f(p)$ for all $p:x=y$ by specifying what the value should be when $p$ is $\refl{x}:x=x$, then the value we specified is in fact the value of $f(\refl{x})$.

We now derive from this induction principle the beginnings of the structure of a higher groupoid.
We begin with symmetry of equality, which, in topological language, means that ``paths can be reversed''.

\begin{lem}\label{lem:opp}
  For every type $A$ and every $x,y:A$ there is a function
  \begin{equation*}
    (x= y)\to(y= x)
  \end{equation*}
  denoted $p\mapsto \opp{p}$, such that $\opp{\refl{x}}\jdeq\refl{x}$ for each $x:A$.
\end{lem}
\begin{proof}[First proof]
  Let $D:\prd{x,y:A}{p:x= y} \type$ be the type family defined by $D(x,y,p)\defeq (y= x)$.
  In other words, $D$ is a function assigning to any $x,y:A$ and $p:x=y$ a type, namely the type $y=x$.
  Then we have an element
  \begin{equation*}
    d\defeq \lam{x} \refl{x}:\prd{x:A} D(x,x,\refl{x}).
  \end{equation*}
  Thus, the eliminator $J$ for identity types gives us an element $J_{D,d}(x,y,p): (y= x)$ for each $p:(x= y)$.
  We can now define the desired function $\opp{(-)}$ to be $\lam{p} J_{D,d}(x,y,p)$, i.e.\ we set $\opp{p} \defeq J_{D,d}(x,y,p)$.
  The conversion rule~\eqref{eq:Jconv} gives $\opp{\refl{x}}\jdeq \refl{x}$, as required.
\end{proof}

We have written out this proof in a very formal style, which may be helpful while the induction rule on identity types is unfamiliar.
However, eventually we prefer to use more natural language, such as in the following equivalent proof.

\begin{proof}[Second proof]
  We want to construct, for each $x,y:A$ and $p:x=y$, an element $\opp{p}:y=x$.
  By induction, it suffices to do this in the case when $y$ is $x$ and $p$ is $\refl{x}$.
  But in this case, the type $x=y$ of $p$ and the type $y=x$ in which we are trying to construct $\opp{p}$ are both simply $x=x$.
  Thus, in the ``reflexivity case'', we can define $\opp{\refl{x}}$ to be simply $\refl{x}$.
  The general case then follows by the induction principle, and the conversion rule $\opp{\refl{x}}\jdeq\refl{x}$ is precisely the proof in the reflexivity case that we gave.
\end{proof}

We will write out the next few proofs in both styles, to help the reader become accustomed to the latter one.
Next we prove the transitivity of equality, or equivalently we ``concatenate paths''.

\begin{lem}\label{lem:concat}
  For every type $A$ and every $x,y,z:A$ there is a function
  \begin{equation*}
  (x= y) \to   \big((y= z)\to (x=  z)\big)
  \end{equation*}
  written $(p,q)\mapsto p\ct q$, such that $\refl{x}\ct \refl{x}\jdeq \refl{x}$ for any $x:A$.
\end{lem}

\begin{proof}[First proof]
  Let $D:\prd{x,y:A}{p:x=y} \type$ be the type family
  \begin{equation*}
    D(x,y,p)\defeq \prd{z:A}{q:y=z} (x=z).
  \end{equation*}
  Note that $D(x,x,\refl x) \jdeq \prd{z:A}{q:x=z} (x=z)$.
  Thus, in order to apply the induction principle for identity types to this $D$, we need a function of type
  \begin{equation}\label{eq:concatD}
    \prd{x:A} D(x,x,\refl{x})
  \end{equation}
  which is to say, of type
  \[ \prd{x,z:A}{q:x=z} (x=z). \]
  Now let $E:\prd{x,z:A}{q:x=z}\type$ be the type family $E(x,z,q)\defeq (x=z)$.
  Note that $E(x,x,\refl x) \jdeq (x=x)$.
  Thus, we have the function
  \begin{equation*}
    e(x) \defeq \refl{x} : E(x,x,\refl{x}).
  \end{equation*}
  By the induction principle for identity types applied to $E$, we obtain a function
  \begin{equation*}
    d(x,z,q) : \prd{x,z:A}{q:x=z} E(x,z,q).
  \end{equation*}
  But $E(x,z,q)\jdeq (x=z)$, so this is~\eqref{eq:concatD}.
  Thus, we can use this function $d$ and apply the induction principle for identity types to $D$, to obtain our desired function of type
  \begin{equation*}
    \prd{x,y,z:A}{q:y=z}{p:x=y} (x=z).
  \end{equation*}
  The conversion rules for the two induction principles give us $\refl{x}\ct \refl{x}\jdeq \refl{x}$ for any $x:A$.
\end{proof}

\begin{proof}[Second proof]
  We want to construct, for every $x,y,z:A$ and every $p:x=y$ and $q:y=z$, an element of $x=z$.
  By induction on $p$, it suffices to assume that $y$ is $x$ and $p$ is $\refl{x}$.
  In this case, the type $y=z$ of $q$ is $x=z$.
  Now by induction on $q$, it suffices to assume also that $z$ is $x$ and $q$ is $\refl{x}$.
  But in this case, $x=z$ is $x=x$, and we have $\refl{x}:(x=x)$.
\end{proof}

The reader may well feel that we have given an overly convoluted proof of this lemma.
In fact, we could stop after the induction on $p$, since at that point what we want to produce is an equality $x=z$, and we already have such an equality, namely $q$.
Why do we go on to do another induction on $q$?

The answer is that, as described in the introduction, we are doing \emph{proof-relevant} mathematics.
When we prove a lemma, we are defining an inhabitant of some type, and it can matter what \emph{specific} element we defined in the course of the proof, not merely the type that that element inhabits (that is, the \emph{statement} of the lemma).
\autoref{lem:concat} has three obvious proofs: we could do induction over $p$, induction over $q$, or induction over both of them.
If we proved it three different ways, we would have three different elements of the same type.
It's not hard to show that these three elements are equal (see \autoref{ex:basics:concat}), but as they are not \emph{definitionally} equal, there can still be reasons to prefer one over another.

In the case of \autoref{lem:concat}, the difference hinges on the computation rule.
If we proved the lemma using a single induction over $p$, then we would end up with a computation rule of the form $\refl{y} \ct q \jdeq q$.
If we proved it with a single induction over $q$, we would have instead $p\ct\refl{x}\jdeq p$, while proving it with a double induction (as we did) gives only $\refl{x}\ct\refl{x} \jdeq \refl{x}$.

The asymmetrical computation rules can sometimes be convenient when doing formalized mathematics, as they allow the computer to simplify more things automatically.
However, in informal mathematics, and arguably even in the formalized case, it can be confusing to have a concatenation operation which behaves asymmetrically and to have to remember which side is the ``special'' one.
Treating both sides symmetrically makes for more robust proofs; this is why we have given the proof that we did.
(However, this is admittedly a stylistic choice.)

The table below summarizes the ``equality'' and ``homotopical'' points of view on what we have done so far.
\begin{center}
  \begin{tabular}{c|c}
    Equality & Homotopy \\\hline
    reflexivity & constant path\\
    symmetry & inversion of paths\\
    transitivity & concatenation of paths
  \end{tabular}
\end{center}

Because of proof-relevance, we can't stop after proving ``symmetry'' and ``transitivity'' of equality: we need to know that these \emph{operations} on equalities are well-behaved.
(This issue is invisible in set theory, where symmetry and transitivity are mere \emph{properties} of equality, rather than structure on
paths.)
From the homotopy-theoretic point of view, concatenation and inversion are just the ``first level'' of higher groupoid structure --- we also need coherence laws on these operations, and analogous operations at higher dimensions.
For instance, we need to know that concatenation is \emph{associative}, and that inversion provides \emph{inverses} with respect to concatenation.

\begin{lem}\label{thm:omg}%[The $\omega$-groupoid structure of types]
  Suppose $A:\type$, that $x,y,z,w:A$ and that $p:x= y$ and $q:y = z$ and $r:z=w$.
  We have the following:
  \begin{enumerate}
  \item $p= p\ct \refl{y}$ and $p = \refl{x} \ct p$.\label{item:omg1}
  \item $\opp{p}\ct p=  \refl{y}$ and $p\ct \opp{p}= \refl{x}$.
  \item $\opp{(\opp{p})}= p$.
  \item $p\ct (q\ct r)=  (p\ct q)\ct r$.\label{item:omg4}
  \end{enumerate}
\end{lem}

Note, in particular, that~\ref{item:omg1}--\ref{item:omg4} are themselves propositional equalities, living in the identity types \emph{of} identity types, such as $p=_{x=y}q$ for $p,q:x=y$.
Topologically, they are \emph{paths of paths}, i.e.\ homotopies.
It is a familiar fact in topology that when we concatenate a path $p$ with the reversed path $\opp p$, we don't literally obtain a constant path (which corresponds to the equality $\refl{}$ in type theory) --- instead we have a homotopy, or higher path, from $p\ct\opp p$ to the constant path.

\begin{proof}[Proof of~\autoref{thm:omg}]
  All the proofs use the induction principle for equalities.
  \begin{enumerate}
  \item \emph{(First proof)} Let $D:\prd{x,y:A}{p:x=y} \type$ be the type family given by 
    \begin{equation*}
      D(x,y,p)\defeq (p= p\ct \refl{y}).
    \end{equation*}
    Then $D(x,x,\refl{x})$ is $\refl{x}=\refl{x}\ct\refl{x}$.
    Since $\refl{x}\ct\refl{x}\jdeq\refl{x}$, it follows that $D(x,x,\refl{x})\jdeq (\refl{x}=\refl{x})$.
    Thus, there is a function
    \begin{equation*}
      d\defeq\lam{x} \refl{\refl{x}}:\prd{x:A} D(x,x,\refl{x}).
    \end{equation*}
    Now the induction principle for identity types gives an element $J(D,d,p):(p= p\ct\refl{y})$ for each $p:x= y$.
    The other equality is proven similarly.

    \noindent
    \emph{(Second proof)} By induction on $p$, it suffices to assume that $y$ is $x$ and that $p$ is $\refl x$.
    But in this case, we have $\refl{x}\ct\refl{x}\jdeq\refl{x}$.
  \item \emph{(First proof)} Let $D:\prd{x,y:A}{p:x=y} \type$ be the type family given by 
    \begin{equation*}
      D(x,y,p)\defeq (\opp{p}\ct p=  \refl{y}).
    \end{equation*}
    Then $D(x,x,\refl{x})$ is $\opp{\refl{x}}\ct\refl{x}=\refl{x}$.
    Since $\opp{\refl{x}}\jdeq\refl{x}$ and $\refl{x}\ct\refl{x}\jdeq\refl{x}$, we get that $D(x,x,\refl{x})\jdeq (\refl{x}=\refl{x})$.
    Hence we find the function
    \begin{equation*}
      d\defeq\lam{x} \refl{\refl{x}}:\prd{x:A} D(x,x,\refl{x}).
    \end{equation*}
    Now path induction gives an element $J(D,d,p):\opp{p}\ct p=\refl{y}$ for each $p:x= y$ in $A$.
    The other equality is similar.

    \noindent \emph{(Second proof)} By induction, it suffices to assume $p$ is $\refl x$.
    But in this case, we have $\opp{p} \ct p \jdeq \opp{\refl x} \ct \refl x \jdeq \refl x$.
  \item \emph{(First proof)} Let $D:\prd{x,y:A}{p:x=y} \type$ be the type family given by
    \begin{equation*}
      D(x,y,p)\defeq (\opp{\opp{p}}= p).
    \end{equation*}
    Then $D(x,x,\refl{x})$ is the type $(\opp{\opp{\refl x}}=\refl{x})$.
    But since $\opp{\refl{x}}\jdeq \refl{x}$ for each $x:A$, we have $\opp{\opp{\refl{x}}}\jdeq \opp{\refl{x}} \jdeq\refl{x}$, and thus $D(x,x,\refl{x})\jdeq(\refl{x}=\refl{x})$.
    Hence we find the function
    \begin{equation*}
      d\defeq\lam{x} \refl{\refl{x}}:\prd{x:A} D(x,x,\refl{x}).
    \end{equation*}
    Now path induction gives an element $J(D,d,p):\opp{\opp{p}}= p$ for each $p:x= y$.

    \noindent \emph{(Second proof)} By induction, it suffices to assume $p$ is $\refl x$.
    But in this case, we have $\opp{\opp{p}}\jdeq \opp{\opp{\refl x}} \jdeq \refl x$.
  \item \emph{(First proof)} Let $D_1:\prd{x,y:A}{p:x=y} \type$ be the type family given by
    \begin{equation*}
      D_1(x,y,p)\defeq\prd{z,w:A}{q:y= z}{r:z= w} \big(p\ct (q\ct r)=  (p\ct q)\ct r\big).
    \end{equation*}
    Then $D_1(x,x,\refl{x})$ is
    \begin{equation*}
      \prd{z,w:A}{q:x= z}{r:z= w} \big(\refl{x}\ct(q\ct r)= (\refl{x}\ct q)\ct r\big).
    \end{equation*}
    To construct an element of this type, let $D_2:\prd{x,z:A}{q:x=z} \type$ be the type family
    \begin{equation*}
      D_2 (x,z,q) \defeq \prd{w:A}{r:z=w} \big(\refl{x}\ct(q\ct r)= (\refl{x}\ct q)\ct r\big).
    \end{equation*}
    Then $D_2(x,x,\refl{x})$ is
    \begin{equation*}
      \prd{w:A}{r:x=w} \big(\refl{x}\ct(\refl{x}\ct r)= (\refl{x}\ct \refl{x})\ct r\big).
    \end{equation*}
    To construct an element of \emph{this} type, let $D_3:\prd{x,w:A}{r:x=w} \type$ be the type family
    \begin{equation*}
      D_3(x,w,r) \defeq \big(\refl{x}\ct(\refl{x}\ct r)= (\refl{x}\ct \refl{x})\ct r\big).
    \end{equation*}
    Then $D_3(x,x,\refl{x})$ is
    \begin{equation*}
      \big(\refl{x}\ct(\refl{x}\ct \refl{x})= (\refl{x}\ct \refl{x})\ct \refl{x}\big)
    \end{equation*}
    which is definitionally equal to the type $(\refl{x} = \refl{x})$, and is therefore inhabited by $\refl{\refl{x}}$.
    Applying the identity elimination rule three times, therefore, we obtain an element of the overall desired type.

    \noindent \emph{(Second proof)} By induction, it suffices to assume $p$, $q$, and $r$ are all $\refl x$.
    But in this case, we have
    \begin{align*}
      p\ct (q\ct r)
      &\jdeq \refl{x}\ct(\refl{x}\ct \refl{x})\\
      &\jdeq \refl{x}\\
      &\jdeq (\refl{x}\ct \refl x)\ct \refl x\\
      &\jdeq (p\ct q)\ct r.
    \end{align*}
    Thus, we have $\refl{\refl{x}}$ inhabiting this type.\qedhere
  \end{enumerate}
\end{proof}

\begin{rmk}
  There are other ways to define all of these higher paths.
  For instance, in \autoref{thm:omg}\ref{item:omg4} we might do induction only over one or two paths rather than all three.
  Each possibility will produce a \emph{definitionally} different proof, but they will all be equal to each other.
  Such an equality between any two particular proofs can, again, be proven by induction, reducing all the paths in question to reflexivities and then observing that both proofs reduce themselves to reflexivities.
\end{rmk}

We are still not really done with the higher groupoid structure: the paths~\ref{item:omg1}--\ref{item:omg4} must also satisfy their own higher coherence laws, which are themselves higher paths, and so on ``all the way up to infinity" (this can be made precise using e.g.\ the notion of a globular operad).
However, for most purposes it is unnecessary to make the whole infinite-dimensional structure explicit.
One of the nice things about homotopy type theory is that all of this structure can be \emph{proven} starting from only the inductive property of identity types, so we can make explicit as much or as little of it as we need.
In particular, in this book we will not need any of the complicated combinatorics involved in making precise notions such as ``coherent structure at all higher levels''.

One particularly important case of iterated identity types (path types) is when the start and end points are the same.
In set theory, the propositon $a=a$ is entirely uninteresting, but in homotopy theory paths from a point to itself are called \emph{loops} and carry lots of interesting higher structure.
Thus, given a type $A$ with a point $a:A$, we define its \emph{loop space} $\Omega(A,a)$ to be the type $\id[A]{a}{a}$.
We may sometimes write simply $\Omega A$ if the point $a$ is understood from context.

Since any two elements of $\Omega A$ are paths with the same start and end points, they can be concatenated;
thus we have an operation $\Omega A\times \Omega A\to \Omega A$.
More generally, the higher groupoid structure of $A$ gives $\Omega A$ the analogous structure of a ``higher group''.

It can also be useful to consider the loop space \emph{of} the loop space of $A$, which is the space of 2-dimensional loops on the identity loop in $a$.
This is written $\Omega^2(A,a)$ and represented in type theory by the type $\id[({\id[A]{a}{a}})]{\refl{a}}{\refl{a}}$.
While $\Omega^2(A,a)$, as a loop space, is again a ``higher group", it now also has some additional structure resulting from the fact that its elements are 2-dimensional loops between 1-dimensional loops.  

\begin{thm}[Eckmann-Hilton]\label{thm:EckmannHilton}
The composition operation on the second loop space $$\Omega^2(A)\times \Omega^2(A)\to \Omega^2(A)$$  is commutative: $\alpha\ct\beta = \beta\ct\alpha$, for any $\alpha, \beta:\Omega^2(A)$.
\end{thm}

\begin{proof}
First, observe that the composition of $1$-loops $\Omega A\times \Omega A\to \Omega A$ induces an operation 
$$
\star : \Omega^2(A)\times \Omega^2(A)\to \Omega^2(A)
$$
 as follows: consider elements $a, b, c : A$ and 1- and 2-paths,
%
$$
\begin{array}{ccc}
p: a = b & , & r : b = c \\
q : a = b & , & s : b = c \\
\alpha : p = q & , & \beta : r = s
\end{array}
$$
%
as depicted in the following diagram.
$$
\ \xy
(0,0)*+{a}="a";
(45,0)*+{b}="b";
(90,0)*+{c}="c";
{\ar@/^3pc/^{p} "a";"b"};
{\ar@/_3pc/_{q} "a";"b"};
{\ar@/^3pc/^{r} "b";"c"};
{\ar@/_3pc/_{s} "b";"c"};
{\ar@{=>} (22,10)*{};(22,-10)*{}};
(28,0)*{\alpha};
{\ar@{=>} (67,10)*{};(67,-10)*{}};
(72,0)*{\beta};
\endxy \ 
$$

Composing the upper and lower 1-paths, respectively, we get two paths $p\ct r,\ q\ct s : a = c$, and there is then a ``horizontal composition" $$\alpha\star\beta : p\ct r = q\ct s$$ between them,
defined as follows: first let $\alpha \rightwhisker r : p\ct r = q\ct r$ be determined by path induction on $r$, then let $q\leftwhisker \beta : q\ct r = q\ct s$ be given by induction on $q$.  Since these paths are composable in the type of 2-paths, we can define the \emph{horizontal composition}  by:
\[
\alpha\star\beta\ \defeq\ (\alpha\rightwhisker r) \ct (q\leftwhisker \beta)\, .
\]
Now suppose that $a \jdeq  b \jdeq  c$, so that all the above 1-paths are in $\Omega(A,a)$, and assume moreover that $q \jdeq  \refl{a}\jdeq  r$, so that $\alpha$ and $\beta$ become composable.  In that case, we therefore have
\[
\alpha\star\beta\ =\ (\alpha\rightwhisker\refl{a}) \ct (\refl{a}\leftwhisker \beta) = \alpha \ct \beta\, .
\]
On the other hand, we can define another horizontal composition analogously by
\[
\alpha\star'\beta\ \defeq\ (p\leftwhisker \beta)\ct (\alpha\rightwhisker s)\, .
\]
and setting $p \jdeq  \refl{a}\jdeq  s$ we learn that in that case 
\[
\alpha\star'\beta\ =\ (\refl{a}\leftwhisker \beta)\ct (\alpha\rightwhisker \refl{a}) = \beta\ct\alpha\, .
\]
But, in general, the two ways of defining horizontal composition agree, $\alpha\star\beta = \alpha\star'\beta$, as we can see by induction on $\alpha$ and $\beta$.  Thus when $p \jdeq  q \jdeq  \refl{a} \jdeq  r\jdeq  s$ we have
\[\alpha \ct \beta = \alpha\star\beta = \alpha\star'\beta = \beta\ct\alpha\,.
\qedhere
\]
\end{proof}

The foregoing fact, which is known as the \emph{Eckmann-Hilton argument}, comes from classical homotopy theory,  and indeed it is used in \autoref{cha:homotopy} below to show that the higher homotopy groups of a type are always abelian groups. 

As this example suggests, the algebra of higher path types is much more intricate than just the groupoid-like structure at each level; the levels interact to give many further operations and laws, as in the study of iterated loop spaces in homotopy theory.
Indeed, as in classical homotopy theory, we can make the following general definitions:

\begin{defn} \label{def:pointedtype}
  A \textbf{pointed type} $(A,a)$ is a type $A:\type$ together with a point $a:A$.
  We write $\pointed{\type} \defeq \sm{A:\type} A$ for the type of pointed types in the universe $\type$.
\end{defn}

\begin{defn} \label{def:loopspace}
  Given a pointed type $(A,a)$, we define the \textbf{loop space} of $(A,a)$ to be the following pointed type:
  \[\Omega(A,a)=((\id[A]aa),\refl{A}(a)).\]
  For $n:\N$, the \textbf{$n$-fold iterated loop space} of a pointed type $(A,a)$ is defined recursively by:
  \begin{align*}
    \Omega^0(A,a)&=(A,a)\\
    \Omega^{n+1}(A,a)&=\Omega^n(\Omega(A,a)).
  \end{align*}
\end{defn}

We will return to iterated loop spaces in \autoref{cha:hlevels,cha:hits,cha:homotopy}.


\section{Functions are functors}
\label{sec:functors}

Now we wish to establish that functions $f:A\to B$ behave functorially on paths.
In traditional type theory, this is equivalently the statement that functions respect equality.
Topologically, this corresponds to saying that every function is ``continuous'', i.e.\ preserves paths.

\begin{lem}\label{lem:map}
  Suppose that $f:A\to B$ is a function.
  Then for any $x,y:A$ there is an operation
  \begin{equation*}
    \apfunc f : (\id[A] x y) \to (\id[B] {f(x)} {f(y)}).
  \end{equation*}
  Moreover, for each $x:A$ we have $\apfunc{f}(\refl{x})\jdeq \refl{f(x)}$.
\end{lem}

The notation $\apfunc f$ can be read either as the \underline{ap}plication of $f$ to a path, or as the \underline{a}ction on \underline{p}aths of $f$.

\begin{proof}[First proof]
  Let $D:\prd{x,y:A}{p:x=y}\type$ be the type family defined by
  \[D(x,y,p)\defeq (f(x)= f(y)).\]
  Then we have
  \begin{equation*}
    d\defeq\lam{x} \refl{f(x)}:\prd{x:A} D(x,x,\refl{x}).
  \end{equation*}
  Applying $J$, we obtain $\apfunc f : \prd{x,y:A}{p:x=y}(f(x)=g(x))$.
  The conversion rule implies $\apfunc f({\refl{x}})\jdeq\refl{f(x)}$ for each $x:A$.
\end{proof}

\begin{proof}[Second proof]
  By induction, it suffices to assume $p$ is $\refl{x}$.
  In this case, we may define $\apfunc f(p) \defeq \refl{f(x)}:f(x)\jdeq f(x)$.
\end{proof}

We will often write $\apfunc f (p)$ as simply $\ap f p$.
This is strictly speaking ambiguous, but generally no confusion arises.
It matches the common convention in category theory of using the same symbol for the application of a functor to objects and to morphisms.

We note that $\apfunc{}$ behaves functorially, in all the ways that one might expect.

\begin{lem}\label{lem:ap-functor}
  For functions $f:A\to B$ and $g:B\to C$ and paths $p:\id[A]xy$ and $q:\id[b]yz$, we have:
  \begin{enumerate}
  \item $\apfunc f(p\ct q) = \apfunc f(p) \ct \apfunc f(q)$.\label{item:apfunctor-ct}
  \item $\apfunc f(\opp p) = \opp{\apfunc f (p)}$.\label{item:apfunctor-opp}
  \item $\apfunc g (\apfunc f(p)) = \apfunc{g\circ f} (p)$.\label{item:apfunctor-compose}
  \end{enumerate}
\end{lem}
\begin{proof}
  Left to the reader.
\end{proof}

Now, since \emph{dependently typed} functions are essential in type theory, we will also need a version of \autoref{lem:map} for these.
However, this is not quite so simple to state, because if $f:\prd{x:A} B(x)$ and $p:x=y$, then $f(x):B(x)$ and $f(y):B(y)$ are elements of distinct types, so that \emph{a priori} we cannot even ask whether they are equal.
The missing ingredient is that $p$ itself gives us a way to relate the types $B(x)$ and $B(y)$.

\begin{lem}[Transport]\label{lem:transport}
  Suppose that $P$ is a type family over $A$ and that $p:\id[A]xy$.
  Then there is a function $\transf{p}:P(x)\to P(y)$.
\end{lem}

\begin{proof}[First proof]
  Let $D:\prd{x,y:A}{p:\id{x}{y}} \type$ be the type family defined by
  \[D(x,y,p)\defeq P(x)\to P(y).\]
  Then we have the function
  \begin{equation*}
    d\defeq\lam{x} \idfunc[P(x)]:\prd{x:A} D(x,x,\refl{x}),
  \end{equation*}
  so that the induction principle gives us $J_{D,d}(x,y,p):P(x)\to P(y)$ for $p:x= y$, which we define to be $\transf p$.
\end{proof}

\begin{proof}[Second proof]
  By induction, it suffices to assume $p$ is $\refl x$.
  But in this case, we can take $\transf{(\refl x)}:P(x)\to P(x)$ to be the identity function.
\end{proof}

Sometimes, it is necessary to notate the type family $P$ in which the transport operation happens.
In this case, we may write
\[\transfib P p - : P(x) \to P(y).\]

Recall that a type family $P$ over a type $A$ can be seen as a property of elements of $A$, which holds at $x$ in $A$ if $P(x)$ is inhabited.
Then the transportation lemma says that $P$ respects equality, in the sense that if $x$ is equal to $y$, then $P(x)$ holds if and only if $P(y)$ holds.
In fact, we will see later on that if $x=y$ then actually $P(x)$ and $P(y)$ are \emph{equivalent}.

Topologically, the transportation lemma can be viewed as a ``path lifting'' operation in a fibration.
We think of a type family $P:A\to \type$ as a fibration with base $A$ and total space $\sm{x:A}P(x)$, with $P(x)$ being the fiber over $x$.
The defining property of a fibration is that given a path $p:x=y$ in the base space $A$ and a point $u:P(x)$ in the fiber over $x$, we may lift the path $p$ to a path in the total space starting at $u$.
The point $\trans p u$ can be thought of as the other endpoint of this lifted path.
We can also define the path itself in type theory:

\begin{lem}[Path lifting property]\label{thm:path-lifting}
  Let $P:A\to\type$ be a type family over $A$ and assume we have $u:P(x)$ for some $x:A$.
  Then for any $p:x=y$, we have
  \begin{equation*}
    \mathsf{lift}(u,p):(x,u)=(y,\trans{p}{u})
  \end{equation*}
  in $\sm{x:A}P(x)$.
\end{lem}
\begin{proof}
  Left to the reader.
  We will prove a more general theorem in \autoref{sec:compute-sigma}.
\end{proof}

% \begin{proof}[First proof]
% Let $D:\prd{x,y:A}{p:x=y}\type$ be defined by
% \begin{equation*}
% D(x,y,p)\defeq (x,u)=(y,\trans{p}{u}).
% \end{equation*}
% Then $D(x,x,\refl{x})\defeq (x,u)=(x,\trans{\refl{x}}{u})$. By the conversion rule we have $\trans{\refl{x}}{u}\defeq u$, so we see that $D(x,x,\refl{x})\defeq (x,u)=(x,u)$. Therefore we find $d(x)\defeq\refl{(x,u)}:D(x,x,\refl{x})$. Now path induction gives a function of type $\prd{x,y:A}{p:x=y}(x,u)=(y,\trans{p}{u})$.
% \end{proof}
% \begin{proof}[Second proof] 
%   By induction, it suffices to find an element of $(x,u)=(x,\trans{\refl{x}}{u})$.
%   Note that $\trans{\refl{x}}{u}\jdeq u$, so we really need to find an element of $(x,u)=(x,u)$.
%   But here we can use reflexivity.
% \end{proof}

Now we can prove the dependent version of \autoref{lem:map}.
The topological intuition is that given $f:\prd{x:A} P(x)$ and a path $p:\id[A]xy$, we ought to be able to apply $f$ to $p$ and obtain a path in the total space of $P$ which ``lies over'' $p$, as shown below.

\begin{center}
  \begin{tikzpicture}[yscale=.5,xscale=2]
    \draw (0,0) arc (-90:170:1cm) node[anchor=south east] {$A$} arc (170:270:1cm);
    \draw (0,4) arc (-90:170:1cm) node[anchor=south east] {$\sm{x:A} P(x)$} arc (170:270:1cm);
    \draw[->] (0,3.8) -- node[auto] {$\proj1$} (0,2.2);
    \node[circle,fill,inner sep=1pt,label=left:{$x$}] (b1) at (-.5,1) {};
    \node[circle,fill,inner sep=1pt,label=right:{$y$}] (b2) at (.5,1) {};
    \draw[decorate,decoration={snake,amplitude=1}] (b1) -- node[auto,swap] {$p$} (b2);
    \node[circle,fill,inner sep=1pt,label=left:{$f(x)$}] (b1) at (-.5,5) {};
    \node[circle,fill,inner sep=1pt,label=right:{$f(y)$}] (b2) at (.5,5) {};
    \draw[decorate,decoration={snake,amplitude=1}] (b1) -- node[auto] {$f(p)$} (b2);
  \end{tikzpicture}
\end{center}

We \emph{can} obtain such a thing from \autoref{lem:map}.
Given $f:\prd{x:A} P(x)$, we can define a non-dependent function $f':A\to \sm{x:A} P(x)$ by setting $f'(x)\defeq (x,f(x))$, and then consider $\ap{f'}{p} : f'(x) = f'(y)$.
However, it is not obvious from the type of such a path that it lies over a specific path in $A$ (in this case, $p$), which is sometimes important.

The solution is to use the transport lemma.
Since there is a canonical path from $u:P(x)$ to $\trans p u :P(y)$ which (at least intuitively) lies over $p$, any path from $u$ to $v:P(y)$ lying over $p$ should factor through this path, essentially uniquely, by a path from $\trans p u$ to $v$ lying entirely in the fiber $P(y)$.
Thus, up to equivalence, it makes sense to define ``a path from $u$ to $v$ lying over $p:x=y$'' to mean a path $\trans p u = v$ in $P(y)$.
And, indeed, we can show that dependent functions produce such paths.

\begin{lem}[Dependent map]\label{lem:mapdep}
  Suppose $f:\prd{x: A} P(x)$; then we have a map
  \[\apdfunc f : \prd{p:x=y}\big(\id[P(y)]{\trans p{f(x)}}{f(y)}\big).\]
\end{lem}

\begin{proof}[First proof]
  Let $D:\prd{x,y:A}{p:\id{x}{y}} \type$ be the type family defined by
  \begin{equation*}
    D(x,y,p)\defeq \trans p {f(x)}= f(y).
  \end{equation*}
  Then $D(x,x,\refl{x})$ is $\trans{(\refl{x})}{f(x)}= f(x)$.
  But since $\trans{(\refl{x})}{f(x)}\jdeq f(x)$, we get that $D(x,x,\refl{x})\jdeq (f(x)= f(x))$.
  Thus, we find the function
  \begin{equation*}
    d\defeq\lam{x} \refl{f(x)}:\prd{x:A} D(x,x,\refl{x})
  \end{equation*}
  and now $J$ gives us $\apdfunc f(p):\trans p{f(x)}= f(y)$ for each $p:x= y$.
\end{proof}

\begin{proof}[Second proof]
  By induction, it suffices to assume $p$ is $\refl x$.
  But in this case, the desired equation is $\trans{(\refl{x})}{f(x)}\jdeq f(x)$, which holds judgmentally.
\end{proof}

We will refer generally to paths which ``lie over other paths'' in this sense as \emph{dependent paths}.
They will play an increasingly important role starting in \autoref{cha:hits}.
In \autoref{sec:computational} we will see that for a few particular kinds of type families, there are equivalent ways to represent the notion of dependent paths that are sometimes more convenient.

Now recall from section \ref{sec:pi-types} that a non-dependently typed function $f:A\to B$ is just the special case of a dependently typed function $f:\prd{x:A} P(x)$ when $P$ is a constant type family, $P(x) \defeq B$.
In this case, $\apdfunc{f}$ and $\apfunc{f}$ are closely related, because of the following lemma:

\begin{lem}\label{thm:trans-trivial}
  If $P:A\to\type$ is defined by $P(x) \defeq B$ for a fixed $B:\type$, then for any $x,y:A$ and $p:x=y$ and $b:B$ we have a path
  \[ \transconst Bpb : \transfib P p b = b. \]
\end{lem}
\begin{proof}[First proof]
  Fix a $b:B$, and let $D:\prd{x,y:A}{p:\id{x}{y}} \type$ be the type family defined by
  \[ D(x,y,p) \defeq (\transfib P p b = b). \]
  Then $D(x,x,\refl x)$ is $(\transfib P{\refl{x}}{b} = b)$, which is judgmentally equal to $(b=b)$ by the computation rule for transporting.
  Thus, we have the function
  \[ d \defeq \lam{x} \refl{b} : \prd{x:A} D(x,x,\refl x). \]
  Now path induction gives us an element of $\prd{x,y:A}{p:x=y}(\transfib P p b = b)$, as desired.
\end{proof}
\begin{proof}[Second proof]
  By induction, it suffices to assume $y$ is $x$ and $p$ is $\refl x$.
  But $\transfib P {\refl x} b \jdeq b$, so in this case what we have to prove is $b=b$, and we have $\refl{b}$ for this.
\end{proof}

Thus, by concatenating with $\transconst B p b$, for any $x,y:A$ and $p:x=y$ and $f:A\to B$ we obtain functions
\begin{align}
  \big(f(x) = f(y)\big) &\to \big(\trans{p}{f(x)} = f(y)\big)\label{eq:ap-to-apd}
  \qquad\text{and} \\
  \big(\trans{p}{f(x)} = f(y)\big) &\to \big(f(x) = f(y)\big).\label{eq:apd-to-ap}
\end{align}
In fact, these functions are inverse equivalences (in the sense to be introduced in \autoref{sec:basics-equivalences}), and they relate $\apfunc f (p)$  to $\apdfunc f (p)$.

\begin{lem}\label{thm:apd-const}
  For $f:A\to B$ and $p:\id[A]xy$, we have
  \[ \apdfunc f(p) = \transconst B p{f(x)} \ct \apfunc f (p) \]
\end{lem}
\begin{proof}[First proof]
  Let $D:\prd{x,y:A}{p:\id xy} \type$ be the type family defined by
  \[ D(x,y,p) \defeq \big(\apdfunc f (p) = \transconst Bp{f(x)} \ct \apfunc f (p)\big). \]
  Thus, we have
  \[D(x,x,\refl x) \jdeq \big(\apdfunc f (\refl x) = \transconst B{\refl x}{f(x)} \ct \apfunc f ({\refl x})\big).\]
  But by definition, all three paths appearing in this type are $\refl{f(x)}$, so we have
  \[ \refl{\refl{f(x)}} : D(x,x,\refl x). \]
  Thus, path induction gives us an element of $\prd{x,y:A}{p:x=y} D(x,y,p)$, which is what we wanted.
\end{proof}
\begin{proof}[Second proof]
  By induction, it suffices to assume $y$ is $x$ and $p$ is $\refl x$.
  In this case, what we have to prove is $\refl{f(x)} = \refl{f(x)} \ct \refl{f(x)}$, which is true judgmentally.
\end{proof}

Because the types of $\apdfunc{f}$ and $\apfunc{f}$ are different, it is often clearer to use different notations for them.
% We may sometimes use a notation $\apd f p$ for $\apdfunc{f}(p)$, which is similar to the notation $\ap f p$ for $\apfunc{f}(p)$.

At this point, we hope the reader is starting to get a feel for proofs by induction on identity types.
From now on we stop giving both styles of proofs, allowing ourselves to use whatever is most clear and convenient (and often the second, more concise one).
Here are a few other useful lemmas about transport; we leave it to the reader to give the proofs (in either style).

\begin{lem}\label{thm:transport-concat}
  Given $P:A\to\type$ with $p:\id[A]xy$ and $q:\id[A]yz$ while $u:P(x)$, we have
  \[ \trans{q}{\trans{p}{u}} = \trans{(p\ct q)}{u}. \]
\end{lem}

\begin{lem}\label{thm:transport-compose}
  For a function $f:A\to B$ and a type family $P:B\to\type$, and any $p:\id[A]xy$ and $u:P(f(x))$, we have
  \[ \transfib{P\circ f}{p}{u} = \transfib{P}{\apfunc f(p)}{u}. \]
\end{lem}

\begin{lem}\label{thm:ap-transport}
  For $P,Q:A\to \type$ and a family of functions $f:\prd{x:A} P(x)\to Q(x)$, and any $p:\id[A]xy$ and $u:P(x)$, we have
  \[ \transfib{Q}{p}{f_x(u)} = f_y(\transfib{P}{p}{u}). \]
\end{lem}


\section{Summary of the basic higher structure}
\label{sec:basics-summary}

Here we summarize the basic definitions made in the previous two sections.

\begin{itemize}
\item $\opp{p} : y=x$, for $p:x=y$, defined by
  \[\opp{\;\refl{x}}\jdeq \refl{x}.\]
\item $p\ct q :y=z$, for $p:x=y$ and $q:y=z$, defined by
  \[ \refl{x}\ct\refl{x}\jdeq\refl{x}.\]
\item If $P$ is a type family over $A$ then $\transf{p}:P(x)\to P(y)$, for $p:x=y$, defined by
  \[\transf{(\refl{x})}\jdeq \idfunc[P(x)].\]
\item If $f:A\to B$ then $\map{f}{p}:f(x)=f(y)$, for $p:x=y$, defined by
  \[\map{f}{\refl{x}}\jdeq \refl{f(x)}.\]
\item If $f:\prod_{x:A}P(x)$ then $\mapdep{f}{p}:\trans{p}{f(x)}=f(y)$, for $p:x=y$, defined by
  \[\mapdep{f}{\refl{x}}\jdeq \refl{f(x)}.\]
\end{itemize}


\section{Homotopies and equivalences}
\label{sec:basics-equivalences}

So far, we have seen how the identity type $\id[A]xy$ can be regarded as a type of \emph{identifications}, \emph{paths}, or \emph{equivalences} between two elements $x$ and $y$ of a type $A$.
Now we investigate the appropriate notions of ``identification'' or ``sameness'' between \emph{functions} and between \emph{types}.
In \autoref{sec:computational}, we will see that homotopy type theory allows us to identify these with instances of the identity type, but before we can do that we need to understand them in their own right.

Traditionally, we regard two functions as the same if they take equal values on all inputs.
Under the propositions-as-types interpretation, this suggests that two functions $f$ and $g$ (perhaps dependently typed) should be the same if the type $\prd{x:A} (f(x)=g(x))$ is inhabited.
Under the homotopical interpretation, this dependent function type consists of \emph{continuous} paths or \emph{functorial} equivalences, and thus may be regarded as the type of \emph{homotopies} or of \emph{natural isomorphisms}.
We will adopt the topological terminology for this.

\begin{defn}
  Let $f,g:\prd{x:A} P(x)$ be two sections of a type family $P:A\to\type$.
  A \textbf{homotopy} from $f$ to $g$ is a term of type
  \begin{equation*}
    (f\htpy g)\;\defeq\; \prd{x:A} (f(x)=g(x))
  \end{equation*}
\end{defn}

Note that a homotopy is not the same as an identification $(f=g)$.
In \autoref{sec:compute-pi} we will show that homotopies and identifications are nevertheless ``equivalent''.

The following proofs are left to the reader.

\begin{lem}\label{lem:homotopy-props}
  Homotopy is an equivalence relation on each function type $A\ra B$.
  That is, we have elements of the types
  \begin{gather*}
    \prd{f:A\to B} (f\htpy f)\\
    \prd{f,g:A\to B} (f\htpy g) \to (g\htpy f)\\
    \prd{f,g,h:A\to B} (f\htpy g) \to (g\htpy h) \to (f\htpy h).
  \end{gather*}
\end{lem}

\begin{lem}
  Composition is associative and unital up to homotopy.
  That is:
  \begin{enumerate}
  \item If $f:A\ra B$ then $f\circ \idfunc[A]\htpy f\htpy \idfunc[B]\circ f$.
  \item If $f:A\ra B, g:B\ra C$ and $h:C\ra D$ then $h\circ (g\circ f)\;\htpy\; (h\circ g)\circ f$.
  \end{enumerate}
\end{lem}

The first level of the continuity/naturality of homotopies can be expressed as follows:

\begin{lem}\label{lem:htpy-natural}
  Suppose $H:f\htpy g$ is a homotopy between functions $f,g:A\to B$ and let $p:\id[A]xy$.  Then there is a term of type
  \begin{equation*}
    H(x)\ct\ap{g}{p}=\ap{f}{p}\ct H(y).
  \end{equation*}
  We may also draw this as a commutative diagram:
  \begin{align*}
    \xymatrix{
      f(x) \ar[r]^{\ap fp} \ar[d]_{H(x)} & f(y) \ar[d]^{H(y)} \\
      g(x) \ar[r]_{\ap gp} & g(y)
    }
  \end{align*}
\end{lem}
\begin{proof}
  By induction, we may assume $p$ is $\refl x$.
  Since $\apfunc{f}$ and $\apfunc g$ compute on reflexivity, in this case what we must show is
  \[ H(x) \ct \refl{g(x)} = \refl{f(x)} \ct H(x). \]
  But this follows since both sides are equal to $H(x)$.
\end{proof}

\begin{cor}\label{cor:hom-fg}
Let $H : f \htpy \idfunc[A]$ be a homotopy, with $f : A \to A$. Then for any $x : A$ we have \[ H(f(x)) = \ap f{H(x)}. \] The above path will be denoted by $\com{H}{f}{x}$.
\end{cor}
\begin{proof}
By naturality of $H$, the following diagram commutes:
\begin{align*}
\xymatrix{
ffx \ar[r]^{\ap f{Hx}} \ar[d]_{H(fx)} & fx \ar[d]^{Hx} \\
fx \ar[r]_{Hx} & x
}
\end{align*}
Canceling $H(x)$, we see that $H(f(x)) = f(H(x))$ as desired.
\end{proof}

Moving on to types, from a traditional perspective one may say that a function $f:A\to B$ is an \emph{isomorphism} if there is a function $g:B\to A$ such that both composites $f\circ g$ and $g\circ f$ are pointwise equal to the identity, i.e.\ such that $f \circ g \htpy \idfunc[B]$ and $g\circ f \htpy \idfunc[A]$.
A homotopical perspective suggests that this should be called a \emph{homotopy equivalence}, and from a categorical one, it should be called an \emph{equivalence of (higher) groupoids}.
However, when doing proof-relevant mathematics, the corresponding type
\begin{equation}
  \sm{g:B\to A} \big((f \circ g \htpy \idfunc[B]) \times (g\circ f \htpy \idfunc[A])\big)\label{eq:qinvtype}
\end{equation}
is poorly behaved.
For instance, for a single function $f:A\to B$ there may be multiple unequal inhabitants of~\eqref{eq:qinvtype}.
(This is closely related to the observation in higher category theory that often one needs to consider \emph{adjoint} equivalences rather than plain equivalences.)
For this reason, we give~\eqref{eq:qinvtype} the following historically accurate, but slightly derogatory-sounding name instead.

\begin{defn}
  For a function $f:A\to B$, a \textbf{quasi-inverse} of $f$ is a triple $(g,\alpha,\beta)$ consisting of a function $g:B\to A$ and homotopies
$\alpha:f\circ g\htpy \idfunc[B]$ and $\beta:g\circ f\htpy \idfunc[A]$.
\end{defn}

Thus,~\eqref{eq:qinvtype} is \emph{the type of quasi-inverses of $f$}; we may denote it by $\qinv(f)$.

\begin{eg}\label{eg:idequiv}
  The identity function $\idfunc[A]:A\to A$ has a quasi-inverse given by $\idfunc[A]$ itself, together with homotopies defined by $\alpha(y) \defeq \refl{y}$ and $\beta(x) \defeq \refl{x}$.
\end{eg}

\begin{eg}\label{eg:concatequiv}
  For any $p:\id[A]xy$ and $z:A$, the functions
  \begin{align*}
    (p\ct -)&:(\id[A]yz) \to (\id[A]xz) \qquad\text{and}\\
    (-\ct p)&:(\id[A]zx) \to (\id[A]zy)
  \end{align*}
  have quasi-inverses given by $(\opp p \ct -)$ and $(-\ct \opp p)$, respectively.
\end{eg}

\begin{eg}\label{thm:transportequiv}
  For any $p:\id[A]xy$ and $P:A\to\type$, the function
  \[\transfib{P}{p}{-}:P(x) \to P(y)\]
  has a quasi-inverse given by $\transfib{P}{\opp p}{-}$.
\end{eg}

In general, we will only use the word \emph{isomorphism} (and similar words such as \emph{bijection}) in the special case when the types $A$ and $B$ ``behave like sets'' (see \autoref{sec:basics-sets}).
In this case, the type~\eqref{eq:qinvtype} is unproblematic.
We will reserve the word \emph{equivalence} for an improved notion with the following properties:
\begin{enumerate}
\item For each $f:A\to B$ there is a function $\qinv(f) \to \isequiv (f)$.\label{item:be1}
\item Similarly, for each $f$ we have $\isequiv (f) \to \qinv(f)$; thus the two are ``logically equivalent''.\label{item:be2}
\item For any two inhabitants $e_1,e_2:\isequiv(f)$ we have $e_1=e_2$.\label{item:be3}
\end{enumerate}
In \autoref{cha:equivalences} we will see that there are many different definitions of $\isequiv(f)$ which satisfy these three properties, but that all of them are equivalent.
For now, to convince the reader that such things exist, we mention only the easiest such definition (though it is not the one we will eventually settle on in \autoref{cha:equivalences}):
\begin{equation}
  \isequiv(f) \;\defeq\;
  \big(\sm{g:B\to A} (f\circ g \htpy \idfunc[B])\big)
  \times
  \big(\sm{h:B\to A} (h\circ f \htpy \idfunc[A])\big)\label{eq:isequiv-invertible}
\end{equation}
We can show~\ref{item:be1} and~\ref{item:be2} for this definition now.
A function $\qinv(f) \to \isequiv (f)$ is easy to define by taking $(g,\alpha,\beta)$ to $(g,\alpha,g,\beta)$.
In the other direction, given $(g,\alpha,h,\beta)$, let $\gamma$ be the composite homotopy
\[ g \overset{\beta}{\htpy} h\circ f\circ g \overset{\alpha}{\htpy} h \]
and let $\beta':g\circ f\htpy \idfunc[A]$ be obtained from $\gamma$ and $\beta$.
Then $(g,\alpha,\beta'):\qinv(f)$.

Property~\ref{item:be3} for this definition is not too hard to prove either, but it requires identifying the identity types of cartesian products and dependent pair types, which we will discuss in \autoref{sec:computational}.
Thus, we postpone it as well.
At this point, the main thing to take away is that there is a well-behaved type which we can pronounce as ``$f$ is an equivalence'', and that we can prove $f$ to be an equivalence by exhibiting a quasi-inverse to it.
In practice, this is the most common approach.

In accord with the proof-relevant philosophy, \emph{an equivalence} from $A$ to $B$ is defined to be a function $f:A\to B$ together with an inhabitant of $\isequiv (f)$, i.e.\ a proof that it is an equivalence.
We write $(\eqv A B)$ for the type of equivalences from $A$ to $B$, i.e.\ the type
\[ (\eqv A B) \;\defeq \; \sm{f:A\to B} \isequiv(f). \]
Property~\ref{item:be3} above will ensure that if two equivalences are equal as functions (that is, the underlying elements of $A\to B$ are equal), then they are also equal as equivalences (see \autoref{sec:compute-sigma}).

We conclude by observing:

\begin{lem}\label{thm:equiv-eqrel}
  Type equivalence is an equivalence relation on \type.
  More specifically:
  \begin{enumerate}
  \item For any $A$, the identity function $\idfunc[A]$ is an equivalence; hence $\eqv A A$.
  \item For any $f:\eqv A B$, we have an equivalence $f^{-1} : \eqv B A$.
  \item For any $f:\eqv A B$ and $g:\eqv B C$, we have $g\circ f : \eqv A C$.
  \end{enumerate}
\end{lem}
\begin{proof}
  The identity function is clearly its own quasi-inverse; hence it is an equivalence.

  If $f:A\to B$ is an equivalence, then it has a quasi-inverse, say $f^{-1}:B\to A$.
  Then $f$ is also a quasi-inverse of $f^{-1}$, so $f^{-1}$ is an equivalence $B\to A$.

  Finally, given $f:\eqv A B$ and $g:\eqv B C$ with quasi-inverses $f^{-1}$ and $g^{-1}$, say, then for any $a:A$ we have $f^{-1} g^{-1} g f a = f^{-1} f a = a$, and for any $c:C$ we have $g f f^{-1} g^{-1} c = g g^{-1} c = c$.
  Thus $f^{-1} \circ g^{-1}$ is a quasi-inverse to $g\circ f$, hence the latter is an equivalence.
\end{proof}

% Local Variables:
% TeX-master: "main"
% End:


\newcommand\ua[1]{\ensuremath{\mathsf{ua}} \: #1}

\section{The identity structure of specific types}
\label{sec:computational}

In Chapter~\ref{cha:introduction} we introduced many ways to form new types: cartesian products, disjoint unions, dependent products, dependent sums, etc.
In the previous sections of this chapter, we have seen that \emph{all} types in homotopy type theory behave like spaces or higher groupoids.
Our goal in this section is to make explicit how this higher structure behaves, for particular types defined as in Chapter~\ref{cha:typetheory}.

It turns out that for many types $A$, the equality types $\id[A]xy$ can be characterized, up to equivalence, in terms of whatever data was used to construct $A$.
For instance, if $A$ is a cartesian product $B\times C$, and $x\jdeq (b,c)$ and $y\jdeq(b',c')$, then we have an equivalence
\begin{equation}\label{eq:prodeqv}
  \eqv{\Big((b,c)=(b',c')\Big)}{\Big((b=b')\times (c=c')\Big)}.
\end{equation}
In more traditional language, two ordered pairs are equal just when their components are equal (but the equivalence~\eqref{eq:prodeqv} says rather more than this).
The higher structure of the identity types can also be expressed in terms of these equivalences; for instance, concatenating two equalities between pairs corresponds to pairwise concatenation.

Similarly, when a dependent type $P:A\to\type$ is built up fiberwise using the type forming rules from Chapter~\ref{cha:typetheory}, the operation $\transfib{P}{p}{-}$ can be characterized, up to homotopy, in terms of the corresponding operations on the data that went into $P$.
For instance, if $P(x) \jdeq B(x)\times C(x)$, then we have
\[\transfib{P}{p}{(b,c)} = \left(\transfib{B}{p}{b},\transfib{C}{p}{c}\right).\]

Finally, the type forming rules are also functorial, and if a function $f$ is built from this functoriality, then the operations $\apfunc f$ and $\apdfunc f$ can be computed based on the corresponding ones on the data going into $f$.
For instance, if $g:B\to B'$ and $h:C\to C'$ and we define $f:B\times C \to B'\times C'$ by $f(b,c)\defeq (g(b),h(c))$, then modulo the equivalence~\eqref{eq:prodeqv}, we can identify $\apfunc f$ with ``$(\apfunc g,\apfunc h)$''.

In this section, we will state and prove theorems of this sort for all the basic type forming rules.

\begin{rmk}
  In the type theory we are working with, identity types are defined simultaneously for all types by the inductive $J$-rule.
  The characterizations for particular types to be discussed in this chapter are then theorems which we have to discover and prove.
  An alternative presentation of type theory might take these characterizations as \emph{definitions} of the identity types (by induction over the construction of types), with the inductive $J$-rule then being provable.
  While such a type theory has not yet been made precise except in very simple cases, it is still helpful to think of the rules to be presented in this section as ``computation'' rules for ``evaluating'' identity types, transport, and function application.
\end{rmk}

\subsection{Cartesian product types}
\label{sec:compute-cartprod}

Given types $A$ and $B$, consider the cartesian product type $A \times B$.  
For any elements $x,y:A\times B$ and a path $p:\id[A\times B]{x}{y}$, by functoriality we can extract paths $\ap{\proj1}p:\id[A]{\proj1(x)}{\proj1(y)}$ and $\ap{\proj2}p:\id[B]{\proj2(x)}{\proj2(y)}$.
Thus, we have a function
\begin{equation}
  (\id[A\times B]{x}{y}) \;\to\; (\id[A]{\proj1(x)}{\proj1(y)}) \times (\id[B]{\proj2(x)}{\proj2(y)}).\label{eq:path-prod}
\end{equation}

\begin{thm}\label{thm:path-prod}
  For any $x$ and $y$, the function~\eqref{eq:path-prod} is an equivalence.
\end{thm}

Read logically, this says that two pairs are equal if they are equal
componentwise.  Read category-theoretically, this says that the
morphisms in a product groupoid are pairs of morphisms.  Read
homotopy-theoretically, this says that the paths in a product
space are pairs of paths.

\begin{proof}
  We need a function in the other direction:
  \begin{equation}
    (\id[A]{\proj1(x)}{\proj1(y)}) \times (\id[B]{\proj2(x)}{\proj2(y)}) \;\to\; (\id[A\times B]{x}{y}) .\label{eq:path-prod-inverse}
  \end{equation}
  By the induction rule for cartesian products, we may assume that $x$ and $y$ are both pairs, i.e.\ $x\jdeq (a,b)$ and $y\jdeq (a',b')$ for some $a,a':A$ and $b,b':B$.
  In this case, what we want is a function
  \begin{equation*}
    (\id[A]{a}{a'}) \times (\id[B]{b}{b'}) \;\to\; \big(\id[A\times B]{(a,b)}{(a',b')}\big).
  \end{equation*}
  Now by induction for the cartesian product in its domain, we may assume given $p:a=a'$ and $q:b=b'$.
  And by two path inductions, we may assume that $a\jdeq a'$ and $b\jdeq b'$ and both $p$ and $q$ are reflexivity.
  But in this case, we have $(a,b)\jdeq(a',b')$ and so we can take the output to also be reflexivity.

  It remains to prove that~\eqref{eq:path-prod-inverse} is quasi-inverse to~\eqref{eq:path-prod}.
  This is a simple sequence of inductions, but they have to be done in the right order.

  If we start with $r:\id[A\times B]{x}{y}$, then we first do a path induction on $r$ in order to assume that $x\jdeq y$ and $r$ is reflexivity.
  In this case, since $\apfunc{\proj1}$ and $\apfunc{\proj2}$ are defined by path induction,~\eqref{eq:path-prod} takes $r\jdeq \refl{x}$ to the pair $(\refl{\proj1x},\refl{\proj2x})$.
  Now by induction on $x$, we may assume $x\jdeq (a,b)$, so that this is $(\refl a, \refl b)$.
  Thus,~\eqref{eq:path-prod-inverse} takes it by definition to $\refl{(a,b)}$, which (under our current assumptions) is $r$.
  
  In the other direction, if we start with $s:(\id[A]{\proj1(x)}{\proj1(y)}) \times (\id[B]{\proj2(x)}{\proj2(y)})$, then we first do induction on $x$ and $y$ to assume that they are pairs $(a,b)$ and $(a',b')$, and then induction on $s:(\id[A]{a}{a'}) \times (\id[B]{b}{b'})$ to reduce it to a pair $(p,q)$ where $p:a=a'$ and $q:b=b'$.
  Now by induction on $p$ and $q$, we may assume they are reflexivities $\refl a$ and $\refl b$, in which case~\eqref{eq:path-prod-inverse} yields $\refl{(a,b)}$ and then~\eqref{eq:path-prod} returns us to $(\refl a,\refl b)\jdeq (p,q)\jdeq s$.
\end{proof}

From a programming perspective, it's useful to unpack this equivalence into the following data:

\newcommand{\pairpath}{\mathsf{pair}^{\mathord{=}}}
\newcommand{\projpath}[1]{\proj{#1}^{\mathord{=}}}

\begin{itemize}
\item An introduction rule for $(\id[A \times B]{x}{y})$ (this is~\eqref{eq:path-prod-inverse})
  \[
  \pairpath : (\id{\proj{1} x}{\proj{1} y}) \times (\id{\proj{1} x}{\proj{1} y}) \to {(\id x y)}
  \]
\item Elimination rules (these are the two components of~\eqref{eq:path-prod}):
  \begin{align*}
    \projpath{1} &: (\id{x}{y}) \to (\id{\proj{1} x}{\proj{1} y})\\
    \projpath{2} &: (\id{x}{y}) \to (\id{\proj{2} x}{\proj{2} y})
  \end{align*}
\item $\beta$-reduction:
  \begin{align*}
    {\projpath{1}{(\pairpath(p, q)})}
    &=_{(\id{\proj{1} x}{\proj{1} y})}
    {p} \\
    {\projpath{2}{(\pairpath(p,q)})}
    &=_{(\id{\proj{2} x}{\proj{2} y})}
    {q}
  \end{align*}
\item $\eta$-equivalence: For any $r : \id[A \times B] x y$
  \[
  \id{r}{\pairpath(\projpath{1} (r), \projpath{2} (r)) }
  \]
\end{itemize}
Moreover, reflexivity, inverses, and composition are defined componentwise:
\begin{align*}
  {\refl{(z : A \times B)}}
  &= {\pairpath (\refl{\proj{1} z},\refl{\proj{2} z})} \\
  {\opp{p}}
  &= {\pairpath \big(\opp{(\projpath{1} p)},\, \opp{(\projpath{2} p)}\big)} \\
  {{p \ct q}}
  &= {\pairpath \big({\projpath{1} p} \ct {\projpath{1} q},\,{\projpath{2} p} \ct {\projpath{2} q}\big)}
\end{align*}
The same is true for all the higher groupoid structure considered in \S\ref{sec:equality}.
All of these equations can be derived by using path induction on the given paths and then returning reflexivity.  

We now consider transport in a product of dependent types.
Given dependent types $ A, B : Z \to \type$, we abusively write $A\times B:Z\to \type$ for the dependent type defined by $(A\times B)(z) \defeq A(z) \times B(z)$.
Now given $p : \id[Z]{z}{w}$ and $x : A(z) \times B(z)$, we can transport $x$ along $p$ to obtain an element of $A(w)\times B(w)$.

\begin{thm}\label{thm:trans-prod}
  In the above situation, we have
  \[
  \id[A(y) \times B(y)]
  {\transfib{A\times B}px}
  {(\transfib{A}{p}{\proj{1}x}, \transfib{B}{p}{\proj{2}x})}
  \]
\end{thm}
\begin{proof}
  By path induction, we may assume $p$ is reflexivity, in which case we have
  \begin{align*}
    \transfib{A\times B}px&\jdeq x\\
    \transfib{A}{p}{\proj{1}x}&\jdeq \proj1x\\
    \transfib{A}{p}{\proj{2}x}&\jdeq \proj2x.
  \end{align*}
  Thus, it remains to show $x = (\proj1 x, \proj2x)$, which follows by induction on $x$.
\end{proof}

Finally, we consider the functoriality of $\apfunc{}$ under cartesian products.
Suppose given types $A,B,A',B'$ and functions $g:A\to A'$ and $h:B\to B'$; then we can define a function $f:A\times B\to A'\times B'$ by $f(x) \defeq (g(\proj1x),h(\proj2x))$.

\begin{thm}\label{thm:ap-prod}
  In the above situation, given $x,y:A\times B$ and $p:\proj1x=\proj1y$ and $q:\proj2x=\proj2y$, we have
  \[ \id[(f(x)=f(y))]{\ap{f}{\pairpath(p,q)}} {\pairpath(\ap{g}{p},\ap{h}{q})}. \]
\end{thm}
\begin{proof}
  Note first that the above equation is well-typed.
  On the one hand, since $\pairpath(p,q):x=y$ we have $\ap{f}{\pairpath(p,q)}:f(x)=f(y)$.
  On the other hand, since $\proj1(f(x))\jdeq g(\proj1x)$ and $\proj2(f(x))\jdeq h(\proj2x)$, we also have $\pairpath(\ap{g}{p},\ap{h}{q}):f(x)=f(y)$.

  Now, by induction, we may assume $x\jdeq(a,b)$ and $y\jdeq(a',b')$, in which case we have $p:a=a'$ and $q:b=b'$.
  Thus, by path induction, we may assume $p$ and $q$ are reflexivity, in which case the desired equation holds judgmentally.
\end{proof}


\subsection{$\Sigma$-types}
\label{sec:compute-sigma}

To find out what it means to be a path in a $\Sigma$-type suppose that we have a path $p:w=w'$ in $\sm{x:A}P(x)$. Then we get $\ap{\proj{1}}{p}:\proj{1}(w)=\proj{1}(w')$. We cannot directly ask whether $\proj{2}(w)$ is identical to $\proj{2}(w')$ since they don't have to be in the same type. However, we can transport $\proj{2}(w)$ along the path $\ap{\proj{1}}(p)$, which does give us a term of the same type as $\proj{2}(w')$. By path induction, we see that we also obtain a path $\trans{\ap{\proj{1}}{p}}{\proj{2}(w)}=\proj{2}(w')$. The next theorem states that we can reverse this process.

\begin{thm}
Suppose that $P:A\to\type$ is a dependent type over a type $A$ and let $w,w^\prime:\sm{x:A}P(x)$. Then there is an equivalence
\begin{equation*}
\eqv{w=w'}{\sm{p:\proj{1}(w)=\proj{1}(w')}\trans{p}{\proj{2}(w)}=\proj{2}(w^\prime)}.
\end{equation*}
\end{thm}

\begin{proof}
We define for any $w,w':\sm{x:A}P(x)$ a function $f(w,w')$ of type
\begin{equation*}
(w=w')\to\sm{p:\proj{1}(w)=\proj{1}(w')}\trans{p}{\proj{2}(w)}=\proj{2}(w^\prime)
\end{equation*}
by path induction, with 
\begin{equation*}
f(w,w,\refl{w})\defeq(\refl{\proj{1}(w)},\refl{\proj{2}(w)}).
\end{equation*}
We want to show that $f$ is an isomorphism, so we can find a section $g(w,w')$ of $f(w,w')$ by finding a function $G(w,w')$ of type
\begin{equation*}
\prd{p:\proj 1(w)=\proj 1(w')}{q:\trans p{\proj 2(w)}=\proj 2(w')}\mathsf{hFiber}(f(w,w'),(p,q))
\end{equation*}
for each $w,w':\sm{x:A}P(x)$. To do this we will first use the induction principle of dependent sums, i.e.\ it is enough to find a function $G((x,u),(y,v))$ of type
\begin{equation*}
\prd{p:x=y}{q:\trans{p}{u}=v}\hfiber{f((x,u),(y,v))}{(p,q)}
\end{equation*}
This is obvious by induction on $p$ followed by induction on $q$ and we get
\begin{equation*}
G((x,u),(x,u),(\refl{x},\refl{u}))\defeq(\refl{(x,u)},\refl{(\refl{x},\refl{u})}).
\end{equation*}
This gives the function $g(w,w')\defeq\proj{1}\circ G(w,w')$ which is a section of $f(w,w')$ by construction. To show that there is a path $g(w,w',f(w,w',\alpha))=\alpha$ for any $\alpha:w=w'$ we use path induction on $\alpha$ and destruction on $w$. The result follows immediately.
\end{proof}

\begin{thm}
Suppose we have a dependent type $P:A\to\type$ and a dependent type $Q:(\sum(x:A),\ P(x))\to\type$. Then we get the dependent type
\begin{equation*}
x:A\vdash \sum(u:P(x)),\ Q(x,u):\type
\end{equation*}
For any path $p:x=y$ and any $(u,z):\sum(u:P(x)),\ Q(x,u)$ we have
\begin{equation*}
p\cdot(u,z)=(\trans{p}{u},\trans{\mathsf{lift}(u,p)}{z}).
\end{equation*}
The path $\mathsf{lift}(u,p)$ is defined in theorem \ref{thm:path_lifting}.
\end{thm}

\begin{proof}
Immediate by path induction.
\end{proof}

\subsection{$\Pi$-types}

Given $A$ and $B : A \to \type$, consider the dependent function type $\prd{x:A}B(x)$.
A path in $\prd{x:A} B(x)$ is given by a homotopy:  

\[
\eqv{(\id[\prd{x:A} B(x)]{f}{g})}{\prd{x:A} \id[B(x)]{(f x)}{(g x)}}
\]

Again it is useful to break this into 

\begin{itemize}
\item An introduction rule for {(\id[\prd{x:A} B(x)]{f}{g})}, function extensionality
  \[
  \funext : (\prd{x:A} \id[B(x)]{(f x)}{(g x)}) \to {(\id[\prd{x:A} B(x)]{f}{g})}
  \]
\item An elimination rule: for all $f,g : \prd{x:A} B(x)$
  \[
  \happly{-}{-} : \id{f}{g} \to \prd{\alpha : \id{x}{y}} \id{\transport{B}{\alpha} \: {(f x)}}{(g y)}
  \]
\item $\beta$-reduction: 
  \[
  \begin{array}{l}
  \id{\happly \: {\funext{(x \mapsto \alpha(x))}} \: {\refl{a}}}{\alpha(a)}
  \end{array}
  \]
\item $\eta$-equivalence: For any $\alpha : \id[\prd{x:A} B(x)] f g$
  \[
  \id{\alpha}{\funext (x \mapsto \happly \: \alpha \: x)}
  \]
\end{itemize}

%% FIXME: where do the rules for \alpha[\delta] go in this style?

Identity, inverses, and composition:
\[
\begin{array}{l}
\refl{(f : \prd{x:A} B)} = \funext(x \mapsto \refl{f x}) \\
\opp{\alpha} = \funext (x \mapsto \opp{(\happly \: {\alpha} \: {\refl x})})  \\
{\alpha} \ct \beta = \funext (x \mapsto {(\happly \: {\alpha} \: {\refl x}) \ct (\happly \: {\beta} \: {\refl x})})  \\
\end{array}
\]

\newcommand{\fcomp}{\circ}

Transport, first for non-dependent functions: given $\alpha : \id {a_0} {a_0'}$
and $f : A(a_0) \to B(a_0)$, 
\[
\begin{array}{l}
\transport{x_0:A_0 \mapsto A(x_0) \to B(x_0)}{\alpha} \: f = 
   (\transport{x_0:A_0 \mapsto B(x_0)}{\alpha}) \fcomp f \fcomp (\transport{x:A_0 \mapsto A(x_0)}{\opp \alpha})
\end{array}
\]

Transport for $\Pi$:  
\[
\begin{array}{l}
\transport{x_0:A_0 \mapsto \prd{x : A(x_0)} B(x_0)}{\alpha} \: f =  \\
   x \mapsto 
   \transport{(p : \sm {x_0:A_0} A) \mapsto B(\proj{1} \: p, \proj{2} \: p)}{\alpha^{\mathord{-}}(x) } 
      \: (f (\transport{x:A_0 \mapsto A(x)}{\opp \alpha}\:  x))
\end{array}
\]
where 
\[
\alpha^-(x) : \id[\sm {x_0:A_0} A(x_0)] {(a_0 , \transport{x_0 \mapsto A(x_0)} {\opp \alpha} \: x)} {(a_0' , x)}
\]
can be defined by path induction on $\alpha$.  
%%FIXME: should discuss this with sigma types

\subsection{The Universe}

A path in $\type$ is given by univalence

\[
\eqv{(\id[\type]{A}{B})}{\eqv A B}
\]

Again it is useful to break this into 

\newcommand\isequiv{\mathsf{isEquiv}}

\begin{itemize}
\item An introduction rule for {(\id[\type]{A}{B})}:
  \[
  \ua{} : {\eqv A B} \to (\id[\type]{A}{B})
  \]
\item The elimination rule is transport at $X:\type \mapsto X$:
  \[
  \transport{X \mapsto X}{} : \id{A}{B} \to (A \to B)
  \]
\item $\beta$-reduction: 
  \[
  \begin{array}{l}
  \id{\transport{X \mapsto X} \: (\ua {(f, fIsEquiv)})}{f}
  \end{array}
  \]
\item $\eta$-equivalence: For any $\alpha : \id A B$
  \[
  \id{\alpha}{\ua {(\transport{X \mapsto X}(\alpha) , \beta)}}
  \]
  where $\mathsf{transportIsEquiv} : \isequiv{\transport{X \mapsto X}(\alpha)}$ can be
  defined by doing path induction on $\alpha$, at which point it
  suffices to show that the identity function is an equivalence.  
\end{itemize}

Identity, inverses, and composition: (FIXME: pick a specific definition
of equivalence?)
\[
\begin{array}{l}
\refl{A} = \ua{\idfunc} \\
\opp{\alpha} = \ua {(\transport {X \mapsto X} {\opp \alpha}, \mathsf{transportIsEquiv} \: (\opp \alpha))} \\ 
{\alpha} \ct \beta = ? \\
\end{array}
\]

\subsection{Identity Type}

When we know what \id[A]{}{} is, \id[ {\id[A]{}{}} ]{}{} follows:

Transport, when $A$ is non-dependent:
\[
\begin{array}{l}
\transport{x_0:A_0 \mapsto \id[A] {a_1(x_0)}{a_2(x_0)}} {\alpha_0}{\alpha} = 
\opp{(\map{a_1}{\alpha_0})} \ct \alpha \ct \map{a_2}{\alpha_0}
\end{array}
\]

Useful special cases:
\[
\begin{array}{l}
\transport{x:A \mapsto \id[A] {a}{x}} {\alpha_0} \: {\alpha} = \alpha \ct \alpha_0 \\
\transport{x:A \mapsto \id[A] {x_0}{a}} {\alpha_0} \: {\alpha} = \opp {\alpha_0} \ct \alpha \\
\end{array}
\]

Transport, when $A$ is dependent:
\[
\begin{array}{l}
\transport{x_0:A_0 \mapsto \id[A(x_0)] {a_1(x_0)}{a_2(x_0)}} {\alpha_0} \: {\alpha} = \\
\opp{(\map{a_1}{\alpha_0})} \ct \mapdep{(\transport{A}{\alpha_0})}{\alpha} \ct \map{a_2}{\alpha_0}
\end{array}
\]

\subsection{Higher Inductives}

\newcommand{\sone}{\mathsf{S^1}}

Consider a higher inductive type such as $\sone$.  The definition of the
higher inductive type does not immediately characterize
\id[\sone]{x}{y}---which is good, because the calculation of higher
homotopy groups can be a significant theorem, so we don't want it to be
baked into the definitions.  However, we will often be able to prove a
theorem characterizing the loop space, which follows the above form.
For example, the proof in Chapter~\ref{cha:homotopy} that the fundamental
group of the circle is the integers plays this role:

\begin{itemize}
\item An introduction rule for \id[\sone]{\mathsf{base}}{\mathsf{base}}:
  \[
  \mathsf{loopToThe} : \mathbb{Z} \to \id{\mathsf{base}}{\mathsf{base}}
  \]
\item An elimination rule:
  \[
  \mathsf{encode} : \id{\mathsf{base}}{\mathsf{base}} \to \mathbb{Z}
  \]
\item With $\beta$ and $\eta$ rules stating that these are mutually inverse.
\end{itemize}

It's less clear that you want to think about identity, inverses, and
composition as being defined through this encoding (rather than thinking
of them as constructors), but you can:

\[
\begin{array}{l}
\refl{\mathsf{base}} = \mathsf{loopToThe} \: 0 \\
\opp{\alpha} = \mathsf{loopToThe} \: (- (\mathsf{encode} \: \alpha)) \\
\alpha \ct \beta = \mathsf{loopToThe} \: ((\mathsf{encode} \: \alpha) + (\mathsf{encode} \: \beta)) \\
\end{array}
\]

This changes the representation of the group structure
from the identity type to an explicit representation, as the free group
on one generator (the additive group on the integers).  

FIXME: say something about map

\section{Examples}

\subsection{Monoid}


% Local Variables:
% TeX-master: "main"
% End:



\section{Universal properties}
\label{sec:universal-properties}

By combining the path computation rules described in \autoref{sec:computational}, we can show that various type forming operations satisfy the expected universal properties.
For instance, given types $X,A,B$, we have a function
\begin{equation}
  (X\to A\times B) \;\to \; (X\to A)\times (X\to B)\label{eq:prod-ump-map}
\end{equation}
defined by $f \mapsto (\proj1 \circ f, \proj2\circ f)$.

\begin{thm}\label{thm:prod-ump}
  \eqref{eq:prod-ump-map} is an equivalence.
\end{thm}
\begin{proof}
  We define a quasi-inverse to send $(g,h)$ to the function $x\mapsto (g(x),h(x))$.
  (Technically, we have used the induction principle for the cartesian product $(X\to A)\times (X\to B)$, to reduce to the case of a pair.)

  Now given $f:X\to A\times B$, the round-trip composite yields the function
  \begin{equation}
    x\mapsto (\proj1(f(x)),\proj2(f(x))).\label{eq:prod-ump-rt1}
  \end{equation}
  By \autoref{thm:path-prod}, for any $x:X$ we have $(\proj1(f(x)),\proj2(f(x))) = f(x)$.
  Thus, by function extensionality, the function~\eqref{eq:prod-ump-rt1} is equal to $f$.

  On the other hand, given $(g,h)$, the round-trip composite yields the pair $(x\mapsto g(x),x\mapsto h(x))$.
  By function extensionaility, the two components of this are equal to $g$ and $h$ respectively, so by \autoref{thm:path-prod}, the pair is equal to $(g,h)$.
\end{proof}

In fact, we also have a dependently typed version of this universal property.
Suppose given a type $X$ and type families $A,B:X\to \type$.
Then we have a function
\begin{equation}\label{eq:prod-umpd-map}
  \Big(\prd{x:X} (A(x)\times B(x))\Big) \;\to\; \Big(\prd{x:X} A(x)\Big) \times \Big(\prd{x:X} B(x)\Big)
\end{equation}
defined as before by $f \mapsto (\proj1 \circ f, \proj2\circ f)$.

\begin{thm}\label{thm:prod-umpd}
  \eqref{eq:prod-umpd-map} is an equivalence.
\end{thm}
\begin{proof}
  Left to the reader.
\end{proof}

Just as $\Sigma$-types are a generalization of cartesian products, they satisfy a generalized version of this universal property.
Jumping right to the dependently typed version, suppose we have a type $X$ and type families $A:X\to \type$ and $P:\prd{x:X} A(x)\to\type$.
Then we have a function
\begin{equation}
  \label{eq:sigma-ump-map}
  \Big(\dprd{x:X}\dsm{a:A(x)} P(x,a)\Big)  \;\to\;
  \Big(\dsm{g:\prd{x:X} A(x)} \dprd{x:X} P(x,g(x))\Big)
\end{equation}
Note that if we have $P(x,a) \defeq B(x)$ for some $B:X\to\type$, then~\eqref{eq:sigma-ump-map} reduces to~\eqref{eq:prod-umpd-map}.

\begin{thm}\label{thm:ttac}
  \eqref{eq:sigma-ump-map} is an equivalence.
\end{thm}
\begin{proof}
  As before, we define a quasi-inverse to send $(g,h)$ to the function $x\mapsto (g(x),h(x))$.
  Now given $f:\prd{x:X} \sm{a:A(x)} P(x,a)$, the round-trip composite yields the function
  \begin{equation}
    x\mapsto (\proj1(f(x)),\proj2(f(x))).\label{eq:prod-ump-rt2}
  \end{equation}
  Now for any $x:X$, by \autoref{thm:eta-sigma} ($\eta$-equivalence for $\Sigma$-types) we have $$(\proj1(f(x)),\proj2(f(x))) = f(x).$$
  Thus, by function extensionality,~\eqref{eq:prod-ump-rt2} is equal to $f$.

  On the other hand, given $(g,h)$, the round-trip composite yields the pair $(x\mapsto g(x),x\mapsto h(x))$.
  But $x\mapsto g(x)$ and $x\mapsto h(x)$ are judgmentally equal to $g$ and $h$, respectively, and hence this pair of functions is also equal to $(g,h)$.
\end{proof}

This is noteworthy because the propositions-as-types interpretation of~\eqref{eq:sigma-ump-map} is ``the axiom of choice''.
If we read $\Sigma$ as ``there exists'' and $\Pi$ (sometimes) as ``for all'', we can pronounce:
\begin{itemize}
\item $\prd{x:X} \sm{a:A(x)} P(x,a)$ as ``for all $x:X$ there exists an $a:A(x)$ such that $P(x,a)$'', and
\item $\sm{g:\prd{x:X} A(x)} \prd{x:X} P(x,g(x))$ as ``there exists a choice function $g:\prd{x:X} A(x)$ such that for all $x:X$ we have $P(x,g(x))$''.
\end{itemize}
Thus, \autoref{thm:ttac} says that not only is the axiom of choice ``true'', it hypotheses are equivalent to its conclusion.
(On the other hand, the classical mathematician may find that~\eqref{eq:sigma-ump-map} does not carry the usual meaning of the axiom of choice, since we have already specified the values of $g$, and there are no choices left to be made.
We will return to this point in \autoref{sec:logic}.)

The above universal property for pair types is for ``mapping in'', which is familiar from the category-theoretic notion of products.
However, pair types also have a universal property for ``mapping out'', which may look less familiar.
In the case of cartesian products, the non-dependent version simply expresses the cartesian closedness adjunction:
\[ \eqvspaced{\big((A\times B) \to C\big)}{\big(A\to (B\to C)\big)}.\]
The dependent version of this is formulated for a type family $C:A\times B\to \type$:
\[ \eqvspaced{\Big(\prd{w:A\times B} C(w)\Big)}{\Big(\prd{x:A}{y:B} C(x,y)\Big)}. \]
Here the left-to-right function is simply the induction principle for $A\times B$, while the right-to-left is evaluation at a pair.
We leave it to the reader to prove that these are quasi-inverses.
There is also a version for $\Sigma$-types:
\begin{equation}
  \eqvspaced{\Big(\prd{w:\sm{x:A} B(x)} C(w)\Big)}{\Big(\prd{x:A}{y:B(x)} C(x,y)\Big)}\label{eq:sigma-lump}
\end{equation}
Again, the left-to-right function is the induction principle.

Some other induction principles are also part of universal properties of this sort.
For instance, path induction is the right-to-left direction of an equivalence as follows:
\begin{equation}
  \label{eq:path-lump}
  \eqvspaced{\Big(\prd{x:A}{p:a=x} B(x,p)\Big)}{B(a,\refl a)}
\end{equation}
for any $a:A$ and type family $B:\prd{x:A} (a=x) \to\type$.
However, inductive types with recursion, such as the natural numbers, have more complicated universal properties; see \autoref{cha:induction}.


\section{Sets and \texorpdfstring{$n$}{n}-types}
\label{sec:basics-sets}

While types in general behave like spaces or higher groupoids, there is a subclass of them that behave more like the sets in a traditional set-theoretic system.
Categorically, we may consider \emph{discrete} groupoids, which are determined by a set of objects and only identity morphisms and higher morphisms; while topologically, we may consider spaces having the discrete topology.
More generally, we may consider groupoids or spaces that are \emph{equivalent} to ones of this sort; since everything we do in type theory is up to homotopy, we can't expect to tell the difference.

Intuitively, we would expect a type to ``be a set'' in this sense if it has no higher homotopical information: any two parallel paths are equal (up to homotopy), and similarly for parallel higher paths at all dimensions.
Fortunately, because everything in homotopy type theory is automatically functorial/continuous, it turns out to be sufficient to ask this at the bottom level.

\begin{defn}\label{defn:set}
  A type $A$ is a \textbf{set} if for all $x,y:A$ and all $p,q:x=y$, we have $p=q$.
\end{defn}

More precisely, the proposition $\isset(A)$ is defined to be the type
\[ \isset(A) \defeq \prd{x,y:A}{p,q:x=y} (p=q). \]

As mentioned in \autoref{sec:types-vs-sets},
the sets in homotopy type theory are not like the sets in ZF set theory, in that there is no global ``membership predicate'' $\in$.
They are more like the sets used in structural mathematics and in category theory, whose elements are ``abstract points'' to which we give structure with functions and relations.
This is all we need in order to use them as a foundational system for most set-based mathematics; we will see some examples in \autoref{cha:set-math}.

Which types are sets?
In \autoref{cha:hlevels} we will study a more general form of this question in depth, but for now we can observe some easy examples.

\begin{eg}
  The type \unit is a set.
  For by \autoref{thm:path-unit}, for any $x,y:\unit$ the type $(x=y)$ is equivalent to \unit.
  Since any two elements of \unit are equal, this implies that any two elements of $x=y$ are equal.
\end{eg}

\begin{eg}
  The type $\emptyt$ is a set, for given any $x,y:\emptyt$ we may deduce anything we like by contradiction.
\end{eg}

\begin{eg}\label{thm:nat-set}
  The type \nat of natural numbers is also a set.
  This follows from \autoref{thm:path-nat}, since all equality types $\id[\nat]xy$ are equivalent to either \unit or \emptyt, and any two inhabitants of \unit or \emptyt are equal.
  We will see another proof of this fact in \autoref{cha:hlevels}.
\end{eg}

Most of the type forming operations we have considered so far also preserve sets.

\begin{eg}\label{thm:isset-prod}
  If $A$ and $B$ are sets, then so is $A\times B$.
  For given $x,y:A\times B$ and $p,q:x=y$, by \autoref{thm:path-prod} we have $p= \pairpath(\projpath1(p),\projpath2(p))$ and $q= \pairpath(\projpath1(q),\projpath2(q))$.
  But $\projpath1(p)=\projpath1(q)$ since $A$ is a set, and $\projpath2(p)=\projpath2(q)$ since $B$ is a set; hence $p=q$.

  Similarly, if $A$ is a set and $B:A\to\type$ is such that each $B(x)$ is a set, then $\sm{x:A} B(x)$ is a set.
\end{eg}

\begin{eg}\label{thm:isset-forall}
  If $A$ is \emph{any} type and $B:A\to \type$ is such that each $B(x)$ is a set, then the type $\prd{x:A} B(x)$ is a set.
  For suppose $f,g:\prd{x:A} B(x)$ and $p,q:f=g$.
  By function extensionality, we have $p = {\funext (x \mapsto \happly(p,x))}$ and likewise $q = {\funext (x \mapsto \happly(q,x))}$.
  But for any $x:A$, we have $\happly(p,x):f(x)=g(x)$ and also $\happly(q,x):f(x)=g(x)$, so since $B(x)$ is a set we have $\happly(p,x) = \happly(q,x)$.
  Now using function extensionality again, the dependent functions $(x \mapsto \happly(p,x))$ and $(x \mapsto \happly(q,x))$ are equal, and hence (applying $\apfunc{\funext}$) so are $p$ and $q$.
\end{eg}

For more examples, see \autoref{ex:isset-coprod,ex:isset-sigma}.  For a more systematic investigation of the subsystem (category) of all sets in homotopy type theory, see~\autoref{cha:set-math}.

Sets are just the first rung on a ladder of what are called \emph{homotopy $n$-types}.
The next rung consists of \emph{$1$-types}, which are analogous to $1$-groupoids in category theory.
The defining property of a set (which we may also call a \emph{$0$-type}) is that it has no non-trivial paths.
Similarly, the defining property of a $1$-type is that it has no non-trivial paths between paths:

\begin{defn}\label{defn:1type}
  A type $A$ is a \textbf{1-type} if for all $x,y:A$ and $p,q:x=y$ and $r,s:p=q$, we have $r=s$.
\end{defn}

Similary, we can define $2$-types, $3$-types, and so on.
We will define the general notion of $n$-type inductively in \autoref{cha:hlevels}, and study the relationships between them.

However, for now it is useful to have two facts in mind.
First, the levels are upward-closed: if $A$ is an $n$-type then $A$ is an $(n+1)$-type.
For example:
%%  will make precise the sense in which this
%% ``suffices for all higher levels'', but as an example, we observe that
%% it suffices for the next level up.

\begin{lem}\label{thm:isset-is1type}
  If $A$ is a set (that is, $\isset(A)$ is inhabited), then $A$ is a 1-type.
\end{lem}
\begin{proof}
  Suppose $f:\isset(A)$; then for any $x,y:A$ and $p,q:x=y$ we have $f(x,y,p,q):p=q$.
  Fix $x$, $y$, and $p$, and define $g: \prd{q:x=y} (p=q)$ by $g(q) \defeq f (x,y,p,q)$.
  Then for any $r:q=q'$, we have $\apdfunc{g}(r) : \trans{r}{g(q)} = g(q')$.
  By \autoref{cor:transport-path-prepost}, therefore, we have $g(q) \ct r = g(q')$.

  In particular, suppose given $x,y,p,q$ and $r,s:p=q$, as in \autoref{defn:1type}, and define $g$ as above.
  Then $g(p) \ct r = g(q)$ and also $g(p) \ct s = g(q)$, hence by cancellation $r=s$.
\end{proof}

Second, this stratification of types by level is not degenerate, in the
sense that not all types are sets:  

\begin{eg}\label{thm:type-is-not-a-set}
  The universe \type is not a set.
  To prove this, it suffices to exhibit a type $A$ and a path $p:A=A$ which is not equal to $\refl A$.
  Take $A=\bool$, and let $f:A\to A$ be defined by $f(\btrue)\defeq \bfalse$ and $f(\bfalse)\defeq \btrue$.
  Then $f(f(x))=x$ for all $x$ (by an easy case analysis), so $f$ is an equivalence.
  Hence, by univalence, $f$ gives rise to a path $p:A=A$.

  If $p$ were equal to $\refl A$, then (again by univalence) $f$ would equal the identity function of $A$.
  But this would imply that $\btrue=\bfalse$, contradicting \autoref{rmk:true-neq-false}.
\end{eg}

In \autoref{cha:hits,cha:homotopy} we will show that for any $n$, there are types which are not $n$-types.

Note that $A$ is a 1-type exactly when for any $x,y:A$, the identity type $\id[A]xy$ is a set.
(Thus, \autoref{thm:isset-is1type} could equivalently be read as saying that the identity types of a set are also sets.)
This will be the basis of the inductive definition of $n$-types we will give in \autoref{cha:hlevels}.

We can also extend this characterization ``downwards'' from sets.
That is, a type $A$ is a set just when for any $x,y:A$, any two elements of $\id[A]xy$ are equal.
Since sets are equivalently 0-types, it is natural to call a type a \emph{$(-1)$-type} if it has this latter property (any two elements of it are equal).
Such types may be regarded as \emph{propositions in a narrow sense}, and their study is just what is usually called ``logic''.


%%%%%%%%%%%%%%%%%%%%%%%%%%%
\section{Logic}
\label{sec:logic}

Type theory, formal or informal, is a collection of rules for manipulating types and their elements.
But when writing mathematics informally in natural language, we generally use familiar words, particularly logical connectives such as ``and'' and ``or'', and logical quantifiers such as ``for all'' and ``there exists''.
In contrast to set theory, type theory offers us more than one way to regard these English phrases as operations on types.
This potential ambiguity needs to be resolved, by setting out local or global conventions, by introducing new annotations to informal mathematics, or both.
This requires some getting used to, but is offset by the fact that because type theory permits this finer analysis of logic, we can represent mathematics more faithfully, with fewer ``abuses of language'' than in set-theoretic foundations.
In this section we will explain the issues involved, and justify the choices we have made.


\subsection{Propositions as types?}
\label{subsec:pat?}

Until now, we have been following the straightforward ``propositions as types'' philosophy described in \autoref{sec:pat}, according to which English phrases such as ``there exists an $x:A$ such that $P(x)$'' are interpreted by corresponding types such as $\sm{x:A} P(x)$, with the proof of a statement being regarded as judging some specific element to inhabit that type.
However, we have also seen some ways in which the ``logic'' resulting from this reading seems unfamiliar to a classical mathematician.
For instance, in \autoref{thm:ttac} we saw that the statement
\begin{equation}\label{eq:english-ac}
  \parbox{\textwidth-2cm}{``If for all $x:X$ there exists an $a:A(x)$ such that $P(x,a)$, then there exists a function $g:\prd{x:A} A(x)$ such that for all $x:X$ we have $P(x,g(x))$,''}
\end{equation}
which looks like the classical \emph{axiom of choice}, is always true under this reading. This is a noteworthy, and often useful, feature of the propositions-as-types logic, but it also illustrates how significantly it differs from the classical interpretation of logic, under which the axiom of choice is not a logical truth, but an additional ``axiom".

On the other hand, we can now also show that corresponding statements looking like the classical \emph{law of double negation} and \emph{law of excluded middle} are incompatible with the univalence axiom.

\begin{thm}\label{thm:not-dneg}
  It is not the case that for all $A:\UU$ we have $\neg(\neg A) \to A$.
\end{thm}
\begin{proof}
  Recall that $\neg A \jdeq (A\to\emptyt)$.
  We also read ``it is not the case that \dots'' as the operator $\neg$.
  Thus, in order to prove this statement, it suffices to assume given some $f:\prd{A:\UU} (\neg\neg A \to A)$ and construct an element of \emptyt.

  The idea of the following proof is to observe that $f$, like any function in type theory, is ``continuous''.
  By univalence, this implies that $f$ is \emph{natural} with respect to equivalences of types.
  From this, and a fixed-point-free autoequivalence, we will be able to extract a contradiction.

  Let $e:\eqv\bool\bool$ be the equivalence defined by $e(\bfalse)\defeq\btrue$ and $e(\btrue)\defeq\bfalse$, as in \autoref{thm:type-is-not-a-set}.
  Let $p:\bool=\bool$ be the path corresponding to $e$ by univalence, i.e.\ $p\defeq \ua(e)$.
  Then we have $f(\bool) : \neg\neg\bool \to\bool$ and
  \[\apd f p : \transfib{A\mapsto (\neg\neg A \to A)}{p}{f(\bool)} = f(\bool).\]
  Hence, for any $u:\neg\neg\bool$, we have
  \[\happly(\apd f p,u) : \transfib{A\mapsto (\neg\neg A \to A)}{p}{f(\bool)}(u) = f(\bool)(u).\]

  Now by~\eqref{eq:transport-arrow}, transporting $f(\bool):\neg\neg\bool\to\bool$ along $p$ in the type family ${A\mapsto (\neg\neg A \to A)}$ is equal to the function which transports its argument along $\opp p$ in the type family $A\mapsto \neg\neg A$, applies $f(\bool)$, then transports the result along $p$ in the type family $A\mapsto A$:
  \[ \transfib{A\mapsto (\neg\neg A \to A)}{p}{f(\bool)}(u) =
  \transfib{A\mapsto A}{p}{f(\bool) (\transfib{A\mapsto \neg\neg A}{\opp{p}}{u})}
  \]
  However, any two points $u,v:\neg\neg\bool$ are equal by function extensionality, since for any $x:\neg\bool$ we have $u(x):\emptyt$ and thus we can derive any conclusion, in particular $u(x)=v(x)$.
  Thus, we have $\transfib{A\mapsto \neg\neg A}{\opp{p}}{u} = u$, and so from $\happly(\apd f p,u)$ we obtain an equality
  \[ \transfib{A\mapsto A}{p}{f(\bool)(u)} = f(\bool)(u).\]
  Finally, as discussed in \autoref{sec:compute-universe}, transporting in the type family $A\mapsto A$ along the path $p\jdeq \ua(e)$ is equivalent to applying the equivalence $e$; thus we have
  \begin{equation}
    e(f(\bool)(u)) = f(\bool)(u).\label{eq:fpaut}
  \end{equation}

  However, we can also prove that
  \begin{equation}
    \prd{x:\bool} \neg(e(x)=x).\label{eq:fpfaut}
  \end{equation}
  This follows from a case analysis on $x$: both cases are immediate from the definition of $e$ and the injectivity of $\inl$ and $\inr$ which we proved in \autoref{sec:compute-coprod}.
  Thus, applying~\eqref{eq:fpfaut} to $f(\bool)(u)$ and~\eqref{eq:fpaut}, we obtain an element of $\emptyt$.
\end{proof}

\begin{rmk}
  In particular, this implies that there can be no Hilbert-style ``choice operator'' which selects an element of every nonempty type.
  The point is that no such operator can be \emph{natural}, and under the univalence axiom, all functions acting on types must be natural with respect to equivalences.
\end{rmk}

\begin{rmk}
  It is, however, still the case that $\neg\neg\neg A \to \neg A$ for any $A$; see \autoref{ex:neg-ldn}.
\end{rmk}

\begin{cor}\label{thm:not-lem}
  It is not the case that for all $A:\UU$ we have $A+(\neg A)$.
\end{cor}
\begin{proof}
  Suppose we had $g:\prd{A:\UU} (A+(\neg A))$.
  We will show that then $\prd{A:\UU} (\neg\neg A \to A)$, so that we can apply \autoref{thm:not-dneg}.
  Thus, suppose $A:\UU$ and $u:\neg\neg A$; we want to construct an element of $A$.

  Now $g(A):A+(\neg A)$, so by case analysis, we may assume either $g(A)\jdeq \inl(a)$ for some $a:A$, or $g(A)\jdeq \inr(w)$ for some $w:\neg A$.
  In the first case, we have $a:A$, while in the second case we have $u(w):\emptyt$ and so we can obtain anything we wish (such as $A$).
  Thus, in both cases we have an element of $A$, as desired.
\end{proof}

Thus, if we want to assume the univalence axiom (which, of course, we do) and still leave ourselves the option of classical reasoning (which is also desirable), we cannot use the unmodified propositions-as-types principle to interpret \emph{all} informal mathematical statements into type theory, since then the law of excluded middle would be false.
However, neither do we want to discard propositions-as-types entirely, because of its many good properties (such as simplicity, constructivity, and computability).
We now discuss a modification of propositions-as-types which resolves these problems; in \autoref{subsec:when-trunc} we will return to the question of which logic to use when.


\subsection{Mere propositions}
\label{subsec:hprops}

We have seen that the propositions-as-types logic has both good and bad properties.
Both have a common cause: when types are viewed as propositions, they can contain more information than mere truth or falsity, and all ``logical'' constructions on them must respect this additional information.
This suggests that we could obtain a more conventional logic by restricting attention to types that do \emph{not} contain any more information than a truth value, and only regarding these as logical propositions.

Such a type $A$ will be ``true'' if it is inhabited, and ``false'' if its inhabitation yields a contradiction (i.e.\ if $\neg A \jdeq (A\to\emptyt)$ is inhabited).
What we want to avoid, in order to obtain a more traditional sort of logic, is treating as logical propositions those types for which giving an element of them gives more information than simply knowing that the type is inhabited.
For instance, if we are given an element of \bool, then we receive more information than the mere fact that \bool contains some element.
Indeed, we receive exactly \emph{one bit} more information: we know \emph{which} element of \bool we were given.
By contrast, if we are given an element of \unit, then we receive no more information than the mere fact that \unit contains an element, since any two elements of \unit are equal to each other.
This suggests the following definition.

\begin{defn}
  A type $P$ is a \textbf{mere proposition} if for all $x,y:P$ we have $x=y$.
\end{defn}

Note that since we are still doing mathematics \emph{in} type theory, this is a definition \emph{in} type theory, which means it is a type --- or, rather, a type family.
Specifically, for any $P:\type$, the type $\isprop(P)$ is defined to be
\[ \isprop(P) \defeq \prd{x,y:P} (x=y). \]
Thus, to assert that ``$P$ is a mere proposition'' means to exhibit an inhabitant of $\isprop(P)$, which is a dependent function connecting any two elements of $P$ by a path.
The continuity/naturality of this function implies that not only are any two elements of $P$ equal, but $P$ contains no higher homotopy either.

\begin{lem}\label{thm:inhabprop-eqvunit}
  If $P$ is a mere proposition and $x_0:P$, then $\eqv P \unit$.
\end{lem}
\begin{proof}
  Define $f:P\to\unit$ by $f(x)\defeq \ttt$, and $g:\unit\to P$ by $g(u)\defeq x_0$.
  The claim follows from the next lemma, and the observation that \unit is a mere proposition by \autoref{thm:path-unit}.
\end{proof}

\begin{lem}\label{lem:equiv-iff-hprop}
  If $P$ and $Q$ are mere propositions such that $P\to Q$ and $Q\to P$, then $\eqv P Q$.
\end{lem}
\begin{proof}
  Suppose given $f:P\to Q$ and $g:Q\to P$.
  Then for any $x:P$, we have $g(f(x))=x$ since $P$ is a mere proposition.
  Similarly, for any $y:Q$ we have $f(g(y))=y$ since $Q$ is a mere proposition; thus $f$ and $g$ are quasi-inverses.
\end{proof}

In homotopy theory, a space that is homotopy equivalent to \unit is said to be \emph{contractible}.
Thus, any mere proposition which is inhabited is contractible (see also \autoref{sec:contractibility}).
On the other hand, the uninhabited type \emptyt is also (vacuously) a mere proposition.
In classical mathematics, at least, these are the only two possibilities.

Mere propositions are also called \emph{subterminal objects} (if thinking categorically), \emph{subsingletons} (if thinking set-theoretically), or \emph{h-propositions}.
The discussion in \autoref{sec:basics-sets} suggests we should also call them \emph{$(-1)$-types}; we will return to this in \autoref{cha:hlevels}.
The adjective ``mere'' emphasizes that although any type may be regarded as a proposition (which we prove by giving an inhabitant of it), a type that is a mere proposition cannot usefully be regarded as any \emph{more} than a proposition: there is no additional information contained in a witness of its truth.

Note that a type $A$ is a set if and only if for all $x,y:A$, the identity type $\id[A]xy$ is a mere proposition.
On the other hand, by copying and simplifying the proof of \autoref{thm:isset-is1type}, we have:

\begin{lem}\label{thm:prop-set}
  Every mere proposition is a set.
\end{lem}
\begin{proof}
  Suppose $f:\isprop(A)$; thus for all $x,y:A$ we have $f(x,y):x=y$.  Fix $x:A$
  and define $g(y)\defeq f(x,y)$.   Then for any $y,z:A$ and $p:y=z$ we have $\apd
  g p : \trans{p}{g(y)}={g(z)}$.  Hence by \autoref{cor:transport-path-prepost}, we have
  $g(y)\ct p = g(z)$, which is to say that $p=\opp{g(y)}\ct g(z)$.  Thus, for
  any $p,q:x=y$, we have $p = \opp{g(x)}\ct g(y) = q$.
\end{proof}

In particular, this implies:

\begin{lem}\label{thm:isprop-isprop}\label{thm:isprop-isset}
  For any type $A$, the types $\isprop(A)$ and $\isset(A)$ are mere propositions.
\end{lem}
\begin{proof}
  Suppose $f,g:\isprop(A)$.  By function extensionality, to show $f=g$ it
  suffices to show $f(x,y)=g(x,y)$ for any $x,y:A$.  But $f(x,y)$ and $g(x,y)$
  are both paths in $A$, and hence are equal because, by either $f$ or $g$, we
  have that $A$ is a proposition, and hence by Lemma~\ref{thm:prop-set} is a
  set.  Similarly, suppose $f,g:\isset(A)$, which is to say that for all
  $a,b:A$, $f(a,b):a=b$ and $g(a,b):a=b$.  But by then since $A$ is a set (by
  either $f$ or $g$), it follows that $f(a,b)=g(a,b)$, and hence $f=g$ by
  function extensionality.
\end{proof}

We have seen one other example so far: condition~\ref{item:be3} in \autoref{sec:basics-equivalences} asserts that for any function $f$, the type $\isequiv (f)$ should be a mere proposition.


\subsection{Classical vs.\ intuitionistic logic}
\label{sec:intuitionism}

With the notion of mere proposition in hand, we can now give the proper formulation of the \emph{law of excluded middle} in homotopy type theory:
\begin{equation}
  \label{eq:lem}
  \mathsf{LEM}\;\defeq\;
  \prd{A:\UU} \Big(\isprop(A) \to (A + \neg A)\Big).
\end{equation}
Similarly, the \emph{law of double negation} is
\begin{equation}
  \label{eq:ldn}
  \mathsf{DN}\;\defeq\;
  \prd{A:\UU} \Big(\isprop(A) \to (\neg\neg A \to A)\Big).
\end{equation}
These formulations avoid the paradoxes of \autoref{thm:not-dneg,thm:not-lem}, since \bool is not a mere proposition.
In order to distinguish these from the more general Propositions-as-Types formulations, we rename the latter:
\begin{align*}
  \mathsf{LEM}_\infty\;\defeq\;&
  \prd{A:\UU} (A + \neg A)\\
  \mathsf{DN}_\infty\;\defeq\;&
  \prd{A:\UU}(\neg\neg A \to A).
\end{align*}


Although $\mathsf{LEM}$ and $\mathsf{DN}$  are not consequences of the basic type theory described in \autoref{cha:typetheory}, they may be consistently assumed as axioms (unlike their $\infty$ counterparts).
For instance, we will assume them in \autoref{sec:wellorderings}.
(The two are also easily seen to be equivalent to each other; see \autoref{ex:lem-ldn}.)

However, it can be surprising how far we can get without using such axioms.
Quite often, a simple reformulation of a definition or theorem enables us to avoid invoking excluded middle or double negation.
While this takes a little getting used to sometimes, it is often worth the hassle, resulting in more elegant and more general proofs.
We discussed some of the benefits of this in the introduction.

For instance, in classical mathematics, double negations are frequently used unnecessarily.
A very simple example is the common assumption that a set $A$ is ``nonempty'', which literally means it is \emph{not} the case that $A$ contains \emph{no} elements.
Almost always what is really meant is the positive assertion that $A$ \emph{does} contain at least one element, and by removing the double negation we make the statement less dependent on LEM.
Thus we say that a type $A$ is \textbf{inhabited} to mean that we assert $A$ itself as a proposition (i.e.\ we construct an element of $A$, usually unnamed).

Similarly, it is not uncommon in classical mathematics to find unnecessary proofs by contradiction.
Of course, proof by contradiction proceeds by way of the law of double negation: we assume $\neg A$ and derive a contradiction, thereby deducing $\neg \neg A$, and thus by DN we obtain $A$.
However, often the derivation of a contradiction from $\neg A$ can be rephrased slightly so as to yield a direct proof of $A$, avoiding the need for DN.

It is also important to note that if the goal is to prove a \emph{negation}, then ``proof by contradiction'' does not involve DN.
In fact, since $\neg A$ is by definition the type $A\to\emptyt$, by definition to prove $\neg A$ is to prove a contradiction (\emptyt) under the assumption of $A$.
Similarly, the law of double negation does hold for negated propositions: $\neg\neg\neg A \to \neg A$.
With practice, one learns to distinguish more carefully between negated and un-negated propositions and to notice when LEM and DN are being used and when they are not.

Thus, contrary to how it may appear on the surface, doing mathematics ``constructively'' does not usually involve giving up important theorems, but rather finding the best way to state the definitions so as to make the important theorems constructively provable.
That is, we may freely use the LEM when first investigating a subject, but once that subject is better understood, we can hope to refine its definitions and proofs so as to avoid that axiom.
% For instance, the theory of ordinal numbers, which classically makes heavy use of LEM, works quite well constructively once we choose the correct definition of ``ordinal''; see \autoref{sec:ordinals}.
This sort of observation is even more pronounced in \emph{homotopy} type theory, where the powerful tools of univalence and higher inductive types allow us to constructively attack many problems that traditionally would require classical reasoning.
We will see several examples of this in \autoref{part:mathematics}.
% For instance, none of the ``synthetic'' homotopy theory we will develop in \autoref{cha:homotopy} requires LEM or DN --- despite the fact that classical homotopy theory (formulated using topological spaces or simplicial sets) makes heavy use of them (as well as the axiom of choice).



\subsection{Subsets}
\label{subsec:prop-subsets}

As another example of the usefulness of mere propositions, we discuss subsets (and more generally subtypes).
Suppose $P:A\to\type$ is a type family, with each type $P(x)$ regarded as a proposition.
Then $P$ itself is a \emph{predicate} on $A$, or a \emph{property} of elements of $A$.

In set theory, whenever we have a predicate on $P$ on a set $A$, we may form the subset $\setof{x\in A | P(x)}$.
In type theory, the obvious analogue is the $\Sigma$-type $\sm{x:A} P(x)$.
An inhabitant of $\sm{x:A} P(x)$ is, of course, a pair $(x,p)$ where $x:A$ and $p$ is a proof of $P(x)$.
However, for general $P$, an element $a:A$ might give rise to more than one distinct element of $\sm{x:A} P(x)$, if the proposition $P(a)$ has more than one distinct proof.
This is counter to the usual intuition of a \emph{subset}.
But if $P$ is a \emph{mere} proposition, then this cannot happen.

\begin{lem}\label{thm:path-subset}
  Suppose $P:A\to\type$ is a type family such that $P(x)$ is a mere proposition for all $x:A$.
  If $u,v:\sm{x:A} P(x)$ are such that $\proj1(u) = \proj1(v)$, then $u=v$.
\end{lem}
\begin{proof}
  Suppose $p:\proj1(u) = \proj1(v)$.
  By \autoref{thm:path-sigma}, to show $u=v$ it suffices to show $\trans{p}{\proj2(u)} = \proj2(v)$.
  But $\trans{p}{\proj2(u)}$ and $\proj2(v)$ are both elements of $P(\proj1(v))$, which is a mere proposition; hence they are equal.
\end{proof}

For instance, recall that in \autoref{sec:basics-equivalences} we defined
\[(\eqv A B) \;\defeq\; \sm{f:A\to B} \isequiv (f),\]
where each type $\isequiv (f)$ was supposed to be a mere proposition.
It follows that if two equivalences have equal underlying functions, then they are equal as equivalences.

Henceforth, if $P:A\to \type$ is a family of mere propositions, we may write
\[\setof{x:A | P(x)}\]
as an alternative notation for $\sm{x:A} P(x)$.
If $A$ is a set, we call $\setof{x:A | P(x)}$ a \textbf{subset} of $A$; for general $A$ we might call it a \textbf{subtype}.

As another example, we may define the ``subuniverses'' of sets and of mere propositions in a universe \UU:
\begin{align*}
  \set_\UU &\defeq \setof{A:\UU | \isset(A) }\\
  \prop_\UU &\defeq \setof{A:\UU | \isprop(A) }.
\end{align*}
An element of $\set_\UU$ is a type $A:\UU$ together with evidence $s:\isset(A)$, and similarly for $\prop_\UU$.
\autoref{thm:path-subset} implies that $\id[\set_\UU]{(A,s)}{(B,t)}$ is equivalent to $\id[\UU]AB$ (and hence to $\eqv AB$).
Thus, we will frequently abuse notation and write simply $A:\set_\UU$ instead of $(A,s):\set_\UU$.
We may also drop the subscript \UU if there is no need to specify the universe in question.

Recall that for any two universes $\UU_i$ and $\UU_{i+1}$, if $A:\UU_i$ then also $A:\UU_{i+1}$.
Thus, for any $(A,s):\set_{\UU_i}$ we also have $(A,s):\set_{\UU_{i+1}}$, and similarly for $\prop_{\UU_i}$, giving natural maps
\begin{align}
  \set_{\UU_i} &\to \set_{\UU_{i+1}}\label{eq:set-up}\\
  \prop_{\UU_i} &\to \prop_{\UU_{i+1}}.\label{eq:prop-up}
\end{align}
The map~\eqref{eq:set-up} cannot be an equivalence, since then we could reproduce the paradoxes of Cantorian set theory.
However, although~\eqref{eq:prop-up} is not automatically an equivalence in the type theory we have presented so far, it is consistent to suppose that it is.
This axiom is called \textbf{impredicativity for mere propositions}.
It follows automatically if $\UU_{i+1}$ satisfies LEM (see \autoref{ex:lem-impred}).

One use for impredicativity is to define powersets.
It is natural to define the powerset of a set $A$ to be $A\to\prop_\UU$; but in the absence of impredicativity, this definition depends (even up to equivalence) on the choice of the universe \UU.
Impredicativity means that we may as well take it to be the smallest universe $\UU_0$, since any other universe would give an equivalent result.
See also \autoref{subsec:piw}.


\subsection{The logic of mere propositions}
\label{subsec:logic-hprop}

We mentioned in \autoref{sec:types-vs-sets} that in contrast to type theory, which has only one basic notion (types), set-theoretic foundations have two basic notions: sets and propositions.
Thus, a classical mathematician is accustomed to manipulating these two kinds of objects separately.

It is possible to recover a similar dichotomy in type theory, with the role of the set-theoretic propositions being played by the types (and type families) that are \emph{mere} propositions.
In many cases, the logical connectives and quantifiers can be represented in this logic by simply restricting the corresponding type-former to the mere propositions.
Of course, this requires knowing that the type-former in question preserves mere propositions.

\begin{eg}
  If $A$ and $B$ are mere propositions, so is $A\times B$.
  This is easy to show using the characterization of paths in products, just like \autoref{thm:isset-prod} but simpler.
  Thus, the connective ``and'' preserves mere propositions.
\end{eg}

\begin{eg}\label{thm:isprop-forall}
  If $A$ is any type and $B:A\to \type$ is such that for all $x:A$, the type $B(x)$ is a mere proposition, then $\prd{x:A} B(x)$ is a mere proposition.
  The proof is just like \autoref{thm:isset-forall} but simpler: given $f,g:\prd{x:A} B(x)$, for any $x:A$ we have $f(x)=g(x)$ since $B(x)$ is a mere proposition.
  But then by function extensionality, we have $f=g$.

  In particular, if $B$ is a mere proposition, then so is $A\to B$ regardless of what $A$ is.
  In even more particular, since \emptyt is a mere proposition, so is $\neg A \jdeq (A\to\emptyt)$.
  Thus, the connectives ``implies'' and ``not'' preserve mere propositions, as does the quantifier ``for all''.
\end{eg}

On the other hand, some type formers do not preserve mere propositions.
Even if $A$ and $B$ are mere propositions, $A+B$ will not in general be.
For instance, \unit is a mere proposition, but $\bool=\unit+\unit$ is not.
Logically speaking, $A+B$ is a ``purely constructive'' sort of ``or'': a witness of it contains the additional information of \emph{which} disjunct is true.
Sometimes this is very useful, but if we want a more classical sort of ``or'' that preserves mere propositions, we need a way to ``truncate'' this type into a mere proposition by forgetting this additional information.

The same issue arises with the $\Sigma$-type $\sm{x:A} P(x)$.
This is a purely constructive interpretation of ``there exists an $x:A$ such that $P(x)$'' which remembers the witness $x$, and hence is not generally a mere proposition even if each type $P(x)$ is.
(Recall that we observed in \autoref{subsec:prop-subsets} that $\sm{x:A} P(x)$ can also be regarded as ``the subset of those $x:A$ such that $P(x)$''.)


\subsection{Propositional truncation}
\label{subsec:prop-trunc}

The \emph{propositional truncation}, also called the \emph{$(-1)$-truncation}, \emph{bracket type}, or \emph{squash type}, is an additional type former which ``truncates'' or ``squashes'' a type down to a mere proposition, forgetting all information contained in inhabitants of that type other than their existence.

More precisely, for any type $A$, there is a type $\brck{A}$.
It has two constructors:
\begin{itemize}
\item For any $a:A$ we have $\bproj a : \brck A$.
\item For any $x,y:\brck A$, we have $x=y$.
\end{itemize}
The first constructor means that if $A$ is inhabited, so is $\brck A$.
The second ensures that $\brck A$ is a mere proposition; usually we leave the witness of this fact nameless.

The induction principle of $\brck A$ says that:
\begin{itemize}
\item If $B$ is a mere proposition and we have $f:A\to B$, then there is an induced $g:\brck A \to B$ such that $g(\bproj a) \jdeq f(a)$ for all $a:A$.
\end{itemize}
In other words, any mere proposition which follows from (the inhabitedness of) $A$ already follows from $\brck A$.
Thus, $\brck A$, as a mere proposition, contains no more information than the inhabitedness of $A$.

In \autoref{ex:lem-brck,ex:impred-brck,sec:hittruncations} we will describe some ways to construct $\brck{A}$ in terms of more general things.
For now, we simply assume it as an additional rule alongside those of \autoref{cha:typetheory}.

With the propositional truncation, we can extend the ``logic of mere propositions'' to cover disjunction and the existential quantifier.
Specifically, $\brck{A+B}$ is a mere propositional version of ``$A$ or $B$'', which does not ``remember'' the information of which disjunct is true.

The induction principle of truncation implies that we can still do a case analysis on $\brck{A+B}$ \emph{when attempting to prove a mere proposition}.
That is, suppose we have an assumption $u:\brck{A+B}$ and we are trying to prove a mere proposition $Q$.
In other words, we are trying to define an element of $\brck{A+B} \to Q$.
Since $Q$ is a mere proposition, by the induction principle for propositional truncation, it suffices to construct a function $A+B\to Q$.
But now we can use case analysis on $A+B$.

Similarly, for a type family $P:A\to\type$, we can consider $\brck{\sm{x:A} P(x)}$, which is a mere propositional version of ``there exists an $x:A$ such that $P(x)$''.
As for disjunction, by combining the induction principles of truncation and $\Sigma$-types, if we have an assumption of type $\brck{\sm{x:A} P(x)}$, we may introduce new assumptions $x:A$ and $y:P(x)$ \emph{when attempting to prove a mere proposition}.
In other words, if we know that there exists some $x:A$ such that $P(x)$, but we don't have a particular such $x$ in hand, then we are free to make use of such an $x$ as long as we aren't trying to construct anything which might depend on the particular value of $x$.
Requiring the codomain to be a mere proposition expresses this independence of the result on the witness, since all possible inhabitants of such a type must be equal.

For the purposes of set-level mathematics in \autoref{cha:real-numbers,cha:set-math},
where we deal mostly with sets and mere propositions, it is convenient to use the
traditional logical notations to refer only to ``propositionally truncated logic''.

\begin{defn} \label{defn:logical-notation}
  We define \emph{traditional logical notation} using truncation as follows, where $P$ and $Q$ denote mere propositions (or families thereof):
  \begin{align*}
    \top            &\ \defeq \ \unit \\
    \bot            &\ \defeq \ \emptyt \\
    P \land Q       &\ \defeq \ P \times Q \\
    P \Rightarrow Q &\ \defeq \ P \to Q \\
    P \Leftrightarrow Q &\ \defeq \ P = Q \\
    \neg P          &\ \defeq \ P \to \emptyt \\
    P \lor Q        &\ \defeq \ \brck{P + Q} \\
    \fall{x : A} P(x) &\ \defeq \ \prd{x : A} P(x) \\
    \exis{x : A} P(x) &\ \defeq \ \brck{\sm{x : A} P(x)}
  \end{align*}
\end{defn}

The notations $\land$ and $\lor$ are also used in homotopy theory for the smash product and the wedge of pointed spaces, which we will introduce in \autoref{cha:hits}.
This technically creates a potential for conflict, but no confusion will generally arise.


\subsection{The Axiom of Choice}
\label{sec:axiom-choice}

We can now properly formulate the \emph{axiom of choice} in homotopy type theory.
Assume a type $X$ and type families $A:X\to\type$ and $P:\prd{x:X} A(x)\to\type$, and moreover that
\begin{itemize}
\item $X$ is a set,
\item $A(x)$ is a set for all $x:X$, and
\item $P(x,a)$ is a mere proposition for all $x:X$ and $a:A(x)$.
\end{itemize}
The \textbf{axiom of choice} asserts that under these assumptions,
\begin{equation}\label{eq:ac}
  \left(\prd{x:X} \brck{\sm{a:A(x)} P(x,a)}\right)
  \to
  \brck{\sm{g:\prd{x:X} A(x)} \prd{x:X} P(x,g(x))}
\end{equation}
Of course, this is a direct translation of~\eqref{eq:english-ac} where we read ``there exists $x:A$ such that $B(x)$'' as $\brck{\sm{x:A}B(x)}$, so we could have written the statement in the familiar logical notation as
\begin{equation*}
  \textstyle
  \Big(\fall{x:X}\exis{a:A(x)} P(x,a)\Big)
  \Rightarrow
  \Big(\exis{g : (\prd{x:X} A(x))} \fall{x : X} P(x,g(x))\Big).
\end{equation*}
%
In particular, note that the propositional truncation appears twice.
The truncation in the domain means we assume that for every $x$ there exists some $a:A(x)$ such that $P(x,a)$, but that these values are not chosen or specified in any known way.
The truncation in the codomain means we conclude that there exists some function $g$, but this function is not determined or specified in any known way.

In fact, because of \autoref{thm:ttac}, this axiom can also be expressed in a simpler form.

\begin{lem}\label{thm:ac-epis-split}
  The axiom of choice~\eqref{eq:ac} is equivalent to the statement that for any set $X$ and any $Y:X\to\type$ such that each $Y(x)$ is a set, we have
  \begin{equation}
    \left(\prd{x:X} \brck{Y(x)}\right)
    \to
    \brck{\prd{x:X} Y(x)}.\label{eq:epis-split}
  \end{equation}
\end{lem}

This corresponds to a well-known equivalent form of the classical axiom of choice, namely ``the cartesian product of a family of nonempty sets is nonempty.''

\begin{proof}
  By \autoref{thm:ttac}, the codomain of~\eqref{eq:ac} is equivalent to
  \[\brck{\prd{x:X} \sm{a:A(x)} P(x,a)}.\]
  Thus,~\eqref{eq:ac} is equivalent to the instance of~\eqref{eq:epis-split} where $Y(x) \defeq \sm{a:A(x)} P(x,a)$.
  Conversely,~\eqref{eq:epis-split} is equivalent to the instance of~\eqref{eq:ac} where $A(x)\defeq Y(x)$ and $P(x,a)\defeq\unit$.
  Thus, the two are logically equivalent.
  Since both are mere propositions, by \autoref{lem:equiv-iff-hprop} they are equivalent types.
\end{proof}

As with LEM and DN, the equivalent forms~\eqref{eq:ac} and~\eqref{eq:epis-split} are not a consequence of our basic type theory, but they may consistently be assumed as axioms.

\begin{rmk}
  It is easy to show that the right side of~\eqref{eq:epis-split} always implies the left.
  Since both are mere propositions, by \autoref{lem:equiv-iff-hprop} the axiom of choice is also equivalent to asking for an equivalence
  \[ \eqv{\left(\prd{x:X} \brck{Y(x)}\right)}{\brck{\prd{x:X} Y(x)}} \]
  This illustrates a common pitfall: although dependent function types preserve mere propositions (\autoref{thm:isprop-forall}), they do not commute with truncation: $\brck{\prd{x:A} P(x)}$ is not generally equivalent to $\prd{x:A} \brck{P(x)}$.
  The axiom of choice, if we assume it, says that this is true \emph{for sets}; as we will see below, it fails in general.
\end{rmk}

The restriction in the axiom of choice to types that are sets can be relaxed to a certain extent.
For instance, we may allow $A$ and $P$ in~\eqref{eq:ac}, or $Y$ in~\eqref{eq:epis-split}, to be arbitrary type families; this results in a seemingly stronger statement that is equally consistent.
We may also replace the propositional truncation by the more general $n$-truncations to be considered in \autoref{cha:hlevels}, obtaining a spectrum of axioms AC$_n$ interpolating between~\eqref{eq:ac}, which we call simply AC, and \autoref{thm:ttac}, which we shall call AC$_\infty$.  However, observe that we cannot relax the requirement that $X$ be a set.  

\begin{lem}\label{thm:no-higher-ac}
  There exists a type $X$ and a family $Y:X\to \type$ such that each $Y(x)$ is a set, but such that~\eqref{eq:epis-split} is false.
\end{lem}
\begin{proof}
  Define $X\defeq \sm{A:\type} \brck{\bool = A}$, and let $x_0 \defeq (\bool, \bproj{\refl{\bool}}) : X$.
  Then by the identification of paths in $\Sigma$-types, the fact that $\brck{A=\bool}$ is a mere proposition, and univalence, we have $\eqv{(\id[X]{(A,p)}{(B,p)})}{(\eqv AB)}$.
  In particular, $\eqv{(\id[X]{x_0}{x_0})}{(\eqv \bool\bool)}$, so as in \autoref{thm:type-is-not-a-set}, $X$ is not a set.
  But if we define $Y(x) \defeq (x_0=x)$, then each $Y(x)$ is a set.

  Now by definition, for any $(A,p):X$ we have $\brck{\bool=A}$, and hence $\brck{x_0 = (A,p)}$.
  Thus, we have $\prd{x:X} \brck{Y(x)}$.
  If~\eqref{eq:epis-split} held for this $X$ and $Y$, then we would also have $\brck{\prd{x:X} Y(x)}$.
  Since we are trying to derive a contradiction ($\emptyt$), which is a mere proposition, we may assume $\prd{x:X} Y(x)$, i.e.\ that $\prd{x:X} (x_0=x)$.
  But this implies $X$ is a mere proposition, and hence a set, which is a contradiction.
\end{proof}

\subsection{The principle of unique choice}
\label{sec:unique-choice}

The following observation is trivial, but very useful.

\begin{lem}
  If $P$ is a mere proposition, then $\eqv P {\brck P}$.
\end{lem}
\begin{proof}
  Of course, we have $P\to \brck{P}$ by definition.
  And since $P$ is a mere proposition, the universal property of $\brck P$ applied to $\idfunc[P] :P\to P$ yields $\brck P \to P$.
  These functions are quasi-inverses by \autoref{lem:equiv-iff-hprop}.
\end{proof}

Among its important consequences is the following.

\begin{cor}[The principle of unique choice]\label{cor:UC}
  Suppose a type family $P:A\to \type$ such that
  \begin{enumerate}
  \item For each $x$, the type $P(x)$ is a mere proposition, and
  \item For each $x$ we have $\brck {P(x)}$.
  \end{enumerate}
  Then we have $f:\prd{x:A} P(x)$.
\end{cor}
\begin{proof}
  Immediate from the two assumptions and the previous lemma.
\end{proof}

The corollary also encapsulates a very useful technique of reasoning.
Namely, suppose we know that $\brck A$, and we want to use this to construct an element of some other type $B$.
We would like to use an element of $A$ in our construction of an element of $B$, but this is allowed only if $B$ is a mere proposition, so that we can apply the induction principle for the propositional truncation $\brck A$; the most we could hope to do in general is to show $\brck B$.
%
Instead, we can extend $B$ with additional data which characterizes \emph{uniquely} the object we wish to construct.
Specifically, we define a predicate $Q:B\to\type$ such that $\sm{x:B} Q(x)$ is a mere proposition.
Then from an element of $A$ we construct an element $b:B$ such that $Q(b)$, hence from $\brck A$ we can construct $\brck{\sm{x:B} Q(x)}$, and because $\brck{\sm{x:B} Q(x)}$ is equivalent to $\sm{x:B} Q(x)$ an element of $B$ may be projected from it.
We provide an example below.

A similar issue arises in set-theoretic mathematics, although it manifests slightly
differently. If we are trying to define a function $f: A \to B$, and depending on an
element $a : A$ we are able to prove mere existence of some $b : B$, we are not done yet
because we need to actually pinpoint an element of~$B$, not just prove its existence.
One option is of course to refine the argument to unique existence of $b : B$, like we did in type theory. But in set theory the problem can often be avoided more simply by an application of the axiom of choice, which picks the required elements for us.
In homotopy type theory, however, quite apart from any desire to avoid choice, the available forms of choice are simply less applicable, since they require that the domain of choice be a \emph{set}.
Thus, if $A$ is not a set (such as perhaps a universe $\UU$), there is no consistent form of choice that will allow us to simply pick an element of $B$ for each $a : A$ to use in defining $f(a)$.


\begin{thm}
  Suppose $P:\nat\to\type$ is such that each $P(n)$ is a mere proposition, and that $\prd{n:\nat} (P(n) + \neg P(n))$ (such a predicate is called \emph{decidable}).
  Then
  \[ \brck{\sm{n:\nat} P(n)} \;\to\; \sm{n:\nat}P(n).\]
\end{thm}
\begin{proof}[Sketch of proof]
  The hypotheses imply that
  \[ \Big(\sm{n:\nat}P(n)\Big) \;\to\; \sm{n:\nat}\Big(P(n) \times \prd{m:\nat} \big((m<n) \to \neg P(m)\big)\Big). \]
  In words, given $n$ such that $P(n)$, we can find the least such $n$: we test every $m<n$ in turn, using decidability to do a case analysis, until we find the first one that satisfies $P(m)$.
  However, the right-hand side of the above implication is a mere proposition: if both $n$ and $n'$ are least numbers satisfying~$P$ then they must be equal.
  Therefore, we also have
  \[ \Brck{\sm{n:\nat}P(n)} \;\to\; \sm{n:\nat}\Big(P(n) \times \prd{m:\nat} \big((m<n) \to \neg P(m)\big)\Big) \]
  from which the claim follows.
\end{proof}

\subsection{When are propositions truncated?}
\label{subsec:when-trunc}

At first glance, it may seem that the truncated versions of $+$ and $\Sigma$ are actually closer to the informal mathematical meaning of ``or'' and ``there exists'' than the untruncated ones.
Certainly, they are closer to the \emph{precise} meaning of ``or'' and ``there exists'' in the first-order logic which underlies formal set theory, since the latter makes no attempt to remember any witnesses to the truth of propositions.
However, it may come as a surprise to realize that the practice of \emph{informal} mathematics is often more accurately described by the untruncated forms.

For example, consider a statement like ``every prime number is either $2$ or odd.''
The working mathematician feels no compunction about using this fact not only to prove \emph{theorems} about prime numbers, but also to perform \emph{constructions} on prime numbers, perhaps doing one thing in the case of $2$ and another in the case of an odd prime.
The end result of the construction is not merely the truth of some statement, but a piece of data which may depend on the parity of the prime number.
Thus, from a type-theoretic perspective, such a construction is naturally phrased using the induction principle for the coproduct type ``$(p=2)+(p\text{ is odd})$'', not its propositional truncation.

Admittedly, this is not an ideal example, since ``$p=2$'' and ``$p$ is odd'' are mutually exclusive, so that $(p=2)+(p\text{ is odd})$ is in fact already a mere proposition and hence equivalent to its truncation (see \autoref{ex:disjoint-or}). % and~\ref{ex:prop-eqvtrunc}).
More compelling examples come from the existential quantifier.
It is not uncommon to prove a theorem of the form ``there exists an $x$ such that \dots'' and then refer later on to ``the $x$ constructed in Theorem Y'' (note the definite article).
Moreover, when deriving further properties of this $x$, one may use phrases such as ``by the construction of $x$ in the proof of Theorem Y''.

A very common example is ``$A$ is isomorphic to $B$'', which strictly speaking means only that there exists \emph{some} isomorphism between $A$ and $B$.
But almost invariably, when proving such a statement, one exhibits a specific isomorphism or proves that some previously known map is an isomorphism, and it often matters later on what particular isomorphism was given.

Set-theoretically trained mathematicians often feel a twinge of guilt at such ``abuses of language''.
We may attempt to apologize for them, expunge them from final drafts, or weasel out of them with vague words like ``canonical''.
The problem is exacerbated by the fact that in formalized set theory, there is technically no way to ``construct'' objects at all --- we can only prove that an object with certain properties exists.
Untruncated logic in type theory thus captures some common practices of informal mathematics that the set theoretic reconstruction obscures.
(This is similar to how the univalence axiom validates the common, but formally unjustified, practice of identifying isomorphic objects.)

On the other hand, sometimes truncated logic is essential.
We have seen this in the statements of LEM and AC; some other examples will appear later on in the book.
Thus, we are faced with the problem: when writing informal type theory, what should we mean by the words ``or'' and ``there exists'' (along with common synonyms such as ``there is'' and ``we have'')?

A universal consensus may not be possible.
Perhaps depending on the sort of mathematics being done, one convention or the other may be more useful --- or, perhaps, the choice of convention may be irrelevant.
In this case, a remark at the beginning of a mathematical paper may suffice to inform the reader of the linguistic conventions in use therein.
However, even after one overall convention is chosen, the other sort of logic will usually arise at least occasionally, so we need a way to refer to it.
More generally, one may consider replacing the propositional truncation with another operation on types that behaves similarly, such as the double negation operation $A\mapsto \neg\neg A$, or the $n$-truncations to be considered in \autoref{cha:hlevels}.
As an experiment in exposition,  in what follows we will occasionally use \emph{adverbs} to denote the application of such ``modalities'' as propositional truncation.

For instance, if untruncated logic is the default convention, we may use the adverb \textbf{merely} to denote propositional truncation.
Thus the phrase
\begin{center}
  ``there merely exists an $x:A$ such that $P(x)$''
\end{center}
indicates the type $\brck{\sm{x:A} P(x)}$.
Similarly, we will say that a type $A$ is \textbf{merely inhabited} to mean that its propositional truncation $\brck A$ is inhabited (i.e.\ that we have an unnamed element of it).
Note that this is a \emph{definition} of the adverb ``merely'' as it is to be used in our informal mathematical English, in the same way that we define nouns like ``group'' and ``ring'', and adjectives like ``regular'' and ``normal'', to have precise mathematical meanings.
We are not claiming that the dictionary definition of ``merely'' refers to propositional truncation; the choice of word is meant only to remind the mathematician reader that a mere proposition contains ``merely'' the information of a truth value and nothing more.

On the other hand, if truncated logic is the current default convention, we may use an adverb such as \textbf{purely} or \textbf{constructively} to indicate its absence, so that
\begin{center}
``there purely exists an $x:A$ such that $P(x)$''
\end{center}
would denote the type $\sm{x:A} P(x)$.
We may also use ``purely'' or ``actually'' just to emphasize the absence of truncation, even when that is the default convention.

In this book we will continue using untruncated logic as the default convention, for a number of reasons.
\begin{enumerate}[label=(\arabic*)]
\item We want to encourage the newcomer to experiment with it, rather than sticking to truncated logic simply because it is more familiar.
\item Using truncated logic as the default in type theory suffers from the same sort of ``abuse of language'' problems as set-theoretic foundations, which untruncated logic avoids.
  For instance, our definition of ``$\eqv A B$'' as the type of equivalences between $A$ and $B$, rather than its propositional truncation, means that to prove a theorem of the form ``$\eqv A B$'' is literally to construct a particular such equivalence.
  This specific equivalence can then be referred to later on.
\item We want to emphasize that the notion of ``mere proposition'' is not a fundamental part of type theory.
  As we will see in \autoref{cha:hlevels}, mere propositions are just the second rung on an infinite ladder, and there are also many other modalities not lying on this ladder at all.
\item Many statements that classically are mere propositions are no longer so in homotopy type theory.
  Of course, foremost among these is equality.
\item On the other hand, one of the most interesting observations of homotopy type theory is that a surprising number of types are \emph{automatically} mere propositions, or can be slightly modified to become so, without the need for any truncation.
  (See \autoref{thm:isprop-isprop} and Chapters~\ref{cha:equivalences}, \ref{cha:hlevels}, \ref{cha:category-theory}, and~\ref{cha:set-math}.)
  Thus, although these types contain no data beyond a truth value, we can nevertheless use them to construct untruncated objects, since there is no need to use the induction principle of propositional truncation.
  This useful fact is more clumsy to express if propositional truncation is applied to all statements by default.
\item Finally, truncations are not very useful for most of the mathematics we will be doing in this book, so it is simpler to notate them explicitly when they occur.
\end{enumerate}

\section{Contractibility}
\label{sec:contractibility}

In \autoref{thm:inhabprop-eqvunit} we observed that a mere proposition which is inhabited must be equivalent to unit, and it is not hard to see that the converse also holds.
A type with this property is called \emph{contractible}.
Another equivalent definition of contractibility, which is also sometimes convenient, is the following.

\begin{defn}\label{defn:contractible}
  A type $A$ is \textbf{contractible}, or a \textbf{singleton}, if there is $a:A$, called the \textbf{center of contraction}, such that $a=x$ for all $x:A$.
  We denote the specified path $a=x$ by $\contr_x$.
\end{defn}

In other words, the type $\iscontr(A)$ is defined to be
\[ \iscontr(A) \defeq \sm{a:A} \prd{x:A}(a=x). \]
Note that under the usual propositions-as-types reading, we can pronounce $\iscontr(A)$ as ``$A$ contains exactly one element'', or more precisely ``$A$ contains an element, and every element of $A$ is equal to that element''.

\begin{rmk}
  We can also pronounce $\iscontr(A)$ more topologically as ``there is a point $a:A$ such that for all $x:A$ there exists a path from $a$ to $x$''.
  Note that to a classical ear, this sounds like a definition of \emph{connectedness} rather than contractibility.
  The point is that the meaning of ``there exists'' in this sentence is a continuous/natural one.
  A more correct way to express connectedness would be $\sm{a:A}\prd{x:A} \brck{a=x}$; we will come back to this later.
\end{rmk}

\begin{lem}\label{thm:contr-unit}
  For a type $A$, the following are logically equivalent.
  \begin{enumerate}
  \item $A$ is contractible in the sense of \autoref{defn:contractible}.\label{item:contr}
  \item $A$ is a mere proposition, and there is a point $a:A$.\label{item:contr-inhabited-prop}
  \item $A$ is equivalent to \unit.\label{item:contr-eqv-unit}
  \end{enumerate}
\end{lem}
\begin{proof}
  If $A$ is contractible, then it certainly has a point $a:A$ (the center of contraction), while for any $x,y:A$ we have $x=a=y$; thus $A$ is a mere proposition.
  Conversely, if we have $a:A$ and $A$ is a mere proposition, then for any $x:A$ we have $x=a$; thus $A$ is contractible.
  And we showed~\ref{item:contr-inhabited-prop}$\Rightarrow$\ref{item:contr-eqv-unit} in \autoref{thm:inhabprop-eqvunit}, while the converse follows since \unit easily has property~\ref{item:contr-inhabited-prop}.
\end{proof}

\begin{lem}\label{thm:isprop-iscontr}
  For any type $A$, the type $\iscontr(A)$ is a mere proposition.
\end{lem}
\begin{proof}
  Suppose given $c,c':\iscontr(A)$.
  We may assume $c\jdeq(a,p)$ and $c'\jdeq(a',p')$ for $a,a':A$ and $p:\prd{x:A} (a=x)$ and $p':\prd{x:A} (a'=x)$.
  By the characterization of paths in $\Sigma$-types, to show $c=c'$ it suffices to exhibit $q:a=a'$ such that $\trans{q}{p}=p'$.

  We choose $q\defeq p(a')$.
  For the other equality, by function extensionality we must show that $(\trans q p)(x)=p'(x)$ for any $x:A$.
  For this, it will suffice to show that for any $x,y:A$ and $u:x=y$ we have $u= \opp{p(x)} \ct p(y)$, since then we would have $(\trans q p)(x) = \opp{p(x)} \ct p(y) = p'(x)$.
  But now we can invoke path induction to assume that $x\jdeq y$ and $u\jdeq \refl{x}$.
  In this case our goal is to show that $\refl x = \opp{p(x)} \ct p(x)$, which is just the inversion law for paths.
\end{proof}

\begin{cor}\label{thm:contr-contr}
  If $A$ is contractible, then so is $\iscontr(A)$.
\end{cor}
\begin{proof}
  By \autoref{thm:isprop-iscontr} and \autoref{thm:contr-unit}\ref{item:contr-inhabited-prop}.
\end{proof}

Like mere propositions, contractible types are preserved by many type constructors.
For instance, we have:

\begin{lem}\label{thm:contr-forall}
  If $P:A\to\type$ is a type family such that each $P(a)$ is contractible, then $\prd{x:A} P(x)$ is contractible.
\end{lem}
\begin{proof}
  By \autoref{thm:isprop-forall}, $\prd{x:A} P(x)$ is a mere proposition since each $P(x)$ is.
  But it also has an element, namely the function sending each $x:A$ to the center of contraction of $P(x)$.
  Thus by \autoref{thm:contr-unit}\ref{item:contr-inhabited-prop}, $\prd{x:A} P(x)$ is contractible.
\end{proof}

(In fact, the statement of \autoref{thm:contr-forall} is equivalent to the function extensionality axiom.
See Appendix~[?].)

Of course, if $A$ is equivalent to $B$ and $A$ is contractible, then so is $B$.
More generally, it suffices for $B$ to be a \emph{retract} of $A$.
By definition, a \define{retraction} is a function $r : A \to B$ such that there exists a function $s : B \to A$, called its \define{section}, and a homotopy $\epsilon:\prd{y:Y} (r(s(y))=y)$; then we say that $B$ is a \define{retract} of $A$.

\begin{lem}\label{thm:retract-contr}
  If $B$ is a retract of $A$, and $A$ is contractible, then so is $B$.
\end{lem}
\begin{proof}
  Let $a_0 : A$ be the center of contraction.
  We claim that $b_0 \defeq p(a_0) : B$ is a center of contraction for $B$.
  Let $b : B$; we need a path $b = b_0$.
  But we have $\epsilon_b : p(s(b)) = b$ and $\contr_{s(b)} : s(b) = a_0$, so by composition
  \[ \opp{\epsilon_b} \ct \ap{p}{\contr_{s(b)}} : b = p(a_0) \jdeq b_0 . \qedhere\]
\end{proof}

Contractible types may not seem very interesting, since they are all equivalent to \unit.
One reason the notion is useful is that sometimes a collection of individually nontrivial data will collectively form a contractible type.
An important example is the space of paths with one free endpoint.
As we will see in \autoref{sec:identity-systems}, this fact essentially encapsulates the Paulin-Mohring induction principle for paths.

\begin{lem}\label{thm:contr-paths}
  For any $A$ and any $a:A$, the type $\sm{x:A} (a=x)$ is contractible.
\end{lem}
\begin{proof}
  We choose as center the point $(a,\refl a)$.
  Now suppose $(x,p):\sm{x:A}(a=x)$; we must show $(a,\refl a) = (x,p)$.
  By the characterization of paths in $\Sigma$-types, it suffices to exhibit $q:a=x$ such that $\trans{q}{\refl a} = p$.
  But we can take $q\defeq p$, in which case $\trans{q}{\refl a} = p$ follows from the characterization of transport in path types.
\end{proof}

When this happens, it can allows us to simplify a complicated construction up to equivalence, using the informal principle that contractible data can be freely ignored.
This principle consists of many lemmas, most of which we leave to the reader; the following is an example.

\begin{lem}\label{thm:omit-contr}
  Let $P:A\to\type$ be a type family.
  \begin{enumerate}
  \item If each $P(x)$ is contractible, then $\sm{x:A} P(x)$ is equivalent to $A$.\label{item:omitcontr1}
  \item If $A$ is contractible with center $a$, then $\sm{x:A} P(x)$ is equivalent to $P(a)$.\label{item:omitcontr2}
  \end{enumerate}
\end{lem}
\begin{proof}
  In the situation of~\ref{item:omitcontr1}, we show that $\proj1:\sm{x:A} P(x) \to A$ is an equivalence.
  For quasi-inverse we define $g(x)\defeq (x,c_x)$ where $c_x$ is the center of $P(x)$.
  The composite $\proj1 \circ g$ is obviously $\idfunc[A]$, whereas the opposite composite is homotopic to the identity by using the contractions of each $P(x)$.

  We leave the proof of~\ref{item:omitcontr2} to the reader (see \autoref{ex:omit-contr2}).
\end{proof}

Another reason contractible types are interesting is that they extend the ladder of $n$-types mentioned in \autoref{sec:basics-sets} downwards one more step.

\begin{lem}\label{thm:prop-minusonetype}
  A type $A$ is a mere proposition if and only if for all $x,y:A$, the type $\id[A]xy$ is contractible.
\end{lem}
\begin{proof}
  For ``if'', we simply observe that any contractible type is inhabited.
  For ``only if'', we observed in \autoref{subsec:hprops} that every mere proposition is a set, so that each type $\id[A]xy$ is a mere proposition.
  But it is also inhabited (since $A$ is a mere proposition), and hence by \autoref{thm:contr-unit}\ref{item:contr-inhabited-prop} it is contractible.
\end{proof}

Thus, contractible types may also be called \emph{$(-2)$-types}.
They are the bottom rung of the ladder of $n$-types, and will be the base case of the inductive definition of $n$-types in \autoref{cha:hlevels}.


%%%%%%%%%%%%%%%%%%%%%%%%%%%
\sectionNotes

The definition of identity types and the elimination rule $J$ are due to Martin-L\"of \cite{ml:itt}.
Our identity types are generally called \emph{intensional}, by contrast with the \emph{extensional} case which would have an additional ``reflection rule'' saying that if $p:x=y$, then in fact $x\jdeq y$.
This reflection rule implies that all the higher groupoid structure collapses, so for nontrivial homotopy we must use the intensional version. 
One may argue, however, that homotopy type theory is more ``extensional'' than traditional extensional type theory, because of the function extensionality and univalence rules.  

The proofs of symmetry (inversion) and transitivity (concatenation) for equalities are well-known in type theory.
The fact that these make each type into a 1-groupoid (up to homotopy) is also folklore, and was exploited in~\cite{hs:gpd-typethy} to give the first ``homotopy" style semantics for type theory.  

The actual homotopical interpretation, with identity types as path spaces, and dependent types as fibrations, is due to \cite{aw:hiit}, who used the formalism of Quillen model categories.  An interpretation in (strict) $\infty$-groupoids was also given in the thesis \cite{mw:thesis}.
For a construction of \emph{all} the higher operations and coherences of an $\infty$-groupoid in type theory, see~\cite{pll:wkom-type} and~\cite{bg:type-wkom}.

Operations such as $\transfib{P}{p}{-}$ and $\apfunc{f}$, and one good notion of equivalence, were first studied extensively in type theory by Voevodsky, using the proof assistant Coq.
Subsequent researchers have found many other equivalent definitions of equivalence, which we will compare in \autoref{cha:equivalences}.

The ``computational'' interpretation of identity types, transport, and so on described in \autoref{sec:computational} has been emphasized by~\cite{lh:canonicity}.
They also described a ``1-truncated'' type theory (see \autoref{cha:hlevels}) in which these rules really are computation steps (that is, definitional equalities which a computer can ``evaluate'').
The possibility of extending this to the full untruncated theory is a subject of current research.

The naive form of function extensionality which says that ``if two functions are pointwise equal, then they are equal'' is a common axiom in type theory.
Some stronger forms of function extensionality were considered in~\cite{garner:depprod}.
The version we have used, which identifies the identity types of function types up to equivalence, was first studied by Voevodsky, who also proved that it is implied by the naive version.

The univalence axiom is also due to Voevodsky.
It was originally motivated by semantic considerations; see~\cite{klv:ssetmodel}.

The simple conclusions in \crefrange{sec:compute-coprod}{sec:compute-nat} such as ``coproduct injections are injective and disjoint'' are well-known in type theory, and the construction of the function \encode is the usual way to prove them.
The more refined approach we have described, which characterizing the entire identity type of a positive type (up to equivalence), is a more recent development; see e.g.~\cite{ls:pi1s1}.

The type-theoretic axiom of choice~\eqref{eq:sigma-ump-map} was noticed in William Howard's original paper~\cite{howard:pat} on the propositions-as-types correspondence, and was studied further by Martin-L\"of with the introduction of his dependent type theory.  It is mentioned as a ``distributivity law" in Bourbaki's set theory \cite{Bourbaki}.

The fact that it is possible to define sets, mere propositions, and contractible types in type theory, with all higher homotopies automatically taken care of as in \autoref{sec:basics-sets,subsec:hprops,sec:contractibility}, was first observed by Voevodsky.
In fact, he defined the entire hierarchy of $n$-types by induction, as we will do in \autoref{cha:hlevels}.

\autoref{thm:not-dneg,thm:not-lem} rely in essense on a classical theorem of Hedberg, which we will prove in \autoref{cha:hlevels}.
The implication that the propositions-as-types form of LEM contradicts univalence was observed by Martin Escardo on the Agda mailing list.
The proof we have given of \autoref{thm:not-dneg} is due to Thierry Coquand.

The propositional truncation was introduced in the extensional type theory of
NuPRL in 1983 by Constable~\cite{Con85} as an
application of ``subset'' and ``quotient'' types.  What is here called the
``propositional truncation'' was called ``squashing'' in the NuPRL type theory~\cite{constable+86nuprl-book}.
Rules characterizing the propositional truncation directly, still in extensional type theory, were given in~\cite{ab:bracket-types}.
The intensional version in homotopy type theory was constructed by Voevodsky using an impredicative quantification, and later by Lumsdaine using higher inductive types (see \autoref{sec:hittruncations}).

The adverb ``purely'' as used to refer to untruncated logic is a reference to the use of monadic modalities to model effects in programming languages.
A computation is said to be \emph{pure} if its execution results in no side effects (such as printing a message to the screen, playing music, or sending data over the Internet).
There exist ``purely functional'' programming languages, such as Haskell, in which it is technically only possible to write pure functions: side effects are represented by applying ``monads'' to output types.
For instance, a function of type $\mathsf{Int}\to\mathsf{Int}$ is pure, while a function of type $\mathsf{Int}\to \mathsf{IO}(\mathsf{Int})$ may perform input and output along the way to computing its result.
Inside of type theory, the propositional truncation $\brck-$ is also a monad, as are the more general modalities one might consider replacing it with; thus it makes sense to call a type \emph{pure} when no such modality is present.


\sectionExercises

\begin{ex}\label{ex:basics:concat}
  Show that the three obvious proofs of \autoref{lem:concat} are pairwise equal.
\end{ex}

\begin{ex}
  Show that the three equalities of proofs constructed in the previous exercise form a commutative triangle.
  In other words, if the three definitions of concatenation are denoted by $(p \ct_1 q)$, $(p\ct_2 q)$, and $(p\ct_3 q)$, then the concatenated equality
  \[(p\ct_1 q) = (p\ct_2 q) = (p\ct_3 q)\]
  is equal to the equality $(p\ct_1 q) = (p\ct_3 q)$.
\end{ex}

\begin{ex}
  Give a fourth, different, proof of \autoref{lem:concat}, and prove that it is equal to the others.
\end{ex}

\begin{ex}
  Prove that the functions~\eqref{eq:ap-to-apd} and~\eqref{eq:apd-to-ap} are inverse equivalences.
  % and that they take $\apfunc f(p)$ to $\apdfunc f (p)$ and vice versa. (that was \autoref{thm:apd-const})
\end{ex}

\begin{ex}\label{ex:equiv-concat}
  Prove that if $p:x=y$, then the function $(p\ct -):(y=z) \to (x=z)$ is an equivalence.
\end{ex}

\begin{ex}\label{ex:ap-sigma}
  State and prove a generalization of \autoref{thm:ap-prod} from cartesian products to $\Sigma$-types.
\end{ex}

\begin{ex}
  State and prove an analogue of \autoref{thm:ap-prod} for coproducts.
\end{ex}

\begin{ex}\label{ex:coprod-ump}
  Prove that coproducts have the expected universal property:
  \[ \eqv{(A+B \to X)}{(A\to X)\times (B\to X)} \]
  Can you generalize this to an equivalence involving dependent functions?
\end{ex}

\begin{ex}
  Prove that if $\eqv A B$ and $A$ is a set, then so is $B$.
\end{ex}

\begin{ex}\label{ex:isset-coprod}
  Prove that if $A$ and $B$ are sets, then so is $A+B$.
\end{ex}

\begin{ex}\label{ex:isset-sigma}
  Prove that if $A$ is a set and $B:A\to \type$ is a type family such that $B(x)$ is a set for all $x:A$, then $\sm{x:A} B(x)$ is a set.
\end{ex}

\begin{ex}\label{ex:neg-ldn}
  Show that for any type $A$, we have $\neg\neg\neg A \to \neg A$.
\end{ex}

\begin{ex}\label{ex:eqvboolbool}
  Show that $\eqv{(\eqv\bool\bool)}{\bool}$.
\end{ex}

\begin{ex}\label{ex:prop-endocontr}
  Show that $A$ is a mere proposition if and only if $A\to A$ is contractible.
\end{ex}

\begin{ex}
  Show that if $A$ is a mere proposition, then so is $A+(\neg A)$.
  Thus, there is no need to insert a propositional truncation in~\eqref{eq:lem}.
\end{ex}

\begin{ex}\label{ex:disjoint-or}
  More generally, show that if $A$ and $B$ are mere propositions and $\neg(A\times B)$, then $A+B$ is also a mere proposition.
\end{ex}

% \begin{ex}\label{ex:hprop-iff-equiv}
%   Show that if $A$ and $B$ are mere propositions such that $A\to B$ and $B\to A$, then $\eqv A B$.
% \end{ex}

% \begin{ex}\label{ex:isprop-isprop}
%   Show that for any type $A$, the types $\isprop(A)$ and $\isset(A)$ are mere propositions.
% \end{ex}

% \begin{ex}\label{ex:prop-eqvtrunc}
%   Show that if $A$ is already a mere proposition, then $\eqv A{\brck{A}}$.
% \end{ex}

\begin{ex}\label{ex:brck-qinv}
  Assuming that some type $\isequiv(f)$ satisfies conditions~\ref{item:be1}--\ref{item:be3} of \autoref{sec:basics-equivalences}, show that the type $\brck{\qinv(f)}$ satisfies the same conditions and is equivalent to $\isequiv(f)$.
\end{ex}

\begin{ex}
  Show that if LEM holds, then the type $\prop \defeq \sm{A:\type} \isprop(A)$ is equivalent to \bool.
\end{ex}

\begin{ex}\label{ex:lem-impred}
  Show that if $\UU_{i+1}$ satisfies LEM, then the canonical inclusion $\prop_{\UU_i} \to \prop_{\UU_{i+1}}$ is an equivalence.
\end{ex}

\begin{ex}
  Show that it is not the case that for all $A:\type$ we have $\brck{A} \to A$.
  (However, there can be particular types for which $\brck{A}\to A$.
  \autoref{ex:brck-qinv} implies that $\qinv(f)$ is such.)
\end{ex}

\begin{ex}
  Show that if LEM holds, then for all $A:\type$ we have $\bbrck{(\brck A \to A)}$.
  (This property is a very simple form of the axiom of choice, which can fail in the absence of LEM; see~\cite{krausgeneralizations}.)
\end{ex}

\begin{ex}
  We showed in \autoref{thm:not-lem} that the following naive form of LEM is inconsistent with univalence:
  \[ \prd{A:\type} (A+(\neg A)) \]
  In the absence of univalence, this axiom is consistent.
  However, show that it implies the axiom of choice~\eqref{eq:ac}.
\end{ex}

\begin{ex}\label{ex:lem-brck}
  Show that assuming LEM, the double negation $\neg \neg A$ has the same universal property as the propositional truncation $\brck A$, and is therefore equivalent to it.
  Thus, under LEM, the propositional truncation can be defined rather than taken as a separate type former.
\end{ex}

\begin{ex}\label{ex:impred-brck}
  Show that if we assume impredicativity of mere propositions as in \autoref{subsec:prop-subsets}, then the type
  \[\prd{P:\prop} \big((A\to P)\to P\big)\]
  has the same universal property as $\brck A$.
  Thus, we can also define the propositional truncation in this case.
\end{ex}

\begin{ex}
  Assuming LEM, show that double negation commutes with universal quantification of mere propositions over sets.
  That is, show that if $X$ is a set and each $Y(x)$ is a mere proposition, then LEM implies
  \begin{equation}
    \eqv{\big(\prd{x:X} \neg\neg Y(x)\big)}{\big(\neg\neg \prd{x:X} Y(x)\big)}.\label{eq:dnshift}
  \end{equation}
  Observe that if we assume instead that each $Y(x)$ is a set, then~\eqref{eq:dnshift} becomes equivalent to the axiom of choice~\eqref{eq:epis-split}.
\end{ex}

\begin{ex}\label{ex:prop-trunc-ind}
  Show that the rules for the propositional truncation given in \autoref{subsec:prop-trunc} are sufficient to imply a dependent version of the induction principle: for any type family $B:\brck A \to \type$ such that each $B(x)$ is a mere proposition, if for every $a:A$ we have $B(\bproj a)$, then for every $x:\brck A$ we have $B(x)$.
\end{ex}

\begin{ex}\label{ex:lem-ldn}
  Show that the law of excluded middle~\eqref{eq:lem} and the law of double negation~\eqref{eq:ldn} are logically equivalent.
\end{ex}

\begin{ex}\label{ex:omit-contr2}
  Prove \autoref{thm:omit-contr}\ref{item:omitcontr2}: if $A$ is contractible with center $a$, then $\sm{x:A} P(x)$ is equivalent to $P(a)$.
\end{ex}

% Local Variables:
% TeX-master: "main"
% End:


\chapter{Logic}
\label{cha:logic}

Type theory, formal or informal, is a collection of rules for manipulating types and their elements.
But when writing mathematics informally in human language, we generally use familiar words, particularly connectives such as ``and'' and ``or'', and quantifiers such as ``for all'' and ``there exists''.
In contrast to set theory, type theory offers us more than one choice for how to regard these English phrases as operations on types.
This potential ambiguity needs to be resolved, by setting out local or global conventions, by introducing new annotations to informal mathematics, or both.
This requires some getting used to, but is offset by the fact that because type theory permits this finer analysis of logic, we can represent mathematics more faithfully, with fewer ``abuses of language'' than in set-theoretic foundations.
In this chapter we will explain the issues involved, and justify the choices we have made in this book.


\section{The failure of classical logic}
\label{sec:patnonclass}

Until now, we have been following the straightforward ``propositions as types'' philosophy described in \S\ref{sec:pat}, according to which English phrases such as ``there exists an $x:A$ such that $P(x)$'' are interpreted by corresponding types such as $\sm{x:A} P(x)$, with the proof of a statement being regarded as judging some specific term to inhabit that type.
However, we have also seen some ways in which the ``logic'' resulting from this reading seems unfamiliar to a classical mathematician.
For instance, in \autoref{thm:ttac} we saw that the statement
\begin{quote}
  ``If for all $x:X$ there exists an $a:A(x)$ such that $P(x,a)$, then there exists a function $g:\prd{x:A} A(x)$ such that for all $x:X$ we have $P(x,g(x))$,''
\end{quote}
which looks like the classical \emph{axiom of choice}, is always true under this reading.
On the other hand, we can now show that corresponding statements looking like the classical \emph{law of double negation} and \emph{law of excluded middle} are incompatible with the univalence axiom.

\begin{thm}\label{thm:not-dneg}
  It is not the case that for all $A:\UU$ we have $\neg(\neg A) \to A$.
\end{thm}
\begin{proof}
  Recall that $\neg A \jdeq (A\to\emptyt)$.
  We also read ``it is not the case that \dots'' as the operator $\neg$.
  Thus, in order to prove this statement, it suffices to assume given some $f:\prd{A:\UU} (\neg\neg A \to A)$ and construct an element of \emptyt.

  The idea of the following proof is to observe that $f$, like any function, is automatically ``continuous'' with respect to paths in its domain.
  By univalence, this implies that $f$ is \emph{natural} with respect to equivalences of types.
  From this, and a fixed-point-free autoequivalence, we will be able to extract a contradiction.

  Recall that $\bool\defeq \unit+\unit$, and let $e:\eqv\bool\bool$ be the equivalence defined by $e(\inl(\ttt))\defeq\inr(\ttt)$ and $e(\inr(\ttt))\defeq\inl(\ttt)$, as in \autoref{thm:type-is-not-a-set}.
  Let $p:\bool=\bool$ be the path corresponding to $e$ by univalence, i.e.\ $p\defeq \ua(e)$.
  Then we have $f(\bool) : \neg\neg\bool \to\bool$ and
  \[\apd f p : \transfib{A\mapsto (\neg\neg A \to A)}{p}{f(\bool)} = f(\bool).\]
  Hence, for any $u:\neg\neg\bool$, we have
  \[\happly(\apd f p,u) : \transfib{A\mapsto (\neg\neg A \to A)}{p}{f(\bool)}(x) = f(\bool)(u).\]

  Now by~\eqref{eq:transport-arrow}, transporting $f(\bool):\neg\neg\bool\to\bool$ along $p$ in the type family ${A\mapsto (\neg\neg A \to A)}$ is equal to the function which transports its argument along $\opp p$ in the type family $A\mapsto \neg\neg A$, applies $f(\bool)$, then transports the result along $p$ in the type family $A\mapsto A$:
  \[ \transfib{A\mapsto (\neg\neg A \to A)}{p}{f(\bool)}(u) =
  \transfib{A\mapsto A}{p}{f(\bool) (\transfib{A\mapsto \neg\neg A}{\opp{p}}{u})}
  \]
  However, any two points $u,v:\neg\neg\bool$ are equal by function extensionality, since for any $x:\neg\bool$ we have $u(x):\emptyt$ and thus we can derive any conclusion, in particular $u(x)=v(x)$.
  Thus, we have $\transfib{A\mapsto \neg\neg A}{\opp{p}}{u} = u$, and so from $\happly(\apd f p,u)$ we obtain an equality
  \[ \transfib{A\mapsto A}{p}{f(\bool)(u)} = f(\bool)(u).\]
  Finally, as discussed in \S\ref{sec:compute-universe}, transporting in the type family $A\mapsto A$ along the path $p\jdeq \ua(e)$ is equivalent to applying the equivalence $e$; thus we have
  \begin{equation}
    e(f(\bool)(u)) = f(\bool)(u).\label{eq:fpaut}
  \end{equation}

  However, we can also prove that
  \begin{equation}
    \prd{x:\bool} \neg(e(x)=x).\label{eq:fpfaut}
  \end{equation}
  This follows from a case analysis on $x$: both cases are immediate from the definition of $e$ and the injectivity of $\inl$ and $\inr$ which we proved in \S\ref{sec:compute-coprod}.
  Thus, applying~\eqref{eq:fpfaut} to $f(\bool)(u)$ and~\eqref{eq:fpaut}, we obtain an element of $\emptyt$.
\end{proof}

\begin{cor}\label{thm:not-lem}
  It is not the case that for all $A:\UU$ we have $A+(\neg A)$.
\end{cor}
\begin{proof}
  Suppose we had $g:\prd{A:\UU} (A+(\neg A))$.
  We will show that then $\prd{A:\UU} (\neg\neg A \to A)$, so that we can apply \autoref{thm:not-dneg}.
  Thus, suppose $A:\UU$ and $u:\neg\neg A$; we want to construct an element of $A$.

  Now $g(A):A+(\neg A)$, so by case analysis, we may assume either $g(A)\jdeq \inl(a)$ for some $a:A$, or $g(A)\jdeq \inr(w)$ for some $w:\neg A$.
  In the first case, we have $a:A$, while in the second case we have $u(w):\emptyt$ and so we can obtain anything we wish (such as $A$).
  Thus, in both cases we have an element of $A$, as desired.
\end{proof}

In conclusion, although the propositions-as-types logic has many good properties (such as simplicity, constructivity, and computability), it is an uncomfortable place in which to try to do classical mathematics, especially from our homotopical point of view.


\section{Mere propositions}
\label{sec:hprops}

The good and bad things about propositions-as-types logic have a common cause: when types are viewed as propositions, they can contain more information than mere truth or falsity, and all ``logical'' constructions on them must respect this additional information.
This suggests that we could obtain a more classical logic by restricting attention to types that do \emph{not} contain any more information than truth or falsity.

Such a type will be ``true'' if it is inhabited and ``false'' if it is not inhabited.
What we want to avoid are types for which giving an element of them gives more information than simply knowing that the type is inhabited.
For instance, if we are given an element of \bool, then we receive more information than the mere fact that \bool contains some element.
Indeed, we receive exactly \emph{one bit} more information: we know \emph{which} element of \bool we were given.
By contrast, if we are given an element of \unit, then we receive no more information than the mere fact that \unit contains an element, since any two elements of \unit are equal to each other.
This suggests the following definition.

\begin{defn}
  A type $P$ is a \textbf{mere proposition} if for all $x,y:P$ we have $x=y$.
\end{defn}

Note that since we are still doing mathematics \emph{in} type theory, this is a definition \emph{in} type theory, which means it is a type --- or, rather, a type family.
Specifically, for any $P:\type$, the type $\isprop(P)$ is defined to be
\[ \isprop(P) \defeq \prd{x,y:P} (x=y). \]
Thus, to assert that ``$P$ is a mere proposition'' means to exhibit an inhabitant of $\isprop(P)$, which is a dependent function connecting any two elements of $P$ by a path.
The continuity/naturality of this function implies that not only are any two elements of $P$ equal, but $P$ contains no higher homotopy either.

\begin{lem}
  If $P$ is a mere proposition and $x_0:P$, then $\eqv P \unit$.
\end{lem}
\begin{proof}
  Define $f:P\to\unit$ by $f(x)\defeq \ttt$, and $g:\unit\to P$ by $g(u)\defeq x_0$.
  Then for any $u:\unit$ we have $\eqv{(f(g(u))=u)}{\unit}$, hence $f(g(u))=u$, while for any $x:P$ we have $g(f(x))=x$ since $P$ is a mere proposition.
\end{proof}

In homotopy theory, a space that is homotopy equivalent to \unit is said to be \emph{contractible}.
Thus, any mere proposition which is inhabited is contractible.
On the other hand, the uninhabited type \emptyt is also (vacuously) a mere proposition.
In classical mathematics, at least, these are the only two possibilities.

Mere propositions are also called \emph{subterminal objects} (if thinking categorically), \emph{subsingletons} (if thinking set-theoretically), or \emph{h-propositions}.
In Chapter~\ref{cha:hlevels} we will learn to also call them \emph{$(-1)$-truncated types}.
The adjective ``mere'' emphasizes that although any type may be regarded as a proposition (which we prove by giving an inhabitant of it), a type that is a mere proposition cannot usefully be regarded as any \emph{more} than a proposition: there is no additional information contained in a witness of its truth.

Note that a type $A$ is a set if and only if for all $x,y:A$, the identity type $\id[A]xy$ is a mere proposition.
On the other hand, by copying (and simplifying) the proof of \autoref{thm:isset-is1type}, we see that every mere proposition is a set.
We have seen one other example so far: condition~\ref{item:be3} in \S\ref{sec:basics-equivalences} asserts that for any function $f$, the type $\isequiv (f)$ should be a mere proposition.


We can now give the proper formulation of the \emph{law of excluded middle} in homotopy type theory:
\begin{equation}
  \label{eq:lem}
  \mathsf{LEM}\;\defeq\;
  \prd{A:\UU} \Big(\isprop(A) \to (A + \neg A)\Big).
\end{equation}
Similarly, the \emph{law of double negation} is
\begin{equation}
  \label{eq:ldn}
  \mathsf{DN}\;\defeq\;
  \prd{A:\UU} \Big(\isprop(A) \to (\neg\neg A \to A)\Big).
\end{equation}
These formulations avoid the paradoxes of \autoref{thm:not-dneg} and \autoref{thm:not-lem}, since \bool is not a mere proposition.
Although they are not consequences of the basic type theory described in Chapter~\ref{cha:typetheory}, they may be consistently assumed as axioms.
For instance, we will assume them in \S\ref{sec:wellorderings}.

However, it can be surprising how far we can get without using such axioms.
Quite often, a simple reformulation of a definition or theorem enables us to avoid invoking excluded middle or double negation.
This is even more pronounced in \emph{homotopy} type theory.
For instance, none of the homotopy theory we will develop in Chapter~\ref{cha:homotopy} requires LEM or DN, despite the fact that classical homotopy theory (formulated using topological spaces or simplicial sets) makes heavy use of them (as well as the axiom of choice).


\section{Subsets}
\label{sec:prop-subsets}

As another example of the usefulness of mere propositions, we discuss subsets (and more generally subtypes).
Suppose $P:A\to\type$ is a type family, with each type $P(x)$ regarded as a proposition.
Then $P$ itself is a \emph{predicate} on $A$, or a \emph{property} of elements of $A$.

In set theory, whenever we have a predicate on $P$ on a set $A$, we may form the subset $\setof{x\in A | P(x)}$.
In type theory, the obvious analogue is the $\Sigma$-type $\sm{x:A} P(x)$.
An inhabitant of $\sm{x:A} P(x)$ is, of course, a pair $(x,p)$ where $x:A$ and $p$ is a proof of $P(x)$.
However, for general $P$, an element $a:A$ might give rise to more than one distinct element of $\sm{x:A} P(x)$, if the proposition $P(a)$ has more than one distinct proof.
This is counter to the usual intuition of a \emph{subset}.
But if $P$ is a \emph{mere} proposition, then this cannot happen.

\begin{lem}
  Suppose $P:A\to\type$ is a type family such that $P(x)$ is a mere proposition for all $x:A$.
  If $u,v:\sm{x:A} P(x)$ are such that $\proj1(u) = \proj1(v)$, then $u=v$.
\end{lem}
\begin{proof}
  Suppose $p:\proj1(u) = \proj1(v)$.
  By \autoref{thm:path-sigma}, to show $u=v$ it suffices to show $\trans{p}{\proj2(u)} = \proj2(v)$.
  But $\trans{p}{\proj2(u)}$ and $\proj2(v)$ are both elements of $P(\proj1(v))$, which is a mere proposition; hence they are equal.
\end{proof}

For instance, recall that in \S\ref{sec:basics-equivalences} we defined
\[(\eqv A B) \;\defeq\; \sm{f:A\to B} \isequiv (f),\]
and that each type $\isequiv (f)$ is a mere proposition.
It follows that if two equivalences have equal underlying functions, then they are equal as equivalences.

Henceforth, if $P:A\to \type$ is a family of mere propositions, we will allow ourselves to write $\setof{x:A | P(x)}$ as an alternative notation for $\sm{x:A} P(x)$.


\section{The logic of mere propositions}
\label{sec:logic-hprop}

We mentioned in \S\ref{sec:types-vs-sets} that in contrast to type theory, which has only one basic notion (types), set-theoretic foundations have two basic notions: sets and propositions.
Thus, a classical mathematician is accustomed to manipulating these two kinds of objects separately.

It is possible to recover a similar dichotomy in type theory, with the role of the set-theoretic propositions being played by the types (and type families) that are \emph{mere} propositions.
In many cases, the logical connectives and quantifiers can be represented in this logic by simply restricting the corresponding type-former to the mere propositions.
Of course, this requires knowing that the type-former in question preserves mere propositions.

\begin{eg}
  If $A$ and $B$ are mere propositions, so is $A\times B$.
  This is easy to show using the characterization of paths in products, just like \autoref{thm:isset-prod} but simpler.
  Thus, the connective ``and'' preserves mere propositions.
\end{eg}

\begin{eg}\label{thm:isprop-forall}
  If $A$ is any type and $B:A\to \type$ is such that for all $x:A$, the type $B(x)$ is a mere proposition, then $\prd{x:A} B(x)$ is a mere proposition.
  The proof is just like \autoref{thm:isset-forall} but simpler: given $f,g:\prd{x:A} B(x)$, for any $x:A$ we have $f(x)=g(x)$ since $B(x)$ is a mere proposition.
  But then by function extensionality, we have $f=g$.

  In particular, if $B$ is a mere proposition, then so is $A\to B$ regardless of what $A$ is.
  In even more particular, since \emptyt is a mere proposition, so is $\neg A \jdeq (A\to\emptyt)$.
  Thus, the connectives ``implies'' and ``not'' preserve mere propositions, as does the quantifier ``for all''.
\end{eg}

On the other hand, some type formers do not preserve mere propositions.
Even if $A$ and $B$ are mere propositions, $A+B$ will not in general be.
For instance, \unit is a mere proposition, but $\bool\defeq\unit+\unit$ is not.
Logically speaking, $A+B$ is a ``purely constructive'' sort of ``or'': a witness of it contains the additional information of \emph{which} disjunct is true.
Sometimes this is very useful, but if we want a more classical sort of ``or'' that preserves mere propositions, we need a way to ``truncate'' this type into a mere proposition by forgetting this additional information.

The same issue arises with the $\Sigma$-type $\sm{x:A} P(x)$.
This is a purely constructive interpretation of ``there exists an $x:A$ such that $P(x)$'' which remembers the witness $x$, and hence is not generally a mere proposition even if each type $P(x)$ is.
(Recall that we observed in \S\ref{sec:prop-subsets} that $\sm{x:A} P(x)$ can also be regarded as ``the subset of those $x:A$ such that $P(x)$''.
The tension between these two interpretations is exactly the point.)


\section{Propositional truncation}
\label{sec:prop-trunc}

The \emph{propositional truncation}, also called the \emph{$(-1)$-truncation}, \emph{bracket type}, or \emph{squash type}, is an additional type former which ``truncates'' or ``squashes'' a type down to a mere proposition, forgetting all information contained in inhabitants of that type other than their existence.

More precisely, for any type $A$, there is a type $\brck{A}$.
It has two constructors:
\begin{itemize}
\item For any $a:A$ we have $\bproj a : \brck A$.
  Thus, if $A$ is inhabited, so is $\brck A$.
\item For any $x,y:\brck A$, we have $x=y$.
  In other words, $\brck A$ is a mere proposition; usually we leave the witness of this fact nameless.
\end{itemize}
The induction principle of $\brck A$ says that:
\begin{itemize}
\item If $B$ is a mere proposition and we have $f:A\to B$, then there is an induced $g:\brck A \to B$ such that $g(\bproj a) \jdeq f(a)$ for all $a:A$.
\end{itemize}
Thus, $\brck A$, as a mere proposition, contains no more information than the inhabitedness of $A$, since any mere proposition which follows from the inhabitedness of $A$ already follows from $\brck A$.

With the propositional truncation, we can extend the ``logic of mere propositions'' to cover disjunction and the existential quantifier.
Specifically, $\brck{A+B}$ is a mere propositional version of ``$A$ or $B$'', which does not ``remember'' the information of which disjunct is true.

The induction principle of truncation implies that we can still do a case analysis on $\brck{A+B}$ \emph{when attempting to prove a mere proposition}.
That is, suppose we have an assumption $u:\brck{A+B}$ and we are trying to prove a mere proposition $Q$.
In other words, we are trying to define an element of $\brck{A+B} \to Q$.
Since $Q$ is a mere propositon, by the induction principle for propositional truncation, it suffices to construct a function $A+B\to Q$.
But now we can use case analysis on $A+B$.

Similarly, for a type family $P:A\to\type$, we can consider $\brck{\sm{x:A} P(x)}$, which is a mere propositional version of ``there exists an $x:A$ such that $P(x)$''.
As for disjunction, by combining the induction principles of truncation and $\Sigma$-types, if we have an assumption of type $\brck{\sm{x:A} P(x)}$, we may introduce new assumptions $x:A$ and $y:P(x)$ \emph{when attempting to prove a mere proposition}.
In other words, if we know that there exists some $x:A$ such that $P(x)$, but we don't have a particular such $x$ in hand, then we are free to make use of such an $x$ as long as we aren't trying to construct anything which might depend on the particular value of $x$.
Requiring the codomain to be a mere proposition expresses this independence of the result on the witness, since all possible inhabitants of such a type must be equal.

We can now properly formulate the \emph{axiom of choice} in homotopy type theory.
Assume a type $X$ and type families $A:X\to\type$ and $P:\prd{x:X} A(x)\to\type$, and moreover that
\begin{itemize}
\item $X$ is a set,
\item $A(x)$ is a set for all $x:X$, and
\item $P(x,a)$ is a mere proposition for all $x:X$ and $a:A(x)$.
\end{itemize}
The axiom of choice asserts that under these assumptions,
\begin{multline}\label{eq:ac}
  \left(\prd{x:X} \Brck{\sm{a:A(x)} P(x,a)}\right)
  \to\\
  \Brck{\sm{g:\prd{x:X} A(x)} \prd{x:X} P(x,g(x))}
\end{multline}
Note that the propositional truncation appears twice.
The truncation in the domain means we assume that for every $x$ there exists some $a:A(x)$ such that $P(x,a)$, but that these values are not chosen or specified in any known way.
The truncation in the codomain means we conclude that there exists some function $g$, but this function is not determined or specified in any known way.

As with LEM and DN,~\eqref{eq:ac} is not a consequence of our basic type theory, but it may consistently be assumed as an axiom.
Note the restriction to types that are sets; because of the continuity/functoriality of all functions, it is unreasonable to assert such a statement without some such restriction.

\begin{rmk}
  A word of caution about a common pitfall: although dependent function types preserve mere propositions (\autoref{thm:isprop-forall}), they do not commute with truncation: $\brck{\prd{x:A} P(x)}$ is not generally equivalent to $\prd{x:A} \brck{P(x)}$.
  Indeed, while the version of the ``axiom of choice'' which is provable in type theory is really just a statement about how $\Sigma$'s and $\Pi$'s commute, the proper axiom of choice~\eqref{eq:ac} is arguably a statement about how $\Pi$'s commute with truncation.
\end{rmk}


\section{When are propositions truncated?}
\label{sec:when-trunc}

At first glance, it may seem that the truncated versions of $+$ and $\Sigma$ are closer to the informal mathematical meaning of ``or'' and ``there exists''.
Certainly, they are closer to the \emph{precise} meaning of ``or'' and ``there exists'' in the first-order logic which underlies formal set theory, since the latter makes no attempt to remember any witnesses to the truth of propositions.
However, frequently the practice of \emph{informal} mathematics is more accurately described by the untruncated forms.

For example, consider a statement like ``every prime number is either $2$ or odd.''
The working mathematician feels no compunction about using this fact not only to prove \emph{theorems} about prime numbers, but also to perform \emph{constructions} on prime numbers, perhaps doing one thing in the case of $2$ and another in the case of an odd prime.
Since the end result of the construction is not merely the truth of some statement, but a piece of data which may depend on the parity of the prime number, from a type-theoretic perspective such a construction is naturally phrased using the induction principle for the coproduct type ``$(p=2)+(p\text{ is odd})$'', not its propositional truncation.

Admittedly, this is not an ideal example, since ``$p=2$'' and ``$p$ is odd'' are mutually exclusive, so that $(p=2)+(p\text{ is odd})$ is in fact already a mere proposition and hence equivalent to its truncation (see Exercises~\ref{ex:disjoint-or} and~\ref{ex:prop-eqvtrunc}).
More compelling examples come from the existential quantifier.
It is not uncommon to prove a theorem of the form ``there exists an $x$ such that \dots'' and then refer later on to ``the $x$ constructed in Theorem Y'' (note the definite article).
Moreover, when deriving further properties of this $x$, one may use phrases such as ``by the construction of $x$ in the proof of Theorem Y''.

A very common example is ``$A$ is isomorphic to $B$'', which strictly speaking means only that there exists \emph{some} isomorphism between $A$ and $B$.
But almost invariably, when proving such a statement one exhibits a specific isomorphism, or proves that some previously known map is an isomorphism, and it often matters later on what particular isomorphism was given.

Honest, set-theoretically trained mathematicians generally feel a twinge of guilt at such ``abuses of language''.
We may attempt to apologize for them, expunge them from final drafts, or weasel out of them with vague words like ``canonical''.
The problem is exacerbated by the fact that in formalized set theory, there is technically no way to ``construct'' objects at all --- we can only prove that an object with certain properties exists.
Untruncated logic in type theory thus validates common practices of informal mathematics which set theory disparages.
(This is similar to how the univalence axiom validates the common, but formally unjustified, practice of identifying isomorphic objects.)

On the other hand, sometimes truncated logic is essential.
We have seen this in the statements of LEM and AC; some other examples will appear later on in the book.
Thus, we are faced with the problem: when writing informal type theory, what should we mean by the words ``or'' and ``there exists'' (along with common synonyms such as ``there is'' and ``we have'')?

A universal consensus may not be possible.
Perhaps depending on the sort of mathematics being done, one convention or the other may be more useful --- or, perhaps, the choice of convention may be irrelevant.
In this case, a remark at the beginning of a mathematical paper may suffice to inform the reader of the linguistic conventions in use therein.
However, even after one overall convention is chosen, the other sort of logic will usually arise at least occasionally, so we need a way to refer to it.
More generally, one may consider replacing the propositional truncation with another operation on types that behaves similarly, such as the double negation $A\mapsto \neg\neg A$, or the $n$-truncations to be considered in Chapter~\ref{cha:hlevels}.
We propose the use of \emph{adverbs} to denote the application of such ``modalities''.

For instance, if untruncated logic is the overall convention, we may use the adverb \textbf{merely} to denote propositional truncation.
Thus the phrase
\begin{center}
  ``there merely exists an $x:A$ such that $P(x)$''
\end{center}
indicates the type $\brck{\sm{x:A} P(x)}$.
On the other hand, if truncated logic is the current default convention, we may use an adverb such as \textbf{purely} or \textbf{constructively} to indicate its absence, so that
\begin{center}
``there purely exists an $x:A$ such that $P(x)$''
\end{center}
would denote the type $\sm{x:A} P(x)$.
We may also use ``purely'' just to emphasize the absence of truncation, even when that is the default convention.

In this book we will continue using untruncated logic as the default convention, for a number of reasons.
\begin{enumerate}[label=(\arabic*)]
\item We want to encourage the newcomer to experiment with it, rather than sticking to truncated logic simply because it is more familiar.
\item Using truncated logic as the default in type theory suffers from the same sort of ``abuse of language'' problems as set-theoretic foundations, which untruncated logic avoids.
  For instance, our definition of ``$\eqv A B$'' as the type of equivalences between $A$ and $B$, rather than its propositional truncation, means that to prove a theorem of the form ``$\eqv A B$'' is literally to construct a particular such equivalence.
  This specific equivalence can then be referred to later on.
\item We want to emphasize that the notion of ``mere proposition'' is not a fundamental part of type theory.
  As we will see in Chapter~\ref{cha:hlevels}, mere propositions are just the second rung on an infinite ladder, and there are also many other modalities not lying on this ladder at all.
\item Many statements that classically are mere propositions are no longer so in homotopy type theory.
  Of course, foremost among these is equality.
\item On the other hand, one of the most interesting observations of homotopy type theory is that a surprising number of types are \emph{automatically} mere propositions, or can be slightly modified to become so, without the need for any truncation.
  (See \autoref{ex:isprop-isprop} and Chapters~\ref{cha:equivalences}, \ref{cha:hlevels}, \ref{cha:category-theory}, and~\ref{cha:set-math}.)
  Thus, although these types contain no data beyond a truth value, we can nevertheless use them to construct untruncated objects, since there is no need to use the induction principle of propositional truncation.
  This useful fact is more clumsy to express if propositional truncation is applied to all statements by default.
\item Finally, truncations are not very useful for most of the mathematics we will be doing in this book, so it is simpler to notate them explicitly when they occur.
\end{enumerate}


\section*{Notes}

\autoref{thm:not-dneg} and \autoref{thm:not-lem} can be traced back to a classical theorem of Hedberg, which we will prove in Chapter~\ref{cha:hlevels}.
The proof we have given of \autoref{thm:not-dneg} is due to Thierry Coquand.

Mere propositions were first defined in type theory by Voevodsky.
His original definition was slightly more complicated than ours, but fits into the more general framework of Chapter~\ref{cha:hlevels}.

The propositional truncation was introduced, in extensional type theory, by~\cite{ab:bracket-types}.
The intensional version was constructed by Voevodsky using an impredicative quantification, and later by Lumsdaine using higher inductive types (see Chapter~\ref{cha:hits}).


\section*{Exercises}
\label{sec:exercises}

\begin{ex}
  Show that if $A$ is a mere proposition, then so is $A+(\neg A)$.
  Thus, there is no need to insert a propositional truncation in~\eqref{eq:lem}.
\end{ex}

\begin{ex}\label{ex:disjoint-or}
  More generally, show that if $A$ and $B$ are mere propositions and $\neg(A\times B)$, then $A+B$ is also a mere proposition.
\end{ex}

\begin{ex}
  Show that if $A$ and $B$ are mere propositions such that $A\to B$ and $B\to A$, then $\eqv A B$.
\end{ex}

\begin{ex}\label{ex:isprop-isprop}
  Show that for any type $A$, the types $\isprop(A)$ and $\isset(A)$ are mere propositions.
\end{ex}

\begin{ex}\label{ex:prop-eqvtrunc}
  Show that if $A$ is already a mere proposition, then $\eqv A{\brck{A}}$.
\end{ex}

\begin{ex}\label{ex:brck-qinv}
  Assuming that some type $\isequiv(f)$ satisfies conditions~\ref{item:be1}--\ref{item:be3} of \S\ref{sec:basics-equivalences}, show that the type $\brck{\mathsf{qinv}(f)}$ satisfies the same conditions and is equivalent to $\isequiv(f)$.
\end{ex}

\begin{ex}
  Show that it is not the case that for all $A:\type$ we have $\brck{A} \to A$.
  (However, there can be particular types for which $\brck{A}\to A$.
  \autoref{ex:brck-qinv} implies that $\mathsf{qinv}(f)$ is such.)
\end{ex}

\begin{ex}
  Show that the rules for the propositional truncation given in \S\ref{sec:prop-trunc} are sufficient to imply a dependent version of the induction principle: for any type family $B:\brck A \to \type$ such that each $B(x)$ is a mere proposition, if for every $a:A$ we have $B(\bproj a)$, then for every $x:\brck A$ we have $B(x)$.
\end{ex}

% Local Variables:
% TeX-master: "main"
% End:


\chapter{Equivalences}
\label{cha:equivalences}

We now study in more detail the notion of \emph{equivalence of types} that was introduced briefly in \autoref{sec:basics-equivalences}.
Specifically, we will give several different ways to define a type $\isequiv(f)$ having the properties mentioned there.
Recall that we wanted $\isequiv(f)$ to have the following properties, which we restate here:
\begin{enumerate}
\item $\qinv(f) \to \isequiv (f)$.\label{item:beb1}
\item $\isequiv (f) \to \qinv(f)$.\label{item:beb2}
\item $\isequiv(f)$ is a mere proposition.\label{item:beb3}
\end{enumerate}
Here $\qinv(f)$ denotes the type of quasi-inverses to $f$:
\begin{equation*}
  \sm{g:B\to A} \big((f \circ g \htpy \idfunc[B]) \times (g\circ f \htpy \idfunc[A])\big).
\end{equation*}
By function extensionality, it follows that $\qinv(f)$ is equivalent to the type
\begin{equation*}
  \sm{g:B\to A} \big((f \circ g = \idfunc[B]) \times (g\circ f = \idfunc[A])\big).
\end{equation*}
We will define three different types having properties~\ref{item:beb1}--\ref{item:beb3}, which we call
\begin{itemize}
\item half adjoint equivalences,
\item bi-invertible maps,
  \index{function!bi-invertible}
  and
\item contractible functions.
\end{itemize}
We will also show that all these types are equivalent.
These names are intentionally somewhat cumbersome, because after we know that they are all equivalent and have properties~\ref{item:beb1}--\ref{item:beb3}, we will revert to saying simply ``equivalence'' without needing to specify which particular definition we choose.
But for purposes of the comparisons in this chapter, we need different names for each definition.

Before we examine the different notions of equivalence, however, we give a little more explanation of why a different concept than quasi-invertibility is needed.

\section{Quasi-inverses}
\label{sec:quasi-inverses}

\index{quasi-inverse|(}%
We have said that $\qinv(f)$ is unsatisfactory because it is not a mere proposition, whereas we would rather that a given function can ``be an equivalence'' in at most one way.
However, we have given no evidence that $\qinv(f)$ is not a mere proposition.
In this section we exhibit a specific counterexample.

\begin{lem}\label{lem:qinv-autohtpy}
  If $f:A\to B$ is such that $\qinv (f)$ is inhabited, then
  \[\eqv{\qinv(f)}{\Parens{\prd{x:A}(x=x)}}.\]
\end{lem}
\begin{proof}
  By assumption, $f$ is an equivalence; that is, we have $e:\isequiv(f)$ and so $(f,e):\eqv A B$.
  By univalence, $\idtoeqv:(A=B) \to (\eqv A B)$ is an equivalence, so we may assume that $(f,e)$ is of the form $\idtoeqv(p)$ for some $p:A=B$.
  Then by path induction, we may assume $p$ is $\refl{A}$, in which case $\idtoeqv(p)$ is $\idfunc[A]$.
  Thus we are reduced to proving $\eqv{\qinv(\idfunc[A])}{(\prd{x:A}(x=x))}$.
  Now by definition we have
  \[ \qinv(\idfunc[A]) \jdeq
  \sm{g:A\to A} \big((g \htpy \idfunc[A]) \times (g \htpy \idfunc[A])\big).
  \]
  By function extensionality, this is equivalent to
  \[ \sm{g:A\to A} \big((g = \idfunc[A]) \times (g = \idfunc[A])\big).
  \]
  And by \autoref{ex:sigma-assoc}, this is equivalent to
  \[ \sm{h:\sm{g:A\to A} (g = \idfunc[A])} (\proj1(h) = \idfunc[A])
  \]
  However, by \autoref{thm:contr-paths}, $\sm{g:A\to A} (g = \idfunc[A])$ is contractible with center $\idfunc[A]$; therefore by \autoref{thm:omit-contr} this type is equivalent to $\idfunc[A] = \idfunc[A]$.
  And by function extensionality, $\idfunc[A] = \idfunc[A]$ is equivalent to $\prd{x:A} x=x$.
\end{proof}

\noindent
We remark that \autoref{ex:qinv-autohtpy-no-univalence} asks for a proof of the above lemma which avoids univalence.

Thus, what we need is some $A$ which admits a nontrivial element of $\prd{x:A}(x=x)$.
Thinking of $A$ as a higher groupoid, an inhabitant of $\prd{x:A}(x=x)$ is a natural transformation\index{natural!transformation} from the identity functor of $A$ to itself.
Such transformations are said to form the \define{center of a category},
\index{center!of a category}%
\index{category!center of}%
since the naturality axiom requires that they commute with all morphisms.
Classically, if $A$ is simply a group regarded as a one-object groupoid, then this yields precisely its center in the usual group-theoretic sense.
This provides some motivation for the following.

\begin{lem}\label{lem:autohtpy}
  Suppose we have a type $A$ with $a:A$ and $q:a=a$ such that
  \begin{enumerate}
  \item The type $a=a$ is a set.\label{item:autohtpy1}
  \item For all $x:A$ we have $\brck{a=x}$.\label{item:autohtpy2}
  \item For all $p:a=a$ we have $p\ct q = q \ct p$.\label{item:autohtpy3}
  \end{enumerate}
  Then there exists $f:\prd{x:A} (x=x)$ with $f(a)=q$.
\end{lem}
\begin{proof}
  Let $g:\prd{x:A} \brck{a=x}$ be as given by~\ref{item:autohtpy2}.  First we
  observe that each type $\id[A]xy$ is a set.  For since being a set is a mere
  proposition, we may apply the induction principle of propositional truncation, and assume that $g(x)=\bproj
  p$ and $g(y)=\bproj q$ for $p:a=x$ and $q:a=y$.  In this case, composing with
  $p$ and $\opp{q}$ yields an equivalence $\eqv{(x=y)}{(a=a)}$.  But $(a=a)$ is
  a set by~\ref{item:autohtpy1}, so $(x=y)$ is also a set.

  Now, we would like to define $f$ by assigning to each $x$ the path $\opp{g(x)}
  \ct q \ct g(x)$, but this does not work because $g(x)$ does not inhabit $a=x$
  but rather $\brck{a=x}$, and the type $(x=x)$ may not be a mere proposition,
  so we cannot use induction on propositional truncation.  Instead we can apply
  the technique mentioned in \autoref{sec:unique-choice}: we characterize
  uniquely the object we wish to construct.  Let us define, for each $x:A$, the
  type
  \[ B(x) \defeq \sm{r:x=x} \prd{s:a=x} (r = \opp s \ct q\ct s).\]
  We claim that $B(x)$ is a mere proposition for each $x:A$.
  Since this claim is itself a mere proposition, we may again apply induction on
  truncation and assume that $g(x) = \bproj p$ for some $p:a=x$.
  Now suppose given $(r,h)$ and $(r',h')$ in $B(x)$; then we have
  \[ h(p) \ct \opp{h'(p)} : r = r'. \]
  It remains to show that $h$ is identified with $h'$ when transported along this equality, which by transport in identity types and function types (\autoref{sec:compute-paths,sec:compute-pi}), reduces to showing
  \[ h(s) = h(p) \ct \opp{h'(p)} \ct h'(s) \]
  for any $s:a=x$.
  But each side of this is an equality between elements of $(x=x)$, so it follows from our above observation that $(x=x)$ is a set.

  Thus, each $B(x)$ is a mere proposition; we claim that $\prd{x:A} B(x)$.
  Given $x:A$, we may now invoke the induction principle of propositional truncation to assume that $g(x) = \bproj p$ for $p:a=x$.
  We define $r \defeq \opp p \ct q \ct p$; to inhabit $B(x)$ it remains to show that for any $s:a=x$ we have
  $r = \opp s \ct q \ct s$.
  Manipulating paths, this reduces to showing that $q\ct (p\ct \opp s) = (p\ct \opp s) \ct q$.
  But this is just an instance of~\ref{item:autohtpy3}.
\end{proof}

\begin{thm}\label{thm:qinv-notprop}
  There exist types $A$ and $B$ and a function $f:A\to B$ such that $\qinv(f)$ is not a mere proposition.
\end{thm}
\begin{proof}
  It suffices to exhibit a type $X$ such that $\prd{x:X} (x=x)$ is not a mere proposition.
  Define $X\defeq \sm{A:\type} \brck{\bool=A}$, as in the proof of \autoref{thm:no-higher-ac}.
  It will suffice to exhibit an $f:\prd{x:X} (x=x)$ which is unequal to $\lam{x} \refl{x}$.

  Let $a \defeq (\bool,\bproj{\refl{\bool}}) : X$, and let $q:a=a$ be the path corresponding to the nonidentity equivalence $e:\eqv\bool\bool$ defined by $e(\bfalse)\defeq\btrue$ and $e(\btrue)\defeq\bfalse$.
  We would like to apply \autoref{lem:autohtpy} to build an $f$.
  By definition of $X$, equalities in subset types (\autoref{subsec:prop-subsets}), and univalence, we have $\eqv{(a=a)}{(\eqv{\bool}{\bool})}$, which is a set, so~\ref{item:autohtpy1} holds.
  Similarly, by definition of $X$ and equalities in subset types we have~\ref{item:autohtpy2}.
  Finally, \autoref{ex:eqvboolbool} implies that every equivalence $\eqv\bool\bool$ is equal to either $\idfunc[\bool]$ or $e$, so we can show~\ref{item:autohtpy3} by a four-way case analysis.

  Thus, we have $f:\prd{x:X} (x=x)$ such that $f(a) = q$.
  Since $e$ is not equal to $\idfunc[\bool]$, $q$ is not equal to $\refl{a}$, and thus $f$ is not equal to $\lam{x} \refl{x}$.
  Therefore, $\prd{x:X} (x=x)$ is not a mere proposition.
\end{proof}

More generally, \autoref{lem:autohtpy} implies that any ``Eilenberg--Mac Lane space'' $K(G,1)$, where $G$ is a nontrivial abelian\index{group!abelian} group, will provide a counterexample; see \autoref{cha:homotopy}.
The type $X$ we used turns out to be equivalent to $K(\mathbb{Z}_2,1)$.
In \autoref{cha:hits} we will see that the circle $\Sn^1 = K(\mathbb{Z},1)$ is another easy-to-describe example.

We now move on to describing better notions of equivalence.

\index{quasi-inverse|)}%

%%%%%%%%%%%%%%%%%%%%%%%%%%%%%%%%%%%%%%
\section{Half adjoint equivalences}
\label{sec:hae}
%%%%%%%%%%%%%%%%%%%%%%%%%%%%%%%%%%%%%%

\index{equivalence!half adjoint|(defstyle}%
\index{half adjoint equivalence|(defstyle}%
\index{adjoint!equivalence!of types, half|(defstyle}%

In \autoref{sec:quasi-inverses} we concluded that $\qinv(f)$ is equivalent to $\prd{x:A} (x=x)$ by discarding a contractible type.
Roughly, the type $\qinv(f)$ contains three data $g$, $\eta$, and $\epsilon$, of which two ($g$ and $\eta$) could together be seen to be contractible when $f$ is an equivalence.
The problem is that removing these data left one remaining ($\epsilon$).
In order to solve this problem, the idea is to add one \emph{additional} datum which, together with $\epsilon$, forms a contractible type.

\begin{defn}\label{defn:ishae}
  A function $f:A\to B$ is a \define{half adjoint equivalence}
  if there are $g:B\to A$ and homotopies $\eta: g \circ f \htpy \idfunc[A]$ and $\epsilon:f \circ g \htpy \idfunc[B]$ such that there exists a homotopy
  \[\tau : \prd{x:A} \map{f}{\eta x} = \epsilon(fx).\]
\end{defn}

Thus we have a type $\ishae(f)$, defined to be
\begin{equation*}
  \sm{g:B\to A}{\eta: g \circ f \htpy \idfunc[A]}{\epsilon:f \circ g \htpy \idfunc[B]} \prd{x:A} \map{f}{\eta x} = \epsilon(fx).
\end{equation*}
Note that in the above definition, the coherence\index{coherence} condition relating $\eta$ and $\epsilon$ only involves $f$.
We might consider instead an analogous coherence condition involving $g$:
\[\upsilon : \prd{y:B} \map{g}{\epsilon y} = \eta(gy)\]
and a resulting analogous definition $\ishae'(f)$.

Fortunately, it turns out each of the conditions implies the other one:

\begin{lem}\label{lem:coh-equiv}
For functions $f : A \to B$ and $g:B\to A$ and homotopies $\eta: g \circ f \htpy \idfunc[A]$ and $\epsilon:f \circ g \htpy \idfunc[B]$, the following conditions are logically equivalent:
\begin{itemize}
\item $\prd{x:A} \map{f}{\eta x} = \epsilon(fx)$
\item $\prd{y:B} \map{g}{\epsilon y} = \eta(gy)$
\end{itemize}
\end{lem}
\begin{proof}
  It suffices to show one direction; the other one is obtained by replacing $A$, $f$, and $\eta$ by $B$, $g$, and $\epsilon$ respectively.
  Let $\tau : \prd{x:A}\;\map{f}{\eta x} = \epsilon(fx)$.
  Fix $y : B$.
  Using naturality of $\epsilon$ and applying $g$, we get the following commuting diagram of paths:
\[\uppercurveobject{{ }}\lowercurveobject{{ }}\twocellhead{{ }}
  \xymatrix@C=3pc{gfgfgy \ar@{=}^-{gfg(\epsilon y)}[r] \ar@{=}_{g(\epsilon (fgy))}[d] & gfgy \ar@{=}^{g(\epsilon y)}[d] \\ gfgy \ar@{=}_{g(\epsilon y)}[r] & gy
  }\]
Using $\tau(gy)$ on the left side of the diagram gives us
\[\uppercurveobject{{ }}\lowercurveobject{{ }}\twocellhead{{ }}
  \xymatrix@C=3pc{gfgfgy \ar@{=}^-{gfg(\epsilon y)}[r] \ar@{=}_{gf(\eta (gy))}[d] & gfgy \ar@{=}^{g(\epsilon y)}[d] \\ gfgy \ar@{=}_{g(\epsilon y)}[r] & gy
  }\]
Using the commutativity of $\eta$ with $g \circ f$ (\autoref{cor:hom-fg}), we have
\[\uppercurveobject{{ }}\lowercurveobject{{ }}\twocellhead{{ }}
  \xymatrix@C=3pc{gfgfgy \ar@{=}^-{gfg(\epsilon y)}[r] \ar@{=}_{\eta (gfgy)}[d] & gfgy \ar@{=}^{g(\epsilon y)}[d] \\ gfgy \ar@{=}_{g(\epsilon y)}[r] & gy
  }\]
However, by naturality of $\eta$ we also have
\[\uppercurveobject{{ }}\lowercurveobject{{ }}\twocellhead{{ }}
  \xymatrix@C=3pc{gfgfgy \ar@{=}^-{gfg(\epsilon y)}[r] \ar@{=}_{\eta (gfgy)}[d] & gfgy \ar@{=}^{\eta(gy)}[d] \\ gfgy \ar@{=}_{g(\epsilon y)}[r] & gy
  }\]
Thus, canceling all but the right-hand homotopy, we have $g(\epsilon y) = \eta(g y)$ as desired.
\end{proof}

However, it is important that we do not include \emph{both} $\tau$ and $\upsilon$ in the definition of $\ishae (f)$ (whence the name ``\emph{half} adjoint equivalence'').
If we did, then after canceling contractible types we would still have one remaining datum --- unless we added another higher coherence condition.
In general, we expect to get a well-behaved type if we cut off after an odd number of coherences.

Of course, it is obvious that $\ishae(f) \to\qinv(f)$: simply forget the coherence datum.
The other direction is a version of a standard argument from homotopy theory and category theory.

\begin{thm}\label{thm:equiv-iso-adj}
  For any $f:A\to B$ we have $\qinv(f)\to\ishae(f)$.
\end{thm}
\begin{proof}
Suppose that $(g,\eta,\epsilon)$ is a quasi-inverse for $f$. We have to provide
a quadruple $(g',\eta',\epsilon',\tau)$ witnessing that $f$ is a half adjoint equivalence. To
define $g'$ and $\eta'$, we can just make the obvious choice by setting $g'
\defeq g$ and $\eta'\defeq \eta$. However, in the definition of $\epsilon'$ we
need start worrying about the construction of $\tau$, so we cannot just follow our nose
and take $\epsilon'$ to be $\epsilon$. Instead, we take
\begin{equation*}
\epsilon'(b) \defeq \opp{\epsilon(f(g(b)))}\ct (\ap{f}{\eta(g(b))}\ct \epsilon(b)).
\end{equation*}
Now we need to find
\begin{equation*}
\tau(a): \opp{\epsilon(f(g(f(a))))}\ct (\ap{f}{\eta(g(f(a)))}\ct \epsilon(f(a)))=\ap{f}{\eta(a)}.
\end{equation*}
Note first that by \autoref{cor:hom-fg}, we have
%$\eta(g(f(a)))\ct\eta(a)=\ap{g}{\ap{f}{\eta(a)}}\ct\eta(a)$ and hence it follows that
$\eta(g(f(a)))=\ap{g}{\ap{f}{\eta(a)}}$. Therefore, we can apply
\autoref{lem:htpy-natural} to compute
\begin{align*}
\ap{f}{\eta(g(f(a)))}\ct \epsilon(f(a))
& = \ap{f}{\ap{g}{\ap{f}{\eta(a)}}}\ct \epsilon(f(a))\\
& = \epsilon(f(g(f(a))))\ct \ap{f}{\eta(a)}
\end{align*}
from which we get the desired path $\tau(a)$.
\end{proof}

Combining this with \autoref{lem:coh-equiv} (or symmetrizing the proof), we also have $\qinv(f)\to\ishae'(f)$.

It remains to show that $\ishae(f)$ is a mere proposition.
For this, we will need to know that the fibers of an equivalence are contractible.

\begin{defn}\label{defn:homotopy-fiber}
  The \define{fiber}
  \indexdef{fiber}%
  \indexsee{function!fiber of}{fiber}%
  of a map $f:A\to B$ over a point $y:B$ is
  \[ \hfib f y \defeq \sm{x:A} (f(x) = y).\]
\end{defn}

In homotopy theory, this is what would be called the \emph{homotopy fiber} of $f$.
The path lemmas in \autoref{sec:computational} yield the following characterization of paths in fibers:

\begin{lem}\label{lem:hfib}
  For any $f : A \to B$, $y : B$, and $(x,p),(x',p') : \hfib{f}{y}$, we have
  \[ \big((x,p) = (x',p')\big) \eqvsym \Parens{\sm{\gamma : x = x'} f(\gamma) \ct p' = p} \qedhere\]
\end{lem}

\begin{thm}\label{thm:contr-hae}
  If $f:A\to B$ is a half adjoint equivalence, then for any $y:B$ the fiber $\hfib f y$ is contractible.
\end{thm}
\begin{proof}
  Let $(g,\eta,\epsilon,\tau) : \ishae(f)$, and fix $y : B$.
  As our center of contraction for $\hfib{f}{y}$ we choose $(gy, \epsilon y)$.
  Now take any $(x,p) : \hfib{f}{y}$; we want to construct a path from $(gy, \epsilon y)$ to $(x,p)$.
  By \autoref{lem:hfib}, it suffices to give a path $\gamma : \id{gy}{x}$ such that $\ap f\gamma \ct p = \epsilon y$.
  We put $\gamma \defeq \opp{g(p)} \ct \eta x$.
  Then we have
  \begin{align*}
    f(\gamma) \ct p & = \opp{fg(p)} \ct f (\eta x) \ct p \\
    & = \opp{fg(p)} \ct \epsilon(fx) \ct p \\
    & = \epsilon y
  \end{align*}
  where the second equality follows by $\tau x$ and the third equality is naturality of $\epsilon$.
\end{proof}

We now define the types which encapsulate contractible pairs of data.
The following types put together the quasi-inverse $g$ with one of the homotopies.

\begin{defn}\label{defn:linv-rinv}
  Given a function $f:A\to B$, we define the types
    \begin{align*}
      \linv(f) &\defeq \sm{g:B\to A} (g\circ f\htpy \idfunc[A])\\
      \rinv(f) &\defeq \sm{g:B\to A} (f\circ g\htpy \idfunc[B])
    \end{align*}
  of \define{left inverses}
  \indexdef{left!inverse}%
  \indexdef{inverse!left}%
  and \define{right inverses}
  \indexdef{right!inverse}%
  \indexdef{inverse!right}%
  to $f$, respectively.
  We call $f$ \define{left invertible}
  \indexdef{function!left invertible}%
  \indexdef{function!right invertible}%
  if $\linv(f)$ is inhabited, and similarly \define{right invertible}
  \indexdef{left!invertible function}%
  \indexdef{right!invertible function}%
  if $\rinv(f)$ is inhabited.
\end{defn}

\begin{lem}\label{thm:equiv-compose-equiv}
  If $f:A\to B$ has a quasi-inverse, then so do
  \begin{align*}
    (f\circ \blank) &: (C\to A) \to (C\to B)\\
    (\blank\circ f) &: (B\to C) \to (A\to C).
  \end{align*}
\end{lem}
\begin{proof}
  If $g$ is a quasi-inverse of $f$, then $(g\circ \blank)$ and $(\blank\circ g)$ are quasi-inverses of $(f\circ \blank)$ and $(\blank\circ f)$ respectively.
\end{proof}

\begin{lem}\label{lem:inv-hprop}
  If $f : A \to B$ has a quasi-inverse, then the types $\rinv(f)$ and $\linv(f)$ are contractible.
\end{lem}
\begin{proof}
  By function extensionality, we have
  \[\eqv{\linv(f)}{\sm{g:B\to A} (g\circ f = \idfunc[A])}.\]
  But this is the fiber of $(\blank\circ f)$ over $\idfunc[A]$, and so
  by \cref{thm:equiv-compose-equiv,thm:equiv-iso-adj,thm:contr-hae}, it is contractible.
  Similarly, $\rinv(f)$ is equivalent to the fiber of $(f\circ \blank)$ over $\idfunc[B]$ and hence contractible.
\end{proof}

Next we define the types which put together the other homotopy with the additional coherence datum.\index{coherence}%

\begin{defn}\label{defn:lcoh-rcoh}
For $f : A \to B$, a left inverse $(g,\eta) : \linv(f)$, and a right inverse $(g,\epsilon) : \rinv(f)$, we denote
\begin{align*}
\lcoh{f}{g}{\eta} & \defeq \sm{\epsilon : f\circ g \htpy \idfunc[B]} \prd{y:B} g(\epsilon y) = \eta (gy), \\
\rcoh{f}{g}{\epsilon} & \defeq \sm{\eta : g\circ f \htpy \idfunc[A]} \prd{x:A} f(\eta x) = \epsilon (fx).
\end{align*}
\end{defn}

\begin{lem}\label{lem:coh-hfib}
For any $f,g,\epsilon,\eta$, we have
\begin{align*}
\lcoh{f}{g}{\eta} & \eqvsym {\prd{y:B} \id[\hfib{g}{gy}]{(fgy,\eta(gy))}{(y,\refl{gy})}}, \\
\rcoh{f}{g}{\epsilon} & \eqvsym {\prd{x:A} \id[\hfib{f}{fx}]{(gfx,\epsilon(fx))}{(x,\refl{fx})}}.
\end{align*}
\end{lem}
\begin{proof}
Using \autoref{lem:hfib}.
\end{proof}

\begin{lem}\label{lem:coh-hprop}
  If $f$ is a half adjoint equivalence, then for any $(g,\epsilon) : \rinv(f)$, the type $\rcoh{f}{g}{\epsilon}$ is contractible.
\end{lem}
\begin{proof}
  By \autoref{lem:coh-hfib} and the fact that dependent function types preserve contractible spaces, it suffices to show that for each $x:A$, the type $\id[\hfib{f}{fx}]{(gfx,\epsilon(fx))}{(x,\refl{fx})}$ is contractible.
  But by \autoref{thm:contr-hae}, $\hfib{f}{fx}$ is contractible, and any path space of a contractible space is itself contractible.
\end{proof}

\begin{thm}\label{thm:hae-hprop}
  For any $f : A \to B$, the type $\ishae(f)$ is a mere proposition.
\end{thm}
\begin{proof}
  By \autoref{ex:prop-inhabcontr} it suffices to assume $f$ to be a half adjoint equivalence and show that $\ishae(f)$ is contractible.
  Now by associativity of $\Sigma$ (\autoref{ex:sigma-assoc}), the type $\ishae(f)$ is equivalent to
  \[\sm{u : \rinv(f)} \rcoh{f}{\proj{1}(u)}{\proj{2}(u)}.\]
  But by \cref{lem:inv-hprop,lem:coh-hprop} and the fact that $\Sigma$ preserves contractibility, the latter type is also contractible.
\end{proof}

Thus, we have shown that $\ishae(f)$ has all three desiderata for the type $\isequiv(f)$.
In the next two sections we consider a couple of other possibilities.

\index{equivalence!half adjoint|)}%
\index{half adjoint equivalence|)}%
\index{adjoint!equivalence!of types, half|)}%

\section{Bi-invertible maps}
\label{sec:biinv}

\index{function!bi-invertible|(defstyle}%
\index{bi-invertible function|(defstyle}%
\index{equivalence!as bi-invertible function|(defstyle}%

Using the language introduced in \autoref{sec:hae}, we can restate the definition proposed in \autoref{sec:basics-equivalences} as follows.

\begin{defn}\label{defn:biinv}
  We say $f:A\to B$ is \define{bi-invertible}
  if it has both a left inverse and a right inverse:
  \[ \biinv (f) \defeq \linv(f) \times \rinv(f). \]
\end{defn}

In \autoref{sec:basics-equivalences} we proved that $\qinv(f)\to\biinv(f)$ and $\biinv(f)\to\qinv(f)$.
What remains is the following.

\begin{thm}\label{thm:isprop-biinv}
  For any $f:A\to B$, the type $\biinv(f)$ is a mere proposition.
\end{thm}
\begin{proof}
  We may suppose $f$ to be bi-invertible and show that $\biinv(f)$ is contractible.
  But since $\biinv(f)\to\qinv(f)$, by \autoref{lem:inv-hprop} in this case both $\linv(f)$ and $\rinv(f)$ are contractible, and the product of contractible types is contractible.
\end{proof}

Note that this also fits the proposal made at the beginning of \autoref{sec:hae}: we combine $g$ and $\eta$ into a contractible type and add an additional datum which combines with $\epsilon$ into a contractible type.
The difference is that instead of adding a \emph{higher} datum (a 2-dimensional path) to combine with $\epsilon$, we add a \emph{lower} one (a right inverse that is separate from the left inverse).

\begin{cor}\label{thm:equiv-biinv-isequiv}
  For any $f:A\to B$ we have $\eqv{\biinv(f)}{\ishae(f)}$.
\end{cor}
\begin{proof}
  We have $\biinv(f) \to \qinv(f) \to \ishae(f)$ and $\ishae(f) \to \qinv(f) \to \biinv(f)$.
  Since both $\ishae(f)$ and $\biinv(f)$ are mere propositions, the equivalence follows from \autoref{lem:equiv-iff-hprop}.
\end{proof}

\index{function!bi-invertible|)}%
\index{bi-invertible function|)}%
\index{equivalence!as bi-invertible function|)}%

\section{Contractible fibers}
\label{sec:contrf}

\index{function!contractible|(defstyle}%
\index{contractible!function|(defstyle}%
\index{equivalence!as contractible function|(defstyle}%

Note that our proofs about $\ishae(f)$ and $\biinv(f)$ made essential use of the fact that the fibers of an equivalence are contractible.
In fact, it turns out that this property is itself a sufficient definition of equivalence.

\begin{defn}[Contractible maps] \label{defn:equivalence}
  A map $f:A\to B$ is \define{contractible}
  if for all $y:B$, the fiber $\hfib f y$ is contractible.
\end{defn}

Thus, the type $\iscontr(f)$ is defined to be
\begin{align}
  \iscontr(f) &\defeq \prd{y:B} \iscontr(\hfib f y)\label{eq:iscontrf}
  % \\
  % &\defeq \prd{y:B} \iscontr (\setof{x:A | f(x) = y}).
\end{align}
Note that in \autoref{sec:contractibility} we defined what it means for a \emph{type} to be contractible.
Here we are defining what it means for a \emph{map} to be contractible.
Our terminology follows the general homotopy-theoretic practice of saying that a map has a certain property if all of its (homotopy) fibers have that property.
Thus, a type $A$ is contractible just when the map $A\to\unit$ is contractible.
From \autoref{cha:hlevels} onwards we will also call contractible maps and types \emph{$(-2)$-truncated}.

We have already shown in \autoref{thm:contr-hae} that $\ishae(f) \to \iscontr(f)$.
Conversely:

\begin{thm}\label{thm:lequiv-contr-hae}
For any $f:A\to B$ we have ${\iscontr(f)} \to {\ishae(f)}$.
\end{thm}
\begin{proof}
Let $P : \iscontr(f)$. We define an inverse mapping $g : B \to A$ by sending each $y : B$ to the center of contraction of the fiber at $y$:
\[ g(y) \defeq \proj{1}(\proj{1}(Py)) \]
We can thus define the homotopy $\epsilon$ by mapping $y$ to the witness that $g(y)$ indeed belongs to the fiber at $y$:
\[ \epsilon(y) \defeq \proj{2}(\proj{1}(P y)) \]
It remains to define $\eta$ and $\tau$. This of course amounts to giving an element of $\rcoh{f}{g}{\epsilon}$. By \autoref{lem:coh-hfib}, this is the same as giving for each $x:A$ a path from $(gfx,\epsilon(fx))$ to $(x,\refl{fx})$ in the fiber of $f$ over $fx$. But this is easy: for any $x : A$, the type $\hfib{f}{fx}$
is contractible by assumption, hence such a path must exist. We can construct it explicitly as
\[\opp{\big(\proj{2}(P(fx))(gfx,\epsilon(fx))\big)} \ct \big(\proj{2}(P(fx)) (x,\refl{fx})\big). \qedhere \]
\end{proof}

It is also easy to see:

\begin{lem}\label{thm:contr-hprop}
  For any $f$, the type $\iscontr(f)$ is a mere proposition.
\end{lem}
\begin{proof}
  By \autoref{thm:isprop-iscontr}, each type $\iscontr (\hfib f y)$ is a mere proposition.
  Thus, by \autoref{thm:isprop-forall}, so is~\eqref{eq:iscontrf}.
\end{proof}

\begin{thm}\label{thm:equiv-contr-hae}
  For any $f:A\to B$ we have $\eqv{\iscontr(f)}{\ishae(f)}$.
\end{thm}
\begin{proof}
  We have already established a logical equivalence ${\iscontr(f)} \Leftrightarrow {\ishae(f)}$, and both are mere propositions (\cref{thm:contr-hprop,thm:hae-hprop}).
  Thus, \autoref{lem:equiv-iff-hprop} applies.
\end{proof}

Usually, we prove that a function is an equivalence by exhibiting a quasi-inverse, but sometimes this definition is more convenient.
For instance, it implies that when proving a function to be an equivalence, we are free to assume that its codomain is inhabited.

\begin{cor}\label{thm:equiv-inhabcod}
  If $f:A\to B$ is such that $B\to \isequiv(f)$, then $f$ is an equivalence.
\end{cor}
\begin{proof}
  To show $f$ is an equivalence, it suffices to show that $\hfib f y$ is contractible for any $y:B$.
  But if $e:B\to \isequiv(f)$, then given any such $y$ we have $e(y):\isequiv(f)$, so that $f$ is an equivalence and hence $\hfib f y$ is contractible, as desired.
\end{proof}

\index{function!contractible|)}%
\index{contractible!function|)}%
\index{equivalence!as contractible function|)}%

\section{On the definition of equivalences}
\label{sec:concluding-remarks}

\indexdef{equivalence}
We have shown that all three definitions of equivalence satisfy the three desirable properties and are pairwise equivalent:
\[ \iscontr(f) \eqvsym \ishae(f) \eqvsym \biinv(f). \]
(There are yet more possible definitions of equivalence, but we will stop with these three.
See \autoref{ex:brck-qinv} and the exercises in this chapter for some more.)
Thus, we may choose any one of them as ``the'' definition of $\isequiv (f)$.
For definiteness, we choose to define
\[ \isequiv(f) \defeq \ishae(f).\]
\index{mathematics!formalized}%
This choice is advantageous for formalization, since $\ishae(f)$ contains the most directly useful data.
On the other hand, for other purposes, $\biinv(f)$ is often easier to deal with, since it contains no 2-dimensional paths and its two symmetrical halves can be treated independently.
However, for purposes of this book, the specific choice will make little difference.

In the rest of this chapter, we study some other properties and characterizations of equivalences.
\index{equivalence!properties of}%


\section{Surjections and embeddings}
\label{sec:mono-surj}

\index{set}
When $A$ and $B$ are sets and $f:A\to B$ is an equivalence, we also call it as \define{isomorphism}
\indexdef{isomorphism!of sets}%
or a \define{bijection}.
\indexdef{bijection}%
\indexsee{function!bijective}{bijection}%
(We avoid these words for types that are not sets, since in homotopy theory and higher category theory they often denote a stricter notion of ``sameness'' than homotopy equivalence.)
In set theory, a function is a bijection just when it is both injective and surjective.
The same is true in type theory, if we formulate these conditions appropriately.
For clarity, when dealing with types that are not sets, we will speak of \emph{embeddings} instead of injections.

\begin{defn}
  Let $f:A\to B$.
  \begin{enumerate}
  \item We say $f$ is \define{surjective}
    \indexsee{surjective!function}{function, surjective}%
    \indexdef{function!surjective}%
    (or a \define{surjection})
    \indexsee{surjection}{function, surjective}%
    if for every $b:B$ we have $\brck{\hfib f b}$.
  \item We say $f$ is an \define{embedding}
    \indexdef{function!embedding}%
    \indexsee{embedding}{function, embedding}%
    if for every $x,y:A$ the function $\apfunc f : (\id[A]xy) \to (\id[B]{f(x)}{f(y)})$ is an equivalence.
  \end{enumerate}
\end{defn}

In other words, $f$ is surjective if every fiber of $f$ is merely inhabited, or equivalently if for all $b:B$ there merely exists an $a:A$ such that $f(a)=b$.
In traditional logical notation, $f$ is surjective if $\fall{b:B}\exis{a:A} (f(a)=b)$.
This must be distinguished from the stronger assertion that $\prd{b:B}\sm{a:A} (f(a)=b)$; if this holds we say that $f$ is a \define{split surjection}.
\indexsee{split!surjection}{function, split surjective}%
\indexsee{surjection!split}{function, split surjective}%
\indexsee{surjective!function!split}{function, split surjective}%
\indexdef{function!split surjective}%

If $A$ and $B$ are sets, then by \autoref{lem:equiv-iff-hprop}, $f$ is an embedding just when
\begin{equation}
  \prd{x,y:A} (\id[B]{f(x)}{f(y)}) \to (\id[A]xy).\label{eq:injective}
\end{equation}
In this case we say that $f$ is \define{injective},
\indexsee{injective function}{function, injective}%
\indexdef{function!injective}%
or an \define{injection}.
\indexsee{injection}{function, injective}%
We avoid these word for types that are not sets, because they might be interpreted as~\eqref{eq:injective}, which is an ill-behaved notion for non-sets.
It is also true that any function between sets is surjective if and only if it is an \emph{epimorphism} in a suitable sense, but this also fails for more general types, and surjectivity is generally the more important notion.

\begin{thm}\label{thm:mono-surj-equiv}
  A function $f:A\to B$ is an equivalence if and only if it is both surjective and an embedding.
\end{thm}
\begin{proof}
  If $f$ is an equivalence, then each $\hfib f b$ is contractible, hence so is $\brck{\hfib f b}$, so $f$ is surjective.
  And we showed in \autoref{thm:paths-respects-equiv} that any equivalence is an embedding.

  Conversely, suppose $f$ is a surjective embedding.
  Let $b:B$; we show that $\sm{x:A}(f(x)=b)$ is contractible.
  Since $f$ is surjective, there merely exists an $a:A$ such that $f(a)=b$.
  Thus, the fiber of $f$ over $b$ is inhabited; it remains to show it is a mere proposition.
  For this, suppose given $x,y:A$ with $p:f(x)=b$ and $q:f(y)=b$.
  Then since $\apfunc f$ is an equivalence, there exists $r:x=y$ with $\apfunc f (r) = p \ct \opp q$.
  However, using the characterization of paths in $\Sigma$-types, the latter equality rearranges to $\trans{r}{p} = q$.
  Thus, together with $r$ it exhibits $(x,p) = (y,q)$ in the fiber of $f$ over $b$.
\end{proof}

\begin{cor}
  For any $f:A\to B$ we have
  \[ \isequiv(f) \eqvsym (\mathsf{isEmbedding}(f) \times \mathsf{isSurjective}(f)).\]
\end{cor}
\begin{proof}
  Being a surjection and an embedding are both mere propositions; now apply \autoref{lem:equiv-iff-hprop}.
\end{proof}

Of course, this cannot be used as a definition of ``equivalence'', since the definition of embeddings refers to equivalences.
However, this characterization can still be useful; see \autoref{sec:whitehead}.
We will generalize it in \autoref{cha:hlevels}.


% \section{Fiberwise equivalences}
\section{Closure properties of equivalences}
\label{sec:equiv-closures}
\label{sec:fiberwise-equivalences}
\index{equivalence!properties of}%


% We end this chapter by observing some important closure properties of equivalences.
We have already seen in \autoref{thm:equiv-eqrel} that equivalences are closed under composition.
Furthermore, we have:

\begin{thm}[The 2-out-of-3 property]\label{thm:two-out-of-three}
  \index{2-out-of-3 property}%
  Suppose $f:A\to B$ and $g:B\to C$.
  If any two of $f$, $g$, and $g\circ f$ are equivalences, so is the third.
\end{thm}
\begin{proof}
  If $g\circ f$ and $g$ are equivalences, then $\opp{(g\circ f)} \circ g$ is a quasi-inverse to $f$.
  On the one hand, we have $\opp{(g\circ f)} \circ g \circ f \htpy \idfunc[A]$, while on the other we have
  \begin{align*}
    f \circ \opp{(g\circ f)} \circ g
    &\htpy \opp g \circ g \circ f \circ \opp{(g\circ f)} \circ g\\
    &\htpy \opp g \circ g\\
    &\htpy \idfunc[B].
  \end{align*}
  Similarly, if $g\circ f$ and $f$ are equivalences, then $f\circ \opp{(g\circ f)}$ is a quasi-inverse to $g$.
\end{proof}

This is a standard closure condition on equivalences from homotopy theory.
Also well-known is that they are closed under retracts, in the following sense.

\index{retract!of a function|(defstyle}%

\begin{defn}\label{defn:retract}
A function $g:A\to B$ is said to be a \define{retract}
of a function $f:X\to Y$ if there is a diagram
\begin{equation*}
  \xymatrix{
    {A} \ar[r]^{s} \ar[d]_{g}
    &
    {X} \ar[r]^{r} \ar[d]_{f}
    &
    {A} \ar[d]^{g}
    \\
    {B} \ar[r]_{s'}
    &
    {Y} \ar[r]_{r'}
    &
    {B}
  }
\end{equation*}
for which there are
\begin{enumerate}
\item a homotopy $R:r\circ s \htpy \idfunc[A]$.
\item a homotopy $R':r'\circ s' \htpy\idfunc[B]$.
\item a homotopy $L:f\circ s\htpy s'\circ g$.
\item a homotopy $K:g\circ r\htpy r'\circ f$.
\item for every $a:A$, a path $H(a)$ witnessing the commutativity of the square
\begin{equation*}
  \xymatrix@C=3pc{
    {g(r(s(a)))} \ar@{=}[r]^-{K(s(a))} \ar@{=}[d]_{\ap g{R(a)}}
    &
    {r'(f(s(a)))} \ar@{=}[d]^{\ap{r'}{L(a)}}
    \\
    {g(a)} \ar@{=}[r]_-{\opp{R'(g(a))}}
    &
    {r'(s'(g(a)))}
  }
\end{equation*}
\end{enumerate}
\end{defn}

Recall that in \autoref{sec:contractibility} we defined what it means for a type to be a retract of another.
This is a special case of the above definition where $B$ and $Y$ are $\unit$.
Conversely, just as with contractibility, retractions of maps induce retractions of their fibers.

\begin{lem}\label{lem:func_retract_to_fiber_retract}
If a function $g:A\to B$ is a retract of a function $f:X\to Y$, then $\hfib{g}b$ is a retract of $\hfib{f}{s'(b)}$
for every $b:B$, where $s':B\to Y$ is as in \autoref{defn:retract}.
\end{lem}

\begin{proof}
Suppose that $g:A\to B$ is a retract of $f:X\to Y$. Then for any $b:B$ we have the functions
\begin{align*}
\varphi_b &:\hfiber{g}b\to\hfib{f}{s'(b)}, &
\varphi_b(a,p) & \defeq \pairr{s(a),L(a)\ct s'(p)},\\
\psi_b &:\hfib{f}{s'(b)}\to\hfib{g}b, &
\psi_b(x,q) &\defeq \pairr{r(x),K(x)\ct r'(q)\ct R'(b)}.
\end{align*}
Then we have $\psi_b(\varphi_b({a,p}))\equiv\pairr{r(s(a)),K(s(a))\ct r'(L(a)\ct s'(p))\ct R'(b)}$.
We claim $\psi_b$ is a retraction with section $\varphi_b$ for all $b:B$, which is to say that for all $(a,p):\hfib g b$ we have $\psi_b(\varphi_b({a,p}))= \pairr{a,p}$.
In other words, we want to show
\begin{equation*}
\prd{b:B}{a:A}{p:g(a)=b} \psi_b(\varphi_b({a,p}))= \pairr{a,p}.
\end{equation*}
By reordering the first two $\Pi$s and applying a version of \autoref{thm:omit-contr}, this is equivalent to
\begin{equation*}
\prd{a:A}\psi_{g(a)}(\varphi_{g(a)}({a,\refl{g(a)}}))=\pairr{a,\refl{g(a)}}.
\end{equation*}
For any $a$, by \autoref{thm:path-sigma}, this equality of pairs is equivalent to a pair of equalities. The first components are equal by $R(a):r(s(a))= a$, so we need only show
\begin{equation*}
\trans{R(a)}{K(s(a))\ct r'(L(a))\ct R'(g(a))} = \refl{g(a)}.
\end{equation*}
But this transportation computes as $\opp{g(R(a))}\ct K(s(a))\ct r'(L(a))\ct R'(g(a))$, so the required path is given by $H(a)$.
\end{proof}

\begin{thm}\label{thm:retract-equiv}
  If $g$ is a retract of an equivalence $f$, then $g$ is also an equivalence.
\end{thm}
\begin{proof}
  By \autoref{lem:func_retract_to_fiber_retract}, every fiber of $g$ is a retract of a fiber of $f$.
  Thus, by \autoref{thm:retract-contr}, if the latter are all contractible, so are the former.
\end{proof}

\index{retract!of a function|)}%

\index{fibration}%
\index{total!space}%
Finally, we show that fiberwise equivalences can be characterized in terms of equivalences of total spaces.
To explain the terminology, recall from \autoref{sec:fibrations} that a type family $P:A\to\type$ can be viewed as a fibration over $A$ with total space $\sm{x:A} P(x)$, the fibration being is the projection $\proj1:\sm{x:A} P(x) \to A$.
From this point of view, given two type families $P,Q:A\to\type$, we may refer to a function $f:\prd{x:A} (P(x)\to Q(x))$ as a \define{fiberwise map} or a \define{fiberwise transformation}.
\indexsee{transformation!fiberwise}{fiberwise transformation}%
\indexsee{function!fiberwise}{fiberwise transformation}%
\index{fiberwise!transformation|(defstyle}%
\indexsee{fiberwise!map}{fiberwise transformation}%
\indexsee{map!fiberwise}{fiberwise transformation}
Such a map induces a function on total spaces:

\begin{defn}\label{defn:total-map}
  Given type families $P,Q:A\to\type$ and a map $f:\prd{x:A} P(x)\to Q(x)$, we define
  \begin{equation*}
    \total f  \defeq \lam{w}\pairr{\proj{1}w,f(\proj{1}w,\proj{2}w)} : \sm{x:A}P(x)\to\sm{x:A}Q(x).
  \end{equation*}
\end{defn}

\begin{thm}\label{fibwise-fiber-total-fiber-equiv}
Suppose that $f$ is a fiberwise transformation between families $P$ and
$Q$ over a type $A$ and let $x:A$ and $v:Q(x)$. Then we have an equivalence
\begin{equation*}
\eqv{\hfib{\total{f}}{\pairr{x,v}}}{\hfib{f(x)}{v}}.
\end{equation*}
\end{thm}
\begin{proof}
  We calculate:
\begin{align}
  \hfib{\total{f}}{\pairr{x,v}}
  & \jdeq \sm{w:\sm{x:A}P(x)}\pairr{\proj{1}w,f(\proj{1}w,\proj{2}w)}=\pairr{x,v}
  \notag \\
  & \eqv{}{} \sm{a:A}{u:P(a)}\pairr{a,f(a,u)}=\pairr{x,v}
  \tag{by~\autoref{ex:sigma-assoc}} \\
  & \eqv{}{} \sm{a:A}{u:P(a)}{p:a=x}\trans{p}{f(a,u)}=v
  \tag{by \autoref{thm:path-sigma}} \\
  & \eqv{}{} \sm{a:A}{p:a=x}{u:P(a)}\trans{p}{f(a,u)}=v
  \notag \\
  & \eqv{}{} \sm{u:P(x)}f(x,u)=v
  \tag{$*$}\label{eq:uses-sum-over-paths} \\
  & \jdeq \hfib{f(x)}{v}. \notag
\end{align}
The equivalence~\eqref{eq:uses-sum-over-paths} follows from \autoref{thm:omit-contr,thm:contr-paths,ex:sigma-assoc}.
\end{proof}

We say that a fiberwise transformation $f:\prd{x:A} P(x)\to Q(x)$ is a \define{fiberwise equivalence}%
\indexdef{fiberwise!equivalence}%
\indexdef{equivalence!fiberwise}
if each $f(x):P(x) \to Q(x)$ is an equivalence.

\begin{thm}\label{thm:total-fiber-equiv}
Suppose that $f$ is a fiberwise transformation between families
$P$ and $Q$ over a type $A$.
Then $f$ is a fiberwise equivalence if and only if $\total{f}$ is an equivalence.
\end{thm}

\begin{proof}
Let $f$, $P$, $Q$ and $A$ be as in the statement of the theorem.
By \autoref{fibwise-fiber-total-fiber-equiv} it follows for all
$x:A$ and $v:Q(x)$ that
$\hfib{\total{f}}{\pairr{x,v}}$ is contractible if and only if
$\hfib{f(x)}{v}$ is contractible.
Thus, $\hfib{\total{f}}{w}$ is contractible for all $w:\sm{x:A}Q(x)$ if and only if $\hfib{f(x)}{v}$ is contractible for all $x:A$ and $v:Q(x)$.
\end{proof}

\index{fiberwise!transformation|)}%


\section{The object classifier}
\label{sec:object-classification}

In type theory we have a basic notion of \emph{family of types}, namely a function $B:A\to\type$.
We have seen that such families behave somewhat like \emph{fibrations} in homotopy theory, with the fibration being the projection $\proj1:\sm{a:A} B(a) \to A$.
A basic fact in homotopy theory is that every map is equivalent to a fibration.
With univalence at our disposal, we can prove the same thing in type theory.

\begin{lem}\label{thm:fiber-of-a-fibration}
  For any type family $B:A\to\type$, the fiber of $\proj1:\sm{x:A} B(x) \to A$ over $a:A$ is equivalent to $B(a)$:
  \[ \eqv{\hfib{\proj1}{a}}{B(a)} \]
\end{lem}
\begin{proof}
  We have
  \begin{align*}
    \hfib{\proj1}{a} &\defeq \sm{u:\sm{x:A} B(x)} \proj1(u)=a\\
    &\eqvsym \sm{x:A}{b:B(x)} (x=a)\\
    &\eqvsym \sm{x:A}{p:x=a} B(x)\\
    &\eqvsym B(a)
  \end{align*}
  using the left universal property of identity types.
\end{proof}

\begin{lem}\label{thm:total-space-of-the-fibers}
  For any function $f:A\to B$, we have $\eqv{A}{\sm{b:B}\hfib{f}{b}}$.
\end{lem}
\begin{proof}
  We have
  \begin{align*}
    \sm{b:B}\hfib{f}{b} &\defeq \sm{b:B}{a:A} (f(a)=b)\\
    &\eqvsym \sm{a:A}{b:B} (f(a)=b)\\
    &\eqvsym A
  \end{align*}
  using the fact that $\sm{b:B} (f(a)=b)$ is contractible.
\end{proof}

\begin{thm}\label{thm:nobject-classifier-appetizer}
For any type $B$ there is an equivalence
\begin{equation*}
\chi:\Parens{\sm{A:\type} (A\to B)}\eqvsym (B\to\type).
\end{equation*}
\end{thm}
\begin{proof}
We have to construct quasi-inverses
\begin{align*}
\chi & : \Parens{\sm{A:\type} (A\to B)}\to B\to\type\\
\psi & : (B\to\type)\to\Parens{\sm{A:\type} (A\to B)}.
\end{align*}
We define $\chi$ by $\chi((A,f),b)\defeq\hfiber{f}b$, and $\psi$ by $\psi(P)\defeq\Pairr{(\sm{b:B} P(b)),\proj1}$.
Now we have to verify that $\chi\circ\psi\htpy\idfunc{}$ and that $\psi\circ\chi \htpy\idfunc{}$.
\begin{enumerate}
\item Let $P:B\to\type$.
  By \autoref{thm:fiber-of-a-fibration},
$\hfiber{\proj1}{b}\eqvsym P(b)$ for any $b:B$, so it follows immediately
that $P\htpy\chi(\psi(P))$.
\item Let $f:A\to B$ be a function. We have to find a path
\begin{equation*}
\Pairr{\tsm{b:B} \hfiber{f}b,\,\proj1}=\pairr{A,f}.
\end{equation*}
First note that by \autoref{thm:total-space-of-the-fibers}, we have
$e:\sm{b:B} \hfiber{f}b\eqvsym A$ with $e(b,a,p)\defeq a$ and $e^{-1}(a)
\defeq(f(a),a,\refl{f(a)})$.
By \autoref{thm:path-sigma}, it remains to show $\trans{(\ua(e))}{\proj1} = f$.
But by the computation rule for univalence and~\eqref{eq:transport-arrow}, we have $\trans{(\ua(e))}{\proj1} = \proj1\circ e^{-1}$, and the definition of $e^{-1}$ immediately yields $\proj1 \circ e^{-1} \jdeq f$.\qedhere
\end{enumerate}
\end{proof}

\noindent
\indexdef{object!classifier}%
\indexdef{classifier!object}%
\index{.infinity1-topos@$(\infty,1)$-topos}%
In particular, this implies that we have an \emph{object classifier} in the sense of higher topos theory.
Recall from \autoref{def:pointedtype} that $\pointed\type$ denotes the type $\sm{A:\type} A$ of pointed types.

\begin{thm}\label{thm:object-classifier}
Let $f:A\to B$ be a function. Then the diagram
\begin{equation*}
  \vcenter{\xymatrix{
      A\ar[r]^-{\vartheta_f} \ar[d]_{f} &
      \pointed{\type}\ar[d]^{\proj1}\\
      B\ar[r]_{\chi_f} &
      \type
      }}
\end{equation*}
is a pullback\index{pullback} square (see \autoref{ex:pullback}).
Here the function $\vartheta_f$ is defined by
\begin{equation*}
 \lam{a} \pairr{\hfiber{f}{f(a)},\pairr{a,\refl{f(a)}}}.
\end{equation*}
\end{thm}
\begin{proof}
Note that we have the equivalences
\begin{align*}
A & \eqvsym \sm{b:B} \hfiber{f}b\\
& \eqvsym \sm{b:B}{X:\type}{p:\hfiber{f}b= X} X\\
& \eqvsym \sm{b:B}{X:\type}{x:X} \hfiber{f}b= X\\
& \eqvsym \sm{b:B}{Y:\pointed{\type}} \hfiber{f}b = \proj1 Y\\
& \jdeq B\times_{\type}\pointed{\type}.
\end{align*}
which gives us a composite equivalence $e:A\eqvsym B\times_\type\pointed{\type}$.
We may display the action of this composite equivalence step by step by
\begin{align*}
a & \mapsto \pairr{f(a),\; \pairr{a,\refl{f(a)}}}\\
& \mapsto \pairr{f(a), \; \hfiber{f}{f(a)}, \; \refl{\hfiber{f}{f(a)}}, \; \pairr{a,\refl{f(a)}}}\\
& \mapsto \pairr{f(a), \; \hfiber{f}{f(a)}, \; \pairr{a,\refl{f(a)}}, \; \refl{\hfiber{f}{f(a)}}}.
\end{align*}
Therefore, we get homotopies $f\htpy\proj1\circ e$ and $\vartheta_f\htpy \proj2\circ e$.
\end{proof}



\section{Univalence implies function extensionality}
\label{sec:univalence-implies-funext}

\index{function extensionality!proof from univalence}%
In the last section of this chapter we include a proof that the univalence axiom implies function
extensionality. Thus, in this section we work \emph{without} the function extensionality axiom.
The proof consists of two steps. First we show
in \autoref{uatowfe} that the univalence
axiom implies a weak form of function extensionality, defined in \autoref{weakfunext} below. The
principle of weak function extensionality in turn implies the usual function extensionality,
and it does so without the univalence axiom (\autoref{wfetofe}).

\index{univalence axiom}%
Let $\type$ be a universe; we will explicitly indicate where we assume that it is univalent.

\begin{defn}\label{weakfunext}
The \define{weak function extensionality principle}
\indexdef{function extensionality!weak}%
asserts that there is a function
\begin{equation*}
\Parens{\prd{x:A}\iscontr(P(x))} \to\iscontr\Parens{\prd{x:A}P(x)}
\end{equation*}
for any family $P:A\to\type$ of types over any type $A$.
\end{defn}

The following lemma is easy to prove using function extensionality; the point here is that it also follows from univalence without assuming function extensionality separately.

\begin{lem} \label{UA-eqv-hom-eqv}
Assuming $\type$ is univalent, for any $A,B,X:\type$ and any $e:\eqv{A}{B}$, there is an equivalence
\begin{equation*}
\eqv{(X\to A)}{(X\to B)}
\end{equation*}
of which the underlying map is given by post-composition with the underlying function of $e$.
\end{lem}

\begin{proof}
  % Immediate by induction on $\eqv{}{}$ (see \autoref{thm:equiv-induction}).
  As in the proof of \autoref{lem:qinv-autohtpy}, we may assume that $e = \idtoeqv(p)$ for some $p:A=B$.
  Then by path induction, we may assume $p$ is $\refl{A}$, so that $e = \idfunc[A]$.
  But in this case, post-composition with $e$ is the identity, hence an equivalence.
\end{proof}

\begin{cor}\label{contrfamtotalpostcompequiv}
Let $P:A\to\type$ be a family of contractible types, i.e.\ \narrowequation{\prd{x:A}\iscontr(P(x)).}
Then the projection $\proj{1}:(\sm{x:A}P(x))\to A$ is an equivalence. Assuming $\type$ is univalent, it follows immediately that post-composition with $\proj{1}$ gives an equivalence
\begin{equation*}
\alpha : \eqv{\Parens{A\to\sm{x:A}P(x)}}{(A\to A)}.
\end{equation*}
\end{cor}

\begin{proof}
  By \autoref{thm:fiber-of-a-fibration}, for $\proj{1}:\sm{x:A}P(X)\to A$ and $x:A$ we have an equivalence
  \begin{equation*}
    \eqv{\hfiber{\proj{1}}{x}}{P(x)}.
  \end{equation*}
  Therefore $\proj{1}$ is an equivalence whenever each $P(x)$ is contractible. The assertion is now a consequence of  \autoref{UA-eqv-hom-eqv}.
\end{proof}

In particular, the homotopy fiber of the above equivalence at $\idfunc[A]$ is contractible. Therefore, we can show that univalence implies weak function extensionality by showing that the dependent function type $\prd{x:A}P(x)$ is a retract of $\hfiber{\alpha}{\idfunc[A]}$.

\begin{thm}\label{uatowfe}
In a univalent universe $\type$, suppose that $P:A\to\type$ is a family of contractible types
and let $\alpha$ be the function of \autoref{contrfamtotalpostcompequiv}.
Then $\prd{x:A}P(x)$ is a retract of $\hfiber{\alpha}{\idfunc[A]}$. As a consequence, $\prd{x:A}P(x)$ is contractible. In other words, the univalence axiom implies the weak function extensionality principle.
\end{thm}

\begin{proof}
Define the functions
\begin{align*}
  \varphi &: \tprd{x:A}P(x)\to\hfiber{\alpha}{\idfunc[A]},\\
  \varphi(f) &\defeq (\lam{x} (x,f(x)),\refl{\idfunc[A]}),
\intertext{and}
  \psi &: \hfiber{\alpha}{\idfunc[A]}\to \tprd{x:A}P(x), \\
  \psi(g,p) &\defeq \lam{x} \trans p{\proj{2} (g(x))}.
\end{align*}
Then $\psi(\varphi(f))=\lam{x} f(x)$, which is $f$, by the uniqueness principle for dependent function types.
\end{proof}

We now show that weak function extensionality implies the usual function extensionality.
Recall from~\eqref{eq:happly} the function $\happly (f,g) : (f = g)\to(f\htpy g)$ which
converts equality of functions to homotopy. In the proof that follows, the univalence
axiom is not used.

\begin{thm}\label{wfetofe}
  \index{function extensionality}%
Weak function extensionality implies the function extensionality \autoref{axiom:funext}.
\end{thm}

\begin{proof}
We want to show that
\begin{equation*}
\prd{A:\type}{P:A\to\type}{f,g:\prd{x:A}P(x)}\isequiv(\happly (f,g)).
\end{equation*}
Since a fiberwise map induces an equivalence on total spaces if and only if it is fiberwise an equivalence by \autoref{thm:total-fiber-equiv}, it suffices to show that the function of type
\begin{equation*}
\Parens{\sm{g:\prd{x:A}P(x)}(f= g)} \to \sm{g:\prd{x:A}P(x)}(f\htpy g)
\end{equation*}
induced by $\lam{g:\prd{x:A}P(x)} \happly (f,g)$ is an equivalence.
Since the type on the left is contractible by \autoref{thm:contr-paths}, it suffices to show that the type on the right:
\begin{equation}\label{eq:uatofesp}
\sm{g:\prd{x:A}P(x)}\prd{x:A}f(x)= g(x)
\end{equation}
is contractible.
Now \autoref{thm:ttac} says that this is equivalent to
\begin{equation}\label{eq:uatofeps}
\prd{x:A}\sm{u:P(x)}f(x)= u.
\end{equation}
The proof of \autoref{thm:ttac} uses function extensionality, but only for one of the composites.
Thus, without assuming function extensionality, we can conclude that~\eqref{eq:uatofesp} is a retract\index{retract!of a type} of~\eqref{eq:uatofeps}.
And~\eqref{eq:uatofeps} is a product of contractible types, which is contractible by the weak function extensionality principle; hence~\eqref{eq:uatofesp} is also contractible.
\end{proof}

\sectionNotes

The fact that the space of continuous maps equipped with quasi-inverses has the wrong homotopy type to be the ``space of homotopy equivalences'' is well-known in algebraic topology.
In that context, the ``space of homotopy equivalences'' $(\eqv AB)$ is usually defined simply as the subspace of the function space $(A\to B)$ consisting of the functions that are homotopy equivalences.
In type theory, this would correspond most closely to $\sm{f:A\to B} \brck{\qinv(f)}$; see \autoref{ex:brck-qinv}.

The first definition of equivalence given in homotopy type theory was the one that we have called $\iscontr(f)$, which was due to Voevodsky.
The possibility of the other definitions was subsequently observed by various people.
The basic theorems about adjoint equivalences\index{adjoint!equivalence} such as \autoref{lem:coh-equiv,thm:equiv-iso-adj} are adaptations of standard facts in higher category theory and homotopy theory.
Using bi-invertibility as a definition of equivalences was suggested by Andr\'e Joyal.

The properties of equivalences discussed in \autoref{sec:mono-surj,sec:equiv-closures} are well-known in homotopy theory.
Most of them were first proven in type theory by Voevodsky.

The fact that every function is equivalent to a fibration is a standard fact in homotopy theory.
The notion of object classifier
\index{object!classifier}%
\index{classifier!object}%
in $(\infty,1)$-category
\index{.infinity1-category@$(\infty,1)$-category}%
theory (the categorical analogue of \autoref{thm:nobject-classifier-appetizer}) is due to Rezk (see~\cite{Rezk05,lurie:higher-topoi}).

Finally, the fact that univalence implies function extensionality (\autoref{sec:univalence-implies-funext}) is due to Voevodsky.
Our proof is a simplification of his.

\sectionExercises

\begin{ex}\label{ex:two-sided-adjoint-equivalences}
  Consider the type of ``two-sided adjoint equivalence\index{adjoint!equivalence} data'' for $f:A\to B$,
  \begin{narrowmultline*}
    \sm{g:B\to A}{\eta: g \circ f \htpy \idfunc[A]}{\epsilon:f \circ g \htpy \idfunc[B]}
    \narrowbreak
    \Parens{\prd{x:A} \map{f}{\eta x} = \epsilon(fx)} \times
    \Parens{\prd{y:B} \map{g}{\epsilon y} = \eta(gy) }.
  \end{narrowmultline*}
  By \autoref{lem:coh-equiv}, we know that if $f$ is an equivalence, then this type is inhabited.
  Give a characterization of this type analogous to \autoref{lem:qinv-autohtpy}.

  Can you give an example showing that this type is not generally a mere proposition?
  (This will be easier after \autoref{cha:hits}.)
\end{ex}

\begin{ex}\label{ex:symmetric-equiv}
  Show that for any $A,B:\UU$, the following type is equivalent to $\eqv A B$.
  \begin{equation*}
    \sm{R:A\to B\to \type}
    \Parens{\prd{a:A} \iscontr\Parens{\sm{b:B} R(a,b)}} \times
    \Parens{\prd{b:B} \iscontr\Parens{\sm{a:A} R(a,b)}}.
  \end{equation*}
  Can you extract from this a definition of a type satisfying the three desiderata of $\isequiv(f)$?
\end{ex}

\begin{ex} \label{ex:qinv-autohtpy-no-univalence}
  Reformulate the proof of \autoref{lem:qinv-autohtpy} without using univalence.
\end{ex}

\begin{ex}[The unstable octahedral axiom]\label{ex:unstable-octahedron}
  \index{axiom!unstable octahedral}%
  \index{octahedral axiom, unstable}%
  Suppose $f:A\to B$ and $g:B\to C$ and $b:B$.
  \begin{enumerate}
  \item Show that there is a natural map $\hfib{g\circ f}{g(b)} \to \hfib{g}{g(b)}$ whose fiber over $(b,\refl{g(b)})$ is equivalent to $\hfib f b$.
  \item Show that $\eqv{\hfib{g\circ f}{g(b)}}{\sm{w:\hfib{g}{g(b)}} \hfib f {\proj1 w}}$.
  \end{enumerate}
\end{ex}

\begin{ex}\label{ex:2-out-of-6}
  \index{2-out-of-6 property}%
  Prove that equivalences satisfy the \emph{2-out-of-6 property}: given $f:A\to B$ and $g:B\to C$ and $h:C\to D$, if $g\circ f$ and $h\circ g$ are equivalences, so are $f$, $g$, $h$, and $h\circ g\circ f$.
  Use this to give a higher-level proof of \autoref{thm:paths-respects-equiv}.
\end{ex}

\begin{ex}\label{ex:qinv-univalence}
  For $A,B:\UU$, define
  \[ \mathsf{idtoqinv}_{A,B} :(A=B) \to \sm{f:A\to B}\qinv(f) \]
  by path induction in the obvious way.
  Let \textbf{\textsf{qinv}-univalence} denote the modified form of the univalence axiom which asserts that for all $A,B:\UU$ the function $\mathsf{idtoqinv}_{A,B}$ has a quasi-inverse.
  \begin{enumerate}
  \item Show that \qinv-univalence can be used instead of univalence in the proof of function extensionality in \autoref{sec:univalence-implies-funext}.
  \item Show that \qinv-univalence can be used instead of univalence in the proof of \autoref{thm:qinv-notprop}.
  \item Show that \qinv-univalence is inconsistent (i.e.\ allows construction of an inhabitant of $\emptyt$).
    Thus, the use of a ``good'' version of $\isequiv$ is essential in the statement of univalence.
  \end{enumerate}
\end{ex}

% Local Variables:
% TeX-master: "hott-online"
% End:


\newcommand{\zero}{\ensuremath{\mathbf{0}}\xspace}
\newcommand{\one}{\ensuremath{\mathbf{1}}\xspace}
\newcommand{\two}{\ensuremath{\mathbf{2}}\xspace}
\newcommand{\three}{\ensuremath{\mathbf{3}}\xspace}
\newcommand{\nat}{\ensuremath{\mathbf{N}}\xspace}
\newcommand{\true}{\ensuremath{\mathbf{true}}\xspace}
\newcommand{\false}{\ensuremath{\mathbf{false}}\xspace}
\newcommand{\rec}{\ensuremath{\mathbf{rec}}\xspace}
\newcommand{\z}{\ensuremath{0}\xspace}
\newcommand{\wtype}[1]{\ensuremath{W}(#1)\xspace}
%\newcommand{\inl}{\ensuremath{\mathbf{inl}}\xspace}
%\newcommand{\inr}{\ensuremath{\mathbf{inr}}\xspace}
\newcommand{\s}{\ensuremath{\mathbf{s}}\xspace}
\newcommand{\alt}{\;|\;\;}
\newcommand{\disj}[2]{#1 + #2}
\newcommand{\der}{\vdash}
\newcommand{\dbl}{\ensuremath{\mathbf{double}}}

\chapter{Induction}
\label{cha:induction}

\section{Booleans and natural numbers}

% Local Variables:
% TeX-master: "main"
% End:

An \emph{inductive type} can be intuitively understood as a type generated by a certain finite collection of constructors. To specify an inductive type formally,
we will use a schematic definition, where we list the name and type of each constructor separately, e.g.:
\begin{align*}
  \two \defeq \; & \true : \two \\
         \alt & \false : \two
\end{align*}
The above definition declares the type $\two$ of Booleans to be the inductive type generated by two constant constructors $\true$ and $\false$. We can similarly define the types $\zero$ (aka Empty, Void), $\one$ (aka Unit), $\three$, and so on, with 0, 1, 3, or more constructors respectively. 

Another extremely important inductive type is the type $\nat$ of natural numbers:  
\begin{align*}
  \nat \defeq \; & \z : \nat \\
        \alt & \s : \nat \to \nat 
\end{align*}
As expected, $\nat$ is generated by a constant constructor $\z$ for the natural number zero and a unary constructor $\s$ taking a natural number $n : \nat$ to its successor $\s(n) : \nat$. It is understood that for any inductive type, different constructors construct different terms and each constructor itself is injective. This ensures that as desired, 0 is not a successor of any other natural number and that no two natural numbers have the same successor.

What can we do with such an inductive type? An intuitive understanding of an inductive type $A$ is that it \emph{behaves as if the only inhabitants of $A$ were the terms constructed solely by applying the constructors of $A$}; we refer to these particular terms as the \emph{canonical terms of $A$}. In an empty context,
each term of $A$ is canonical - for instance, any closed term of an identity type is necessarily the identity path. In the setting of nonempty contexts,  
interesting and often nontrivial behavior may occur, such as the one exhibited by identity types which behave much like paths in a space.

In the case of the type $\two$ of Booleans, assuming that each term is canonical simply means that each term $b : \two$ must be either $\true$ or $\false$. In particular, we have the principle of \emph{(dependent) elimination}:

\begin{itemize}
\item When proving a statement $\Gamma \der E : \two \to \type$ about \emph{all} Booleans, it suffices to prove it for $\true$ and $\false$, i.e., give proofs
$\Gamma \der e_t : E(\true)$ and $\Gamma \der e_f : E(\false)$.
\end{itemize}

Furthermore, the resulting proof $\Gamma \der \rec_\two(E,e_t,e_f): \prd{b : \two}E(b)$ behaves as expected when applied to the constructors $\true$ and $\false$; this principle is expressed by the \emph{computation rules}:
\begin{itemize}
\item The proof $\Gamma \der \rec_\two(E,e_t,e_f,\true) : E(\true)$ is identical to $e_t$.
\item The proof $\Gamma \der \rec_\two(E,e_t,e_f,\false) : E(\true)$ is identical to $e_f$.
\end{itemize}
For simplicity we often omit the ambient context $\Gamma$.

The rules for the type $\two$ of Booleans allow us to reason by \emph{case analysis}. Since neither of the two constructors takes any arguments, this is all we need for Booleans. However, for more complex types such as the type $\nat$ of natural numbers, true induction is often needed (hence the name \emph{inductive type}):

\begin{itemize}
\item When proving a statement $E : \nat \to \type$ about \emph{all} natural numbers, it suffices to prove it for $\z$ and for $\s(n)$, assuming it holds
for $n$. This entails giving the proofs $e_z : E(\z)$ and $e_s : \prd{n : \nat}{y : E(n)} E(\s(n))$.
\end{itemize}
The variable $y$ represents our inductive hypothesis. As for Booleans, we also have the associated computation rules for the function $\rec_\nat(E,e_z,e_s) : \prd{x:\nat} E(x)$:
\begin{itemize}
\item The proof $\rec_\nat(E,e_z,e_s,\z) : E(\z)$ is identical to $e_z$.
\item For any $n : \nat$, the proof $\rec_\nat(E,e_z,e_s,\s(n)) : E(\s(n))$ is identical to $e_s(n,\rec_\nat(E,e_z,e_s,n))$.
\end{itemize}
The dependent function $\rec_\nat(E,e_z,e_s)$ can thus be understood as being defined recursively on the argument $x : \nat$, via the recurrences $e_z$ and $e_s$: When $x$ is zero, the function simply returns $e_z$. When $x$ is the successor of another natural number $n$, the result is obtained by taking the recurrence $e_s$ and plugging in the specific predecessor $n$ and the recursive call value $\rec_\nat(E,e_z,e_s,n)$.

As an example we look at how to define a function on natural numbers which doubles its argument. We wish to apply dependent elimination with the constant type family $E \defeq \lambda(x : \nat), \nat$ since the intended type of $\dbl$ is $\nat \to \nat$. We first need to supply the value of $\dbl(\z)$, which is easy: we put $e_z \defeq \z$. Next, to compute the value of $\dbl(\s(n))$ for a natural number $n$, we first compute the value of $\dbl(n)$ and then perform the successor operation twice. This is captured by the recurrence $e_s(n,y) \defeq \s(\s(y))$; the variable $y$ stands for the result of the recursive call $\dbl(n)$. Thus, we define
\[ \dbl \defeq \rec_\nat(\lambda(x :\nat), \nat, \;  \z, \;  \lambda(n: \nat) \lambda (y:\nat), \s(\s(y))) \]
This indeed has the correct computational behavior: for example, we have 
\begin{align*}
\dbl(\s(\s(\z))) & \equiv e_s(\s(\z), \dbl(\s(\z))) \\
                 & \equiv \s(\s(\dbl(\s(\z)))) \\
                 & \equiv \s(\s(e_s(\z,\dbl(\z)))) \\
                 & \equiv \s(\s(\s(\s(\dbl(\z))))) \\
                 & \equiv \s(\s(\s(\s(e_z)))) \\
                 & \equiv \s(\s(\s(\s(\z))))
\end{align*}
as desired.

There are many similar cases when one wants to use dependent elimination with a constant family term. This is referred to as \emph{simple elimination}, which corresponds to recursion. In this case, we omit the $\lambda$-binder for simplicity and write $\dbl \defeq \rec_\nat(\nat, \z, \ldots)$.

The depndent elimination principle is quite strong and allows us to prove a variety of interesting theorems. For example, by specifying the terms $e_z$ and $e_s$, we  uniquely determine how the recursor behaves on canonical terms, and thus on all natural numbers. If we now have another function which obeys the same recurrence, then our intuition suggests these two functions should be equal. It turns out this is indeed the case and we have the following \emph{uniqueness principle}:

\begin{thm}
Let $f,g : \prd{x:\nat} E(x)$ be two functions which satisfy the recurrences $e_z : E(\z)$ and $e_s : \prd{n : \nat}{y : E(n)} E(\s(n))$ up to propositional equality, i.e., such that
\begin{align*}
\id{f(\z)}{e_z} \\ 
\id{g(\z)}{e_z}
\end{align*}
and 
\begin{align*}
\prd{n : \nat} \id{f(\s(n))}{e_s(n, f(n))} \\
\prd{n : \nat} \id{g(\s(n))}{e_s(n, g(n))}
\end{align*}
Then $f$ and $g$ are equal, i.e., we have $\alpha : \id[\prd{x :\nat} E(x)]{f}{g}$. 
\end{thm}

\begin{proof}
We use dependent elimination with the type $E(x) \defeq \id{f(x)}{g(x)}$. For the base case, we have \[f(\z) = e_z = g(\z)\]
For the inductive case, assume $n : \nat$ such that $f(n) = g(n)$. Then
\[ f(\s(n)) = e_s(n, f(n)) = e_s(n, g(n)) = g(\s(n)) \]
The first and last equality follow from the assumptions on $f$ and $g$. The middle equality follows from the inductive hypothesis and the fact that application preserves equality. This gives us pointwise equality between $f$ and $g$; invoking function extensionality finishes the proof.
\end{proof}
We note that the function $f$ is only required to satisfy the recurrences \emph{up to propositional equality}. The theorem itself only asserts propositional equality between functions - indeed, it is possible to construct functions which satisfy the same recurrence but are not definitionally equal (exercise). 

Similar uniqueness theorems can generally be formulated and shown for other inductive types as well. Such uniqueness results are a very useful tool; for instance, using uniqueness we can show that a certain class of inductive types gives rise to \emph{homotopy-initial algebras}. Taking Booleans as the simplest example, we define a \emph{$\two$-algebra} to be any type $C$ with two terms $c_0, c_1 : C$. Thus,

\begin{align*}
\mathtt{2Alg} \defeq \sm {C : \type} C \times C
\end{align*}
Given $\two$-algebras $(C,c_0,c_1)$ and $(D,d_0,d_1)$, we define a $\two$-homomorphism between them as a function $h : C \to D$ mapping $c_0$ to $d_0$ and
$c_1$ to $d_1$ (up to propositional equality). Thus,
\begin{align*}
\mathtt{2Hom \; (C,c_0,c_1) \; (D,d_0,d_1)} \defeq \sm {h : C \to D} \; \id{h(c_0)}{d_0} \times \id{h(c_1)}{d_1}
\end{align*}
A $\two$-algebra $\chi^*$ is then called \emph{homotopy-initial} if for any other $\two$-algebra $\chi$, the type of $\two$-homomorphisms from $\chi^*$ to $\chi$ is contractible. Thus,
\begin{align*}
\mathtt{is\_hinitial \; \chi^* \defeq \fall{\chi : 2Alg} \; is\_contr \; (2Hom \; \chi^*\; \chi)}
\end{align*}

We now have the following theorem; the proof is only sketched as it involves a large amount of technical detail.
\begin{thm}
The $\two$-algebra $(\two, \true, \false)$ is homotopy initial.
\end{thm}
\begin{proof}
(Rough sketch) Fix an arbitrary $\two$-algebra $(C,c_0,c_1)$. Using elimination on $\two$ it is easy to construct a $\two$-homomorphism into $C$. This will be our
center of contraction. To show that any other homomorphism is equal to the chosen one, we appeal to the uniqueness theorem for $\two$.
\end{proof}

\section{Disjoint unions and dependent sums}
A \emph{disjoint union} of two types $A$ and $B$ behaves much like a conjunction: we can choose to supply either a term of type $A$ or a term of type $B$. Schematically, we declare this as follows:

\begin{align*}
  \disj{A}{B} \defeq \; & \inl : A \to (\disj{A}{B}) \\
         \alt & \inr : B \to (\disj{A}{B})
\end{align*}

In this definition, $A$ and $B$ act as \emph{parameters}, i.e., they denote arbitrary types. In other words, the type constructor $+$ is really a function taking the types $A$ and $B$ as an argument and returning a new inductive type $\disj{A}{B}$. Although omitted from the above schema for clarity, the constructors $\inl$ and $\inr$ likewise take the types $A$ and $B$ as the first 2 arguments, followed by a term of type $A$ (for \inl) or a term of type $B$ (for \inr).

Elimination for a disjoint union amounts to case analysis:
\begin{itemize}
\item When proving a statement $E : (\disj{A}{B}) \to \type$ about \emph{all} terms of the disjoint union $\disj{A}{B}$, it suffices to prove it for $\inl(a)$ and $\inr(b)$, i.e., give proofs $e_l : \prd{a:A} E(\inl(a))$ and $e_r : \prd{b:B} E(\inr(b))$.
\end{itemize}
The associated computation rules for the function $\rec_{\disj{A}{B}}(E,e_l,e_r) : \prd{x:\disj{A}{B}} E(x)$ are as expected:
\begin{itemize}
\item For each $a : A$, the proof $\rec_{\disj{A}{B}}(E,e_l,e_r,\inl(a)) : E(\inl(a))$ is identical to $e_l(a)$.
\item For each $b : B$, the proof $\rec_{\disj{A}{B}}(E,e_l,e_r,\inr(b)) : E(\inr(b))$ is identical to $e_r(b)$.
\end{itemize}

A \emph{product} of two types $A$ and $B$ (denoted by $A \times B$) is the type of pairs $(a,b)$, where $a : A$ and $b : B$. A \emph{dependent sum} is a generalization of this concept, where we allow the type $B$ to depend on $A$. A simple example is the type of pairs $(n,(c_1,\ldots,c_n))$, where the first component is a natural number $n : \nat$ and the second component is a vector $(c_1,\ldots,c_n) : \mathbf{Vec}_C(n)$ of length $n$ over another type $C : \type$.
Given $A : \type$ and $B : A \to \type$, the dependent sum of $A$ and $B$ is given schematically as
\begin{align*}
  \sm{a:A} B(a) \defeq \; & \mathbf{pair} : \prd{a:A}{b:B(a)} \sm{a:A} B(a)
\end{align*}
Thus, the parameterized inductive type $\sm{a:A} B(a)$ has a single constructor $\mathbf{pair}$. Its first two arguments (not shown) are the parameters $A$ and $B$; the remaining two arguments are the respective components $a : A$ and $b : B(a)$. We will often denote $\mathbf{pair}(a,b)$ simply by $(a,b)$.

The elimination and computation rules are very simple:

\begin{itemize}
\item When proving a statement $E : \big(\sm{a:A} B(a)\big) \to \type$ about \emph{all} terms of the dependent sum $\sm{a:A} B(a)$, it suffices to prove it for a pair $(a,b)$, i.e., give a proof $e : \prd{a:A}{b:B(a)} E((a,b))$.
\end{itemize}

\begin{itemize}
\item For any terms $a : A$ and $b : B(a)$, the proof $\rec_{\sm{a:A} B(a)}(E,e,(a,b)) : E((a,b))$ is identical to $e(a,b)$.
\end{itemize}
Using the elimination operator, it is very easy to construct the well-know projection functions $\pi_1$ and $\pi_2$, extracting the first resp. the second component of a pair.

\section{W-types}
Martin-L{\"o}f's W-types, also known as the types of well-founded trees, are a generalization of such types as natural numbers, lists, and binary trees. A particular W-type is specified by giving two parameters $A : \type$ and $B : A \to \type$, written $\wtype{a:A} B(a)$.

The type $A$ represents the type of constructors for $\wtype{a :A} B(a)$. For instance, when defining natural numbers as a W-type the type $A$ would be the type $\two$ inhabited by the two terms $\true$ and $\false$, since there are precisely two ways how to construct a natural number - either it will be zero or a successor of another natural number. 

The dependent type family $B : A \to \type$ is used to record the arity of constructors: a constructor $a : A$ will take $B(a)$-many inductive arguments. These arguments are represented as a function $f : B(a) \to \wtype{a :A} B(a)$, with the understanding that for any $b : B(a)$, $f(b)$ is the $b$-th argument to the constructor $a$. 

In the case of natural numbers, the constructor $\true $ has arity 0, since it constucts the constant zero; the constructor $\false$ has arity 1, since it constructs the successor of its argument. We can capture this by using elimination on $\two$ to define a function into a universe of types:

\[ B \defeq \rec_\two(\lambda(b:\two), \; \bbU,\zero,\one) \]

 


The ``dependent`` type $E \defeq 

%\[\nat \stackrel{\text{def}}{=} \wtype{x}{\two}{\ifte{x}{\zero}{\one}} \]
%It is also possible to express parametrized types such as $\listt{\nat}$ as W-types. In this case we would have infinitely many constructors - one for the empty list plus a separate constructor for each $n : \nat$, corresponding to
%the $\mathbf{cons} \; n$ operation. Thus the type of constructors would be $\one + \nat$, with the arities defined in the obvious way.
%
%\subsection{Introduction Rule}
%We have the following introduction rule for W types:
%\begin{mathpar}
%\inferrule{ }{\entails{\Gamma, \term{x}{A}, \term{f}{B \to \wtype{x}{A}{B}}}{\term{\supp{x}{f}}{\wtype{x}{A}{B}}}}(\WI)
%\end{mathpar}
%By substitution, for any two terms $\entails{\Gamma}{\term{a}{A}}$ and $\entails{\Gamma}{\term{h}{\subst{B}{x}{a} \to {\wtype{x}{A}{B}}}}$ we get a term $\entails{\Gamma}{\term{\supp{a}{h}}{\wtype{x}{A}{B}}}$. Intuitively, $a$ determines a constructor for ${\wtype{x}{A}{B}}$ and $h$ determines the arguments for $a$ by mapping each argument index to an element of ${\wtype{x}{A}{B}}$. In the case of natural numbers, for instance, we would have $0 \stackrel{\text{def}}{=} \supp{\true}{\lam{x}{\zero}{\abort{x}}}$, $1 \stackrel{\text{def}}{=} \supp{\false}{\lam{x}{\one}{0}}$, $2 \stackrel{\text{def}}{=} \supp{\false}{\lam{x}{\one}{1}}$, etc.  
%
%\subsection{Elimination Rule}
%We have the following elimination rule for W types, referred to as dependent elimination:
%\begin{mathpar}
%\inferrule{\entails{\Gamma,\term{w}{\wtype{x}{A}{B}}}{\type{C}} \\
%           \entails{\Gamma,\term{x}{A},\term{f}{B \to \wtype{x}{A}{B}},\term{g}{\pitype{b}{B}{\subst{C}{w}{(\app{f}{b})}}}}{\term{H}{\subst{C}{w}{\supp{x}{f}}}}} 
%          {\entails{\Gamma,\term{w}{\wtype{x}{A}{B}}}{\term{\wrec{C}{H}{w}}{C}}}(\WDE)
%\end{mathpar}
%The above rule is the infinitary version of the elimination rule for inductively defined datatypes. If we view the dependent type $C$ as a predicate on $\wtype{x}{A}{B}$, the rule says that in order to show that $C$ holds for all  terms of $\wtype{x}{A}{B}$, it suffices to show that $C$ holds for all terms of $\wtype{x}{A}{B}$ having the form $\supp{x}{f}$ for some $x$ and $f$, thus capturing the notion that $\wtype{x}{A}{B}$ contains no other terms besides those created though one of its constructors. Moreover, when showing that $C$ holds for a term of the form $\supp{x}{f}$, we are allowed to use the induction hypothesis that $C$ holds for all arguments of the constructor $x$, capturing the notion of well-foundedness. In other words, $\wtype{x}{A}{B}$ can be seen as the type of trees of finite depth whose nodes are labeled by terms of $A$ and each node labeled by $a$ has precisely $|\subst{B}{x}{a}|$ children. The elimination rule then allows us to reason by induction on the depth of such a tree. In the case of natural numbers this reduces to ordinary induction.
%
%As an example we consider the function \emph{double} on natural numbers. To define the function recursively we must construct a suitable term 
%\[\entails{\term{x}{\two},\term{f}{B \to \nat},\term{g}{B \to \nat}}{\term{H}{\nat}}\]
%where $B$ denotes the type $\ifte{x}{\zero}{\one}$. The variable $x$ represents an arbitrary constructor, $f$ represents the predecessor (if applicable), and $g$ represents the double of the predecessor. The term $H$ itself then stands for the double of the natural number represented by $x$ and $f$. To construct $H$, we
%first define an auxiliary term $V$ as the case analysis
%\begin{align*}
%& \ifte{x \\ & \; }{\lam{f'}{\zero \to \nat}{\lam{g'}{\zero \to \nat}{0}} \\ & \;}
%  {\lam{f'}{\one \to \nat}{\lam{g'}{\one \to \nat}{\supp{\false}{\lam{-}{\one}{\supp{\false}{\lam{-}{\one}{(\app{g'}{\langle \rangle})}}}}}}}
%\end{align*}
%We then put
%\begin{align*}
%H & \stackrel{\text{def}}{=} V f \; g \\
%\textit{double} & \stackrel{\text{def}}{=} \lam{n}{\nat}{\wrec{\nat}{H}{n}}
%\end{align*}
%The variables $f'$ and $g'$ in the case analysis can be thought of as representing the functions $f$ and $g$ after the type refinement has been performed, i.e. after replacing the type $B$ by either $\zero$ or $\one$, depending on the value of $x$. If $x$ evaluates to $\true$, we are in the branch for $n = 0$ and we return 0. If $x$ evaluates to $\false$, we are in the branch for $n = s(n')$; the expression $g' \; \langle \rangle$ then gives us the result of the recursive call on the predecessor $n'$. The constructor $\false$ is then applied twice, corresponding to a double application of the successor function. This gives us the double of $n$.
% 
%\subsection{Computation Rule}
%We have the following computation rule for W types:
%\begin{align*}
%\inferrule{\entails{\Gamma, \term{w}{\wtype{x}{A}{B}}}{\type{C}} \\
%           \entails{\Gamma,\term{x}{A},\term{f}{B \to \wtype{x}{A}{B}},\term{g}{\pitype{b}{B}{\subst{C}{w}{(\app{f}{b})}}}}
%                    {\term{H}{\subst{C}{w}{\supp{x}{f}}}} \\%\\\\
%          u(w) \stackrel{\text{def}}{=} \wrec{C}{H}{w}}
%          {\entails{\Gamma, \term{x}{A}, \term{f}{B \to \wtype{x}{A}{B}}}
%                   {\term{u(\supp{x}{f}) = \subst{H}{g}{\big(\lam{b}{B}{u(\app{f}{b})}\big)}}{\subst{C}{w}{\supp{x}{f}}}}}(\WDB)      
%\end{align*}
%The rule is a form of $\beta$-reduction and states that the recursor obtained from the elimination rule behaves as specified by $H$. By substitution, we obtain the equality 
%\[\wrec{C}{H}{\supp{a}{h}} = H[a/x, h/f, \big(\lam{b}{\subst{B}{x}{a}}{\app{h}{b}}\big) / g]\]
%for any two terms $a$ and $h$ of suitable types. In the case of the function \textit{double} defined in the previous subsection, it is easy to see that by the computation rule we have $\textit{double} \; 0 = 0$ and $\textit{double} \; 1 = 2$, as expected.
%
%\section{Identity types}

\chapter{Higher inductive types}
\label{cha:hits}

\index{type!higher inductive|(}%
\indexsee{inductive!type!higher}{type, higher inductive}%
\indexsee{higher inductive type}{type, higher inductive}%

\section{Introduction}
\label{sec:intro-hits}

\index{generation!of a type, inductive|(}

Like the general inductive types we discussed in \autoref{cha:induction}, \emph{higher inductive types} are a general schema for defining new types generated by some constructors.
But unlike ordinary inductive types, in defining a higher inductive type we may have ``constructors'' which generate not only \emph{points} of that type, but also \emph{paths} and higher paths in that type.
\index{type!circle}%
\indexsee{circle type}{type,circle}%
For instance, we can consider the higher inductive type $\Sn^1$ generated by
\begin{itemize}
\item A point $\base:\Sn^1$, and
\item A path $\lloop : {\id[\Sn^1]\base\base}$.
\end{itemize}
This should be regarded as entirely analogous to the definition of, for instance, $\bool$, as being generated by
\begin{itemize}
\item A point $\bfalse:\bool$ and
\item A point $\btrue:\bool$,
\end{itemize}
or the definition of $\nat$ as generated by
\begin{itemize}
\item A point $0:\nat$ and
\item A function $\suc:\nat\to\nat$.
\end{itemize}
When we think of types as higher groupoids, the more general notion of ``generation'' is very natural:
since a higher groupoid is a ``multi-sorted object'' with paths and higher paths as well as points, we should allow ``generators'' in all dimensions.

We will refer to the ordinary sort of constructors (such as $\base$) as \define{point constructors}
\indexdef{constructor!point}%
\indexdef{point!constructor}%
or \emph{ordinary constructors}, and to the others (such as $\lloop$) as \define{path constructors}
\indexdef{constructor!path}%
\indexdef{path!constructor}%
or \emph{higher constructors}.
Each path constructor must specify the starting and ending point of the path, which we call its \define{source}
\indexdef{source!of a path constructor}%
and \define{target};
\indexdef{target!of a path constructor}%
for $\lloop$, both source and target are $\base$.

Note that a path constructor such as $\lloop$ generates a \emph{new} inhabitant of an identity type, which is not (at least, not \emph{a priori}) equal to any previously existing such inhabitant.
In particular, $\lloop$ is not \emph{a priori} equal to $\refl{\base}$ (although proving that they are definitely unequal takes a little thought; see \autoref{thm:loop-nontrivial}).
This is what distinguishes $\Sn^1$ from the ordinary inductive type \unit.

There are some important points to be made regarding this generalization.

\index{free!generation of an inductive type}%
First of all, the word ``generation'' should be taken seriously, in the same sense that a group can be freely generated by some set.
In particular, because a higher groupoid comes with \emph{operations} on paths and higher paths, when such an object is ``generated'' by certain constructors, the operations create more paths that do not come directly from the constructors themselves.
For instance, in the higher inductive type $\Sn^1$, the constructor $\lloop$ is not the only nontrivial path from $\base$ to $\base$; we have also ``$\lloop\ct\lloop$'' and ``$\lloop\ct\lloop\ct\lloop$'' and so on, as well as $\opp{\lloop}$, etc., all of which are different.
This may seem so obvious as to be not worth mentioning, but it is a departure from the behavior of ``ordinary'' inductive types, where one can expect to see nothing in the inductive type except what was ``put in'' directly by the constructors.

Secondly, this generation is really \emph{free} generation: higher inductive types do not technically allow us to impose ``axioms'', such as forcing ``$\lloop\ct\lloop$'' to equal $\refl{\base}$.
However, in the world of $\infty$-groupoids,%
\index{.infinity-groupoid@$\infty$-groupoid}
there is little difference between ``free generation'' and ``presentation'',
\index{presentation!of an infinity-groupoid@of an $\infty$-groupoid}%
\index{generation!of an infinity-groupoid@of an $\infty$-groupoid}%
since we can make two paths equal \emph{up to homotopy} by adding a new 2-di\-men\-sion\-al generator relating them (e.g.\ a path $\lloop\ct\lloop = \refl{\base}$ in $\base=\base$).
We do then, of course, have to worry about whether this new generator should satisfy its own ``axioms'', and so on, but in principle any ``presentation'' can be transformed into a ``free'' one by making axioms into constructors.
As we will see, by adding ``truncation constructors'' we can use higher inductive types to express classical notions such as group presentations as well.

Thirdly, even though a higher inductive type contains ``constructors'' which generate \emph{paths in} that type, it is still an inductive definition of a \emph{single} type.
In particular, as we will see, it is the higher inductive type itself which is given a universal property (expressed, as usual, by an induction principle), and \emph{not} its identity types.
The identity type of a higher inductive type retains the usual induction principle of any identity type (i.e.\ path induction), and does not acquire any new induction principle.

Thus, it may be nontrivial to identify the identity types of a higher inductive type in a concrete way, in contrast to how in \autoref{cha:basics} we were able to give explicit descriptions of the behavior of identity types under all the traditional type forming operations.
For instance, are there any paths from $\base$ to $\base$ in $\Sn^1$ which are not simply composites of copies of $\lloop$ and its inverse?
Intuitively, it seems that the answer should be no (and it is), but proving this is not trivial.
Indeed, such questions bring us rapidly to problems such as calculating the homotopy groups of spheres, a long-standing problem in algebraic topology for which no simple formula is known.
Homotopy type theory brings a new and powerful viewpoint to bear on such questions, but it also requires type theory to become as complex as the answers to these questions.

\index{dimension!of path constructors}%
Fourthly, the ``dimension'' of the constructors (i.e.\ whether they output points, paths, paths between paths, etc.)\ does not have a direct connection to which dimensions the resulting type has nontrivial homotopy in.
As a simple example, if an inductive type $B$ has a constructor of type $A\to B$, then any paths and higher paths in $A$ result in paths and higher paths in $B$, even though the constructor is not a ``higher'' constructor at all.
The same thing happens with higher constructors too: having a constructor of type $A\to (\id[B]xy)$ means not only that points of $A$ yield paths from $x$ to $y$ in $B$, but that paths in $A$ yield paths between these paths, and so on.
As we will see, this possibility is responsible for much of the power of higher inductive types.

On the other hand, it is even possible for constructors \emph{without} higher types in their inputs to generate ``unexpected'' higher paths.
For instance, in the 2-dimensional sphere $\Sn^2$ generated by
\symlabel{s2a}
\index{type!2-sphere}%
\begin{itemize}
\item A point $\base:\Sn^2$, and
\item A 2-dimensional path $\surf:\refl{\base} = \refl{\base}$ in ${\base=\base}$,
\end{itemize}
there is a nontrivial \emph{3-dimensional path} from $\refl{\refl{\base}}$ to itself.
Topologists will recognize this path as an incarnation of the \emph{Hopf fibration}.
From a category-theoretic point of view, this is the same sort of phenomenon as the fact mentioned above that $\Sn^1$ contains not only $\lloop$ but also $\lloop\ct\lloop$ and so on: it's just that in a \emph{higher} groupoid, there are \emph{operations} which raise dimension.
Indeed, we saw many of these operations back in \autoref{sec:equality}: the associativity and unit laws are not just properties, but operations, whose inputs are 1-paths and whose outputs are 2-paths.

\index{generation!of a type, inductive|)}%

% In US Trade format it wants a page break here but then it stretches the above itemize,
% so we give it some stretchable space to use if it wants to.
\vspace*{0pt plus 20ex}

\section{Induction principles and dependent paths}
\label{sec:dependent-paths}

When we describe a higher inductive type such as the circle as being generated by certain constructors, we have to explain what this means by giving rules analogous to those for the basic type constructors from \autoref{cha:typetheory}.
The constructors themselves give the \emph{introduction} rules, but it requires a bit more thought to explain the \emph{elimination} rules, i.e.\ the induction and recursion principles.
In this book we do not attempt to give a general formulation of what constitutes a ``higher inductive definition'' and how to extract the elimination rule from such a definition --- indeed, this is a subtle question and the subject of current research.
Instead we will rely on some general informal discussion and numerous examples.

\index{type!circle}%
\index{recursion principle!for S1@for $\Sn^1$}%
The recursion principle is usually easy to describe: given any type equipped with the same structure with which the constructors equip the higher inductive type in question, there is a function which maps the constructors to that structure.
For instance, in the case of $\Sn^1$, the recursion principle says that given any type $B$ equipped with a point $b:B$ and a path $\ell:b=b$, there is a function $f:\Sn^1\to B$ such that $f(\base)=b$ and $\apfunc f (\lloop) = \ell$.

\index{computation rule!for S1@for $\Sn^1$}%
\index{equality!definitional}%
The latter two equalities are the \emph{computation rules}.
\index{computation rule!for higher inductive types|(}%
\index{computation rule!propositional|(}%
There is, however, a question of whether these computation rules are judgmental\index{judgmental equality} equalities or propositional equalities (paths).
For ordinary inductive types, we had no qualms about making them judgmental, although we saw in \autoref{cha:induction} that making them propositional would still yield the same type up to equivalence.
In the ordinary case, one may argue that the computation rules are really \emph{definitional} equalities, in the intuitive sense described in the Introduction.

\index{equality!judgmental}%
For higher inductive types, this is less clear. %, and it is likewise less clear to what extent these equalities can be made judgmental in the known set-theoretic models.
Moreover, since the operation $\apfunc f$ is not really a fundamental part of the type theory, but something that we \emph{defined} using the induction principle of identity types (and which we might have defined in some other, equivalent, way), it seems inappropriate to refer to it explicitly in a \emph{judgmental} equality.
Judgmental equalities are part of the deductive system, which should not depend on particular choices of definitions that we may make \emph{within} that system.
There are also semantic and implementation issues to consider; see the Notes.

It does seem unproblematic to make the computational rules for the \emph{point} constructors of a higher inductive type judgmental.
In the example above, this means we have $f(\base)\jdeq b$, judgmentally.
This choice facilitates a computational view of higher inductive types.
Moreover, it also greatly simplifies our lives, since otherwise the second computation rule $\apfunc f (\lloop) = \ell$ would not even be well-typed as a propositional equality; we would have to compose one side or the other with the specified identification of $f(\base)$ with $b$.
(Such problems do arise eventually, of course, when we come to talk about paths of higher dimension, but that will not be of great concern to us here.
See also \autoref{sec:hubs-spokes}.)
Thus, we take the computation rules for point constructors to be judgmental, and those for paths and higher paths to be propositional.%
\footnote{In particular, in the language of \autoref{sec:types-vs-sets}, this means that our higher inductive types are a mix of \emph{rules} (specifying how we can introduce such types and their elements, their induction principle, and their computation rules for point constructors) and \emph{axioms} (the computation rules for path constructors, which assert that certain identity types are inhabited by otherwise unspecified terms).
We may hope that eventually, there will be a better type theory in which higher inductive types, like univalence, will be presented using only rules and no axioms.%
\indexfoot{axiom!versus rules}%
\indexfoot{rule!versus axioms}%
}

\begin{rmk}\label{rmk:defid}
Recall that for ordinary inductive types, we regard the computation rules for a recursively defined function as not merely judgmental equalities, but \emph{definitional} ones, and thus we may use the notation $\defeq$ for them.
For instance, the truncated predecessor\index{predecessor!function, truncated} function $p:\nat\to\nat$ is defined by $p(0)\defeq 0$ and $p(\suc(n))\defeq n$.
In the case of higher inductive types, this sort of notation is reasonable for the point constructors (e.g.\ $f(\base)\defeq b$), but for the path constructors it could be misleading, since equalities such as $\ap f \lloop = \ell$ are not judgmental.
Thus, we hybridize the notations, writing instead $\ap f \lloop \defid \ell$ for this sort of ``propositional equality by definition''.
\end{rmk}
\index{computation rule!for higher inductive types|)}%
\index{computation rule!propositional|)}%

\index{type!circle|(}%
\index{induction principle!for S1@for $\Sn^1$}%
Now, what about the induction principle (the dependent eliminator)?
Recall that for an ordinary inductive type $W$, to prove by induction that $\prd{x:W} P(x)$, we must specify, for each constructor of $W$, an operation on $P$ which acts on the ``fibers'' above that constructor in $W$.
For instance, if $W$ is the natural numbers \nat, then to prove by induction that $\prd{x:\nat} P(x)$, we must specify
\begin{itemize}
\item An element $b:P(0)$ in the fiber over the constructor $0:\nat$, and
\item For each $n:\nat$, a function $P(n) \to P(\suc(n))$.
\end{itemize}
The second can be viewed as a function ``$P\to P$'' lying \emph{over} the constructor $\suc:\nat\to\nat$, generalizing how $b:P(0)$ lies over the constructor $0:\nat$.

By analogy, therefore, to prove that $\prd{x:\Sn^1} P(x)$, we should specify
\begin{itemize}
\item An element $b:P(\base)$ in the fiber over the constructor $\base:\Sn^1$, and
\item A path from $b$ to $b$ ``lying over the constructor $\lloop:\base=\base$''.
\end{itemize}
Note that even though $\Sn^1$ contains paths other than $\lloop$ (such as $\refl{\base}$ and $\lloop\ct\lloop$), we only need to specify a path lying over the constructor \emph{itself}.
This expresses the intuition that $\Sn^1$ is ``freely generated'' by its constructors.

The question, however, is what it means to have a path ``lying over'' another path.
It definitely does \emph{not} mean simply a path $b=b$, since that would be a path in the fiber $P(\base)$ (topologically, a path lying over the \emph{constant} path at $\base$).
Actually, however, we have already answered this question in \autoref{cha:basics}: in the discussion preceding \autoref{lem:mapdep} we concluded that a path from $u:P(x)$ to $v:P(y)$ lying over $p:x=y$ can be represented by a path $\trans p u = v$ in the fiber $P(y)$.
Since we will have a lot of use for such \define{dependent paths}
\index{path!dependent}%
in this chapter, we introduce a special notation for them:
\begin{equation}
  (\dpath P p u v) \defeq (\transfib{P} p u = v).\label{eq:dpath}
\end{equation}

\begin{rmk}
There are other possible ways to define dependent paths.
For instance, instead of $\trans p u = v$ we could consider $u = \trans{(\opp p)}{v}$.
We could also obtain it as a special case of a more general ``heterogeneous equality'',
\index{heterogeneous equality}%
\index{equality!heterogeneous}%
or with a direct definition as an inductive type family.
All these definitions result in equivalent types, so in that sense it doesn't much matter which we pick.
However, choosing $\trans p u = v$ as the definition makes it easiest to conclude other things about dependent paths, such as the fact that $\apdfunc{f}$ produces them, or that we can compute them in particular type families using the transport lemmas in \autoref{sec:computational}.
\end{rmk}

With the notion of dependent paths in hand, we can now state more precisely the induction principle for $\Sn^1$: given $P:\Sn^1\to\type$ and
\begin{itemize}
\item An element $b:P(\base)$, and
\item A path $\ell : \dpath P \lloop b b$,
\end{itemize}
there is a function $f:\prd{x:\Sn^1} P(x)$ such that $f(\base)\jdeq b$ and $\apd f \lloop = \ell$.
As in the non-dependent case, we speak of defining $f$ by $f(\base)\defeq b$ and $\apd f \lloop \defid \ell$.

\begin{rmk}\label{rmk:varies-along}
  When describing an application of this induction principle informally, we regard it as a splitting of the goal ``$P(x)$ for all $x:\Sn^1$'' into two cases, which we will sometimes introduce with phrases such as ``when $x$ is $\base$'' and ``when $x$ varies along $\lloop$'', respectively.
  \index{vary along a path constructor}%
  There is no specific mathematical meaning assigned to ``varying along a path'': it is just a convenient way to indicate the beginning of the corresponding section of a proof; see \autoref{thm:S1-autohtpy} for an example.
\end{rmk}

Topologically, the induction principle for $\Sn^1$ can be visualized as shown in \autoref{fig:topS1ind}.
Given a fibration over the circle (which in the picture is a torus), to define a section of this fibration is the same as to give a point $b$ in the fiber over $\base$ along with a path from $b$ to $b$ lying over $\lloop$.
The way we interpret this type-theoretically, using our definition of dependent paths, is shown in \autoref{fig:ttS1ind}: the path from $b$ to $b$ over $\lloop$ is represented by a path from $\trans \lloop b$ to $b$ in the fiber over $\base$.

\begin{figure}
  \centering
  \begin{tikzpicture}
    \draw (0,0) ellipse (3 and .5);
    \draw (0,3) ellipse (3.5 and 1.5);
    \begin{scope}[yshift=4]
      \clip (-3,3) -- (-1.8,3) -- (-1.8,3.7) -- (1.8,3.7) -- (1.8,3) -- (3,3) -- (3,0) -- (-3,0) -- cycle;
      \draw[clip] (0,3.5) ellipse (2.25 and 1);
      \draw (0,2.5) ellipse (1.7 and .7);
    \end{scope}
    \node (P) at (4.5,3) {$P$};
    \node (S1) at (4.5,0) {$\Sn^1$};
    \draw[->>,thick] (P) -- (S1);
    \node[fill,circle,inner sep=1pt,label={below right:$\base$}] at (0,-.5) {};
    \node at (-2.6,.6) {$\lloop$};
    \node[fill,circle,\OPTblue,inner sep=1pt] (b) at (0,2.3) {};
    \node[\OPTblue] at (-.2,2.1) {$b$};
      \begin{scope}
        \draw[\OPTblue] (b) to[out=180,in=-150] (-2.7,3.5) to[out=30,in=180] (0,3.35);
        \draw[\OPTblue,dotted] (0,3.35) to[out=0,in=175] (1.4,4.35);
        \draw[\OPTblue] (1.4,4.35) to[out=-5,in=90] (2.5,3) to[out=-90,in=0,looseness=.8] (b);
      \end{scope}
      \node[\OPTblue] at (-2.2, 3.3) {$\ell$};
  \end{tikzpicture}
  \caption{The topological induction principle for $\Sn^1$}
  \label{fig:topS1ind}
\end{figure}

\begin{figure}
  \centering
  \begin{tikzpicture}
    \draw (0,0) ellipse (3 and .5);
    \draw (0,3) ellipse (3.5 and 1.5);
    \begin{scope}[yshift=4]
      \clip (-3,3) -- (-1.8,3) -- (-1.8,3.7) -- (1.8,3.7) -- (1.8,3) -- (3,3) -- (3,0) -- (-3,0) -- cycle;
      \draw[clip] (0,3.5) ellipse (2.25 and 1);
      \draw (0,2.5) ellipse (1.7 and .7);
    \end{scope}
    \node (P) at (4.5,3) {$P$};
    \node (S1) at (4.5,0) {$\Sn^1$};
    \draw[->>,thick] (P) -- (S1);
    \node[fill,circle,inner sep=1pt,label={below right:$\base$}] at (0,-.5) {};
    \node at (-2.6,.6) {$\lloop$};
    \node[fill,circle,\OPTblue,inner sep=1pt] (b) at (0,2.3) {};
      \node[\OPTblue] at (-.3,2.3) {$b$};
      \node[fill,circle,\OPTpurple,inner sep=1pt] (tb) at (0,1.8) {};
      % \draw[\OPTpurple,dashed] (b) to[out=0,in=0,looseness=5] (0,4) to[out=180,in=180] (tb);
      \draw[\OPTpurple,dashed] (b) arc (-90:90:2.9 and 0.85) arc (90:270:2.8 and 1.1);
      \begin{scope}
        \clip (b) -- ++(.1,0) -- (.1,1.8) -- ++(-.2,0) -- ++(0,-1) -- ++(3,2) -- ++(-3,0) -- (-.1,2.3) -- cycle;
        \draw[\OPTred,dotted,thick] (.2,2.07) ellipse (.2 and .57);
        \begin{scope}
          % \draw[clip] (b) -- ++(.1,0) |- (tb) -- ++(-.2,0) -- ++(0,-1) -| ++(3,3) -| (b);
          \clip (.2,0) rectangle (-2,3);
          \draw[\OPTred,thick] (.2,2.07) ellipse (.2 and .57);
        \end{scope}
      \end{scope}
      \node[\OPTred] at (1,1.2) {$\ell: \trans \lloop b=b$};
  \end{tikzpicture}
  \caption{The type-theoretic induction principle for $\Sn^1$}
  \label{fig:ttS1ind}
\end{figure}

Of course, we expect to be able to prove the recursion principle from the induction principle, by taking $P$ to be a constant type family.
This is in fact the case, although deriving the non-dependent computation rule for $\lloop$ (which refers to $\apfunc f$) from the dependent one (which refers to $\apdfunc f$) is surprisingly a little tricky.

\begin{lem}\label{thm:S1rec}
  \index{recursion principle!for S1@for $\Sn^1$}%
  \index{computation rule!for S1@for $\Sn^1$}%
  If $A$ is a type together with $a:A$ and $p:\id[A]aa$, then there is a
  function $f:\Sn^1\to{}A$ with
  \begin{align*}
    f(\base)&\defeq a \\
    \apfunc f(\lloop)&\defid p.
  \end{align*}
\end{lem}
\begin{proof}
  We would like to apply the induction principle of $\Sn^1$ to the constant type family, $(\lam{x} A): \Sn^1\to \UU$.
  The required hypotheses for this are a point of $(\lam{x} A)(\base) \jdeq A$, which we have (namely $a:A$), and a dependent path in $\dpath {x \mapsto A}{\lloop} a a$, or equivalently $\transfib{x \mapsto A}{\lloop} a = a$.
  This latter type is not the same as the type $\id[A]aa$ where $p$ lives, but it is equivalent to it, because by \autoref{thm:trans-trivial} we have $\transconst{A}{\lloop}{a} : \transfib{x \mapsto A}{\lloop} a= a$.
  Thus, given $a:A$ and $p:a=a$, we can consider the composite
  \[\transconst{A}{\lloop}{a} \ct p:(\dpath {x \mapsto A}\lloop aa).\]
  Applying the induction principle, we obtain $f:\Sn^1\to A$ such that
  \begin{align}
    f(\base) &\jdeq a \qquad\text{and}\label{eq:S1recindbase}\\
    \apdfunc f(\lloop) &= \transconst{A}{\lloop}{a} \ct p.\label{eq:S1recindloop}
  \end{align}
  It remains to derive the equality $\apfunc f(\lloop)=p$.
  However, by \autoref{thm:apd-const}, we have
  \[\apdfunc f(\lloop) = \transconst{A}{\lloop}{f(\base)} \ct \apfunc f(\lloop).\]
  Combining this with~\eqref{eq:S1recindloop} and canceling the occurrences of $\transconstf$ (which are the same by~\eqref{eq:S1recindbase}), we obtain $\apfunc f(\lloop)=p$.
\end{proof}

% Similarly, in this case we speak of defining $f$ by $f(\base)\defeq a$ and $\ap f \lloop \defid p$.
We also have a corresponding uniqueness principle.

\begin{lem}\label{thm:uniqueness-for-functions-on-S1}
  \index{uniqueness!principle, propositional!for functions on the circle}%
  If $A$ is a type and $f,g:\Sn^1\to{}A$ are two maps together with two
  equalities $p,q$:
  \begin{align*}
    p:f(\base)&=_Ag(\base),\\
    q:\map{f}\lloop&=^{\lam{x} x=_Ax}_p\map{g}\lloop.
  \end{align*}
  Then for all $x:\Sn^1$ we have $f(x)=g(x)$.
\end{lem}
\begin{proof}
  We apply the induction principle of $\Sn^1$ at the type family $P(x)\defeq(f(x)=g(x))$.
  When $x$ is $\base$, $p$ is exactly what we need.
  And when $x$ varies along $\lloop$, we need
  \(p=^{\lam{x} f(x)=g(x)}_{\lloop} p,\)
  which by \autoref{thm:transport-path,thm:dpath-path} can be reduced to $q$.
\end{proof}

\index{universal!property!of S1@of $\Sn^1$}%
These two lemmas imply the expected universal property of the circle:

\begin{lem}\label{thm:S1ump}
  For any type $A$ we have a natural equivalence
  \[ (\Sn^1 \to A) \;\eqvsym\;
  \sm{x:A} (x=x).
  \]
\end{lem}
\begin{proof}
  We have a canonical function $f:(\Sn^1 \to A) \to \sm{x:A} (x=x)$ defined by $f(g) \defeq (g(\base),\ap g \lloop)$.
  The induction principle shows that the fibers of $f$ are inhabited, while the uniqueness principle shows that they are mere propositions.
  Hence they are contractible, so $f$ is an equivalence.
\end{proof}

\index{type!circle|)}%

As in \autoref{sec:htpy-inductive}, we can show that the conclusion of \autoref{thm:S1ump} is equivalent to having an induction principle with propositional computation rules.
Other higher inductive types also satisfy lemmas analogous to \autoref{thm:S1rec,thm:S1ump}; we will generally leave their proofs to the reader.
We now proceed to consider many examples.


\section{The interval}
\label{sec:interval}

\index{type!interval|(defstyle}%
\indexsee{interval!type}{type, interval}%
The \define{interval}, which we denote $\interval$, is perhaps an even simpler higher inductive type than the circle.
It is generated by:
\begin{itemize}
\item a point $\izero:\interval$,
\item a point $\ione:\interval$, and
\item a path $\seg : \id[\interval]\izero\ione$.
\end{itemize}
\index{recursion principle!for interval type}%
The recursion principle for the interval says that given a type $B$ along with
\begin{itemize}
\item a point $b_0:B$,
\item a point $b_1:B$, and
\item a path $s:b_0=b_1$,
\end{itemize}
there is a function $f:\interval\to B$ such that $f(\izero)\jdeq b_0$, $f(\ione)\jdeq b_1$, and $\ap f \seg = s$.
\index{induction principle!for interval type}%
The induction principle says that given $P:\interval\to\type$ along with
\begin{itemize}
\item a point $b_0:P(\izero)$,
\item a point $b_1:P(\ione)$, and
\item a path $s:\dpath{P}{\seg}{b_0}{b_1}$,
\end{itemize}
there is a function $f:\prd{x:\interval} P(x)$ such that $f(\izero)\jdeq b_0$, $f(\ione)\jdeq b_1$, and $\apd f \seg = s$.

Regarded purely up to homotopy, the interval is not really interesting:

\begin{lem}
  The type $\interval$ is contractible.
\end{lem}

\begin{proof}
  We prove that for all $x:\interval$ we have $x=_\interval\ione$. In other words we want a
  function $f$ of type $\prd{x:\interval}(x=_\interval\ione)$. We begin to define $f$ in the following way:
  \begin{alignat*}{2}
    f(\izero)&\defeq \seg  &:\izero&=_\interval\ione,\\
    f(\ione)&\defeq \refl\ione &:\ione &=_\interval\ione.
  \end{alignat*}
  It remains to define $\apd{f}\seg$, which must have type $\seg =_\seg^{\lam{x} x=_\interval\ione}\refl \ione$.
  By definition this type is $\trans\seg\seg=_{\ione=_\interval\ione}\refl\ione$, which in turn is equivalent to $\rev\seg\ct\seg=\refl\ione$.
  But there is a canonical element of that type, namely the proof that path inverses are in fact inverses.
\end{proof}

However, type-theoretically the interval does still have some interesting features, just like the topological interval in classical homotopy theory.
For instance, it enables us to give an easy proof of function extensionality.
(Of course, as in \autoref{sec:univalence-implies-funext}, for the duration of the following proof we suspend our overall assumption of the function extensionality axiom.)

\begin{lem}\label{thm:interval-funext}
  \index{function extensionality!proof from interval type}%
  If $f,g:A\to{}B$ are two functions such that $f(x)=g(x)$ for every $x:A$, then
  $f=g$ in the type $A\to{}B$.
\end{lem}

\begin{proof}
  Let's call the proof we have $p:\prd{x:A}(f(x)=g(x))$. For all $x:A$ we define
  a function $\widetilde{p}_x:\interval\to{}B$ by
  \begin{align*}
    \widetilde{p}_x(\izero) &\defeq f(x), \\
    \widetilde{p}_x(\ione) &\defeq g(x), \\
    \map{\widetilde{p}_x}\seg &\defid p(x).
  \end{align*}
  We now define $q:\interval\to(A\to{}B)$ by
  \[q(i)\defeq(\lam{x} \widetilde{p}_x(i))\]
  Then $q(\izero)$ is the function $\lam{x} \widetilde{p}_x(\izero)$, which is equal to $f$ because $\widetilde{p}_x(\izero)$ is defined by $f(x)$.
  Similarly, we have $q(\ione)=g$, and hence
  \[\map{q}\seg:f=_{(A\to{}B)}g \qedhere\]
\end{proof}

In \autoref{ex:funext-from-interval} we ask the reader to complete the proof of the full function extensionality axiom from \autoref{thm:interval-funext}.

\index{type!interval|)}%

\section{Circles and spheres}
\label{sec:circle}

\index{type!circle|(}%
We have already discussed the circle $\Sn^1$ as the higher inductive type generated by
\begin{itemize}
\item A point $\base:\Sn^1$, and
\item A path $\lloop : {\id[\Sn^1]\base\base}$.
\end{itemize}
\index{induction principle!for S1@for $\Sn^1$}%
Its induction principle says that given $P:\Sn^1\to\type$ along with $b:P(\base)$ and $\ell :\dpath P \lloop b b$, we have $f:\prd{x:\Sn^1} P(x)$ with $f(\base)\jdeq b$ and $\apd f \lloop = \ell$.
Its non-dependent recursion principle says that given $B$ with $b:B$ and $\ell:b=b$, we have $f:\Sn^1\to B$ with $f(\base)\jdeq b$ and $\ap f \lloop = \ell$.

We observe that the circle is nontrivial.

\begin{lem}\label{thm:loop-nontrivial}
  $\lloop\neq\refl{\base}$.
\end{lem}
\begin{proof}
  Suppose that $\lloop=\refl{\base}$.
  Then since for any type $A$ with $x:A$ and $p:x=x$, there is a function $f:\Sn^1\to A$ defined by $f(\base)\defeq x$ and $\ap f \lloop \defid p$, we have
  \[p = f(\lloop) = f(\refl{\base}) = \refl{x}.\]
  But this implies that every type is a set, which as we have seen is not the case (see \autoref{thm:type-is-not-a-set}).
\end{proof}

The circle also has the following interesting property, which is useful as a source of counterexamples.

\begin{lem}\label{thm:S1-autohtpy}
  There exists an element of $\prd{x:\Sn^1} (x=x)$ which is not equal to $x\mapsto \refl{x}$.
\end{lem}
\begin{proof}
  We define $f:\prd{x:\Sn^1} (x=x)$ by $\Sn^1$-induction.
  When $x$ is $\base$, we let $f(\base)\defeq \lloop$.
  Now when $x$ varies along $\lloop$ (see \autoref{rmk:varies-along}), we must show that $\transfib{x\mapsto x=x}{\lloop}{\lloop} = \lloop$.
  However, in \autoref{sec:compute-paths} we observed that $\transfib{x\mapsto x=x}{p}{q} = \opp{p} \ct q \ct p$, so what we have to show is that $\opp{\lloop} \ct \lloop \ct \lloop = \lloop$.
  But this is clear by canceling an inverse.

  To show that $f\neq (x\mapsto \refl{x})$, it suffices by function extensionality to show that $f(\base) \neq \refl{\base}$.
  But $f(\base)=\lloop$, so this is just the previous lemma.
\end{proof}

For instance, this enables us to extend \autoref{thm:type-is-not-a-set} by showing that any universe which contains the circle cannot be a 1-type.

\begin{cor}
  If the type $\Sn^1$ belongs to some universe \type, then \type is not a 1-type.
\end{cor}
\begin{proof}
  The type $\Sn^1=\Sn^1$ in \type is, by univalence, equivalent to the type $\eqv{\Sn^1}{\Sn^1}$ of auto\-equivalences of $\Sn^1$, so it suffices to show that $\eqv{\Sn^1}{\Sn^1}$ is not a set.
  \index{automorphism!of S1@of $\Sn^1$}%
  For this, it suffices to show that its equality type $\id[(\eqv{\Sn^1}{\Sn^1})]{\idfunc[\Sn^1]}{\idfunc[\Sn^1]}$ is not a mere proposition.
  Since being an equivalence is a mere proposition, this type is equivalent to $\id[(\Sn^1\to\Sn^1)]{\idfunc[\Sn^1]}{\idfunc[\Sn^1]}$.
  But by function extensionality, this is equivalent to $\prd{x:\Sn^1} (x=x)$, which as we have seen in \autoref{thm:S1-autohtpy} contains two unequal elements.
\end{proof}

\index{type!circle|)}%

\index{type!2-sphere|(}%
\indexsee{sphere type}{type, sphere}%
We have also mentioned that the 2-sphere $\Sn^2$ should be the higher inductive type generated by
\symlabel{s2b}
\begin{itemize}
\item A point $\base:\Sn^2$, and
\item A 2-dimensional path $\surf:\refl{\base} = \refl{\base}$ in ${\base=\base}$.
\end{itemize}
\index{recursion principle!for S2@for $\Sn^2$}%
The recursion principle for $\Sn^2$ is not hard: it says that given $B$ with $b:B$ and $s:\refl b = \refl b$, we have $f:\Sn^2\to B$ with $f(\base)\jdeq b$ and $\aptwo f \surf = s$.
Here by ``$\aptwo f \surf$'' we mean an extension of the functorial action of $f$ to two-dimensional paths, which can be stated precisely as follows.

\begin{lem}\label{thm:ap2}
  Given $f:A\to B$ and $x,y:A$ and $p,q:x=y$, and $r:p=q$, we have a path $\aptwo f r : \ap f p = \ap f q$.
\end{lem}
\begin{proof}
  By path induction, we may assume $p\jdeq q$ and $r$ is reflexivity.
  But then we may define $\aptwo f {\refl p} \defeq \refl{\ap f p}$.
\end{proof}

In order to state the general induction principle, we need a version of this lemma for dependent functions, which in turn requires a notion of dependent two-dimensional paths.
As before, there are many ways to define such a thing; one is by way of a two-dimensional version of transport.

\begin{lem}\label{thm:transport2}
  Given $P:A\to\type$ and $x,y:A$ and $p,q:x=y$ and $r:p=q$, for any $u:P(x)$ we have $\transtwo r u : \trans p u = \trans q u$.
\end{lem}
\begin{proof}
  By path induction.
\end{proof}

Now suppose given $x,y:A$ and $p,q:x=y$ and $r:p=q$ and also points $u:P(x)$ and $v:P(y)$ and dependent paths $h:\dpath P p u v$ and $k:\dpath P q u v$.
By our definition of dependent paths, this means $h:\trans p u = v$ and $k:\trans q u = v$.
Thus, it is reasonable to define the type of dependent 2-paths over $r$ to be
\[ (\dpath P r h k )\defeq (h = \transtwo r u \ct k). \]
We can now state the dependent version of \autoref{thm:ap2}.

\begin{lem}\label{thm:apd2}
  Given $P:A\to\type$ and $x,y:A$ and $p,q:x=y$ and $r:p=q$ and a function $f:\prd{x:A} P(x)$, we have
  $\apdtwo f r : \dpath P r {\apd f p}{\apd f q}$.
\end{lem}
\begin{proof}
  Path induction.
\end{proof}

\index{induction principle!for S2@for $\Sn^2$}%
Now we can state the induction principle for $\Sn^2$: given $P:\Sn^2\to\type$ with $b:P(\base)$ and $s:\dpath P \surf {\refl b}{\refl b}$, there is a function $f:\prd{x:\Sn^2} P(x)$ such that $f(\base)\jdeq b$ and $\apdtwo f \surf = s$.

\index{type!2-sphere|)}%

Of course, this explicit approach gets more and more complicated as we go up in dimension.
Thus, if we want to define $n$-spheres for all $n$, we need some more systematic idea.
One approach is to work with $n$-dimensional loops\index{loop!n-@$n$-} directly, rather than general $n$-dimensional paths.\index{path!n-@$n$-}

\index{type!pointed}%
Recall from \autoref{sec:equality} the definitions of \emph{pointed types} $\type_*$, and the $n$-fold loop space\index{loop space!iterated} $\Omega^n : \type_* \to \type_*$
(\cref{def:pointedtype,def:loopspace}).  Now we can define the
$n$-sphere $\Sn^n$ to be the higher inductive type generated by
\index{type!n-sphere@$n$-sphere}%
\begin{itemize}
\item A point $\base:\Sn^n$, and
\item An $n$-loop $\lloop_n : \Omega^n(\Sn^n,\base)$.
\end{itemize}
In order to write down the induction principle for this presentation, we would need to define a notion of ``dependent $n$-loop\indexdef{loop!dependent n-@dependent $n$-}'', along with the action of dependent functions on $n$-loops.
We leave this to the reader (see \autoref{ex:nspheres}); in the next section we will discuss a different way to define the spheres that is sometimes more tractable.


\section{Suspensions}
\label{sec:suspension}

\indexsee{type!suspension of}{suspension}%
\index{suspension|(defstyle}%
The \define{suspension} of a type $A$ is the universal way of making the points of $A$ into paths (and hence the paths in $A$ into 2-paths, and so on).
It is a type $\susp A$ defined by the following generators:\footnote{There is an unfortunate clash of notation with dependent pair types, which of course are also written with a $\Sigma$.
  However, context usually disambiguates.}
\begin{itemize}
\item a point $\north:\susp A$,
\item a point $\south:\susp A$, and
\item a function $\merid:A \to (\id[\susp A]\north\south)$.
\end{itemize}
The names are intended to suggest a ``globe'' of sorts, with a north pole, a south pole, and an $A$'s worth of meridians
\indexdef{pole}%
\indexdef{meridian}%
from one to the other.
Indeed, as we will see, if $A=\Sn^1$, then its suspension is equivalent to the surface of an ordinary sphere, $\Sn^2$.

\index{recursion principle!for suspension}%
The recursion principle for $\susp A$ says that given a type $B$ together with
\begin{itemize}
\item points $n,s:B$ and
\item a function $m:A \to (n=s)$,
\end{itemize}
we have a function $f:\susp A \to B$ such that $f(\north)\jdeq n$ and $f(\south)\jdeq s$, and for all $a:A$ we have $\ap f {\merid(a)} = m(a)$.
\index{induction principle!for suspension}%
Similarly, the induction principle says that given $P:\susp A \to \type$ together with
\begin{itemize}
\item a point $n:P(\north)$,
\item a point $s:P(\south)$, and
\item for each $a:A$, a path $m(a):\dpath P{\merid(a)}ns$,
\end{itemize}
there exists a function $f:\prd{x:\susp A} P(x)$ such that $f(\north)\jdeq n$ and $f(\south)\jdeq s$ and for each $a:A$ we have $\apd f {\merid(a)} = m(a)$.

Our first observation about suspension is that it gives another way to define the circle.

\begin{lem}\label{thm:suspbool}
  \index{type!circle}%
  $\eqv{\susp\bool}{\Sn^1}$.
\end{lem}
\begin{proof}
  Define $f:\susp\bool\to\Sn^1$ by recursion such that $f(\north)\defeq \base$ and $f(\south)\defeq\base$, while $\ap f{\merid(\bfalse)}\defid\lloop$ but $\ap f{\merid(\btrue)} \defid \refl{\base}$.
  Define $g:\Sn^1\to\susp\bool$ by recursion such that $g(\base)\defeq \north$ and $\ap g \lloop \defid \merid(\bfalse) \ct \opp{\merid(\btrue)}$.
  We now show that $f$ and $g$ are quasi-inverses.

  First we show by induction that $g(f(x))=x$ for all $x:\susp \bool$.
  If $x\jdeq\north$, then $g(f(\north)) \jdeq g(\base)\jdeq \north$, so we have $\refl{\north} : g(f(\north))=\north$.
  If $x\jdeq\south$, then $g(f(\south)) \jdeq g(\base)\jdeq \north$, and we choose the equality $\merid(\btrue) : g(f(\south)) = \south$.
  It remains to show that for any $y:\bool$, these equalities are preserved as $x$ varies along $\merid(y)$, which is to say that when $\refl{\north}$ is transported along $\merid(y)$ it yields $\merid(\btrue)$.
  By transport in path spaces and pulled back fibrations, this means we are to show that
  \[ \opp{\ap g {\ap f {\merid(y)}}} \ct \refl{\north} \ct \merid(y) = \merid(\btrue). \]
  Of course, we may cancel $\refl{\north}$.
  Now by \bool-induction, we may assume either $y\jdeq \bfalse$ or $y\jdeq \btrue$.
  If $y\jdeq \bfalse$, then we have
  \begin{align*}
    \opp{\ap g {\ap f {\merid(\bfalse)}}} \ct \merid(\bfalse)
    &= \opp{\ap g {\lloop}} \ct \merid(\bfalse)\\
    &= \opp{(\merid(\bfalse) \ct \opp{\merid(\btrue)})} \ct \merid(\bfalse)\\
    &= \merid(\btrue) \ct \opp{\merid(\bfalse)} \ct \merid(\bfalse)\\
    &= \merid(\btrue)
  \end{align*}
  while if $y\jdeq \btrue$, then we have
  \begin{align*}
    \opp{\ap g {\ap f {\merid(\btrue)}}} \ct \merid(\btrue)
    &= \opp{\ap g {\refl{\base}}} \ct \merid(\btrue)\\
    &= \opp{\refl{\north}} \ct \merid(\btrue)\\
    &= \merid(\btrue).
  \end{align*}
  Thus, for all $x:\susp \bool$, we have $g(f(x))=x$.

  Now we show by induction that $f(g(x))=x$ for all $x:\Sn^1$.
  If $x\jdeq \base$, then $f(g(\base))\jdeq f(\north)\jdeq\base$, so we have $\refl{\base} : f(g(\base))=\base$.
  It remains to show that this equality is preserved as $x$ varies along $\lloop$, which is to say that it is transported along $\lloop$ to itself.
  Again, by transport in path spaces and pulled back fibrations, this means to show that
  \[ \opp{\ap f {\ap g {\lloop}}} \ct \refl{\base} \ct \lloop = \refl{\base}.\]
  However, we have
  \begin{align*}
    \ap f {\ap g {\lloop}} &= \ap f {\merid(\bfalse) \ct \opp{\merid(\btrue)}}\\
    &= \ap f {\merid(\bfalse)} \ct \opp{\ap f {\merid(\btrue)}}\\
    &= \lloop \ct \refl{\base}
  \end{align*}
  so this follows easily.
\end{proof}

Topologically, the two-point space \bool is also known as the \emph{0-dimensional sphere}, $\Sn^0$.
(For instance, it is the space of points at distance $1$ from the origin in $\mathbb{R}^1$, just as the topological 1-sphere is the space of points at distance $1$ from the origin in $\mathbb{R}^2$.)
Thus, \autoref{thm:suspbool} can be phrased suggestively as $\eqv{\susp\Sn^0}{\Sn^1}$.
\index{type!n-sphere@$n$-sphere|defstyle}%
\indexsee{n-sphere@$n$-sphere}{type, $n$-sphere}%
In fact, this pattern continues: we can define all the spheres inductively by
\begin{equation}\label{eq:Snsusp}
  \Sn^0 \defeq \bool
  \qquad\text{and}\qquad
  \Sn^{n+1} \defeq \susp \Sn^n.
\end{equation}
We can even start one dimension lower by defining $\Sn^{-1}\defeq \emptyt$, and observe that $\eqv{\susp\emptyt}{\bool}$.

To prove carefully that this agrees with the definition of $\Sn^n$ from the previous section would require making the latter more explicit.
However, we can show that the recursive definition has the same universal property that we would expect the other one to have.
If $(A,a_0)$ and $(B,b_0)$ are pointed types (with basepoints often left implicit), let $\Map_*(A,B)$ denote the type of based maps:
\index{based map}
\symlabel{based-maps}
\[ \Map_*(A,B) \defeq \sm{f:A\to B} (f(a_0)=b_0). \]
Note that any type $A$ gives rise to a pointed type $A_+ \defeq A+\unit$ with basepoint $\inr(\ttt)$; this is called \emph{adjoining a disjoint basepoint}.
\indexdef{basepoint!adjoining a disjoint}%
\index{disjoint!basepoint}%
\index{adjoining a disjoint basepoint}%

\begin{lem}
  For a type $A$ and a pointed type $(B,b_0)$, we have
  \[ \eqv{\Map_*(A_+,B)}{(A\to B)} \]
\end{lem}
Note that on the right we have the ordinary type of \emph{unbased} functions from $A$ to $B$.
\begin{proof}
  From left to right, given $f:A_+ \to B$ with $p:f(\inr(\ttt)) = b_0$, we have $f\circ \inl : A \to B$.
  And from right to left, given $g:A\to B$ we define $g':A_+ \to B$ by $g'(\inl(a))\defeq g(a)$ and $g'(\inr(u)) \defeq b_0$.
  We leave it to the reader to show that these are quasi-inverse operations.
\end{proof}

In particular, note that $\eqv{\bool}{\unit_+}$.
Thus, for any pointed type $B$ we have
\[{\Map_*(\bool,B)} \eqvsym {(\unit \to B)}\eqvsym B.\]
%
Now recall that the loop space\index{loop space} operation $\Omega$ acts on pointed types, with definition $\Omega(A,a_0) \defeq (\id[A]{a_0}{a_0},\refl{a_0})$.
We can also make the suspension $\susp$ act on pointed types, by $\susp(A,a_0)\defeq (\susp A,\north)$.

\begin{lem}\label{lem:susp-loop-adj}
  \index{universal!property!of suspension}%
  For pointed types $(A,a_0)$ and $(B,b_0)$ we have
  \[ \eqv{\Map_*(\susp A, B)}{\Map_*(A,\Omega B)}.\]
\end{lem}
\begin{proof}
We first observe the following chain of equivalences:
\begin{align*}
\Map_*(\susp A, B) & \defeq \sm{f:\susp A\to B} (f(\north)=b_0) \\
                   & \eqvsym \sm{f:\sm{b_n : B}{b_s : B} (A \to (b_n = b_s))} (\fst(f)=b_0) \\
                   & \eqvsym \sm{b_n : B}{b_s : B} \big(A \to (b_n = b_s)\big) \times (b_n=b_0) \\
                   & \eqvsym \sm{p : \sm{b_n : B} (b_n=b_0)}{b_s : B} (A \to (\fst(p) = b_s)) \\
                   & \eqvsym \sm{b_s : B} (A \to (b_0 = b_s))
\end{align*}
The first equivalence is by the universal property of suspensions, which says that
\[ \Parens{\susp A \to B} \eqvsym \Parens{\sm{b_n : B} \sm{b_s : B} (A \to (b_n = b_s)) } \]
with the function from right to left given by the recursor (see \autoref{ex:susp-lump}).
The second and third equivalences are by \autoref{ex:sigma-assoc}, along with a reordering of components.
Finally, the last equivalence follows from \autoref{thm:omit-contr}, since by \autoref{thm:contr-paths}, $\sm{b_n : B} (b_n=b_0)$ is contractible with center $(b_0, \refl{b_0})$.

The proof is now completed by the following chain of equivalences:
\begin{align*}
  \sm{b_s : B} (A \to (b_0 = b_s))
  &\eqvsym \sm{b_s : B}{g:A \to (b_0 = b_s)}{q:b_0 = b_s} (g(a_0) = q)\\
  &\eqvsym \sm{r : \sm{b_s : B}(b_0 = b_s)}{g:A \to (b_0 = \proj1(r))} (g(a_0) = \proj2(r))\\
  &\eqvsym \sm{g:A \to (b_0 = b_0)} (g(a_0) = \refl{b_0})\\
  &\jdeq \Map_*(A,\Omega B).
\end{align*}
Similar to before, the first and last equivalences are by \autoref{thm:omit-contr,thm:contr-paths}, and the second is by \autoref{ex:sigma-assoc} and reordering of components.
\end{proof}

\index{type!n-sphere@$n$-sphere|defstyle}%
In particular, for the spheres defined as in~\eqref{eq:Snsusp} we have
\index{universal!property!of Sn@of $\Sn^n$}%
\[ \Map_*(\Sn^n,B) \eqvsym \Map_*(\Sn^{n-1}, \Omega B) \eqvsym \cdots \eqvsym \Map_*(\bool,\Omega^n B) \eqvsym \Omega^n B. \]
Thus, these spheres $\Sn^n$ have the universal property that we would expect from the spheres defined directly in terms of $n$-fold loop spaces\index{loop space!iterated} as in \autoref{sec:circle}.

\index{suspension|)}%

\section{Cell complexes}
\label{sec:cell-complexes}

\index{cell complex|(defstyle}%
\index{CW complex|(defstyle}%
In classical topology, a \emph{cell complex} is a space obtained by successively attaching discs along their boundaries.
It is called a \emph{CW complex} if the boundary of an $n$-dimensional disc\index{disc} is constrained to lie in the discs of dimension strictly less than $n$ (the $(n-1)$-skeleton).\index{skeleton!of a CW-complex}

Any finite CW complex can be presented as a higher inductive type, by turning $n$-dimensional discs into $n$-dimensional paths and partitioning the image of the attaching\index{attaching map} map into a source\index{source!of a path constructor} and a target\index{target!of a path constructor}, with each written as a composite of lower dimensional paths.
Our explicit definitions of $\Sn^1$ and $\Sn^2$ in \autoref{sec:circle} had this form.

\index{torus}%
Another example is the torus $T^2$, which is generated by:
\begin{itemize}
\item a point $b:T^2$,
\item a path $p:b=b$,
\item another path $q:b=b$, and
\item a 2-path $t: p\ct q = q \ct p$.
\end{itemize}
Perhaps the easiest way to see that this is a torus is to start with a rectangle, having four corners $a,b,c,d$, four edges $p,q,r,s$, and an interior which is manifestly a 2-path $t$ from $p\ct q$ to $r\ct s$:
\begin{equation*}
  \xymatrix{
      a\ar@{=}[r]^p\ar@{=}[d]_r \ar@{}[dr]|{\Downarrow t} &
      b\ar@{=}[d]^q\\
      c\ar@{=}[r]_s &
      d
      }
\end{equation*}
Now identify the edge $r$ with $q$ and the edge $s$ with $p$, resulting in also identifying all four corners.
Topologically, this identification can be seen to produce a torus.

\index{induction principle!for torus}%
\index{torus!induction principle for}%
The induction principle for the torus is the trickiest of any we've written out so far.
Given $P:T^2\to\type$, for a section $\prd{x:T^2} P(x)$ we require
\begin{itemize}
\item a point $b':P(b)$,
\item a path $p' : \dpath P p {b'} {b'}$,
\item a path $q' : \dpath P q {b'} {b'}$, and
\item a 2-path $t'$ between the ``composites'' $p'\ct q'$ and $q'\ct p'$, lying over $t$.
\end{itemize}
In order to make sense of this last datum, we need a composition operation for dependent paths, but this is not hard to define.
Then the induction principle gives a function $f:\prd{x:T^2} P(x)$ such that $f(b)\jdeq b'$ and $\apd f {p} = p'$ and $\apd f {q} = q'$ and something like ``$\apdtwo f t = t'$''.
However, this is not well-typed as it stands, firstly because the equalities $\apd f {p} = p'$ and $\apd f {q} = q'$ are not judgmental, and secondly because $\apdfunc f$ only preserves path concatenation up to homotopy.
We leave the details to the reader (see \autoref{ex:torus}).

Of course, another definition of the torus is $T^2 \defeq \Sn^1 \times \Sn^1$ (in \autoref{ex:torus-s1-times-s1} we ask the reader to verify the equivalence of the two).
\index{Klein bottle}%
\index{projective plane}%
The cell-complex definition, however, generalizes easily to other spaces without such descriptions, such as the Klein bottle, the projective plane, etc.
But it does get increasingly difficult to write down the induction principles, requiring us to define notions of dependent $n$-paths and of $\apdfunc{}$ acting on $n$-paths.
Fortunately, once we have the spheres in hand, there is a way around this.

\section{Hubs and spokes}
\label{sec:hubs-spokes}

\indexsee{spoke}{hub and spoke}%
\index{hub and spoke|(defstyle}%

In topology, one usually speaks of building CW complexes by attaching $n$-dimensional discs along their $(n-1)$-dimensional boundary spheres.
\index{attaching map}%
However, another way to express this is by gluing in the \emph{cone}\index{cone!of a sphere} on an $(n-1)$-dimensional sphere.
That is, we regard a disc\index{disc} as consisting of a cone point (or ``hub''), with meridians
\index{meridian}%
(or ``spokes'') connecting that point to every point on the boundary, continuously, as shown in \autoref{fig:hub-and-spokes}.

\begin{figure}
  \centering
  \begin{tikzpicture}
    \draw (0,0) circle (2cm);
    \foreach \x in {0,20,...,350}
      \draw[\OPTblue] (0,0) -- (\x:2cm);
    \node[\OPTblue,circle,fill,inner sep=2pt] (hub) at (0,0) {};
  \end{tikzpicture}
  \caption{A 2-disc made out of a hub and spokes}
  \label{fig:hub-and-spokes}
\end{figure}

We can use this idea to express higher inductive types containing $n$-dimensional path-con\-struc\-tors for $n>1$ in terms of ones containing only 1-di\-men\-sion\-al path-con\-struc\-tors.
The point is that we can obtain an $n$-dimensional path as a continuous family of 1-dimensional paths parametrized by an $(n-1)$-di\-men\-sion\-al object.
The simplest $(n-1)$-dimensional object to use is the $(n-1)$-sphere, although in some cases a different one may be preferable.
(Recall that we were able to define the spheres in \autoref{sec:suspension} inductively using suspensions, which involve only 1-dimensional path constructors.
Indeed, suspension can also be regarded as an instance of this idea, since it involves a family of 1-dimensional paths parametrized by the type being suspended.)

\index{torus}
For instance, the torus $T^2$ from the previous section could be defined instead to be generated by:
\begin{itemize}
\item a point $b:T^2$,
\item a path $p:b=b$,
\item another path $q:b=b$,
\item a point $h:T^2$, and
\item for each $x:\Sn^1$, a path $s(x) : f(x)=h$, where $f:\Sn^1\to T^2$ is defined by $f(\base)\defeq b$ and $\ap f \lloop \defid p \ct q \ct \opp p \ct \opp q$.
\end{itemize}
The induction principle for this version of the torus says that given $P:T^2\to\type$, for a section $\prd{x:T^2} P(x)$ we require
\begin{itemize}
\item a point $b':P(b)$,
\item a path $p' : \dpath P p {b'} {b'}$,
\item a path $q' : \dpath P q {b'} {b'}$,
\item a point $h':P(h)$, and
\item for each $x:\Sn^1$, a path $\dpath {P}{s(x)}{g(x)}{h'}$, where $g:\prd{x:\Sn^1} P(f(x))$ is defined by $g(\base)\defeq b'$ and $\apd g \lloop \defid p' \ct q' \ct \opp{(p')} \ct \opp{(q')}$.
\end{itemize}
Note that there is no need for dependent 2-paths or $\apdtwofunc{}$.
We leave it to the reader to write out the computation rules.

\begin{rmk}\label{rmk:spokes-no-hub}
One might question the need for introducing the hub point $h$; why couldn't we instead simply add paths continuously relating the boundary of the disc to a point \emph{on} that boundary, as shown in \autoref{fig:spokes-no-hub}?
However, this does not work without further modification.
For if, given some $f:\Sn^1 \to X$, we give a path constructor connecting each $f(x)$ to $f(\base)$, then what we end up with is more like the picture in \autoref{fig:spokes-no-hub-ii} of a cone whose vertex is twisted around and glued to some point on its base.
The problem is that the specified path from $f(\base)$ to itself may not be reflexivity.
We could remedy the problem by adding a 2-dimensional path constructor to ensure this, but using a separate hub avoids the need for any path constructors of dimension above~$1$.
\end{rmk}

\begin{figure}
  \centering
  \begin{minipage}{2in}
    \begin{center}
      \begin{tikzpicture}
        \draw (0,0) circle (2cm);
        \clip (0,0) circle (2cm);
        \foreach \x in {0,15,...,165}
        \draw[\OPTblue] (0,-2cm) -- (\x:4cm);
      \end{tikzpicture}
    \end{center}
    \caption{Hubless spokes}
    \label{fig:spokes-no-hub}
  \end{minipage}
  \qquad
  \begin{minipage}{2in}
    \begin{center}
      \begin{tikzpicture}[xscale=1.3]
        \draw (0,0) arc (-90:90:.7cm and 2cm) ;
        \draw[dashed] (0,4cm) arc (90:270:.7cm and 2cm) ;
        \draw[\OPTblue] (0,0) to[out=90,in=0] (-1,1) to[out=180,in=180] (0,0);
        \draw[\OPTblue] (0,4cm) to[out=180,in=180,looseness=2] (0,0);
        \path (0,0) arc (-90:-60:.7cm and 2cm) node (a) {};
        \draw[\OPTblue] (a.center) to[out=120,in=10] (-1.2,1.2) to[out=190,in=180] (0,0);
        \path (0,0) arc (-90:-30:.7cm and 2cm) node (b) {};
        \draw[\OPTblue] (b.center) to[out=150,in=20] (-1.4,1.4) to[out=200,in=180] (0,0);
        \path (0,0) arc (-90:0:.7cm and 2cm) node (c) {};
        \draw[\OPTblue] (c.center) to[out=180,in=30] (-1.5,1.5) to[out=210,in=180] (0,0);
        \path (0,0) arc (-90:30:.7cm and 2cm) node (d) {};
        \draw[\OPTblue] (d.center) to[out=190,in=50] (-1.7,1.7) to[out=230,in=180] (0,0);
        \path (0,0) arc (-90:60:.7cm and 2cm) node (e) {};
        \draw[\OPTblue] (e.center) to[out=200,in=70] (-2,2) to[out=250,in=180] (0,0);
        \clip (0,0) to[out=90,in=0] (-1,1) to[out=180,in=180] (0,0);
        \draw (0,4cm) arc (90:270:.7cm and 2cm) ;
      \end{tikzpicture}
    \end{center}
    \caption{Hubless spokes, II}
    \label{fig:spokes-no-hub-ii}
  \end{minipage}
\end{figure}

\begin{rmk}
  \index{computation rule!propositional}%
  Note also that this ``translation'' of higher paths into 1-paths does not preserve judgmental computation rules for these paths, though it does preserve propositional ones.
\end{rmk}

\index{cell complex|)}%
\index{CW complex|)}%

\index{hub and spoke|)}%


\section{Pushouts}
\label{sec:colimits}

\index{type!limit}%
\index{type!colimit}%
\index{limit!of types}%
\index{colimit!of types}%
From a category-theoretic point of view, one of the important aspects of any foundational system is the ability to construct limits and colimits.
In set-theoretic foundations, these are limits and colimits of sets, whereas in our case they are limits and colimits of \emph{types}.
We have seen in \autoref{sec:universal-properties} that cartesian product types have the correct universal property of a categorical product of types, and in \autoref{ex:coprod-ump} that coproduct types likewise have their expected universal property.

As remarked in \autoref{sec:universal-properties}, more general limits can be constructed using identity types and $\Sigma$-types, e.g.\ the pullback\index{pullback} of $f:A\to C$ and $g:B\to C$ is $\sm{a:A}{b:B} (f(a)=g(b))$ (see \autoref{ex:pullback}).
However, more general \emph{colimits} require identifying elements coming from different types, for which higher inductives are well-adapted.
Since all our constructions are homotopy-invariant, all our colimits are necessarily \emph{homotopy colimits}, but we drop the ubiquitous adjective in the interests of concision.

In this section we discuss \emph{pushouts}, as perhaps the simplest and one of the most useful colimits.
Indeed, one expects all finite colimits (for a suitable homotopical definition of ``finite'') to be constructible from pushouts and finite coproducts.
It is also possible to give a direct construction of more general colimits using higher inductive types, but this is somewhat technical, and also not completely satisfactory since we do not yet have a good fully general notion of homotopy coherent diagrams.

\indexsee{type!pushout of}{pushout}%
\index{pushout|(defstyle}%
\index{span}%
Suppose given a span of types and functions:
\[\Ddiag=\;\vcenter{\xymatrix{C \ar^g[r] \ar_f[d] & B \\ A & }}\]
The \define{pushout} of this span is the higher inductive type $A\sqcup^CB$ presented by
\begin{itemize}
\item a function $\inl:A\to A\sqcup^CB$,
\item a function $\inr:B \to A\sqcup^CB$, and
\item for each $c:C$ a path $\glue(c):(\inl(f(c))=\inr(g(c)))$.
\end{itemize}
In other words, $A\sqcup^CB$ is the disjoint union of $A$ and $B$, together with for every $c:C$ a witness that $f(c)$ and $g(c)$ are equal.
The recursion principle says that if $D$ is another type, we can define a map $s:A\sqcup^CB\to{}D$ by defining
\begin{itemize}
\item for each $a:A$, the value of $s(\inl(a)):D$,
\item for each $b:B$, the value of $s(\inr(b)):D$, and
\item for each $c:C$, the value of $\mapfunc{s}(\glue(c)):s(\inl(f(c)))=s(\inr(g(c)))$.
\end{itemize}
We leave it to the reader to formulate the induction principle.
It also implies the uniqueness principle that if $s,s':A\sqcup^CB\to{}D$ are two maps such that
\index{uniqueness!principle, propositional!for functions on a pushout}%
\begin{align*}
  s(\inl(a))&=s'(\inl(a))\\
  s(\inr(b))&=s'(\inr(b))\\
  \mapfunc{s}(\glue(c))&=\mapfunc{s'}(\glue(c))
  \qquad\text{(modulo the previous two equalities)}
\end{align*}
for every $a,b,c$, then $s=s'$.

To formulate the universal property of a pushout, we introduce the following.

\begin{defn}\label{defn:cocone}
  Given a span $\Ddiag= (A \xleftarrow{f} C \xrightarrow{g} B)$ and a type $D$, a \define{cocone under $\Ddiag$ with vertex $D$}
  \indexdef{cocone}%
  \index{vertex of a cocone}%
  consists of functions $i:A\to{}D$ and $j:B\to{}D$ and a homotopy $h : \prd{c:C} (i(f(c))=j(g(c)))$:
  \[\uppercurveobject{{ }}\lowercurveobject{{ }}\twocellhead{{ }}
  \xymatrix{C \ar^g[r] \ar_f[d] \drtwocell{^h} & B \ar^j[d] \\ A \ar_i[r] & D
  }\]
  We denote by $\cocone{\Ddiag}{D}$ the type of all such cocones, i.e.
  \[ \cocone{\Ddiag}{D} \defeq
  \sm{i:A\to D}{j:B\to D} \prd{c:C} (i(f(c))=j(g(c))).
  \]
\end{defn}

Of course, there is a canonical cocone under $\Ddiag$ with vertex $A\sqcup^C B$ consisting of $\inl$, $\inr$, and $\glue$.
\[\uppercurveobject{{ }}\lowercurveobject{{ }}\twocellhead{{ }}
\xymatrix{C \ar^g[r] \ar_f[d] \drtwocell{^\glue\ \ } & B \ar^\inr[d] \\
  A \ar_-\inl[r] & A\sqcup^CB }\]
The following lemma says that this is the universal such cocone.

\begin{lem}\label{thm:pushout-ump}
  \index{universal!property!of pushout}%
  For any type $E$, there is an equivalence
  \[ (A\sqcup^C B \to E) \;\eqvsym\; \cocone{\Ddiag}{E}. \]
\end{lem}
\begin{proof}
  Let's consider an arbitrary type $E:\type$.
  There is a canonical function
  \[\function{(A\sqcup^CB\to{}E)}{\cocone{\Ddiag}{E}}
  {t}{\composecocone{t}c_\sqcup}\]
  defined by sending $(i,j,h)$ to $(t\circ{}i,t\circ{}j,\mapfunc{t}\circ{}h)$.
  We show that this is an equivalence.

  Firstly, given a $c=(i,j,h):\cocone{\mathscr{D}}{E}$, we need to construct a
  map $\mathsf{s}(c)$ from $A\sqcup^CB$ to $E$.
  \[\uppercurveobject{{ }}\lowercurveobject{{ }}\twocellhead{{ }}
  \xymatrix{C \ar^g[r] \ar_f[d] \drtwocell{^h} & B \ar^{j}[d] \\
    A \ar_-{i}[r] & E }\]
 The map $\mathsf{s}(c)$ is defined in the following way
  \begin{align*}
    \mathsf{s}(c)(\inl(a))&\defeq i(a),\\
    \mathsf{s}(c)(\inr(b))&\defeq j(b),\\
    \mapfunc{\mathsf{s}(c)}(\glue(x))&\defid h(x).
  \end{align*}
We have defined a map
\[\function{\cocone{\Ddiag}{E}}{(A\sqcup^BC\to{}E)}{c}{\mathsf{s}(c)}\]
and we need to prove that this map is an inverse to
$t\mapsto{}\composecocone{t}c_\sqcup$.
On the one hand, if $c=(i,j,h):\cocone{\Ddiag}{E}$, we have
\begin{align*}
  \composecocone{\mathsf{s}(c)}c_\sqcup & =
  (\mathsf{s}(c)\circ\inl,\mathsf{s}(c)\circ\inr,
  \mapfunc{\mathsf{s}(c)}\circ\glue) \\
  & = (\lamu{a:A} \mathsf{s}(c)(\inl(a)),\;
  \lamu{b:B} \mathsf{s}(c)(\inr(b)),\;
  \lamu{x:C} \mapfunc{\mathsf{s}(c)}(\glue(x))) \\
  & = (\lamu{a:A} i(a),\;
  \lamu{b:B} j(b),\;
  \lamu{x:C} h(x)) \\
  & \jdeq (i, j, h) \\
  & = c.
\end{align*}
%
On the other hand, if $t:A\sqcup^BC\to{}E$, we want to prove that
$\mathsf{s}(\composecocone{t}c_\sqcup)=t$.
For $a:A$, we have
\[\mathsf{s}(\composecocone{t}c_\sqcup)(\inl(a))=t(\inl(a))\]
because the first component of $\composecocone{t}c_\sqcup$ is $t\circ\inl$. In
the same way, for $b:B$ we have
\[\mathsf{s}(\composecocone{t}c_\sqcup)(\inr(b))=t(\inr(b))\]
and for $x:C$ we have
\[\mapfunc{\mathsf{s}(\composecocone{t}c_\sqcup)}(\glue(x))
=\mapfunc{t}(\glue(x))\]
hence $\mathsf{s}(\composecocone{t}c_\sqcup)=t$.

This proves that $c\mapsto\mathsf{s}(c)$ is a quasi-inverse to $t\mapsto{}\composecocone{t}c_\sqcup$, as desired.
\end{proof}

A number of standard homotopy-theoretic constructions can be expressed as (homotopy) pushouts.
\begin{itemize}
\item The pushout of the span $\unit \leftarrow A \to \unit$ is the \define{suspension} $\susp A$ (see \autoref{sec:suspension}).%
  \index{suspension}
\symlabel{join}
\item The pushout of $A \xleftarrow{\proj1} A\times B \xrightarrow{\proj2} B$ is called the \define{join} of $A$ and $B$, written $A*B$.%
  \indexdef{join!of types}
\item The pushout of $\unit \leftarrow A \xrightarrow{f} B$ is the \define{cone} or \define{cofiber} of $f$.%
  \indexdef{cone!of a function}%
  \indexsee{mapping cone}{cone of a function}%
  \indexdef{cofiber of a function}%
\symlabel{wedge}
\item If $A$ and $B$ are equipped with basepoints $a_0:A$ and $b_0:B$, then the pushout of $A \xleftarrow{a_0} \unit \xrightarrow{b_0} B$ is the \define{wedge} $A\vee B$.%
  \indexdef{wedge}
\symlabel{smash}
\item If $A$ and $B$ are pointed as before, define $f:A\vee B \to A\times B$ by $f(\inl(a))\defeq (a,b_0)$ and $f(\inr(b))\defeq (a_0,b)$, with $\ap f \glue \defid \refl{(a_0,b_0)}$.
  Then the cone of $f$ is called the \define{smash product} $A\wedge B$.%
  \indexdef{smash product}
\end{itemize}
We will discuss pushouts further in \autoref{cha:hlevels,cha:homotopy}.

\begin{rmk}
  As remarked in \autoref{subsec:prop-trunc}, the notations $\wedge$ and $\vee$ for the smash product and wedge of pointed spaces are also used in logic for ``and'' and ``or'', respectively.
  Since types in homotopy type theory can behave either like spaces or like propositions, there is technically a potential for conflict --- but since they rarely do both at once, context generally disambiguates.
  Furthermore, the smash product and wedge only apply to \emph{pointed} spaces, while the only pointed mere proposition is $\top\jdeq\unit$ --- and we have $\unit\wedge \unit = \unit$ and $\unit\vee\unit=\unit$ for either meaning of $\wedge$ and $\vee$.
\end{rmk}

\index{pushout|)}%

\begin{rmk}
  Note that colimits do not in general preserve truncatedness.
  For instance, $\Sn^0$ and \unit are both sets, but the pushout of $\unit \leftarrow \Sn^0 \to \unit$ is $\Sn^1$, which is not a set.
  If we are interested in colimits in the category of $n$-types, therefore (and, in particular, in the category of sets), we need to ``truncate'' the colimit somehow.
  We will return to this point in \autoref{sec:hittruncations,cha:hlevels,cha:set-math}.
\end{rmk}


\section{Truncations}
\label{sec:hittruncations}

\index{truncation!propositional|(}%
In \autoref{subsec:prop-trunc} we introduced the propositional truncation as a new type forming operation;
we now observe that it can be obtained as a special case of higher inductive types.
This reduces the problem of understanding truncations to the problem of understanding higher inductives, which at least are amenable to a systematic treatment.
It is also interesting because it provides our first example of a higher inductive type which is truly \emph{recursive}, in that its constructors take inputs from the type being defined (as does the successor $\suc:\nat\to\nat$).

Let $A$ be a type; we define its propositional truncation $\brck A$ to be the higher inductive type generated by:
\begin{itemize}
\item A function $\bprojf : A \to \brck A$, and
\item For each $x,y:\brck A$, a path $x=y$.
\end{itemize}
Note that the second constructor is by definition the assertion that $\brck A$ is a mere proposition.
Thus, the definition of $\brck A$ can be interpreted as saying that $\brck A$ is freely generated by a function $A\to\brck A$ and the fact that it is a mere proposition.

The recursion principle for this higher inductive definition is easy to write down: it says that given any type $B$ together with
\begin{itemize}
\item A function $g:A\to B$, and
\item For any $x,y:B$, a path $x=_B y$,
\end{itemize}
there exists a function $f:\brck A \to B$ such that
\begin{itemize}
\item $f(\bproj a) \jdeq g(a)$ for all $a:A$, and
\item for any $x,y:\brck A$, the function $\apfunc f$ takes the specified path $x=y$ in $\brck A$ to the specified path $f(x) = f(y)$ in $B$ (propositionally).
\end{itemize}
\index{recursion principle!for truncation}%
These are exactly the hypotheses that we stated in \autoref{subsec:prop-trunc} for the recursion principle of propositional truncation --- a function $A\to B$ such that $B$ is a mere proposition --- and the first part of the conclusion is exactly what we stated there as well.
The second part (the action of $\apfunc f$) was not mentioned previously, but it turns out to be vacuous in this case, because $B$ is a mere proposition, so \emph{any} two paths in it are automatically equal.

\index{induction principle!for truncation}%
There is also an induction principle for $\brck A$, which says that given any $B:\brck A \to \type$ together with
\begin{itemize}
\item a function $g:\prd{a:A} B(\bproj a)$, and
\item for any $x,y:\brck A$ and $u:B(x)$ and $v:B(y)$, a dependent path $q:\dpath{B}{p(x,y)}{u}{v}$, where $p(x,y)$ is the path coming from the second constructor of $\brck A$,
\end{itemize}
there exists $f:\prd{x:\brck A} B(x)$ such that $f(\bproj a)\jdeq g(a)$ for $a:A$, and also another computation rule.
However, because there can be at most one function between any two mere propositions (up to homotopy), this induction principle is not really useful (see also \autoref{ex:prop-trunc-ind}).

\index{truncation!propositional|)}%
\index{truncation!set|(}%

\index{set|(}%
We can, however, extend this idea to construct similar truncations landing in $n$-types, for any $n$.
For instance, we might define the \emph{0-trun\-ca\-tion} $\trunc0A$ to be generated by
\begin{itemize}
\item A function $\tprojf0 : A \to \trunc0 A$, and
\item For each $x,y:\trunc0A$ and each $p,q:x=y$, a path $p=q$.
\end{itemize}
Then $\trunc0A$ would be freely generated by a function $A\to \trunc0A$ together with the assertion that $\trunc0A$ is a set.
A natural induction principle for it would say that given $B:\trunc0 A \to \type$ together with
\begin{itemize}
\item a function $g:\prd{a:A} B(\tproj0a)$, and
\item for any $x,y:\trunc0A$ with $z:B(x)$ and $w:B(y)$, and each $p,q:x=y$ with $r:\dpath{B}{p}{z}{w}$ and $s:\dpath{B}{q}{z}{w}$, a 2-path $v:\dpath{\dpath{B}{-}{z}{w}}{u(x,y,p,q)}{r}{s}$, where $u(x,y,p,q):p=q$ is obtained from the second constructor of $\trunc0A$,
\end{itemize}
there exists $f:\prd{x:\trunc0A} B(x)$ such that $f(\tproj0a)\jdeq g(a)$ for all $a:A$, and also $\apdtwo{f}{u(x,y,p,q)}$ is the 2-path specified above.
(As in the propositional case, the latter condition turns out to be uninteresting.)
From this, however, we can prove a more useful induction principle.

\begin{lem}\label{thm:trunc0-ind}
  Suppose given $B:\trunc0 A \to \type$ together with $g:\prd{a:A} B(\tproj0a)$, and assume that each $B(x)$ is a set.
  Then there exists $f:\prd{x:\trunc0A} B(x)$ such that $f(\tproj0a)\jdeq g(a)$ for all $a:A$.
\end{lem}
\begin{proof}
  It suffices to construct, for any $x,y,z,w,p,q,r,s$ as above, a 2-path $v:\dpath{B}{u(x,y,p,q)}{r}{s}$.
  However, by the definition of dependent 2-paths, this is an ordinary 2-path in the fiber $B(y)$.
  Since $B(y)$ is a set, a 2-path exists between any two parallel paths.
\end{proof}

This implies the expected universal property.

\begin{lem}\label{thm:trunc0-lump}
  \index{universal!property!of truncation}%
  For any set $B$ and any type $A$, composition with $\tprojf0:A\to \trunc0A$ determines an equivalence
  \[ \eqvspaced{(\trunc0A\to B)}{(A\to B)}. \]
\end{lem}
\begin{proof}
  The special case of \autoref{thm:trunc0-ind} when $B$ is the constant family gives a map from right to left, which is a right inverse to the ``compose with $\tprojf0$'' function from left to right.
  To show that it is also a left inverse, let $h:\trunc0A\to B$, and define $h':\trunc0A\to B$ by applying \autoref{thm:trunc0-ind} to the composite $a\mapsto h(\tproj0a)$.
  Thus, $h'(\tproj0a)=h(\tproj0a)$.

  However, since $B$ is a set, for any $x:\trunc0A$ the type $h(x)=h'(x)$ is a mere proposition, and hence also a set.
  Therefore, by \autoref{thm:trunc0-ind}, the observation that $h'(\tproj0a)=h(\tproj0a)$ for any $a:A$ implies $h(x)=h'(x)$ for any $x:\trunc0A$, and hence $h=h'$.
\end{proof}

\index{limit!of sets}%
\index{colimit!of sets}%
For instance, this enables us to construct colimits of sets.
We have seen that if $A \xleftarrow{f} C \xrightarrow{g} B$ is a span of sets, then the pushout $A\sqcup^C B$ may no longer be a set.
(For instance, if $A$ and $B$ are \unit and $C$ is \bool, then the pushout is $\Sn^1$.)
However, we can construct a pushout that is a set, and has the expected universal property with respect to other sets, by truncating.

\begin{lem}\label{thm:set-pushout}
  \index{universal!property!of pushout}%
  Let $A \xleftarrow{f} C \xrightarrow{g} B$ be a span\index{span} of sets.
  Then for any set $E$, there is a canonical equivalence
  \[ \Parens{\trunc0{A\sqcup^C B} \to E} \;\eqvsym\; \cocone{\Ddiag}{E}. \]
\end{lem}
\begin{proof}
  Compose the equivalences in \autoref{thm:pushout-ump,thm:trunc0-lump}.
\end{proof}

We refer to $\trunc0{A\sqcup^C B}$ as the \define{set-pushout}
\indexdef{set-pushout}%
\index{pushout!of sets}
of $f$ and $g$, to distinguish it from the (homotopy) pushout $A\sqcup^C B$.
Alternatively, we could modify the definition of the pushout in \autoref{sec:colimits} to include the $0$-truncation constructor directly, avoiding the need to truncate afterwards.
Similar remarks apply to any sort of colimit of sets; we will explore this further in \autoref{cha:set-math}.

However, while the above definition of the 0-truncation works --- it gives what we want, and is consistent --- it has a couple of issues.
Firstly, it doesn't fit so nicely into the general theory of higher inductive types.
In general, it is tricky to deal directly with constructors such as the second one we have given for $\trunc0A$, whose \emph{inputs} involve not only elements of the type being defined, but paths in it.

This can be gotten round fairly easily, however.
Recall in \autoref{sec:bool-nat} we mentioned that we can allow a constructor of an inductive type $W$ to take ``infinitely many arguments'' of type $W$ by having it take a single argument of type $\nat\to W$.
There is a general principle behind this: to model a constructor with funny-looking inputs, use an auxiliary inductive type (such as \nat) to parametrize them, reducing the input to a simple function with inductive domain.

For the 0-truncation, we can consider the auxiliary \emph{higher} inductive type $S$ generated by two points $a,b:S$ and two paths $p,q:a=b$.
Then the fishy-looking constructor of $\trunc 0A$ can be replaced by the unobjectionable
\begin{itemize}
\item For every $f:S\to A$, a path $\apfunc{f}(p) = \apfunc{f}(q)$.
\end{itemize}
Since to give a map out of $S$ is the same as to give two points and two parallel paths between them, this yields the same induction principle.

\index{set|)}%

\index{truncation!set|)}%
\index{truncation!n-truncation@$n$-truncation}%
A more serious problem with our current definition of $0$-truncation, however, is that it doesn't generalize very well.
If we want to describe a notion of definition of ``$n$-truncation'' into $n$-types uniformly for all $n:\nat$, then this approach is unfeasible, since the second constructor would need a number of arguments that increases with $n$.
In \autoref{sec:truncations}, therefore, we will use a different idea to construct these, based on the observation that the type $S$ introduced above is equivalent to the circle $\Sn^1$.
This includes the 0-truncation as a special case, and satisfies generalized versions of \autoref{thm:trunc0-ind,thm:trunc0-lump}.


\section{Quotients}
\label{sec:set-quotients}

A particularly important sort of colimit of sets is the \emph{quotient} by a relation.
That is, let $A$ be a set and $R:A\times A \to \prop$ a family of mere propositions (a \define{mere relation}).
\indexdef{relation!mere}%
\indexdef{mere relation}%
Its quotient should be the set-coequalizer of the two projections
\[ \tsm{a,b:A} R(a,b) \rightrightarrows A. \]
We can also describe this directly, as the higher inductive type $A/R$ generated by
\index{set-quotient|(defstyle}%
\indexsee{quotient of sets}{set-quotient}%
\indexsee{type!quotient}{set-quotient}%
\begin{itemize}
\item A function $q:A\to A/R$;
\item For each $a,b:A$ such that $R(a,b)$, an equality $q(a)=q(b)$; and
\item The $0$-truncation constructor: for all $x,y:A/R$ and $r,s:x=y$, we have $r=s$.
\end{itemize}
We will sometimes refer to this higher inductive type $A/R$ as the \define{set-quotient} of $A$ by $R$, to emphasize that it produces a set by definition.
(There are more general notions of ``quotient'' in homotopy theory, but they are mostly beyond the scope of this book.
However, in \autoref{sec:rezk} we will consider the ``quotient'' of a type by a 1-groupoid, which is the next level up from set-quotients.)

\begin{rmk}
  It is not actually necessary for the definition of set-quotients, and most of their properties, that $A$ be a set.
  However, this is generally the case of most interest.
\end{rmk}

\begin{lem}\label{thm:quotient-surjective}
  The function $q:A\to A/R$ is surjective.
\end{lem}
\begin{proof}
  We must show that for any $x:A/R$ there merely exists an $a:A$ with $q(a)=x$.
  We use the induction principle of $A/R$.
  The first case is trivial: if $x$ is $q(a)$, then of course there merely exists an $a$ such that $q(a)=q(a)$.
  And since the goal is a mere proposition, it automatically respects all path constructors, so we are done.
\end{proof}

We can now prove that the set-quotient has the expected universal property of a (set-)coequalizer.

\begin{lem}\label{thm:quotient-ump}
  For any set $B$, precomposing with $q$ yields an equivalence
  \[ \eqvspaced{(A/R \to B)}{\Parens{\sm{f:A\to B} \prd{a,b:A} R(a,b) \to (f(a)=f(b))}}.\]
\end{lem}
\begin{proof}
  The quasi-inverse of $\blank\circ q$, going from right to left, is just the recursion principle for $A/R$.
  That is, given $f:A\to B$ such that
  \narrowequation{\prd{a,b:A} R(a,b) \to (f(a)=f(b)),} we define $\bar f:A/R\to B$ by $\bar f(q(a))\defeq f(a)$.
  This defining equation says precisely that $(f\mapsto \bar f)$ is a right inverse to $(\blank\circ q)$.

  For it to also be a left inverse, we must show that for any $g:A/R\to B$ and $x:A/R$ we have $g(x) = \overline{g\circ q}$.
  However, by \autoref{thm:quotient-surjective} there merely exists $a$ such that $q(a)=x$.
  Since our desired equality is a mere proposition, we may assume there purely exists such an $a$, in which case $g(x) = g(q(a)) = \overline{g\circ q}(q(a)) = \overline{g\circ q}(x)$.
\end{proof}

Of course, classically the usual case to consider is when $R$ is an \define{equivalence relation}, i.e.\ we have
\indexdef{relation!equivalence}%
\indexsee{equivalence!relation}{relation, equivalence}%
%
\begin{itemize}
\item \define{reflexivity}: $\prd{a:A} R(a,a)$,
  \indexdef{reflexivity!of a relation}%
  \indexdef{relation!reflexive}%
\item \define{symmetry}: $\prd{a,b:A} R(a,b) \to R(b,a)$, and
  \indexdef{symmetry!of a relation}%
  \indexdef{relation!symmetric}%
\item \define{transitivity}: $\prd{a,b,c:C} R(a,b) \times R(b,c) \to R(a,c)$.
  \indexdef{transitivity!of a relation}%
  \indexdef{relation!transitive}%
\end{itemize}
%
In this case, the set-quotient $A/R$ has additional good properties, as we will see in \autoref{sec:piw-pretopos}: for instance, we have $R(a,b) \eqvsym (\id[A/R]{q(a)}{q(b)})$.
\symlabel{equivalencerelation}
We often write an equivalence relation $R(a,b)$ infix as $a\eqr b$.

The quotient by an equivalence relation can also be constructed in other ways.
The set theoretic approach is to consider the set of equivalence classes, as a subset of the power set\index{power set} of $A$.
We can mimic this ``impredicative'' construction in type theory as well.
\index{impredicative!quotient}

\begin{defn}
  A predicate $P:A\to\prop$ is an \define{equivalence class}
  \indexdef{equivalence!class}%
  of a relation $R : A \times A \to \prop$ if there merely exists an $a:A$ such that for all $b:A$ we have $\eqv{R(a,b)}{P(b)}$.
\end{defn}

As $R$ and $P$ are mere propositions, the equivalence $\eqv{R(a,b)}{P(b)}$ is the same thing as implications $R(a,b) \to P(b)$ and $P(b) \to R(a,b)$.
And of course, for any $a:A$ we have the canonical equivalence class $P_a(b) \defeq R(a,b)$.

\begin{defn}\label{def:VVquotient}
  We define
  \begin{equation*}
    A\sslash R \defeq \setof{ P:A\to\prop | P \text{ is an equivalence class of } R}.
  \end{equation*}
  The function $q':A\to A\sslash R$ is defined by $q'(a) \defeq P_a$.
\end{defn}

\begin{thm}
  For any equivalence relation $R$ on $A$, the type $A\sslash R$ is equivalent to the set-quotient $A/R$.
\end{thm}
\begin{proof}
  First, note that if $R(a,b)$, then since $R$ is an equivalence relation we have $R(a,c) \Leftrightarrow R(b,c)$ for any $c:A$.
  Thus, $R(a,c) = R(b,c)$ by univalence, hence $P_a=P_b$ by function extensionality, i.e.\ $q'(a)=q'(b)$.
  Therefore, by \autoref{thm:quotient-ump} we have an induced map $f:A/R \to A\sslash R$ such that $f\circ q = q'$.

  We show that $f$ is injective and surjective, hence an equivalence.
  Surjectivity follows immediately from the fact that $q'$ is surjective, which in turn is true essentially by definition of $A\sslash R$.
  For injectivity, if $f(x)=f(y)$, then to show the mere proposition $x=y$, by surjectivity of $q$ we may assume $x=q(a)$ and $y=q(b)$ for some $a,b:A$.
  Then $R(a,c) = f(q(a))(c) = f(q(b))(c) = R(b,c)$ for any $c:A$, and in particular $R(a,b) = R(b,b)$.
  But $R(b,b)$ is inhabited, since $R$ is an equivalence relation, hence so is $R(a,b)$.
  Thus $q(a)=q(b)$ and so $x=y$.
\end{proof}

In \autoref{subsec:quotients} we will give an alternative proof of this theorem.
Note that unlike $A/R$, the construction $A\sslash R$ raises universe level: if $A:\UU_i$ and $R:A\to A\to \prop_{\UU_i}$, then in the definition of $A\sslash R$ we must also use $\prop_{\UU_i}$ to include all the equivalence classes, so that $A\sslash R : \UU_{i+1}$.
Of course, we can avoid this if we assume the propositional resizing axiom from \autoref{subsec:prop-subsets}.

\begin{rmk}\label{defn-Z}
The previous two constructions provide quotients in generality, but in particular cases there may be easier constructions.
For instance, we may define the integers \Z as a set-quotient
\indexdef{integers}%
\indexdef{number!integers}%
%
\[ \Z \defeq (\N \times \N)/{\eqr} \]
%
where $\eqr$ is the equivalence relation defined by
%
\[ (a,b) \eqr (c,d) \defeq (a + d = b + c). \]
%
In other words, a pair $(a,b)$ represents the integer $a - b$.
In this case, however, there are \emph{canonical representatives} of the equivalence classes: those of the form $(n,0)$ or $(0,n)$.
\end{rmk}

The following lemma says that when this sort of thing happens, we don't need either general construction of quotients.
(A function $r:A\to A$ is called \define{idempotent}
\indexdef{function!idempotent}%
\indexdef{idempotent!function}%
if $r\circ r = r$.)

\begin{lem}\label{lem:quotient-when-canonical-representatives}
  Suppose $\eqr$ is a relation on a set $A$, and there exists an idempotent $r
  : A \to A$ such that $\eqv{(r(x) = r(y))}{(x \eqr y)}$ for all $x, y: A$.
  (This implies $\eqr$ is an equivalence relation.)
  Then the type
  %
  \begin{equation*}
    (A/{\eqr}) \defeq \Parens{\sm{x : A} r(x) = x}
  \end{equation*}
  %
  satisfies the universal property of the set-quotient of $A$ by~$\eqr$, and hence is equivalent to it.
  In other words, there is a map $q : A \to (A/{\eqr})$ such that for every set $B$, precomposition with $q$ induces an equivalence
  %
  \begin{equation}
    \label{eq:quotient-when-canonical}
    \Parens{(A/{\eqr}) \to B} \eqvsym \Parens{\sm{g : A \to B} \prd{x, y : A} (x \eqr y) \to (g(x) = g(y))}.
  \end{equation}
\end{lem}

\begin{proof}
  Let $i : \prd{x : A} r(r(x)) = r(x)$ witness idempotence of~$r$.
  The map $q : A \to (A/{\eqr})$ is defined by $q(x) \defeq (r(x), i(x))$.
  Note that since $A$ is a set, we have $q(x)=q(y)$ if and only if $r(x)=r(y)$, hence (by assumption) if and only if $x \eqr y$.
  We define a map $e$ from left to right in~\eqref{eq:quotient-when-canonical} by
  \[ e(f) \defeq (f \circ q, \nameless), \]
  %
  where the underscore $\nameless$ denotes the following proof: if $x, y : A$ and $x \eqr y$, then $q(x)=q(y)$ as observed above, hence $f(q(x)) = f(q(y))$.
  To see that $e$ is an equivalence, consider the map $e'$ in the opposite direction defined by
  %
  \[ e'(g, s) (x,p) \defeq g(x). \]
  %
  Given any $f : (A/{\eqr}) \to B$,
  %
  \[ e'(e(f))(x, p) \jdeq f(q(x)) \jdeq f(r(x), i(x)) = f(x, p) \]
  %
  where the last equality holds because $p : r(x) = x$ and so $(x,p) = (r(x), i(x))$
  because $A$ is a set. Similarly we compute
  %
  \[ e(e'(g, s)) \jdeq e(g \circ \proj{1}) \jdeq (g \circ \proj{1} \circ q, {\nameless}). \]
  %
  Because $B$ is a set we need not worry about the $\nameless$ part, while for the first
  component we have
  %
  \[ g(\proj{1}(q(x))) \jdeq g(r(x)) = g(x), \]
  %
  where the last equation holds because $r(x) \eqr x$, and $g$ respects $\eqr$ by
  the assumption $s$.
\end{proof}

\begin{cor}\label{thm:retraction-quotient}
  Suppose $p:A\to B$ is a retraction between sets.
  Then $B$ is the quotient of $A$ by the equivalence relation $\eqr$ defined by
  \[ (a_1 \eqr a_2) \defeq (p(a_1) = p(a_2)). \]
\end{cor}
\begin{proof}
  Suppose $s:B\to A$ is a section of $p$.
  Then $s\circ p : A\to A$ is an idempotent which satisfies the condition of \autoref{lem:quotient-when-canonical-representatives} for this $\eqr$, and $s$ induces an isomorphism from $B$ to its set of fixed points.
\end{proof}

\begin{rmk}\label{Z-quotient-by-canonical-representatives}
\autoref{lem:quotient-when-canonical-representatives} applies to $\Z$ with the idempotent $r : \N \times \N \to \N \times \N$
defined by
%
\begin{equation*}
  r(a, b) =
  \begin{cases}
    (a - b, 0) & \text{if $a \geq b$,} \\
    (0, b - a) & \text{otherwise.}
  \end{cases}
\end{equation*}
%
(This is a valid definition even constructively, since the relation $\geq$ on $\N$ is decidable.)
Thus a non-negative integer is canonically represented as $(k, 0)$ and a non-positive one by $(0, m)$, for $k,m:\N$.
This division into cases implies the following ``induction principle'' for integers, which will be useful in \autoref{cha:homotopy}.
\index{natural numbers}%
(As usual, we identify a natural number $n$ with the corresponding non-negative integer, i.e.\ with the image of $(n,0):\N\times\N$ in $\Z$.)
\end{rmk}

\begin{lem}\label{thm:sign-induction}
  \index{integers!induction principle for}%
  \index{induction principle!for integers}%
  Suppose $P:\Z\to\type$ is a type family and that we have
  \begin{itemize}
  \item $d_0: P(0)$,
  \item $d_+: \prd{n:\N} P(n) \to P(\suc(n))$, and
  \item $d_- : \prd{n:\N} P(-n) \to P(-\suc(n))$.
  \end{itemize}
  Then we have $f:\prd{z:\Z} P(z)$ such that
  \begin{itemize}
    \item $f(0) = d_0$,
    \item $f(\suc(n)) = d_+(n,f(n))$ for all $n:\N$, and
    \item $f(-\suc(n)) = d_-(n,f(-n))$ for all $n:\N$.
  \end{itemize}
\end{lem}
\begin{proof}
  For purposes of this proof, let $\Z$ denote $\sm{x:\N\times\N}(r(x)=x)$, where $r$ is the above idempotent.
  (We can then transport the result to any equivalent definition of $\Z$.)
  Let $q:\N\times\N\to\Z$ be the quotient map, defined by $q(x) = (r(x),i(x))$ as in \autoref{lem:quotient-when-canonical-representatives}.
  Now define $Q\defeq P\circ q:\N\times \N \to \type$.
  By transporting the given data across appropriate equalities, we obtain
  \begin{align*}
    d'_0 &: Q(0,0)\\
    d'_+ &: \prd{n:\N} Q(n,0) \to Q(\suc(n),0)\\
    d'_- &: \prd{n:\N} Q(0,n) \to Q(0,\suc(n)).
  \end{align*}
  Note also that since $q(n,m) = q(\suc(n),\suc(m))$, we have an induced equivalence
  \[e_{n,m}:\eqv{Q(n,m)}{Q(\suc(n),\suc(m))}.\]
  We can then construct $g:\prd{x:\N\times \N} Q(x)$ by double induction on $x$:
  \begin{align*}
    g(0,0) &\defeq d'_0,\\
    g(\suc(n),0) &\defeq d'_+(n,g(n,0)),\\
    g(0,\suc(m)) &\defeq d'_-(m,g(0,m)),\\
    g(\suc(n),\suc(m)) &\defeq e_{n,m}(g(n,m)).
  \end{align*}
  Now we have $\proj1 : \Z \to \N\times\N$, with the property that $q\circ \proj1 = \idfunc$.
  In particular, therefore, we have $Q\circ \proj1 = P$, and hence a family of equivalences $s:\prd{z:\Z} \eqv{Q(\proj1(z))}{P(z)}$.
  Thus, we can define $f(z) = s(z,g(\proj1(z)))$ to obtain $f:\prd{z:\Z} P(z)$, and verify the desired equalities.
\end{proof}

We will sometimes denote a function $f:\prd{z:\Z} P(z)$ obtained from \autoref{thm:sign-induction} with a pattern-matching syntax, involving the three cases $d_0$, $d_+$, and $d_-$:
\begin{align*}
  f(0) &\defid d_0\\
  f(\suc(n)) &\defid d_+(n,f(n))\\
  f(-\suc(n)) &\defid d_-(n,f(-n))
\end{align*}
We use $\defid$ rather than $\defeq$, as we did for the path-constructors of higher inductive types, to indicate that the ``computation'' rules implied by \autoref{thm:sign-induction} are only propositional equalities.
For example, in this way we can define the $n$-fold concatenation of a loop for any integer $n$.

\begin{cor}\label{thm:looptothe}
  \indexdef{path!concatenation!n-fold@$n$-fold}%
  Let $A$ be a type with $a:A$ and $p:a=a$.
  There is a function $\prd{n:\Z} (a=a)$, denoted $n\mapsto p^n$, defined by
  \begin{align*}
    p^0 &\defid \refl{a}\\
    p^{n+1} &\defid p^n \ct p
    & &\text{for $n\ge 0$}\\
    p^{n-1} &\defid p^n \ct \opp p
    & &\text{for $n\le 0$.}
  \end{align*}
\end{cor}

We will discuss the integers further in \autoref{sec:free-algebras,sec:field-rati-numb}.

\index{set-quotient|)}%

\section{Algebra}
\label{sec:free-algebras}

In addition to constructing higher-dimensional objects such as spheres and cell complexes, higher inductive types are also very useful even when working only with sets.
We have seen one example already in \autoref{thm:set-pushout}: they allow us to construct the colimit of any diagram of sets, which is not possible in the base type theory of \autoref{cha:typetheory}.
Higher inductive types are also very useful when we study sets with algebraic structure.

As a running example in this section, we consider \emph{groups}, which are familiar to most mathematicians and exhibit the essential phenomena (and will be needed in later chapters).
However, most of what we say applies equally well to any sort of algebraic structure.

\index{monoid|(}%

\begin{defn}
  A \define{monoid}
  \indexdef{monoid}%
  is a set $G$ together with
  \begin{itemize}
  \item a \emph{multiplication}
    \indexdef{multiplication!in a monoid}%
    \indexdef{multiplication!in a group}%
    function $G\times G\to G$, written infix as $(x,y) \mapsto x\cdot y$; and
  \item a \emph{unit}
    \indexdef{unit!of a monoid}%
    \indexdef{unit!of a group}%
    element $e:G$; such that
  \item for any $x:G$, we have $x\cdot e = x$ and $e\cdot x = x$; and
  \item for any $x,y,z:G$, we have $x\cdot (y\cdot z) = (x\cdot y)\cdot z$.
    \index{associativity!in a monoid}%
    \index{associativity!in a group}%
  \end{itemize}
  A \define{group}
  \indexdef{group}%
  is a monoid $G$ together with
  \begin{itemize}
  \item an \emph{inversion} function $i:G\to G$, written $x\mapsto \opp x$; such that
    \index{inverse!in a group}%
  \item for any $x:G$ we have $x\cdot \opp x = e$ and $\opp x \cdot x = e$.
  \end{itemize}
\end{defn}

\begin{rmk}\label{rmk:infty-group}
Note that we require a group to be a set.
We could consider a more general notion of ``$\infty$-group''%
\index{.infinity-group@$\infty$-group}
which is not a set, but this would take us further afield than is appropriate at the moment.
With our current definition, we may expect the resulting ``group theory'' to behave similarly to the way it does in set-theoretic mathematics (with the caveat that, unless we assume \LEM{}, it will be ``constructive'' group theory).\index{mathematics!constructive}
\end{rmk}

\begin{eg}
  The natural numbers \N are a monoid under addition, with unit $0$, and also under multiplication, with unit $1$.
  If we define the arithmetical operations on the integers \Z in the obvious way, then as usual they are a group under addition and a monoid under multiplication (and, of course, a ring).
  For instance, if $u, v \in \Z$ are represented by $(a,b)$ and $(c,d)$, respectively, then $u + v$ is represented by $(a + c, b + d)$, $-u$ is represented by $(b, a)$, and $u v$ is represented by $(a c + b d, a d + b c)$.
\end{eg}

\begin{eg}\label{thm:homotopy-groups}
  We essentially observed in \autoref{sec:equality} that if $(A,a)$ is a pointed type, then its loop space\index{loop space} $\Omega(A,a)\defeq (\id[A]aa)$ has all the structure of a group, except that it is not in general a set.
  It should be an ``$\infty$-group'' in the sense mentioned in \autoref{rmk:infty-group}, but we can also make it a group by truncation.
  Specifically, we define the \define{fundamental group}
  \indexsee{group!fundamental}{fundamental group}%
  \indexdef{fundamental!group}%
  of $A$ based at $a:A$ to be
  \[\pi_1(A,a)\defeq \trunc0{\Omega(A,a)}.\]
  This inherits a group structure; for instance, the multiplication $\pi_1(A,a) \times \pi_1(A,a) \to \pi_1(A,a)$ is defined by double induction on truncation from the concatenation of paths.

  More generally, the \define{$n^{\mathrm{th}}$ homotopy group}
  \index{homotopy!group}%
  \indexsee{group!homotopy}{homotopy group}%
  of $(A,a)$ is $\pi_n(A,a)\defeq \trunc0{\Omega^n(A,a)}$.
  \index{loop space!iterated}%
  Then $\pi_n(A,a) = \pi_1(\Omega^{n-1}(A,a))$ for $n\ge 1$, so it is also a group.
  (When $n=0$, we have $\pi_0(A) \jdeq \trunc0 A$, which is not a group.)
  Moreover, the Eckmann--Hilton argument \index{Eckmann--Hilton argument} (\autoref{thm:EckmannHilton}) implies that if $n\ge 2$, then $\pi_n(A,a)$ is an \emph{abelian}\index{group!abelian} group, i.e.\ we have $x\cdot y = y\cdot x$ for all $x,y$.
  \autoref{cha:homotopy} will be largely the study of these groups.
\end{eg}

\index{algebra!free}%
\index{free!algebraic structure}%
One important notion in group theory is that of the \emph{free group} generated by a set, or more generally of a group \emph{presented} by generators\index{generator!of a group} and relations.
It is well-known in type theory that \emph{some} free algebraic objects can be defined using \emph{ordinary} inductive types.
\symlabel{lst-freemonoid}%
\indexdef{type!of lists}%
\indexsee{list type}{type, of lists}%
\index{monoid!free|(}%
For instance, the free monoid on a set $A$ can be identified with the type $\lst A$ of \emph{finite lists} \index{finite!lists, type of} of elements of $A$, which is inductively generated by
\begin{itemize}
\item a constructor $\nil:\lst A$, and
\item for each $\ell:\lst A$ and $a:A$, an element $\cons(a,\ell):\lst A$.
\end{itemize}
We have an obvious inclusion $\eta : A\to \lst A$ defined by $a\mapsto \cons(a,\nil)$.
The monoid operation on $\lst A$ is concatenation, defined recursively by
\begin{align*}
  \nil \cdot \ell &\defeq \ell\\
  \cons (a,\ell_1) \cdot \ell_2 &\defeq \cons(a, \ell_1\cdot\ell_2).
\end{align*}
It is straightforward to prove, using the induction principle for $\lst A$, that $\lst A$ is a set and that concatenation of lists is associative
\index{associativity!of list concatenation}%
and has $\nil$ as a unit.
Thus, $\lst A$ is a monoid.

\begin{lem}\label{thm:free-monoid}
  \indexsee{free!monoid}{monoid, free}%
  For any set $A$, the type $\lst A$ is the free monoid on $A$.
  In other words, for any monoid $G$, composition with $\eta$ is an equivalence
  \[ \eqv{\hom_{\mathrm{Monoid}}(\lst A,G)}{(A\to G)}, \]
  where $\hom_{\mathrm{Monoid}}(\blank,\blank)$ denotes the set of monoid homomorphisms (functions which preserve the multiplication and unit).
  \indexdef{homomorphism!monoid}%
  \indexdef{monoid!homomorphism}%
\end{lem}
\begin{proof}
  Given $f:A\to G$, we define $\bar{f}:\lst A \to G$ by recursion:
  \begin{align*}
    \bar{f}(\nil) &\defeq e\\
    \bar{f}(\cons(a,\ell)) &\defeq f(a) \cdot \bar{f}(\ell).
  \end{align*}
  It is straightforward to prove by induction that $\bar{f}$ is a monoid homomorphism, and that $f\mapsto \bar f$ is a quasi-inverse of $(\blank\circ \eta)$; see \autoref{ex:free-monoid}.
\end{proof}

\index{monoid!free|)}%

This construction of the free monoid is possible essentially because elements of the free monoid have computable canonical forms (namely, finite lists).
However, elements of other free (and presented) algebraic structures --- such as groups --- do not in general have \emph{computable} canonical forms.
For instance, equality of words in group presentations is algorithmically\index{algorithm} undecidable.
However, we can still describe free algebraic objects as \emph{higher} inductive types, by simply asserting all the axiomatic equations as path constructors.

\indexsee{free!group}{group, free}%
\index{group!free|(}%
For example, let $A$ be a set, and define a higher inductive type $\freegroup{A}$ with the following generators.
\begin{itemize}
\item A function $\eta:A\to \freegroup{A}$.
\item A function $m: \freegroup{A} \times \freegroup{A} \to \freegroup{A}$.
\item An element $e:\freegroup{A}$.
\item A function $i:\freegroup{A} \to \freegroup{A}$.
\item For each $x,y,z:\freegroup{A}$, an equality $m(x,m(y,z)) = m(m(x,y),z)$.
\item For each $x:\freegroup{A}$, equalities $m(x,e) = x$ and $m(e,x) = x$.
\item For each $x:\freegroup{A}$, equalities $m(x,i(x)) = e$ and $m(i(x),x) = e$.
\item The $0$-truncation constructor: for any $x,y:\freegroup{A}$ and $p,q:x=y$, we have $p=q$.
\end{itemize}
The first constructor says that $A$ maps to $\freegroup{A}$.
The next three give $\freegroup{A}$ the operations of a group: multiplication, an identity element, and inversion.
The three constructors after that assert the axioms of a group: associativity\index{associativity}, unitality, and inverses.
Finally, the last constructor asserts that $\freegroup{A}$ is a set.

Therefore, $\freegroup{A}$ is a group.
It is also straightforward to prove:

\begin{thm}
  \index{universal!property!of free group}%
  $\freegroup{A}$ is the free group on $A$.
  In other words, for any (set) group $G$, composition with $\eta:A\to \freegroup{A}$ determines an equivalence
  \[ \hom_{\mathrm{Group}}(\freegroup{A},G) \eqvsym (A\to G) \]
  where $\hom_{\mathrm{Group}}(\blank,\blank)$ denotes the set of group homomorphisms between two groups.
  \indexdef{group!homomorphism}%
  \indexdef{homomorphism!group}%
\end{thm}
\begin{proof}
  The recursion principle of the higher inductive type $\freegroup{A}$ says \emph{precisely} that if $G$ is a group and we have $f:A\to G$, then we have $\bar{f}:\freegroup{A} \to G$.
  Its computation rules say that $\bar{f}\circ \eta \jdeq f$, and that $\bar f$ is a group homomorphism.
  Thus, $(\blank\circ \eta) :  \hom_{\mathrm{Group}}(\freegroup{A},G) \to (A\to G)$ has a right inverse.
  It is straightforward to use the induction principle of $\freegroup{A}$ to show that this is also a left inverse.
\end{proof}

\index{acceptance}
It is worth taking a step back to consider what we have just done.
We have proven that the free group on any set exists \emph{without} giving an explicit construction of it.
Essentially all we had to do was write down the universal property that it should satisfy.
In set theory, we could achieve a similar result by appealing to black boxes such as the adjoint functor theorem\index{adjoint!functor theorem}; type theory builds such constructions into the foundations of mathematics.

Of course, it is sometimes also useful to have a concrete description of free algebraic structures.
In the case of free groups, we can provide one, using quotients.
Consider $\lst{A+A}$, where in $A+A$ we write $\inl(a)$ as $a$, and $\inr(a)$ as $\hat{a}$ (intended to stand for the formal inverse of $a$).
The elements of $\lst{A+A}$ are \emph{words} for the free group on $A$.

\begin{thm}
  Let $A$ be a set, and let $\freegroupx{A}$ be the set-quotient of $\lst{A+A}$ by the following relations.
  \begin{align*}
    (\dots,a_1,a_2,\widehat{a_2},a_3,\dots) &=
    (\dots,a_1,a_3,\dots)\\
    (\dots,a_1,\widehat{a_2},a_2,a_3,\dots) &=
    (\dots,a_1,a_3,\dots).
  \end{align*}
  Then $\freegroupx{A}$ is also the free group on the set $A$.
\end{thm}
\begin{proof}
  First we show that $\freegroupx{A}$ is a group.
  We have seen that $\lst{A+A}$ is a monoid; we claim that the monoid structure descends to the quotient.
  We define $\freegroupx{A} \times \freegroupx{A} \to \freegroupx{A}$ by double quotient recursion; it suffices to check that the equivalence relation generated by the given relations is preserved by concatenation of lists.
  Similarly, we prove the associativity and unit laws by quotient induction.

  In order to define inverses in $\freegroupx{A}$, we first define $\mathsf{reverse}:\lst B\to\lst B$ by recursion on lists:
  \begin{align*}
    \mathsf{reverse}(\nil) &\defeq \nil,\\
    \mathsf{reverse}(\cons(b,\ell))&\defeq \mathsf{reverse}(\ell)\cdot \cons(b,\nil).
  \end{align*}
  Now we define $i:\freegroupx{A}\to \freegroupx{A}$ by quotient recursion, acting on a list $\ell:\lst{A+A}$ by switching the two copies of $A$ and reversing the list.
  This preserves the relations, hence descends to the quotient.
  And we can prove that $i(x) \cdot x = e$ for $x:\freegroupx{A}$ by induction.
  First, quotient induction allows us to assume $x$ comes from $\ell:\lst{A+A}$, and then we can do list induction:
  \begin{align*}
    i(\nil) \ct \nil &= \nil \ct \nil\\
    &= \nil\\
    i(\cons(a,\ell)) \ct \cons(a,\ell) &= i(\ell) \ct \cons(\hat{a},\nil) \ct \cons(a,\ell)\\
    &= i(\ell) \ct \cons(\hat{a},\cons(a,\ell))\\
    &= i(\ell) \ct \ell\\
    &= \nil. \tag{by the inductive hypothesis}
  \end{align*}
  (We have omitted a number of fairly evident lemmas about the behavior of concatenation of lists, etc.)

  This completes the proof that $\freegroupx{A}$ is a group.
  Now if $G$ is any group with a function $f:A\to G$, we can define $A+A\to G$ to be $f$ on the first copy of $A$ and $f$ composed with the inversion map of $G$ on the second copy.
  Now the fact that $G$ is a monoid yields a monoid homomorphism $\lst{A+A} \to G$.
  And since $G$ is a group, this map respects the relations, hence descends to a map $\freegroupx{A}\to G$.
  It is straightforward to prove that this is a group homomorphism, and the unique one which restricts to $f$ on $A$.
\end{proof}

\index{monoid|)}%

If $A$ has decidable equality\index{decidable!equality} (such as if we assume excluded middle), then the quotient defining $\freegroupx{A}$ can be obtained from an idempotent as in \autoref{lem:quotient-when-canonical-representatives}.
We define a word, which we recall is just an element of $\lst{A+A}$, to be \define{reduced}
\indexdef{reduced word in a free group}
if it contains no adjacent pairs of the form $(a,\hat a)$ or $(\hat a,a)$.
When $A$ has decidable equality, it is straightforward to define the \define{reduction}
\index{reduction!of a word in a free group}%
of a word, which is an idempotent generating the appropriate quotient; we leave the details to the reader.

If $A\defeq \unit$, which has decidable equality, a reduced word must consist either entirely of $\ttt$'s or entirely of $\hat{\ttt}$'s.
Thus, the free group on $\unit$ is equivalent to the integers \Z, with $0$ corresponding to $\nil$, the positive integer $n$ corresponding to a reduced word of $n$ $\ttt$'s, and the negative integer $(-n)$ corresponding to a reduced word of $n$ $\hat{\ttt}$'s.
One could also, of course, show directly that \Z has the universal property of $\freegroup{\unit}$.

\begin{rmk}\label{thm:freegroup-nonset}
  Nowhere in the construction of $\freegroup{A}$ and $\freegroupx{A}$, and the proof of their universal properties, did we use the assumption that $A$ is a set.
  Thus, we can actually construct the free group on an arbitrary type.
  Comparing universal properties, we conclude that $\eqv{\freegroup{A}}{\freegroup{\trunc0A}}$.
\end{rmk}

\index{group!free|)}%

\index{algebra!colimits of}%
We can also use higher inductive types to construct colimits of algebraic objects.
For instance, suppose $f:G\to H$ and $g:G\to K$ are group homomorphisms.
Their pushout in the category of groups, called the \define{amalgamated free product}
\indexdef{amalgamated free product}%
\indexdef{free!product!amalgamated}%
$H *_G K$, can be constructed as the higher inductive type generated by
\begin{itemize}
\item Functions $h:H\to H *_G K$ and $k:K\to H *_G K$.
\item The operations and axioms of a group, as in the definition of $\freegroup{A}$.
\item Axioms asserting that $h$ and $k$ are group homomorphisms.
\item For $x:G$, we have $h(f(x)) = k(g(x))$.
\item The $0$-truncation constructor.
\end{itemize}
On the other hand, it can also be constructed explicitly, as the set-quotient of $\lst{H+K}$ by the following relations:
\begin{align*}
  (\dots, x_1, x_2, \dots) &= (\dots, x_1\cdot x_2, \dots)
  & &\text{for $x_1,x_2:H$}\\
  (\dots, y_1, y_2, \dots) &= (\dots, y_1\cdot y_2, \dots)
  & &\text{for $y_1,y_2:K$}\\
  (\dots, 1_G, \dots) &= (\dots, \dots) &&  \\
  (\dots, 1_H, \dots) &= (\dots, \dots) &&  \\
  (\dots, f(x), \dots) &= (\dots, g(x), \dots)
  & &\text{for $x:G$.}
\end{align*}
We leave the proofs to the reader.
In the special case that $G$ is the trivial group, the last relation is unnecessary, and we obtain the \define{free product}
\indexdef{free!product}%
$H*K$, the coproduct in the category of groups.
(This notation unfortunately clashes with that for the \emph{join} of types, as in \autoref{sec:colimits}, but context generally disambiguates.)

\index{presentation!of a group}%
Note that groups defined by \emph{presentations} can be regarded as a special case of colimits.
Suppose given a set (or more generally a type) $A$, and a pair of functions $R\rightrightarrows \freegroup{A}$.
We regard $R$ as the type of ``relations'', with the two functions assigning to each relation the two words that it sets equal.
For instance, in the presentation $\langle a \mid a^2 = e \rangle$ we would have $A\defeq \unit$ and $R\defeq \unit$, with the two morphisms $R\rightrightarrows \freegroup{A}$ picking out the list $(a,a)$ and the empty list $\nil$, respectively.
Then by the universal property of free groups, we obtain a pair of group homomorphisms $\freegroup{R} \rightrightarrows \freegroup{A}$.
Their coequalizer in the category of groups, which can be built just like the pushout, is the group \emph{presented} by this presentation.

\mentalpause

Note that all these sorts of construction only apply to \emph{algebraic} theories,\index{theory!algebraic} which are theories whose axioms are (universally quantified) equations referring to variables, constants, and operations from a given signature\index{signature!of an algebraic theory}.
They can be modified to apply also to what are called \emph{essentially algebraic theories}:\index{theory!essentially algebraic} those whose operations are partially defined on a domain specified by equalities between previous operations.
They do not apply, for instance, to the theory of fields, in which the ``inversion'' operation is partially defined on a domain $\setof{x | x \mathrel{\#} 0}$ specified by an \emph{apartness} $\#$ between previous operations, see \autoref{RD-inverse-apart-0}.
And indeed, it is well-known that the category of fields has no initial object.
\index{initial!field}%

On the other hand, these constructions do apply just as well to \emph{infinitary}\index{infinitary!algebraic theory} algebraic theories, whose ``operations'' can take infinitely many inputs.
In such cases, there may not be any presentation of free algebras or colimits of algebras as a simple quotient, unless we assume the axiom of choice.
This means that higher inductive types represent a significant strengthening of constructive type theory (not necessarily in terms of proof-theoretic strength, but in terms of practical power), and indeed are stronger in some ways than Zermelo--Fraenkel\index{set theory!Zermelo--Fraenkel} set theory (without choice).
% We will see an example of this in \autoref{sec:ordinals}.


\section{The flattening lemma}
\label{sec:flattening}

As we will see in \autoref{cha:homotopy}, amazing things happen when we combine higher inductive types with univalence.
The principal way this comes about is that if $W$ is a higher inductive type and \UU is a type universe, then we can define a type family $P:W\to \UU$ by using the recursion principle for $W$.
When we come to the clauses of the recursion principle dealing with the path constructors of $W$, we will need to supply paths in \UU, and this is where univalence comes in.

For example, suppose we have a type $X$ and a self-equivalence $e:\eqv X X$.
Then we can define a type family $P:\Sn^1 \to \UU$ by using $\Sn^1$-recursion:
\begin{equation*}
  P(\base) \defeq X
  \qquad\text{and}\qquad
  \ap P\lloop \defid \ua(e).
\end{equation*}
The type $X$ thus appears as the fiber $P(\base)$ of $P$ over the basepoint.
The self-equivalence $e$ is a little more hidden in $P$, but the following lemma says that it can be extracted by transporting along \lloop.

\begin{lem}\label{thm:transport-is-given}
  Given $B:A\to\type$ and $x,y:A$, with a path $p:x=y$ and an equivalence $e:\eqv{P(x)}{P(y)}$ such that $\ap{B}p = \ua(e)$, then for any $u:P(x)$ we have
  \begin{align*}
    \transfib{B}{p}{u} &= e(u).
  \end{align*}
\end{lem}
\begin{proof}
  Applying \autoref{thm:transport-is-ap}, we have
  \begin{align*}
    \transfib{B}{p}{u} &= \idtoeqv(\ap{B}p)(u)\\
    &= \idtoeqv(\ua(e))(u)\\
    &= e(u).\qedhere
  \end{align*}
\end{proof}

We have seen type families defined by recursion before: in \autoref{sec:compute-coprod,sec:compute-nat} we used them to characterize the identity types of (ordinary) inductive types.
In \autoref{cha:homotopy}, we will use similar ideas to calculate homotopy groups of higher inductive types.

In this section, we describe a general lemma about type families of this sort which will be useful later on.
We call it the \define{flattening lemma}:
\indexdef{flattening lemma}%
\indexdef{lemma!flattening}%
it says that if $P:W\to\UU$ is defined recursively as above, then its total space $\sm{x:W} P(x)$ is equivalent to a ``flattened'' higher inductive type, whose constructors may be deduced from those of $W$ and the definition of $P$.
From a category-theoretic point of view, $\sm{x:W} P(x)$ is the ``Grothendieck\index{Grothendieck construction} construction'' of $P$, and this expresses its universal property as a ``lax\index{lax colimit} colimit''.

We prove here one general case of the flattening lemma, which directly implies many particular cases and suggests the method to prove others.
Suppose we have $A,B:\type$ and $f,g:B\to{}A$, and that the higher inductive type $W$ is generated by
\begin{itemize}
\item $\cc:A\to{}W$ and
\item $\pp:\prd{b:B} (\cc(f(b))=_W\cc(g(b)))$.
\end{itemize}
Thus, $W$ is the \define{(homotopy) coequalizer}
\indexdef{coequalizer}%
\indexdef{type!coequalizer}%
of $f$ and $g$.
Using binary sums (coproducts) and dependent sums ($\Sigma$-types), a lot of interesting nonrecursive higher
inductive types can be represented in this form. All point constructors have to
be bundled in the type $A$ and all path constructors in the type $B$.
For instance:
\begin{itemize}
\item The circle $\Sn^1$ can be represented by taking $A\defeq \unit$ and $B\defeq \unit$, with $f$ and $g$ the identity.
\item The pushout of $j:X\to Y$ and $k:X\to Z$ can be represented by taking $A\defeq Y+Z$ and $B\defeq X$, with $f\defeq \inl \circ j$ and $g\defeq \inr\circ k$.
\end{itemize}
Now suppose in addition that
\begin{itemize}
\item $C:A\to\type$ is a family of types over $A$, and
\item $D:\prd{b:B}\eqv{C(f(b))}{C(g(b))}$ is a family of equivalences over $B$.
\end{itemize}
Define a type family $P : W\to\type$ inductively by
\begin{align*}
  P(\cc(a)) &\defeq C(a)\\
  \map{P}{\pp(b)} &\defid \ua(D(b)).
\end{align*}
Let \Wtil be the higher inductive type generated by
\begin{itemize}
\item $\cct:\prd{a:A} C(a) \to \Wtil$ and
\item $\ppt:\prd{b:B}{y:C(f(b))} (\cct(f(b),y)=_{\Wtil}\cct(g(b),D(b)(y)))$.
\end{itemize}

The flattening lemma is:

\begin{lem}[Flattening lemma]\label{thm:flattening}
  In the above situation, we have
  \[ \eqvspaced{\Parens{\sm{x:W} P(x)}}{\widetilde{W}}. \]
\end{lem}

\index{universal!property!of dependent pair type}%
As remarked above, this equivalence can be seen as expressing the universal property of $\sm{x:W} P(x)$ as a ``lax\index{lax colimit} colimit'' of $P$ over $W$.
It can also be seen as part of the \emph{stability and descent} property of colimits, which characterizes higher toposes.%
\index{.infinity1-topos@$(\infty,1)$-topos}%
\index{stability!and descent}%

The proof of \autoref{thm:flattening} occupies the rest of this section.
It is somewhat technical and can be skipped on a first reading.
But it is also a good example of ``proof-relevant mathematics'',
\index{mathematics!proof-relevant}%
so we recommend it on a second reading.

The idea is to show that $\sm{x:W} P(x)$ has the same universal property as \Wtil.
We begin by showing that it comes with analogues of the constructors $\cct$ and $\ppt$.

\begin{lem}\label{thm:flattening-cp}
  There are functions
  \begin{itemize}
  \item $\cct':\prd{a:A} C(a) \to \sm{x:W} P(x)$ and
  \item $\ppt':\prd{b:B}{y:C(f(b))} \Big(\cct'(f(b),y)=_{\sm{w:W}P(w)}\cct'(g(b),D(b)(y))\Big)$.
  \end{itemize}
\end{lem}
\begin{proof}
  The first is easy; define $\cct'(a,x) \defeq (\cc(a),x)$ and note that by definition $P(\cc(a))\jdeq C(a)$.
  For the second, suppose given $b:B$ and $y:C(f(b))$; we must give an equality
  \[ (\cc(f(b)),y) = (\cc(g(b)),D(b)(y)). \]
  Since we have $\pp(b):f(b)=g(b)$, by equalities in $\Sigma$-types it suffices to give an equality $\trans{\pp(b)}{y} = D(b)(y)$.
  But this follows from \autoref{thm:transport-is-given}, using the definition of $P$.
\end{proof}

Now the following lemma says to define a section of a type family over $\sm{w:W} P(w)$, it suffices to give analogous data as in the case of \Wtil.

\begin{lem}\label{thm:flattening-rect}
  Suppose $Q:\big(\sm{x:W} P(x)\big) \to \type$ is a type family and that we have
  \begin{itemize}
  \item $c : \prd{a:A}{x:C(a)} Q(\cct'(a,x))$ and
  \item $p : \prd{b:B}{y:C(f(b))} \Big(\trans{\ppt'(b,y)}{c(f(b),y)} = c(g(b),D(b)(y))\Big)$. %_{Q(\cct'(g(b),D(b)(y)))}
  \end{itemize}
  Then there exists $k:\prd{z:\sm{w:W} P(w)} Q(z)$ such that $k(\cct'(a,x)) \jdeq c(a,x)$.
\end{lem}
\begin{proof}
  Suppose given $w:W$ and $x:P(w)$; we must produce an element $k(w,x):Q(w,x)$.
  By induction on $w$, it suffices to consider two cases.
  When $w\jdeq \cc(a)$, then we have $x:C(a)$, and so $c(a,x):Q(\cc(a),x)$ as desired.
  (This part of the definition also ensures that the stated computational rule holds.)

  Now we must show that this definition is preserved by transporting along $\pp(b)$ for any $b:B$.
  Since what we are defining, for all $w:W$, is a function of type $\prd{x:P(w)} Q(w,x)$, by \autoref{thm:dpath-forall} it suffices to show that for any $y:C(f(b))$, we have
  \[ \transfib{Q}{\pairpath(\pp(b),\refl{\trans{\pp(b)}{y}})}{c(f(b),y)} = c(g(b),\trans{\pp(b)}{y}). \]
  Let $q:\trans{\pp(b)}{y} = D(b)(y)$ be the path obtained from \autoref{thm:transport-is-given}.
  Then we have
  \begin{align}
    c(g(b),\trans{\pp(b)}{y})
    &= \transfib{x\mapsto Q(\cct'(g(b),x))}{\opp{q}}{c(g(b),D(b)(y))}
    \tag{by $\opp{\apdfunc{x\mapsto c(g(b),x)}(\opp q)}$} \\
    &= \transfib{Q}{\apfunc{x\mapsto \cct'(g(b),x)}(\opp q)}{c(g(b),D(b)(y))}
    \tag{by \autoref{thm:transport-compose}}.
  \end{align}
  Thus, it suffices to show
  \begin{multline*}
    \Transfib{Q}{\pairpath(\pp(b),\refl{\trans{\pp(b)}{y}})}{c(f(b),y)} = {}\\
    \Transfib{Q}{\apfunc{x\mapsto c(g(b),x)}(\opp q)}{c(g(b),D(b)(y))}.
  \end{multline*}
  Moving the right-hand transport to the other side, and combining two transports, this is equivalent to
  %
  \begin{narrowmultline*}
    \Transfib{Q}{\apfunc{x\mapsto c(g(b),x)}(q) \ct
      \pairpath(\pp(b),\refl{\trans{\pp(b)}{y}})}{c(f(b),y)} =
    \narrowbreak
    c(g(b),D(b)(y)).
  \end{narrowmultline*}
  %
  However, we have
  \begin{multline*}
    \apfunc{x\mapsto c(g(b),x)}(q) \ct \pairpath(\pp(b),\refl{\trans{\pp(b)}{y}})
    = {} \\
    \pairpath(\refl{g(b)},q) \ct \pairpath(\pp(b),\refl{\trans{\pp(b)}{y}})
    = \pairpath(\pp(b),q)
    = \ppt'(b,y)
  \end{multline*}
  so the construction is completed by the assumption $p(b,y)$ of type
  \[ \transfib{Q}{\ppt'(b,y)}{c(f(b),y)} = c(g(b),D(b)(y)). \qedhere \]
\end{proof}

\autoref{thm:flattening-rect} \emph{almost} gives $\sm{w:W}P(w)$ the same induction principle as \Wtil.
The missing bit is the equality $\apdfunc{f}(\ppt'(b,y)) = p(b,y)$.
In order to prove this, we would need to analyze the proof of \autoref{thm:flattening-rect}, which of course is the definition of $k$.

It should be possible to do this, but it turns out that we only need the computation rule for the non-dependent recursion principle.
Thus, we now give a somewhat simpler direct construction of the recursor, and a proof of its computation rule.

\begin{lem}\label{thm:flattening-rectnd}
  Suppose $Q$ is a type and that we have
  \begin{itemize}
  \item $c : \prd{a:A} C(a) \to Q$ and
  \item $p : \prd{b:B}{y:C(f(b))} \Big(c(f(b),y) =_Q c(g(b),D(b)(y))\Big)$.
  \end{itemize}
  Then there exists $k:\big(\sm{w:W} P(w)\big) \to Q$ such that $k(\cct'(a,x)) \jdeq c(a,x)$.
\end{lem}
\begin{proof}
  As in \autoref{thm:flattening-rect}, we define $k(w,x)$ by induction on $w:W$.
  When $w\jdeq \cc(a)$, we define $k(\cc(a),x)\defeq c(a,x)$.
  Now by \autoref{thm:dpath-arrow}, it suffices to consider, for $b:B$ and $y:C(f(b))$, the composite path
  \begin{equation}\label{eq:flattening-rectnd}
    \transfib{x\mapsto Q}{\pp(b)}{c(f(b),y)}
    = c(g(b),\transfib{P}{\pp(b)}{y})
  \end{equation}
  %
  defined as the composition
  %
  \begin{align}
    \transfib{x\mapsto Q}{\pp(b)}{c(f(b),y)}
    &= c(f(b),y) \tag{by \autoref{thm:trans-trivial}}\\
    &= c(g(b),D(b)(y)) \tag{by $p(b,y)$}\\
    &= c(g(b),\transfib{P}{\pp(b)}{y}). \tag{by \autoref{thm:transport-is-given}}
  \end{align}
  The computation rule $k(\cct'(a,x)) \jdeq c(a,x)$ follows by definition, as before.
\end{proof}

For the second computation rule, we need the following lemma.

\begin{lem}\label{thm:ap-sigma-rect-path-pair}
  Let $Y:X\to\type$ be a type family and let $k:(\sm{x:X}Y(x)) \to Z$ be defined componentwise by $k(x,y) \defeq d(x)(y)$ for a curried function $d:\prd{x:X} Y(x)\to Z$.
  Then for any $s:\id[X]{x_1}{x_2}$ and any $y_1:Y(x_1)$ and $y_2:Y(x_2)$ with a path $r:\trans{s}{y_1}=y_2$, the path
  \[\apfunc k (\pairpath(s,r)) :k(x_1,y_1) = k(x_2,y_2)\]
  is equal to the composite
  \begin{align}
    k(x_1,y_1)
    &\jdeq d(x_1)(y_1) \notag\\
    &= \transfib{x\mapsto Z}{s}{d(x_1)(y_1)}
    \tag{by $\opp{\text{(\autoref{thm:trans-trivial})}}$}\\
    &= \transfib{x\mapsto Z}{s}{d(x_1)(\trans{\opp s}{\trans{s}{y_1}})}
    \notag\\
    &= \big(\transfib{x\mapsto (Y(x)\to Z)}{s}{d(x_1)}\big)(\trans{s}{y_1})
    \tag{by~\eqref{eq:transport-arrow}}\\
    &= d(x_2)(\trans{s}{y_1})
    \tag{by $\happly(\apdfunc{d}(s))(\trans{s}{y_1}$}\\
    &= d(x_2)(y_2)
    \tag{by $\apfunc{d(x_2)}(r)$}\\
    &\jdeq k(x_2,y_2).
    \notag
  \end{align}
\end{lem}
\begin{proof}
  After path induction on $s$ and $r$, both equalities reduce to reflexivities.
\end{proof}

At first it may seem surprising that \autoref{thm:ap-sigma-rect-path-pair} has such a complicated statement, while it can be proven so simply.
The reason for the complication is to ensure that the statement is well-typed: $\apfunc k (\pairpath(s,r))$ and the composite path it is claimed to be equal to must both have the same start and end points.
Once we have managed this, the proof is easy by path induction.

\begin{lem}\label{thm:flattening-rectnd-beta-ppt}
  In the situation of \autoref{thm:flattening-rectnd}, we have $\apfunc{k}(\ppt'(b,y)) = p(b,y)$.
\end{lem}
\begin{proof}
  Recall that $\ppt'(b,y) \defeq \pairpath(\pp(b),q)$ where $q:\trans{\pp(b)}{y} = D(b)(y)$ comes from \autoref{thm:transport-is-given}.
  Thus, since $k$ is defined componentwise, we may compute $\apfunc{k}(\ppt'(b,y))$ by \autoref{thm:ap-sigma-rect-path-pair}, with
  \begin{align*}
    x_1 &\defeq \cc(f(b)) & y_1 &\defeq y\\
    x_2 &\defeq \cc(g(b)) & y_2 &\defeq D(b)(y)\\
    s &\defeq \pp(b)      &   r &\defeq q.
  \end{align*}
  The curried function $d:\prd{w:W} P(w) \to Q$ was defined by induction on $w:W$;
  to apply \autoref{thm:ap-sigma-rect-path-pair} we need to understand $\apfunc{d(x_2)}(r)$ and $\happly(\apdfunc{d}(s),\trans s {y_1})$.

  For the first, since $d(\cc(a),x)\jdeq c(a,x)$, we have
  \[ \apfunc{d(x_2)}(r) \jdeq \apfunc{c(g(b),-)}(q). \]
  For the second, the computation rule for the induction principle of $W$ tells us that $\apdfunc{d}(\pp(b))$ is equal to the composite~\eqref{eq:flattening-rectnd}, passed across the equivalence of \autoref{thm:dpath-arrow}.
  Thus, the computation rule given in \autoref{thm:dpath-arrow} implies that $\happly(\apdfunc{d}(\pp(b)),\trans {\pp(b)}{y})$ is equal to the composite
  \begin{align}
    \big(\trans{\pp(b)}{c(f(b),-)}\big)(\trans {\pp(b)}{y})
    &= \trans{\pp(b)}{c(f(b),\trans{\opp {\pp(b)}}{\trans {\pp(b)}{y}})}
    \tag{by~\eqref{eq:transport-arrow}}\\
    &= \trans{\pp(b)}{c(f(b),y)}
    \notag \\
    &= c(f(b),y)
    \tag{by \autoref{thm:trans-trivial}}\\
    &= c(f(b),D(b)(y))
   \tag{by $p(b,y)$}\\
    &= c(f(b),\trans{\pp(b)}{y}).
    \tag{by $\opp{\apfunc{c(g(b),-)}(q)}$}
  \end{align}
  Finally, substituting these values of $\apfunc{d(x_2)}(r)$ and $\happly(\apdfunc{d}(s),\trans s {y_1})$ into \autoref{thm:ap-sigma-rect-path-pair}, we see that all the paths cancel out in pairs, leaving only $p(b,y)$.
\end{proof}

Now we are finally ready to prove the flattening lemma.

\begin{proof}[Proof of \autoref{thm:flattening}]
  We define $h:\Wtil \to \sm{w:W}P(w)$ by using the recursion principle for \Wtil, with $\cct'$ and $\ppt'$ as input data.
  Similarly, we define $k:(\sm{w:W}P(w)) \to \Wtil$ by using the recursion principle of \autoref{thm:flattening-rectnd}, with $\cct$ and $\ppt$ as input data.

  On the one hand, we must show that for any $z:\Wtil$, we have $k(h(z))=z$.
  By induction on $z$, it suffices to consider the two constructors of \Wtil.
  But we have
  \[k(h(\cct(a,x))) \jdeq k(\cct'(a,x)) \jdeq \cct(a,x)\]
  by definition, while similarly
  \[\ap k{\ap h{\ppt(b,y)}} = \ap k{\ppt'(b,y)} = \ppt(b,y) \]
  using the propositional computation rule for $\Wtil$ and \autoref{thm:flattening-rectnd-beta-ppt}.

  On the other hand, we must show that for any $z:\sm{w:W}P(w)$, we have $h(k(z))=z$.
  But this is essentially identical, using \autoref{thm:flattening-rect} for ``induction on $\sm{w:W}P(w)$'' and the same computation rules.
\end{proof}

\section{The general syntax of higher inductive definitions}
\label{sec:naturality}

In \autoref{sec:strictly-positive}, we discussed the conditions on a putative ``inductive definition'' which make it acceptable, namely that all inductive occurrences of the type in its constructors are ``strictly positive''.\index{strict!positivity}
In this section, we say something about the additional conditions required for \emph{higher} inductive definitions.
Finding a general syntactic description of valid higher inductive definitions is an area of current research, and all of the solutions proposed to date are somewhat technical in nature; thus we only give a general description and not a precise definition.
Fortunately, the corner cases never seem to arise in practice.

Like an ordinary inductive definition, a higher inductive definition is specified by a list of \emph{constructors}, each of which is a (dependent) function.
For simplicity, we may require the inputs of each constructor to satisfy the same condition as the inputs for constructors of ordinary inductive types.
In particular, they may contain the type being defined only strictly positively.
Note that this excludes definitions such as the $0$-truncation as presented in \autoref{sec:hittruncations}, where the input of a constructor contains not only the inductive type being defined, but its identity type as well.
It may be possible to extend the syntax to allow such definitions; but also, in \autoref{sec:truncations} we will give a different construction of the $0$-truncation whose constructors do satisfy the more restrictive condition.

The only difference between an ordinary inductive definition and a higher one, then, is that the \emph{output} type of a constructor may be, not the type being defined ($W$, say), but some identity type of it, such as $\id[W]uv$, or more generally an iterated identity type such as $\id[({\id[W]uv})]pq$.
Thus, when we give a higher inductive definition, we have to specify not only the inputs of each constructor, but the expressions $u$ and $v$ (or $u$, $v$, $p$, and $q$, etc.)\ which determine the source\index{source!of a path constructor} and target\index{target!of a path constructor} of the path being constructed.

Importantly, these expressions may refer to \emph{other} constructors of $W$.
For instance, in the definition of $\Sn^1$, the constructor $\lloop$ has both $u$ and $v$ being $\base$, the previous constructor.
To make sense of this, we require the constructors of a higher inductive type to be specified \emph{in order}, and we allow the source and target expressions $u$ and $v$ of each constructor to refer to previous constructors, but not later ones.
(Of course, in practice the constructors of any inductive definition are written down in some order, but for ordinary inductive types that order is irrelevant.)

Note that this order is not necessarily the order of ``dimension'': in principle, a 1-dimensional path constructor could refer to a 2-dimensional one and hence need to come after it.
However, we have not given the 0-dimensional constructors (point constructors) any way to refer to previous constructors, so they might as well all come first.
And if we use the hub-and-spoke construction (\autoref{sec:hubs-spokes}) to reduce all constructors to points and 1-paths, then we might assume that all point constructors come first, followed by all 1-path constructors --- but the order among the 1-path constructors continues to matter.

The remaining question is, what sort of expressions can $u$ and $v$ be?
We might hope that they could be any expression at all involving the previous constructors.
However, the following example shows that a naive approach to this idea does not work.

\begin{eg}\label{eg:unnatural-hit}
  Consider a family of functions $f:\prd{X:\type} (X\to X)$.
  Of course, $f_X$ might be just $\idfunc[X]$ for all $X$, but other such $f$s may also exist.
  For instance, nothing prevents $f_{\bool}:\bool\to\bool$ from being the nonidentity automorphism\index{automorphism!of 2, nonidentity@of $\bool$, nonidentity} (see \autoref{ex:unnatural-endomorphisms}).

  Now suppose that we attempt to define a higher inductive type $K$ generated by:
  \begin{itemize}
  \item two elements $a,b:K$, and
  \item a path $\sigma:f_K(a)=f_K(b)$.
  \end{itemize}
  What would the induction principle for $K$ say?
  We would assume a type family $P:K\to\type$, and of course we would need $x:P(a)$ and $y:P(b)$.
  The remaining datum should be a dependent path in $P$ living over $\sigma$, which must therefore connect some element of $P(f_K(a))$ to some element of $P(f_K(b))$.
  But what could these elements possibly be?
  We know that $P(a)$ and $P(b)$ are inhabited by $x$ and $y$, respectively, but this tells us nothing about $P(f_K(a))$ and $P(f_K(b))$.
\end{eg}

Clearly some condition on $u$ and $v$ is required in order for the definition to be sensible.
It seems that, just as the domain of each constructor is required to be (among other things) a \emph{covariant functor}, the appropriate condition on the expressions $u$ and $v$ is that they define \emph{natural transformations}.
Making precise sense of this requirement is beyond the scope of this book, but informally it means that $u$ and $v$ must only involve operations which are preserved by all functions between types.

For instance, it is permissible for $u$ and $v$ to refer to concatenation of paths, as in the case of the final constructor of the torus in \autoref{sec:cell-complexes}, since all functions in type theory preserve path concatenation (up to homotopy).
However, it is not permissible for them to refer to an operation like the function $f$ in \autoref{eg:unnatural-hit}, which is not necessarily natural: there might be some function $g:X\to Y$ such that $f_Y \circ g \neq g\circ f_X$.
(Univalence implies that $f_X$ must be natural with respect to all \emph{equivalences}, but not necessarily with respect to functions that are not equivalences.)

The intuition of naturality supplies only a rough guide for when a higher inductive definition is permissible.
Even if it were possible to give a precise specification of permissible forms of such definitions in this book, such a specification would probably be out of date quickly, as new extensions to the theory are constantly being explored.
For instance, the presentation of $n$-spheres in terms of ``dependent $n$-loops\index{loop!dependent n-@dependent $n$-}'' referred to in \autoref{sec:circle}, and the ``higher inductive-recursive definitions'' used in \autoref{cha:real-numbers}, were innovations introduced while this book was being written.
We encourage the reader to experiment --- with caution.


\sectionNotes

The general idea of higher inductive types was conceived in discussions between Andrej Bauer, Peter Lumsdaine, Mike Shulman, and Michael Warren at the Oberwolfach meeting in 2011, although there are some suggestions of some special cases in earlier work.  Subsequently, Guillaume Brunerie and Dan Licata contributed substantially to the general theory, especially by finding convenient ways to represent them in computer proof assistants
\index{proof!assistant}
and do homotopy theory with them (see \autoref{cha:homotopy}).

A general discussion of the syntax of higher inductive types, and their semantics in higher-categorical models, appears in~\cite{ls:hits}.
As with ordinary inductive types, models of higher inductive types can be constructed by transfinite iterative processes; a slogan is that ordinary inductive types describe \emph{free} monads while higher inductive types describe \emph{presentations} of monads.\index{monad}%
The introduction of path constructors also involves the model-category-theoretic equivalence between ``right homotopies'' (defined using path spaces) and ``left homotopies'' (defined using cylinders) --- the fact that this equivalence is generally only up to homotopy provides a semantic reason to prefer propositional computation rules for path constructors.

Another (temporary) reason for this preference comes from the limitations of existing computer implementations.
Proof assistants\index{proof!assistant} like \Coq and \Agda have ordinary inductive types built in, but not yet higher inductive types.
We can of course introduce them by assuming lots of axioms, but this results in only propositional computation rules.
However, there is a trick due to Dan Licata which implements higher inductive types using private data types; this yields judgmental rules for point constructors but not path constructors.

The type-theoretic description of higher spheres using loop spaces and suspensions in \autoref{sec:circle,sec:suspension} is largely due to  Brunerie and  Licata; Hou has given a type-theoretic version of the alternative description that uses $n$-dimensional paths\index{path!n-@$n$-}.
The reduction of higher paths to 1-dimensional paths with hubs and spokes (\autoref{sec:hubs-spokes}) is due to  Lumsdaine and  Shulman.
The description of truncation as a higher inductive type is due to  Lumsdaine; the $(-1)$-truncation is closely related to the ``bracket types'' of~\cite{ab:bracket-types}.
The flattening lemma was first formulated in generality by  Brunerie.

\index{set-quotient}
Quotient types are unproblematic in extensional type theory, such as \NuPRL~\cite{constable+86nuprl-book}.
They are often added by passing to an extended system of setoids.\index{setoid}
However, quotients are a trickier issue in intensional type theory (the starting point for homotopy type theory), because one cannot simply add new propositional equalities without specifying how they are to behave. Some solutions to this problem have been studied~\cite{hofmann:thesis,Altenkirch1999,altenkirch+07ott}, and several different notions of quotient types have been considered.  The construction of set-quotients using higher-inductives provides an argument for our particular approach (which is similar to some that have previously been considered), because it arises as an instance of a general mechanism.  Our construction does not yet provide a new solution to all the computational problems related to quotients, since we still lack a good computational understanding of higher inductive types in general---but it does mean that ongoing work on the computational interpretation of higher inductives applies to the quotients as well.  The construction of quotients in terms of equivalence classes is, of
course, a standard set-theoretic idea, and a well-known aspect of elementary topos theory; its use in type theory (which depends on the univalence axiom, at least for mere propositions) was proposed by Voevodsky.  The fact that quotient types in intensional type theory imply function extensionality was proved by~\cite{hofmann:thesis}, inspired by the work of~\cite{carboni} on exact completions; \autoref{thm:interval-funext} is an adaptation of such arguments.

\sectionExercises

\begin{ex}\label{ex:torus}
  Define concatenation of dependent paths, prove that application of dependent functions preserves concatenation, and write out the precise induction principle for the torus $T^2$ with its computation rules.\index{torus}
\end{ex}

\begin{ex}\label{ex:suspS1}
  Prove that $\eqv{\susp \Sn^1}{\Sn^2}$, using the explicit definition of $\Sn^2$ in terms of $\base$ and $\surf$ given in \autoref{sec:circle}.
\end{ex}

\begin{ex}\label{ex:torus-s1-times-s1}
  Prove that the torus $T^2$ as defined in \autoref{sec:cell-complexes} is equivalent to $\Sn^1\times \Sn^1$.
  (Warning: the path algebra for this is rather difficult.)
\end{ex}

\begin{ex}\label{ex:nspheres}
  Define dependent $n$-loops\index{loop!dependent n-@dependent $n$-} and the action of dependent functions on $n$-loops, and write down the induction principle for the $n$-spheres as defined at the end of \autoref{sec:circle}.
\end{ex}

\begin{ex}\label{ex:susp-spheres-equiv}
  Prove that $\eqv{\susp \Sn^n}{\Sn^{n+1}}$, using the definition of $\Sn^n$ in terms of $\Omega^n$ from \autoref{sec:circle}.
\end{ex}

\begin{ex}\label{ex:spheres-make-U-not-2-type}
  Prove that if the type $\Sn^2$ belongs to some universe \type, then \type is not a 2-type.
\end{ex}

\begin{ex}\label{ex:monoid-eq-prop}
  Prove that if $G$ is a monoid and $x:G$, then $\sm{y:G}((x\cdot y = e) \times (y\cdot x =e))$ is a mere proposition.
  Conclude, using the principle of unique choice (\autoref{cor:UC}), that it would be equivalent to define a group to be a monoid such that for every $x:G$, there merely exists a $y:G$ such that $x\cdot y = e$ and $y\cdot x=e$.
\end{ex}

\begin{ex}\label{ex:free-monoid}
  Prove that if $A$ is a set, then $\lst A$ is a monoid.
  Then complete the proof of \autoref{thm:free-monoid}.\index{monoid!free}
\end{ex}

\begin{ex}\label{ex:unnatural-endomorphisms}
  Assuming \LEM{}, construct a family $f:\prd{X:\type}(X\to X)$ such that $f_\bool:\bool\to\bool$ is the nonidentity automorphism.\index{automorphism!of 2, nonidentity@of $\bool$, nonidentity}
\end{ex}

\begin{ex}\label{ex:funext-from-interval}
  Show that the map constructed in \autoref{thm:interval-funext} is in fact a quasi-inverse to $\happly$, so that an interval type implies the full function extensionality axiom.
  (You may have to use \autoref{ex:strong-from-weak-funext}.)
\end{ex}

\begin{ex}\label{ex:susp-lump}
  Prove the universal property of suspension:
  \[ \Parens{\susp A \to B} \eqvsym \Parens{\sm{b_n : B} \sm{b_s : B} (A \to (b_n = b_s)) } \]
\end{ex}

\begin{ex}\label{ex:alt-integers}
  Show that $\eqv{\Z}{\N+\unit+\N}$.
  Show that if we were to define $\Z$ as $\N+\unit+\N$, then we could obtain \autoref{thm:sign-induction} with judgmental computation rules.
\end{ex}

\index{type!higher inductive|)}%

% Local Variables:
% TeX-master: "hott-online"
% End:


\bgroup % restrict the scope of our macro definitions to this file

\newbox\pbbox
\setbox\pbbox=\hbox{\xy \POS(65,0)\ar@{-} (0,0) \ar@{-} (65,65)\endxy}
\def\pb{\save[]+<3.5mm,-3.5mm>*{\copy\pbbox} \restore}

\newcommand{\comp}[2]{\ensuremath{{#2} \circ {#1}}}
\newcommand{\istype}[1]{\mathsf{is}\mbox{-}{#1}\mbox{-}\mathsf{type}}
% \newcommand{\nplusone}{\ensuremath{(n\mbox{\rm{+}}1)}}
% \newcommand{\nminusone}{\ensuremath{(n\mbox{\rm{-}}1)}}
\newcommand{\nplusone}{\ensuremath{(n+1)}}
\newcommand{\nminusone}{\ensuremath{(n-1)}}
\newcommand{\fact}{\textsf{fact}}


\chapter{\texorpdfstring{$n$}{n}-types}
\label{cha:hlevels}

One of the basic notions of homotopy theory is that of a \emph{homotopy $n$-type}: a space containing no interesting homotopy above dimension $n$.
For instance, a homotopy $0$-type is essentially a set, containing no nontrivial paths, while a homotopy $1$-type may contain nontrivial paths, but no nontrivial paths between paths.
Homotopy $n$-types are also called \emph{$n$-truncated spaces}.
We have mentioned this notion already in \autoref{sec:basics-sets}; our first goal in this chapter is to give it a formal definition in homotopy type theory.

A dual notion to truncatedness is connectedness: a space is \emph{$n$-connected} if it has no interesting homotopy in dimensions $n$ and \emph{below}.
For instance, a space is $0$-connected if it has only one connected component, and $1$-connected (also called ``simply connected'') if it also has no nontrivial loops (though it may have nontrivial higher loops between loops).

The duality between truncatedness and connectedness is most easily seen by extending both notions to maps.
We call a map \emph{$n$-truncated} or \emph{$n$-connected} if all its fibers are so.
Then $n$-connected and $n$-truncated maps form the two classes of maps in an \emph{orthogonal factorization system}, i.e.\ every map factors uniquely as an $n$-connected map followed by an $n$-truncated one.

In the case $n=-1$, the $n$-truncated maps are the monomorphisms and the $n$-connected maps are the surjections, as defined in \autoref{sec:mono-surj}.
Thus, the $n$-connected factorization system is a massive generalization of the standard image factorization of a function between sets into a surjection followed by an injection.
At the end of this chapter, we sketch briefly an even more general theory: any type-theoretic \emph{modality} gives rise to an analogous factorization system.


\section{$n$-types}
\label{sec:n-types}

As mentioned in \autoref{sec:logic,sec:contractibility}, it turns out to be convenient to define $n$-types starting two levels below zero, with the $(-1)$-types being the mere propositions and the $(-2)$-types the contractible ones.

\begin{defn}\label{def:hlevel}
  Define the predicate $\istype{n} : \type \to \type$ for $n \geq -2$ by recursion as follows:
  \[ \istype{n}(X) \defeq
  \begin{cases}
    \iscontr(X) & \text{ if } n = -2 , \\
    \prd{x,y : X} \istype{n'}(\id[X]{x}{y}) & \text{ if } n = n'+1
  \end{cases}
  \]
  We say that $X$ \define{is an $n$-type} if $\istype{n}(X)$ is inhabited.
\end{defn}

\begin{rmk}
  The number $n$ in \autoref{def:hlevel} ranges over all integers greater than or equal to $-2$.
  We could make sense of this formally by defining a type $\Z_{{\geq}-2}$ of such integers (a type whose induction principle is identical to that of $\Nat$), or instead defining a predicate $\istype{(k-2)}$ for $k : \Nat$.
  Either way, we can prove theorems about $n$-types by induction on $n$, with $n = -2$ as the base case.
\end{rmk}

\begin{eg}
  We saw in \autoref{thm:prop-minusonetype} that $X$ is a $(-1)$-type if and only if it is a mere proposition.
  Therefore, $X$ is a $0$-type if and only if it is a set.
\end{eg}

We have also seen that there are types which are not sets (\autoref{thm:type-is-not-a-set}).
So far, however, we have not shown for any $n>0$ that there exist types which are not $n$-types.
In \autoref{cha:homotopy}, however, we will show that the $(n+1)$-sphere $\Sn^{n+1}$ is not an $n$-type.
Moreover, in \autoref{sec:whitehead} will give an example of a type that is not an $n$-type for \emph{any} (finite) number $n$.

We begin the general theory of $n$-types by showing they are closed under certain operations and constructors.

\begin{thm}\label{thm:h-level-retracts}
 Let $p : X \to Y$ be a retraction and suppose that $X$ is an $n$-type, for any $n\geq -2$.
 Then $Y$ is also an $n$-type.
\end{thm}

\begin{proof}
 We proceed by induction on $n$.
 The base case $n=-2$ is \autoref{thm:retract-contr}.

 For the inductive step, assume that any retract of an $n$-type is an $n$-type, and that $X$ is an $\nplusone$-type.
 Let $y, y' : Y$; we must show that $\id{y}{y'}$ is an $n$-type.
 Snce $X$ is an $\nplusone$-type, $\id[X]{s(y)}{s(y')}$ is an $n$-type.
 We claim that $\id{y}{y'}$ is a retract of $\id[X]{s(y)}{s(y')}$.
 As the section, we have
 \[ \apfunc s : (y=y') \to (s(y)=s(y')). \]
 For the retraction, we define $t:(s(y)=s(y'))\to(y=y')$ by
 \[ t(q) \defeq  \opp{\epsilon_y} \ct \ap p q \ct \epsilon_{y'}.\]
 To show that $t$ is a retraction of $\apfunc s$, we must show that
 \[ \opp{\epsilon_y} \ct \ap p {\ap sr} \ct \epsilon_{y'} = q \]
 for any $r:y=y'$.
 But this follows easily from \autoref{lem:htpy-natural}.
\end{proof}

As an immediate corollary we obtain the stability of $n$-types under equivalence (which is also immediate from univalence):

\begin{cor}\label{cor:preservation-hlevels-weq}
 If $\eqv{X}{Y}$ and $X$ is an $n$-type, then so is $Y$.
\end{cor}

Recall also the notion of monomorphism from \autoref{sec:mono-surj}.

\begin{thm}\label{thm:isntype-mono}
  If $f:X\to Y$ is a monomorphism and $Y$ is an $n$-type for some $n\ge -1$, then so is $X$.
\end{thm}
\begin{proof}
  Let $x,x':X$; we must show that $\id[X]{x}{x'}$ is an $\nminusone$-type.
  But since $f$ is a monomorphism, we have $(\id[X]{x}{x'}) \simeq (\id[Y]{f(x)}{f(x')})$, and the latter is an $\nminusone$-type by assumption.
\end{proof}

Note that this theorem fails when $n=-2$: the map $\emptyt \to \unit$ is a monomorphism, but $\unit$ is a $(-2)$-type while $\emptyt$ is not.

\begin{thm}\label{thm:hlevel-cumulative}
 The hierarchy of $n$-types is cumulative in the following sense:
   given a number $n \geq -2$, if $X$ is an $n$-type, then it is also an $\nplusone$-type.
\end{thm}

\begin{proof}
 We proceed by induction on $n$.

 For $n = -2$, we need to show that a contractible type, say, $A$, has contractible path spaces.
       Let $a_0: A$ be the center of contraction of $A$, and let $x, y : A$. We show that $\id[A]{x}{y}$
       is contractible.
       By contractibility of $A$ we have a path $\contr_x \ct \opp{\contr_y} : x = y$, which we choose as
       the center of contraction for $\id{x}{y}$.
       Given any $p : x = y$, we need to show $p = \contr_x \ct \opp{\contr_y}$.
           By identity elimination, it suffices to show that
        $\refl{x} = \contr_x \ct \opp{\contr_x}$, which is trivial.

 For the inductive step, we need to show that $\id[X]{x}{y}$ is an $\nplusone$-type, provided
          that $X$ is an $\nplusone$-type. Applying the induction hypothesis to $\id[X]{x}{y}$
         yields the desired result.
\end{proof}

% \section{Preservation under constructors}
% \label{sec:ntype-pres}

We now show that $n$-types are preserved by most type forming operations.

\begin{thm}\label{thm:ntypes-sigma}
 Let $n \geq -2$, and let $A : \type$ and $B : A \to \type$.
 If $A$ is an $n$-type and for all $a : A$, $B(a)$ is an $n$-type, then so is $\sm{x : A} B(x)$.
\end{thm}

\begin{proof}
 We proceed by induction on $n$.

 For $n = -2$, we choose the center of contraction for $\sm{x : A} B(x)$ to be the pair
       $(a_0, b_0)$, where $a_0 : A$ is the center of contraction of $A$ and $b_0 : B(a_0)$ is the center of contraction of $B(a_0)$.
       Given any other element $(a,b)$ of $\sm{x : A} B(x)$, we provide a path $\id{(a, b)}{(a_0,b_0)}$
       by contractibility of $A$ and $B(a_0)$, respectively.

 For the inductive step, suppose that $A$ is an $\nplusone$-type and
         for any $a : A$, $B(a)$ is an $\nplusone$-type. We show that $\sm{x : A} B(x)$ is an $\nplusone$-type:
      fix $(a_1, b_1)$ and $(a_2,b_2)$ in $\sm{x : A} B(x)$,
     we show that $\id{(a_1, b_1)}{(a_2,b_2)}$ is an $n$-type.
      By \autoref{thm:path-sigma} we have
      \[ \eqvspaced{(\id{(a_1, b_1)}{(a_2,b_2)})}{\sm{p : \id{a_1}{a_2}} (\id[B(a_2)]{\trans{p}{b_1}}{b_2})} \]
   and by preservation of $n$-types under equivalences (\autoref{cor:preservation-hlevels-weq})
   it suffices to prove that the latter is an $n$-type. This follows from the
   induction hypothesis.
\end{proof}

As a special case, if $A$ and $B$ are $n$-types, so is $A\times B$.
Note also that \autoref{thm:hlevel-cumulative} implies that if $A$ is an $n$-type, then so is $\id[A]xy$ for any $x,y:A$.
Combining this with \autoref{thm:ntypes-sigma}, we see that for any functions $f:A\to C$ and $g:B\to C$ between $n$-types, the \define{pullback}
\[ A\times_C B \defeq \sm{x:A}{y:B} (f(x)=g(y)) \]
is also an $n$-type.
More generally, $n$-types are closed under all \emph{limits}.

\begin{thm}\label{thm:hlevel-prod}
 Let $n\geq -2$, and let $A : \type$ and $B : A \to \type$.
 If for all $a : A$, $B(a)$ is an $n$-type, then so is $\prd{x : A} B(x)$.
\end{thm}

\begin{proof}
  We proceed by induction on $n$.
  For $n = -2$, the result is simply \autoref{thm:contr-forall}.

  For the inductive step, let $f, g : \prd{a:A}B(a)$.
  We need to show that $\id{f}{g}$ is an $n$-type.
  By function extensionality and closure of $n$-types under equivalence, it suffices to show that $\prd{a : A} (\id[B(a)]{f(a)}{g(a)})$ is an $n$-type.
  This follows from the inductive hypothesis.
\end{proof}

As a special case of the above theorem, the function space $A \to B$ is an $n$-type provided that $B$ is an $n$-type.
We can now generalize our observations in \autoref{cha:basics} that $\isset(A)$ and $\isprop(A)$ are mere propositions.

\begin{thm}\label{thm:isaprop-isofhlevel}
 For any $n \geq -2$ and any type $X$, the type $\istype{n}(X)$ is a mere proposition.
\end{thm}
\begin{proof}
  We proceed by induction with respect to $n$.

 For the base case, we need to show that for any $X$, the type $\iscontr(X)$ is a mere proposition.
 By \autoref{thm:contr-unit}, it suffices to show that if $X$ is contractible, then $\iscontr(X)$ is a mere proposition.
 But this follows from \autoref{thm:contr-contr}.

For the inductive step we need to show
\[\prd{X : \type} \isprop (\istype{n}(X)) \to \prd{X : \type} \isprop (\istype{\nplusone}(X)) \]
To show the conclusion of this implication, we need to show that for any type $X$, the type
\[\prd{x, x' : X}\istype{n}(x = x')\]
is a mere proposition. By \autoref{thm:hlevel-prod} it suffices to show that for any $x, x' : X$, the type $\istype{n}(x =_X x')$ is a mere
proposition.
But this follows from the induction hypothesis applied to the type $(x =_X x')$.
\end{proof}

Finally, we show that the type of $n$-types is itself an $\nplusone$-type.
We define this to be:
\[\ntype{n} \defeq \sm{X : \type} \istype{n}(X) \]
%If necessary, we may specify the universe $\UU$ by writing ${\ntype{n}}_\UU$.
In particular, we have $\prop \defeq \ntype{(-1)}$ and $\set \defeq \ntype{0}$, as defined in \autoref{cha:basics}.
Note that just as for \prop and \set, because $\istype{n}(X)$ is a mere proposition, by \autoref{thm:path-subset} for any $(X,p), (X',p'):\ntype{n}$ we have
\begin{align*}
  \Big(\id[\ntype{n}]{(X, p)}{(X', p')}\Big) &\simeq (\id[\type] X X')\\
  &\simeq (\eqv{X}{X'}).
\end{align*}

\begin{thm}\label{thm:hleveln-of-hlevelSn}
 For any $n \geq -2$, the type $\ntype{n}$ is an $\nplusone$-type.
\end{thm}
\begin{proof}%[Proof of \autoref{thm:hleveln-of-hlevelSn}]
  Let $(X, p), (X', p') : \ntype{n}$; we need to show that $\id{(X, p)}{(X', p')}$ is an $n$-type.
  By the above observation, this type is equivalent to $\eqv{X}{X'}$.
  Next, we observe that the projection
  \[(\eqv{X}{X'}) \hookrightarrow (X \rightarrow X').\]
  is a monomorphism, so that if $n\geq -1$, then by \autoref{thm:isntype-mono} it suffices to show that $X \rightarrow X'$ is an $n$-type.
  But since $n$-types are preserved under the arrow type, this reduces to an assumption that $X'$ is an $n$-type.

  In the case $n=-2$, this argument shows that $\eqv{X}{X'}$ is a $(-1)$-type --- but it is also inhabited, since any two contractible types
are equivalent to \unit, and hence to each other.
  Thus, $\eqv{X}{X'}$ is also a $(-2)$-type.
\end{proof}

\section{UIP and Hedberg's theorem}
\label{sec:hedberg}

In \autoref{sec:basics-sets} we defined a type $X$ to be a \emph{set} if for all $x, y : X$ and $p, q : x =_X y$ we have $p = q$.
In conventional type theory, this property goes by the name of \emph{uniqueness of identity proofs (UIP)}.
We have seen also that it is equivalent to being a $0$-type in the sense of the previous section.
Here is another equivalent characterization, involving Streicher's ``axiom K" \cite{StreicherK}:

\begin{thm}\label{thm:h-set-uip-K}
 A type $X$ is a set if and only if it satisfies \emph{Axiom K}: for all $x : X$ and $p : (x =_A x)$ we have $p = \refl{x}$.
\end{thm}

\begin{proof}
  Clearly Axiom K is a special case of UIP.
  Conversely, if $X$ satisfies Axiom K, let $x, y : X$ and $p, q : (\id{x}{y})$; we want to show $p=q$.
  But induction on $q$ reduces this goal precisely to Axiom K.
\end{proof}

We stress that \emph{we} are not assuming the K principle as an axiom!
It is simply a property which a particular type may or may not satisfy (which is equivalent to being a set).

The following theorem is another useful way to show that types are sets.

\begin{thm}\label{thm:h-set-refrel-in-paths-sets}
  Suppose $R$ is a reflexive mere relation on a type $X$ implying identity.
  Then $X$ is a set, and $R(x,y)$ is equivalent to $\id[X]{x}{y}$ for all $x,y:X$.
\end{thm}

\begin{proof}
  Let $\rho : \prd{x:X} R(x,x)$ witness the reflexivity of $R$, and let $f : \prd{x,y:X} R(x,y) \to (\id[X]{x}{y})$ be a witness that $R$
implies identity.
  Note first that the two statements in the theorem are equivalent.
  For on one hand, if $X$ is a set, then $\id[X]xy$ is a mere proposition, and since it is logically equivalent to the mere proposition
$R(x,y)$ by hypothesis, it must also be equivalent to it.
  On the other hand, if $\id[X]xy$ is equivalent to $R(x,y)$, then like the latter it is a mere proposition for all $x,y:X$, and hence $X$
is a set.

  We give two proofs of this theorem.
  The first shows directly that $X$ is a set; the second shows directly that $R(x,y)\simeq (x=y)$.

  \textbf{First proof:} we show that $X$ is a set.
  The idea is the same as that of \autoref{thm:prop-set}: the function $f$ must be continuous in its arguments $x$ and $y$.
  However, it is slightly more notationally complicated because we have to deal with the additional argument of type $R(x,y)$.

  Firstly, for any $x:X$ and $p:\id[X]xx$, consider $\apdfunc{f(x)}(p)$.
  This is a dependent path from $f(x,x)$ to itself.
  Since $f(x,x)$ is still a function $R(x,x) \to (\id[X]xy)$, by \autoref{thm:dpath-arrow} this yields a path
  \[\trans{p}{f(x,x,r)} = f(x,x,\trans{p}r).
  \]
  On the left-hand side, we have transport in an identity type, which is concatenation.
  And on the right-hand side, we have $\trans{p}r = r$, since both lie in the mere proposition $R(x,x)$.
  Thus, substituting $r\defeq \rho(x)$, we obtain
  \[ f(x,x,\rho(x)) \ct p = f(x,x,\rho(x)). \]
  By cancellation, $p=\refl{x}$.
  So $X$ satisfies Axiom K, and hence is a set.

  \textbf{Second proof:} we show that each $f(x,y) : R(x,y) \to \id[X]{x}{y}$ is an equivalence.
  By \autoref{thm:total-fiber-equiv}, it suffices to show that $f$ induces an equivalence of total spaces:
  \begin{equation*}
    \eqv{\big(\sm{y:X}R(x,y)\big)}{\big(\sm{y:X}\id[X]{x}{y}\big)}.
  \end{equation*}
  By \autoref{thm:contr-paths}, the type on the right is contractible, so it
  suffices to show that the type on the left is contractible. As the center of
  contraction we take the pair $\pairr{x,\rho(x)}$.  It remains to show, for
  every ${y:X}$ and every ${H:R(x,y)}$ that
  \begin{equation*}
    \id{\pairr{x,\rho(x)}}{\pairr{y,H}}.
  \end{equation*}
  But since $R(x,y)$ is a mere proposition, by \autoref{thm:path-sigma} it suffices to show that
  $\id[X]{x}{y}$, which we get from $f(H)$.
\end{proof}

\begin{cor}\label{notnotstable-equality-to-set}
  If a type $X$ has the property that $\neg\neg(x=y)\to(x=y)$ for any $x,y:X$, then $X$ is a set.
\end{cor}

Another convenient way to show that a type is a set is by way of the following property.

\begin{defn}\label{defn:decidable-equality}
 A type $X$ has \textbf{decidable equality} if for all $x, y : X$ we have
 \[(x =_X y) + \neg (x =_X y).\]
\end{defn}

In the propositions-as-types language, we can say that $X$ has decidable equality if for every $x,y:X$, either $x=y$ or $x\neq y$.
Note that this is a very constructive form of ``or''.
LEM implies (using \autoref{thm:isprop-forall}) that \emph{any} type $X$ has \emph{decidable mere equality}, meaning
\[\prd{x,y:X} \big(\brck{x=y} + \brck{x\neq y}\big).\]
\autoref{defn:decidable-equality} asserts moreover that a path $x=y$ can be chosen, when it exists, continuously (or computably, or
functorially) in $x$ and $y$.
This turns out to imply that $X$ is a set, by way of \autoref{thm:h-set-refrel-in-paths-sets} and the following lemma.

\begin{lem}
For any type $A$ we have $(A+\neg A)\to(\neg\neg A\to A)$.
\end{lem}

\begin{proof}
Suppose $x:A+\neg A$. We have two cases to consider.
If $x$ is $\inl(a)$ for some $a:A$, then we have the constant function $\neg\neg A
\to A$ which maps everything to $a$. If $x$ is $\inr(f)$ for some $f:\neg A$,
we have the term $g(f):\emptyt$ for any $g:\neg\neg A$. Hence we may use
\textit{ex falso quodlibet} to obtain an element of $A$ for any $g:\neg\neg A$.
\end{proof}

\begin{thm}[Hedberg]\label{thm:hedberg}
  If $X$ has decidable equality, then $X$ is a set.
\end{thm}

\begin{proof}
If $X$ has decidable equality, it follows that $\neg\neg(x=y)\to(x=y)$ for any
$x,y:X$. Therefore, Hedbergs theorem follows from 
\autoref{notnotstable-equality-to-set}.
\end{proof}

There is, of course, a strong connection between this theorem and \autoref{thm:not-lem}.
The form of LEM denied by \autoref{thm:not-lem} clearly implies that every type has decidable equality, and hence is a set; which we know is
not the case.

Recall that in \autoref{thm:nat-set} we observed that $\nat$ is a set, using our characterization of its equality types in
\autoref{sec:compute-nat}.
A more traditional proof of this theorem uses only~\eqref{eq:zero-not-succ} and~\eqref{eq:suc-injective}, rather than the full
characterization of \autoref{thm:path-nat}, with \autoref{thm:hedberg} to fill in the blanks.

\begin{thm}\label{prop:nat-is-set}
 The type $\Nat$ of natural numbers has decidable equality, and hence is a set.
\end{thm}

\begin{proof}
  Let $x, y : \Nat$ be given; we proceed by induction on $x$ and case analysis on $y$ to prove $(x=y)+\neg(x=y)$.
  If $x \jdeq 0$ and $y \jdeq 0$, we take $\inl(\refl{}(0))$.
  If $x \jdeq 0$ and $y \jdeq \suc(n)$, then by~\eqref{eq:zero-not-succ} we get $\neg (0 = \suc (n))$.

  For the inductive step, let $x \jdeq \suc (n)$.
  If $y \jdeq 0$, we use~\eqref{eq:zero-not-succ} again.
  Finally, if $y \jdeq \suc (m)$, the induction hypothesis gives $(m = n)+\neg(m = n)$.
  In the first case, if $p:m=n$, then $\ap \suc p:\suc(m)=\suc(n)$.
  And in the second case,~\eqref{eq:suc-injective} yields $\neg(\suc(m)=\suc(n))$.
\end{proof}

Although Hedberg's theorem appears rather special to sets ($0$-types), ``Axiom K'' generalizes naturally to $n$-types.
Note that the ordinary Axiom K (as a property of a type $X$) states that for all $x:X$, the loop space $\Omega(X,x)$ (see \cref{def:loopspace}) is contractible.
Since $\Omega(X,x)$ is always inhabited (by $\refl{x}$), this is equivalent to its being a mere proposition (a $(-1)$-type).
Since $0 = (-1)+1$, this suggests the following generalization.

\begin{thm}\label{thm:hlevel-loops}
  For any $n\geq -1$, a type $X$ type is an $\nplusone$-type if and only if for all $x : X$, the type $\Omega(X, x)$ is an $n$-type.
\end{thm}

Before proving this, we prove an auxiliary lemma:

\begin{lem}\label{lem:hlevel-if-inhab-hlevel}
  Given $n \geq -1$ and $X : \type$.
  If, given any inhabitant of $X$ it follows that $X$ is an $n$-type, then $X$ is an $n$-type.
\end{lem}
\begin{proof}
  Let $f : X \to \istype{n}(X)$ be the given map.
  We need to show that for any $x, x' : X$, the type $\id{x}{x'}$ is an $\nminusone$-type.
  But then $f(x)$ shows that $X$ is an $n$-type, hence all its path spaces are $\nminusone$-types.
\end{proof}

\begin{proof}[Proof of \autoref{thm:hlevel-loops}]
  The ``only if'' direction is obvious, since $\Omega(X,x)\defeq (\id[X]xx)$.
  Conversely, in order to show that $X$ is an $\nplusone$-type, we need to show that for any $x, x' : X$, the type $\id{x}{x'}$ is an
$n$-type.
  Following \autoref{lem:hlevel-if-inhab-hlevel} it suffices to give a map
  \[ (\id{x}{x'}) \to \istype{n}(\id{x}{x'})  .\]
  By path induction, it suffices to do this when $x\jdeq x'$, in which case it follows from the assumption that $\Omega(X, x)$ is an
$n$-type.
\end{proof}

By induction and some slightly clever whiskering, we can obtain a generalization of the K property to $n>0$.

\begin{thm}\label{thm:ntype-nloop}
  For every $n\ge 0$, a type $A$ is an $n$-type if and only if $\Omega^{n+1}(A,a)$ is contractible for all $a:A$.
\end{thm}
\begin{proof}
  The case $n=0$ is \autoref{thm:h-set-uip-K}.
  By induction, suppose the statement holds for $n:\N$.
  By \autoref{thm:hlevel-loops}, $A$ is an $(n+1)$-type iff $\Omega(A,a)$ is an $n$-type for all $a:A$.
  By the inductive hypothesis, the latter is equivalent to saying that $\Omega^{n+1}(\Omega(A,a),p)$ is contractible for all $p:\Omega(A,a)$.

  Since $\Omega^{n+2}(A,a) \defeq \Omega^{n+1}(\Omega(A,a),\refl{a})$, and $\Omega^{n+1} = \Omega^n \circ \Omega$, it will suffice to show that $\Omega(\Omega(A,a),p)$ is equal to $\Omega(\Omega(A,a),\refl{a})$, in the type $\pointed\type$ of pointed types.
  For this, it suffices to give an equivalence
  \[ g : \Omega(\Omega(A,a),p) \simeq \Omega(\Omega(A,a),\refl{a}) \]
  which carries the basepoint $\refl{p}$ to the basepoint $\refl{\refl{a}}$.
  For $q:p=p$, define $g(q):\refl{a} = \refl{a}$ to be the following composite:
  \[ \refl{a} = p\ct \opp p \overset{q}{=} p\ct\opp p = \refl{a}, \]
  where the path labeled ``$q$'' is actually $\apfunc{\lam{r} r\ct\opp p} (q)$.
  Then $g$ is an equivalence because it is a composite of equivalences
  \[ (p=p) \xrightarrow{\apfunc{\lam{r} r\ct\opp p}} (p\ct \opp p = p\ct \opp p) \xrightarrow{i\ct - \ct \opp i} (\refl{a} = \refl{a}). \]
  using \autoref{eg:concatequiv,thm:paths-respects-equiv}, where $i:\refl{a} = p\ct \opp p$ is the canonical equality.
  And it is evident that $g(\refl{p}) = \refl{\refl{a}}$.
\end{proof}

% \begin{defn}
%   A function $f:X\to A$ is said to be \emph{null-homotopic} if there is an $a:A$ such that $f=(\lam{x} a)$.
%   In other words, we define
%   \begin{equation*}
%     \mathsf{hNull}(f)\defeq\sm{a:A} (f=(\lam{x} a)) .
%   \end{equation*}
% \end{defn}

% \begin{lem}\label{lem:hnull_to_map_hnull}
% If a function $f:X\to A$ is null-homotopic, then so is
% \[\apfunc f : (x= y)\to (f(x)= f(y)).\]
% \end{lem}

% \begin{proof}
% Suppose $H:\sm{a:A} (f=\lam{x} a)$.  A proof by path induction on $p:x= y$ reveals that there is a homotopy
% \begin{equation*}
% \apfunc{f} = \lam{p} (\pi_2 H(y))^{-1}\ct (\pi_2 H(x)).\qedhere
% \end{equation*}
% \end{proof}

% Recall from \autoref{sec:suspension} that we can define the $n$-sphere $\Sn^n$ inductively as the suspension of the $(n-1)$-sphere, starting
% with $\Sn^{-1}\defeq \emptyt$.

% \begin{thm}\label{thm:sphere_n_hnull_to_hlevel_sn}
% For every $n:\N$, a type $A$ is an $n$-type if all the functions from $\Sn^{n+1}$ to $A$ are null-homotopic.
% \end{thm}

% \begin{proof}
% The exact statement which we will prove is
% \begin{equation*}
% \prd{n:\N}{A:\type} \big(\prd{f:\Sn^n\to A} \mathsf{hNull}(f)\big)\to\istype{\nminusone}(A).
% \end{equation*}
% The proof is by induction on $n$.  In the case $n\jdeq 0$ we have that
% \begin{equation*}
% \big(\prd{f:\Sn^{0}\to A} \mathsf{hNull}(f)\big)\simeq\prd{x,y:A} (x= y),
% \end{equation*}
% so we see immediately that a type $A$ is a proposition whenever every function $f:\Sn^0\to A$ is null-homotopic.

% Now suppose that we have a function of type
% \begin{equation*}
% \prd{A:\type}{f:\Sn^n\to A}{t:\Sn^n} (f(t)= f(\north^n)\big)\to\istype{\nminusone}(A).
% \end{equation*}
% and let $A$ be a type and let 
% \begin{equation*}
% H:\prd{f:\Sn^{n+1}\to A}{t:\Sn^{n+1}} (f(t)= f(\north^{n+1})).
% \end{equation*}
% We wish to show that $A$ is an $n$-type, which we will do by showing that $x= y$ is an $\nminusone$-type for each $x,y:A$.
% Suppose that $x,y:A$; then by the induction hypothesis it suffices to show that every function $\Sn^n\to (x= y)$ is null-homotopic.
% We have that $\Sn^n\to(x= y)$ is equivalent to
% \[\sm{f:S^{n+1}\to A} (f(\north^{n+1})= x)\times(f(\south^{n+1})= y).\]
% Suppose that $f:\Sn^n\to A$ is a function and suppose that $\tilde{f}$ corresponds to $f$ via the indicated equivalence. Then $\tilde{f}$ is
% null-homotopic, and hence $\lam{t} \tilde{f}(\merid^n(t))$ is null-homotopic (by \autoref{lem:hnull_to_map_hnull}).
% It follows that $f$ is null-homotopic.
% \end{proof}


\section{Truncations}
\label{sec:truncations}

In \autoref{subsec:prop-trunc} we introduced the propositional truncation, which makes the ``best approximation'' of a type that is a mere
proposition, i.e.\ a $(-1)$-type.
In \autoref{sec:hittruncations} we constructed this truncation as a higher inductive type, and gave one way to generalize it to a
0-truncation.
We now explain a better generalization of this, which truncates any type into an $n$-type for any $n\geq -2$.

The idea is to make use of \autoref{thm:ntype-nloop}, which states that $A$ is an $n$-type just when $\Omega^{n+1}(A,a)$ is contractible for
all $a:A$, and \autoref{lem:susp-loop-adj}, which implies that $\Omega^{n+1}(A,a) \simeq \Map_*(\Sn^{n+1},(A,a))$, where $\Sn^{n+1}$ is
equipped with some basepoint which we may as well call \base.
However, contractibility of $\Map_*(\Sn^{n+1},(A,a))$ is something that we can ensure directly by giving path constructors.

We might first of all try to define $\trunc nA$ to be generated by a function $\tproj n- : A \to \trunc n A$, together with for each
$r:\Sn^{n+1} \to \trunc n A$ and each $x:\Sn^{n+1}$, a path $s_r(x):r(x) = r(\base)$.
%
But this does not quite work, for the same reason that \autoref{rmk:spokes-no-hub} fails.
Instead, we use the full ``hub and spoke'' construction as in \autoref{sec:hubs-spokes}.

Thus, we take $\trunc nA$ to be the higher inductive type generated by:
\begin{itemize}
\item a function $\tproj n- : A \to \trunc n A$,
\item for each $r:\Sn^{n+1} \to \trunc n A$, a \emph{hub} point $h(r):\trunc n A$, and
\item for each $r:\Sn^{n+1} \to \trunc n A$ and each $x:\Sn^{n+1}$, a \emph{spoke} path $s_r(x):r(x) = h(r)$.
\end{itemize}

\noindent
The existence of these constructors is now enough to show:

\begin{lem}
  $\trunc n A$ is an $n$-type.
\end{lem}
\begin{proof}
  By \autoref{thm:ntype-nloop}, it suffices to show that $\Omega ^{n+1}(\trunc nA,b)$ is contractible for all $b:\trunc nA$, which by
\autoref{lem:susp-loop-adj} is equivalent to $\Map_*(\Sn^{n+1},(\trunc nA,b))$.
  As center of contraction for the latter, we choose the function $c_b:\Sn^{n+1} \to \trunc nA$ which is constant at $b$, together with
$\refl b : c_b(\base) = b$.

  Now, an arbitrary element of $\Map_*(\Sn^{n+1},(\trunc nA,b))$ consists of a map $r:\Sn^{n+1} \to \trunc n A$ together with a path
$p:r(\base)=b$.
  By function extensionality, to show $r = c_b$ it suffices to give, for each $x:\Sn^{n+1}$, a path $r(x)=c_b(x) \jdeq b$.
  We choose this to be the composite $s_r(x) \ct \opp{s_r(\base)} \ct p$, where $s_r(x)$ is the spoke at $x$.


  Finally, we must show that when transported along this equality $r=c_b$, the path $p$ becomes $\refl b$.
  By transport in path types, this means we need
  \[\opp{(s_r(\base) \ct \opp{s_r(\base)} \ct p)} \ct p = \refl b.\]
  But this is immediate from path operations.
\end{proof}

To show the desired universal property of the $n$-truncation, we need the induction principle.
We extract this from the constructors in the usual way; it says that given $P:\trunc nA\to\type$ together with
\begin{itemize}
\item For each $a:A$, an element $g(a) : P(\tproj na)$,
\item For each $r:\Sn^{n+1} \to \trunc n A$ and $r':\prd{x:\Sn^{n+1}}, P(r(x))$, an element $h'(r,r'):P(h(r))$.
\item For each $r:\Sn^{n+1} \to \trunc n A$ and $r':\prd{x:\Sn^{n+1}}, P(r(x))$, and each $x:\Sn^{n+1}$, a dependent path
$\dpath{P}{s_r(x)}{r'(x)}{h'(r,r')}$.
\end{itemize}
there exists a section $f:\prd{x:\trunc n A} P(x)$ with $f(\tproj n a) \jdeq g(a)$ for all $a:A$.
To make this more useful, we reformulate it as follows.

\begin{thm}\label{thm:truncn-ind}
  For any type family $P:\trunc n A \to \type$ such that each $P(x)$ is an $n$-type, and any function $g : \prd{a:A} P(\tproj n a)$, there
exists a section $f:\prd{x:\trunc n A} P(x)$ such that $f(\tproj n a)\defeq g(a)$ for all $a:A$.
\end{thm}
\begin{proof}
  It will suffice to construct the second and third data listed above, since $g$ has exactly the type of the first datum.
  Given $r:\Sn^{n+1} \to \trunc n A$ and $r':\prd{x:\Sn^{n+1}}, P(r(x))$, we have $h(r):\trunc n A$ and $s_r :\prd{x:\Sn^{n+1}} (r(x) =
h(r))$.
  Define $t:\Sn^{n+1} \to P(h(r))$ by $t(x) \defeq \trans{s_r(x)}{r'(x)}$.
  Then since $P(h(r))$ is $n$-truncated, there exists a point $u:P(h(r))$ and a contraction $v:\prd{x:\Sn^{n+1}} (t(x) = u)$.
  Define $h'(r,r') \defeq u$, giving the second datum.
  Then (recalling the definition of dependent paths), $v$ has exactly the type required of the third datum.
\end{proof}

In particular, if $E$ is some $n$-type, we can consider the constant family of types equal to $E$ for every point of $A$.
Thus, every map $f:A\to{}E$ can be extended to a map $\extend{f}:\trunc nA\to{}E$ defined by $\extend{f}(\tproj na)\defeq f(a)$; this is the \emph{recursion principle} for $\trunc n A$.

The induction principle also implies a uniqueness principle for functions of this form.
Namely, if $E$ is an $n$-type and $g,g':\trunc nA\to{}E$ are such
that $g(\tproj na)=g'(\tproj na)$ for every $a:A$, then $g(x)=g'(x)$ for all $x:\trunc nA$, since the type $g(x)=g'(x)$ is an $n$-type.
Thus, $g=g'$.
This yields the following universal property.

\begin{lem}[Universal property of truncations]\label{thm:trunc-reflective}
  Let $n\ge-2$, $A:\type$ and $B:\typele{n}$. The following map is an
  equivalence:
  \[\function{(\trunc nA\to{}B)}{A\to{}B}{g}{g\circ\tprojf n}\]
\end{lem}

\begin{proof}
  Given that $B$ is $n$-truncated, any $f:A\to{}B$ can be extended to a map $\extend{f}:\trunc nA\to{}B$.
  The map $\extend{f}\circ\tprojf n$ is equal to $f$, because for every $a:A$ we have $\extend{f}(\tproj na)=f(a)$ by definition.
  And the map $\extend{g\circ\tprojf n}$ is equal to $g$, because they both send $\tproj na$ to $g(\tproj na)$.
\end{proof}

In categorical language, this says that the $n$-types form a \emph{reflective subcategory} of the category of types.
(To state this fully precisely, one ought to use the language of $(\infty,1)$-categories.)
In particular, this implies that the $n$-truncation is functorial:
given $f:A\to B$, applying the recursion principle to the composite $A\xrightarrow{f} B \to \trunc n B$ yields a map $\trunc n f: \trunc n A \to \trunc n B$.
By definition, we have a homotopy
\begin{equation}
  \mathsf{nat}^f_n : \prd{a:A} \trunc n f(\tproj n a) = \tproj n {f(a)},\label{eq:trunc-nat}
\end{equation}
expressing \emph{naturality} of the maps $\tproj n-$.

Uniqueness implies functoriality laws such as $\trunc n {g\circ f} = \trunc n g \circ \trunc n f$ and $\trunc n{\idfunc[A]} = \idfunc[\trunc n A]$, with attendant coherence laws.
We also have higher functoriality, for instance:

\begin{lem}\label{thm:trunc-htpy}
  Given $f,g:A\to B$ and a homotopy $h:f\htpy g$, there is an induced homotopy $\trunc n h : \trunc n f \htpy \trunc n g$ such that the composite
  \begin{equation}
    \xymatrix@C=4pc{\tproj n{f(a)} \ar@{=}[r]^-{\opp{\mathsf{nat}^f_n(a)}} &
      \trunc n f(\tproj n a) \ar@{=}[r]^-{\trunc n h(\tproj na)} &
      \trunc n g(\tproj n a) \ar@{=}[r]^-{\mathsf{nat}^g_n(a)} &
      \tproj n{g(a)}}\label{eq:trunc-htpy}
  \end{equation}
  is equal to $\apfunc{\tproj n-}(h(a))$.
\end{lem}
\begin{proof}
  First, we indeed have a homotopy with components $\apfunc{\tproj n-}(h(a)) : \tproj n{f(a)} = \tproj n{g(a)}$.
  Composing on either sides with the paths $\tproj n{f(a)} = \trunc n f(\tproj n a)$ and $\tproj n{g(a)} = \trunc n g(\tproj n a)$, which arise from the definitions of $\trunc n f$ and $\trunc ng$, we obtain a homotopy $(\trunc n f \circ \tproj n-) \htpy (\trunc n g \circ \tproj n-)$, and hence an equality by function extensionality.
  But since $(-\circ \tproj n-)$ is an equivalence, there must be a path $\trunc nf = \trunc ng$ inducing it, and the coherence laws for function extensionality imply~\eqref{eq:trunc-htpy}.
\end{proof}

The following observation about reflective subcategories is also standard.

\begin{cor}
  A type $A$ is an $n$-type if and only if the map $\tproj n- : A \to \trunc n A$ is an equivalence.
\end{cor}
\begin{proof}
  ``If'' follows from closure of $n$-types under equivalence.
  On the other hand, if $A$ is an $n$-type, we can define $\ext(\idfunc[A]):\trunc n A\to{}A$.
  Then we have $\ext(\idfunc[A])\circ\tproj n-=\idfunc[A]:A\to{}A$ by
  definition.  In order to prove that
  $\tproj n-\circ\ext(\idfunc[A])=\idfunc[\trunc nA]$, we only need to prove
  that $\tproj n-\circ\ext(\idfunc[A])\circ\tproj n-=
  \idfunc[\trunc nA]\circ\tproj n-$.
  This is again true:
  \[\xymatrix{
    A \ar^{\tproj n-}[r] \ar_{\idfunc[A]}[rd] &
    \trunc nA \ar^>>>{\ext(\idfunc[A])}[d] \ar@/^40pt/^{\idfunc[\trunc nA]}[dd] \\
    & A \ar_{\tproj n-}[d] \\
    & \trunc nA}\]
\end{proof}

The category of $n$-types also has some special properties not possesed by all reflective subcategories.
For instance, the reflector $\trunc n-$ preserves finite products.

\begin{thm}\label{cor:trunc-prod}
  For any types $A$ and $B$, the induced map $\trunc n{A\times B} \to \trunc nA \times \trunc nB$ is an equivalence.
\end{thm}
\begin{proof}
  It suffices to show that $\trunc nA \times \trunc nB$ has the same universal property as $\trunc n{A\times B}$.
  Thus, let $C$ be an $n$-type; we have
  \begin{align*}
    (\trunc nA \times \trunc nB \to C)
    &= (\trunc nA \to (\trunc nB \to C))\\
    &= (\trunc nA \to (B \to C))\\
    &= (A \to (B \to C))\\
    &= (A \times B \to C)
  \end{align*}
  using the universal properties of $\trunc nB$ and $\trunc nA$, along with the fact that $B\to C$ is an $n$-type since $C$ is.
  It is straightforward to verify that this equivalence is given by composing with $\tproj n- \times \tproj n-$, as needed.
\end{proof}

We can characterize the path spaces of a truncation using the same method that we used in \autoref{sec:compute-coprod,sec:compute-nat} for
coproducts and natural numbers (and which we will use in \autoref{cha:homotopy} to calculate homotopy groups).
Unsurprisingly, the path spaces in the $(n+1)$-truncation of $A$ are the $n$-truncations of the path spaces of $A$.
Indeed, for any $x,y:A$ there is a canonical map
\begin{equation}
  f:\ttrunc n{x=_Ay}\to \Big(\tproj {n+1}x=_{\trunc{n+1}A}\tproj {n+1}y\Big)\label{eq:path-trunc-map}
\end{equation}
defined by
\[f(\tproj n{p})\defeq \apfunc{\tproj {n+1}-}(p). \]
This definition uses the recursion principle for $\trunc n-$, which is correct because $\trunc {n+1}A$ is $(n+1)$-truncated, so that the
codomain of $f$ is $n$-truncated.

\begin{thm} \label{thm:path-truncation}
  For any $A$ and $x,y:A$ and $n\ge -2$, the map~\eqref{eq:path-trunc-map} is an equivalence; thus we have
  \[ \eqv{\ttrunc n{x=_Ay}}{\Big(\tproj {n+1}x=_{\trunc{n+1}A}\tproj {n+1}y\Big)}. \]
\end{thm}

\begin{proof}
  As in previous situations, we cannot directly define a quasi-inverse to~\eqref{eq:path-trunc-map} because there is no way to induct on an
equality between $\tproj {n+1}x$ and $\tproj {n+1}y$.
  Thus, instead we generalize its type, in order to have general elements of the type $\trunc{n+1}A$ instead of $\tproj {n+1}x$ and $\tproj
{n+1}y$.
  Define $P:\trunc {n+1}A\to\trunc {n+1}A\to\typele{n}$ by
  \[P(\tproj {n+1}x,\tproj {n+1}y)\defeq \trunc n{x=_Ay}\]
  This definition is correct because $\trunc n{x=_Ay}$ is $n$-truncated, and $\typele{n}$ is $(n+1)$-truncated by
\autoref{thm:hleveln-of-hlevelSn}.
  Now for every $u,v:\trunc{n+1}A$, there is a map
  \[\encode:P(u,v) \to \big(u=_{\trunc{n+1}A}v\big)\]
  defined for $u=\tproj {n+1}x$ and $v=\tproj {n+1}y$ and $p:x=y$ by
  \[\encode(\tproj n{p})\defeq \apfunc{\tproj{n+1}-} (p).\]
  Since the codomain of $\encode$ is $n$-truncated, it suffices to define it only for $u$ and $v$ of this form, and then it's just the same
definition as before.
  We also define a function
  \[ r : \prd{u:\trunc{n+1} A} P(u,u) \]
  by induction on $u$, where $r(\tproj{n+1} x) \defeq \tproj n {\refl x}$.

  Now we can define an inverse map
  \[\decode: (u=_{\trunc{n+1}A}v) \to P(u,v)\]
  by
  \[\decode(p) \defeq \transfib{v\mapsto P(u,v)}{p}{r(u)}. \]
  To show that the composite
  \[ (u=_{\trunc{n+1}A}v) \xrightarrow{\decode} P(u,v) \xrightarrow{\encode} (u=_{\trunc{n+1}A}v) \]
  is the identity function, by path induction it suffices to check it for $\refl u : u=u$, in which case what we need to know is that
$\decode(r(u)) = \refl{u}$.
  But since this is an $n$-type, hence also an $(n+1)$-type, we may assume $u\jdeq \tproj {n+1} x$, in which case it follows by definition
of $r$ and $\decode$.
  Finally, to show that 
  \[ P(u,v) \xrightarrow{\encode} (u=_{\trunc{n+1}A}v) \xrightarrow{\decode} P(u,v) \]
  is the identity function, since this goal is again an $n$-type, we may assume that $u=\tproj {n+1}x$ and $v=\tproj {n+1}y$ and that we are
considering $\tproj n p:P(\tproj{n+1}x,\tproj{n+1}y)$ for some $p:x=y$.
  Then we have
  \begin{align*}
    \decode(\encode(\tproj n p)) &= \decode(\apfunc{\tproj{n+1}-}(p))\\
    &= \transfib{v\mapsto P(\tproj{n+1}x,v)}{\apfunc{\tproj{n+1}-}(p)}{\tproj n {\refl x}}\\
    &= \transfib{v\mapsto \trunc n{u=v}}{p}{\tproj n {\refl x}}\\
    &= \tproj n {\transfib{v \mapsto (u=v)}{p}{\refl x}}\\
    &= \tproj n p.
  \end{align*}
  This completes the proof that \encode and \decode are quasi-inverses.
  The stated result is then the special case where $u=\tproj {n+1}x$ and $v=\tproj {n+1}y$.
\end{proof}

\begin{cor}
  Let $n\ge-2$ and $(A,a)$ be a pointed type. Then
  \[\trunc n{\Omega(A,a)}=\Omega(\trunc{n+1}{(A,a)})\]
\end{cor}
\begin{proof}
  This is a special case of the previous lemma where $x=y=a$.
\end{proof}

\begin{cor}
  Let $n\ge -2$ and $k\ge 0$ and $(A,a)$ a pointed type.  Then
  \[\trunc n{\Omega^k(A,a)} = \Omega^k(\trunc{n+k}{(A,a)}). \]
\end{cor}
\begin{proof}
  By induction on $k$, using the recursive definition of $\Omega^k$.
\end{proof}

We also observe that ``truncations are cumulative'': if we truncate to an $n$-type and then to a $k$-type with $k\le n$, then we might as
well have truncated directly to a $k$-type.

\begin{lem}
  Let $k,n\ge-2$ with $k\le{}n$ and $A:\type$. Then
  $\trunc k{\trunc nA}=\trunc kA$.
\end{lem}
\begin{proof}
  We define two maps $f:\trunc k{\trunc nA}\to\trunc kA$ and
  $g:\trunc kA\to\trunc k{\trunc nA}$ in the following way:

  \[f(\tproj k{\tproj na})=\tproj ka\]
  \[g(\tproj ka)=\tproj k{\tproj na}\]

  The map $f$ is well-defined because $\trunc kA$ is $k$-truncated and also
  $n$-truncated (because $k\le{}n$), and the map $g$ is well-defined because
  $\trunc k{\trunc nA}$ is $k$-truncated.

  The composition $f\circ{}g:\trunc kA\to\trunc kA$ satisfy
  $(f\circ{}g)(\tproj ka)=\tproj ka$ hence $f\circ{}g=\idfunc[\trunc kA]$, and
  we also have $g\circ{}f=\idfunc[\trunc k{\trunc nA}]$ in the same way.
\end{proof}

% \begin{lem}
%   We have $\trunc n{\unit}=\unit$.
% \end{lem}
% \begin{proof}
%   Indeed, $\unit$ is $n$-truncated for every $n$ hence $\trunc n{\unit}=\unit$ by
%   \autoref{reflectPequiv}.
% \end{proof}


\section{Colimits of $n$-types}
\label{sec:pushouts}

Recall that in \autoref{sec:colimits}, we used higher inductive types to define pushouts of types, and proved their universal property.
In general, a (homotopy) colimit of $n$-types may no longer be an $n$-type.
However, if we $n$-truncate it, we obtain an $n$-type which satisfies the correct universal property with respect to other $n$-types.

In this section we prove this formally for the case of pushouts, which is the most important and nontrivial one.
Recall the following definitions from \autoref{sec:colimits}.

\begin{defn}
  A \define{span} % in $\P$
  is a 5-tuple $\Ddiag=(A,B,C,f,g)$ with % $A,B,C:\P$ and
  $f:C\to{}A$ and $g:C\to{}B$.
  \[\Ddiag=\quad\vcenter{\xymatrix{C \ar^g[r] \ar_f[d] & B \\ A & }}\]
\end{defn}

\begin{defn}
  Given a span $\Ddiag=(A,B,C,f,g)$ and a type $D$, a %$D:\P$, a
  \define{cocone under $\Ddiag$ with base $D$} is a triple $(i, j, h)$ with
  $i:A\to{}D$, $j:B\to{}D$ and $h : \prd{c:C}i(f(c))=j(g(c))$:
  \[\uppercurveobject{{ }}\lowercurveobject{{ }}\twocellhead{{ }}
  \xymatrix{C \ar^g[r] \ar_f[d] \drtwocell{^h} & B \ar^j[d] \\ A \ar_i[r] & D
  }\]
  We denote by $\cocone{\Ddiag}{D}$ the type of all such cocones.
\end{defn}

The type of cocones is (covariantly) functorial.
For instance, given $D,E$ % $D,E:\P$
and a map $t:D\to{}E$, there is a map
  \[\function{\cocone{\Ddiag}{D}}{\cocone{\Ddiag}{E}}{c}{\composecocone{t}c}\]
  defined by:
  \[\composecocone{t}(i,j,h)=(t\circ{}i,t\circ{}j,\mapfunc{t}\circ{}h)\]
  % \[\uppercurveobject{{ }}\lowercurveobject{{ }}\twocellhead{{ }}
  % \xymatrix{C \ar^g[r] \ar_f[d] \drtwocell{^h} & B \ar_j[d]
  %   \ar@/_/^{t\circ{}j}[rdd] & \\
  %   A \ar^i[r] \ar@/^/_{t\circ{}i}[rrd] & D \ar[rd]|<<<<t & \\
  %   & & E }\]
And given $D,E,F$, %$:\P$,
functions $t:D\to{}E$, $u:E\to{}F$ and $c:\cocone{\Ddiag}{D}$, we have
\begin{align}
  \composecocone{\idfunc[D]}c &= c \label{eq:composeconeid}\\
  \composecocone{(u\circ{}t)}c&=\composecocone{u}(\composecocone{t}c)\label{eq:composeconefunc}
\end{align}

\begin{defn}
  Given a span $\Ddiag$ of $n$-types, an $n$-type $D$, and a cocone
  $c:\cocone{\Ddiag}{D}$, the pair $(D,c)$ is said to be a \define{pushout
  of $\Ddiag$ in $n$-types} if for every $n$-type $E$, the map
  \[\function{(D\to{}E)}{\cocone{\Ddiag}{E}}{t}{\composecocone{t}c}\]
  is an equivalence.
\end{defn}

\begin{comment}
We showed in \autoref{thm:pushout-ump} that pushouts exist when $\P$ is \type itself, by giving a direct construction in terms of higher
inductive types.
For a general \P, pushouts may or may not exist, but if they do, then they are unique.

\begin{lem}
  If $(D,c)$ and $(D',c')$ are two pushouts of $\Ddiag$ in $\P$, then
  $(D,c)=(D',c')$.
\end{lem}
\begin{proof}
  We first prove that the two types $D$ and $D'$ are equivalent.

  Using the universal property of $D$ with $D'$, we see that the following map is an
  equivalence
  \[\function{(D\to{}D')}{\cocone{\Ddiag}{D'}}{t}{\composecocone{t}c}\]

  In particular, there is a function $f:D\to{}D'$ satisfying $\composecocone{f}c=c'$. In the
  same way there is a function $g:D'\to{}D$ such that $\composecocone{g}c'=c$.

  In order to prove that $g\circ{}f=\idfunc[D]$ we use the universal property of
  $D$ for $D$, which says that the following map is an equivalence:
  \[\function{(D\to{}D)}{\cocone{\Ddiag}{D}}{t}{\composecocone{t}c}\]

  Using the functoriality of $t\mapsto{}\composecocone{t}c$ we see that
  \begin{align*}
    \composecocone{(g\circ{}f)}c &= \composecocone{g}(\composecocone{f}c) \\
    &= \composecocone{g}c' \\
    &= c \\
    &= \composecocone{\idfunc[D]}c
  \end{align*}
  hence
  $g\circ{}f=\idfunc[D]$, because equivalences are injective. The same argument
  with $D'$ instead of $D$ shows that $f\circ{}g=\idfunc[D']$.

  Hence $D$ and $D'$ are equal, and the fact that $(D,c)=(D',c')$ follows from
  the fact that the equivalence between $D$ and $D'$ we just defined sends $c$
  to $c'$.
\end{proof}

\begin{cor}
  The type of pushouts of $\Ddiag$ in $\P$ is \anhprop. In particular if
  pushouts merely exist then they actually exist.
\end{cor}

As in the case of pullbacks, if \P is reflective, then pushouts in \P always exist.
However, unlike the case of pullbacks, pushouts in \P are not the same as the pushouts in \type: they are obtained by applying the
reflector.
\end{comment}

In order to construct pushouts of $n$-types, we need to explain how to reflect spans and cocones.

\bgroup
\def\reflect(#1){\trunc n{#1}}

\begin{defn}
  Let
  \[\Ddiag=\quad\vcenter{\xymatrix{C \ar^g[r] \ar_f[d] & B \\ A & }}\]
  be a span. We denote by $\reflect(\Ddiag)$ the following
  span of $n$-types:
  \[\reflect(\Ddiag)\defeq\quad \vcenter{\xymatrix{\reflect(C) \ar^{\reflect(g)}[r]
      \ar_{\reflect(f)}[d] & \reflect(B) \\ \reflect(A) & }}\]
\end{defn}

\begin{defn}
  Let $D:\type$ and $c=(i,j,h):\cocone{\Ddiag}{D}$.
  We define
  \[\reflect(c)=(\reflect(i),\reflect(j),\reflect(h)):
  \cocone{\reflect(\Ddiag)}{\reflect(D)}\]
  where $\reflect(h): \reflect(i) \circ \reflect(f) \htpy \reflect(j) \circ \reflect(g)$ is defined as in \autoref{thm:trunc-htpy}.
  % \[\reflect(h):\prd{c:\reflect(C)}\reflect(i)(\reflect(f)(c))=\reflect(j)(\reflect(g)(c))\]
  % is defined in the following way:
\end{defn}

\egroup

We now observe that the maps from each type to its $n$-truncation assemble into a map of spans, in the following sense.

\begin{defn}
  Let 
  \[\Ddiag=\quad\vcenter{\xymatrix{C \ar^g[r] \ar_f[d] & B \\ A & }}
  \qquad\text{and}\qquad
  \Ddiag'=\quad\vcenter{\xymatrix{C' \ar^{g'}[r] \ar_{f'}[d] & B' \\ A' & }}
  \]
  be spans.
  A \define{map of spans} $\Ddiag \to \Ddiag'$ consists of functions $\alpha:A\to A'$, $\beta:B\to B'$, and $\gamma:C\to C'$ and homotopies $\phi: \alpha\circ f \htpy f'\circ \gamma$ and $\psi:\beta\circ g \htpy g' \circ \gamma$.
\end{defn}

Thus, for any span $\Ddiag$, we have a map of spans $\tproj[\Ddiag] n- : \Ddiag \to \trunc n\Ddiag$ consisting of $\tproj[A]n-$, $\tproj[B]n-$, $\tproj[C]n-$, and the naturality homotopies $\mathsf{nat}^f_n$ and $\mathsf{nat}^g_n$ from~\eqref{eq:trunc-nat}.

We also need to know that maps of spans behave functorially.
Namely, if $(\alpha,\beta,\gamma,\phi,\psi):\Ddiag \to \Ddiag'$ is a map of spans and $D$ any type, then we have
\[ \function{\cocone{\Ddiag'}{D}}{\cocone{\Ddiag}{D}}{(i,j,h)}{(i\circ \alpha,j\circ\beta, k)} \]
where $k: \prd{z:C} i(\alpha(f(z))) = j(\beta(g(z)))$ is the composite
\begin{equation}\label{eq:mapofspans-htpy}
\xymatrix{
  i(\alpha(f(z))) \ar@{=}[r]^{\apfunc{i}(\phi)} &
  i(f'(\gamma(z))) \ar@{=}[r]^{h(\gamma(z))} &
  j(g'(\gamma(z))) \ar@{=}[r]^{\apfunc{j}(\psi)} &
  j(\beta(g(z))). }
\end{equation}
We denote this cocone by $(i,j,h) \circ (\alpha,\beta,\gamma,\phi,\psi)$.
Moreover, this functorial action commutes with the other functoriality of cocones:

\begin{lem}\label{thm:conemap-funct}
  Given $(\alpha,\beta,\gamma,\phi,\psi):\Ddiag \to \Ddiag'$ and $t:D\to E$, the following diagram commutes:
  \begin{equation*}
    \vcenter{\xymatrix{
        \cocone{\Ddiag'}{D}\ar[r]^-{t\circ -}\ar[d] &
        \cocone{\Ddiag'}{E}\ar[d]\\
        \cocone{\Ddiag}{D}\ar[r]_-{t\circ -} &
        \cocone{\Ddiag}{E}
      }}
  \end{equation*}
\end{lem}
\begin{proof}
  Given $(i,j,h):\cocone{\Ddiag'}{D}$, note that both composites yield a cocone whose first two components are $t\circ i\circ \alpha$ and $t\circ j\circ\beta$.
  Thus, it remains to verify that the homotopies agree.
  For the top-right composite, the homotopy is~\eqref{eq:mapofspans-htpy} with $(i,j,h)$ being replaced by $(t\circ i, t\circ j, \apfunc{t}\circ h)$:
  \begin{equation*}
    \xymatrix@C=3pc{
      t(i(\alpha(f(z)))) \ar@{=}[r]^{\apfunc{t\circ i}(\phi)} &
      t(i(f'(\gamma(z)))) \ar@{=}[r]^{\apfunc{t}(h(\gamma(z)))} &
      t(j(g'(\gamma(z)))) \ar@{=}[r]^{\apfunc{t\circ j}(\psi)} &
      t(j(\beta(g(z)))). }
  \end{equation*}
  On the other hand, for the left-bottom composite, the homotopy is $\apfunc{t}$ applied to~\eqref{eq:mapofspans-htpy}.
  Since $\apfunc{}$ respects path-concatenation, this is equal to
  \begin{equation*}
    \xymatrix@C=3pc{
      t(i(\alpha(f(z)))) \ar@{=}[r]^{\apfunc{t}(\apfunc{i}(\phi))} &
      t(i(f'(\gamma(z)))) \ar@{=}[r]^{\apfunc{t}(h(\gamma(z)))} &
      t(j(g'(\gamma(z)))) \ar@{=}[r]^{\apfunc{t}(\apfunc{j}(\psi))} &
      t(j(\beta(g(z)))). }
  \end{equation*}
  But $\apfunc{t}\circ \apfunc{i} = \apfunc{t\circ i}$ and similarly for $j$, so these two homotopies are equal.
\end{proof}

Finally, note that since we defined $\trunc nc : \cocone{\trunc n \Ddiag}{\trunc n D}$ using \autoref{thm:trunc-htpy}, the additional condition~\eqref{eq:trunc-htpy} implies
\begin{equation}
  \tproj[D] n- \circ c = \trunc n c \circ \tproj[\Ddiag]n- .\label{eq:conetrunc}
\end{equation}
for any $c:\cocone{\Ddiag}{D}$.
Now we can prove our desired theorem.

\begin{thm}
  \label{reflectcommutespushout}
  Let $\Ddiag$ be a span and $(D,c)$ its pushout.
  Then $(\trunc nD,\trunc n c)$ is a pushout of $\trunc n\Ddiag$ in $n$-types.
\end{thm}
\begin{proof}
  Let $E$ be an $n$-type, and consider the following diagram:
\bgroup
\def\reflect(#1){\trunc n{#1}}
  \begin{equation*}
  \vcenter{\xymatrix{
      (\trunc nD \to E)\ar[r]^-{-\circ \tproj[D] n-}\ar[d]_{-\circ \trunc nc} &
      (D\to E)\ar[d]^{-\circ c}\\
      \cocone{\trunc n \Ddiag}{E}\ar[r]^-{-\circ \tproj[\Ddiag]n-}\ar@{<-}[d]_{\ell_1} &
      \cocone{\Ddiag}{E}\ar@{<-}[d]^{\ell_2}\\
      (\reflect(A)\to{}E)\times_{(\reflect(C)\to{}E)}(\reflect(B)\to{}E)\ar[r] &
      (A\to{}E)\times_{(C\to{}E)}(B\to{}E)
      }}
  \end{equation*}
\egroup
  The upper horizontal arrow is an equivalence since $E$ is an $n$-type, while $-\circ c$ is an equivalence since $c$ is a pushout cocone.
  Thus, by the 2-out-of-3 property, to show that $-\circ \trunc nc$ is an equivalence, it will suffice to show that the upper square commutes and that the middle horizontal arrow is an equivalence.
  To see that the upper square commutes, let $t:\trunc nD \to E$; then
  \begin{alignat*}{2}
    \big(t \circ \trunc n c\big) \circ \tproj[\Ddiag] n-
    &= t \circ \big(\trunc n c \circ \tproj[\Ddiag] n-\big)
    &&\quad\text{by \autoref{thm:conemap-funct}}\\
    &= t\circ \big(\tproj[D]n- \circ c\big)
    &&\quad\text{by~\eqref{eq:conetrunc}}\\
    &= \big(t\circ \tproj[D]n-\big) \circ c
    &&\quad\text{by~\eqref{eq:composeconefunc}}.
  \end{alignat*}
  To show that the middle horizontal arrow is an equivalence, consider the lower square.
  The two lower vertical arrows are simply applications of $\happly$:
  \begin{align*}
    \ell_1(i,j,p) &\defeq (i,j,\happly(p))\\
    \ell_2(i,j,p) &\defeq (i,j,\happly(p))
  \end{align*}
  and hence are equivalences by function extensionality.
  The lowest horizontal arrow is defined by
  \[ (i,j,p) \mapsto \big( i\circ \tproj[A]n- ,\;\; j \circ \tproj[B] n- ,\;\; q\big) \]
  where $q$ is the composite
  \begin{alignat*}{2}
    i\circ \tproj[A]n- \circ f
    &= i\circ \trunc nf \circ \tproj[C]n-
    &&\quad\text{by }{\funext(\lam{z} \apfunc{i}(\mathsf{nat}^f_n(z)))}\\
    &= j\circ \trunc ng \circ \tproj[C]n-
    &&\quad\text{by }\apfunc{-\circ \tproj[C] n-}(p)\\
    &= j\circ \tproj[B]n- \circ g
    &&\quad\text{by }{\funext(\lam{z} \apfunc{j}(\mathsf{nat}^g_n(z)))}.
  \end{alignat*}
  This is an equivalence, because it is induced by an equivalence of cospans.
  Thus, by 2-out-of-3, it will suffice to show that the lower square commutes.
  But the two composites around the lower square agree definitionally on the first two components, so it suffices to show that for $(i,j,p)$ in the lower-left corner and $z:C$, the path
  \[ \happly(q,z) : i(\tproj n{f(z)}) = j(\tproj n{g(z)}) \]
  (with $q$ as above)
  is equal to the composite
  \begin{alignat*}{2}
    i(\tproj n{f(z)})
    &= i(\trunc nf(\tproj nz))
    &&\quad\text{by }{\apfunc{i}(\mathsf{nat}^f_n(z))}\\
    &= j(\trunc ng(\tproj nz))
    &&\quad\text{by }{\happly(p,\tproj nz)}\\
    &= j(\tproj n{g(z)})
    &&\quad\text{by }{\apfunc{j}(\mathsf{nat}^g_n(z))}.
  \end{alignat*}
  However, since $\happly$ is functorial, it suffices to check equality for the three component paths:
  \begin{align*}
    \happly({\funext(\lam{z} \apfunc{i}(\mathsf{nat}^f_n(z)))},z)
    &= {\apfunc{i}(\mathsf{nat}^f_n(z))}\\
    \happly(\apfunc{-\circ \tproj[C] n-}(p), z)
    &= {\happly(p,\tproj nz)}\\
    \happly({\funext(\lam{z} \apfunc{j}(\mathsf{nat}^g_n(z)))},z)
    &= {\apfunc{j}(\mathsf{nat}^g_n(z))}.
  \end{align*}
  The first and third of these are just the fact that $\happly$ is quasi-inverse to $\funext$, while
  the second is an easy general lemma about $\happly$ and precomposition.
\end{proof}


\section{Connectedness}
\label{sec:connectivity}

An $n$-type is one that has no interesting information above dimension $n$.
By contrast, an \emph{$n$-connected type} is one that has no interesting information \emph{below} dimension $n$.
It turns out to be natural to study a more general notion for functions as well.

\begin{defn}
A function $f:A\to B$ is said to be \define{$n$-connected} if for all $b:B$, the type $\trunc n{\hfiber f b}$ is contractible:
\begin{equation*}
  \mathsf{conn}_n(f)\defeq \prd{b:B}\iscontr(\trunc n{\hfiber{f}b}). 
\end{equation*}
A type $A$ is said to be \define{$n$-connected} if the unique function $A\to\unit$ is $n$-connected, i.e.\ if $\trunc nA$ is contractible.
\end{defn}

Thus, a function $f:A\to B$ is $n$-connected if and only if $\hfib{f}b$ is $n$-connected for every $b:B$.
Of course, every function is $(-2)$-connected.
At the next level, we have:

\begin{lem}
  A function $f$ is $(-1)$-connected if and only if it is surjective in the sense of \autoref{sec:mono-surj}.
\end{lem}
\begin{proof}
  We defined $f$ to be surjective if $\brck{\hfiber f b}$ is inhabited for all $b$.
  But since it is a mere proposition, inhabitation is equivalent to contractibility.
\end{proof}

Thus, $n$-connectedness of a function for $n\ge 0$ can be thought of as a strong form of surjectivity.
Category-theoretically, $(-1)$-connectedness corresponds to essential surjectivity on objects, while $n$-connectedness corresponds to essential surjectivity on $k$-morphisms for $k\le n+1$.

\begin{rmk}
  While our notion of $n$-connectedness for types agrees with the standard notion in homotopy theory, our notion of $n$-connectedness for \emph{functions} is off by one from a common indexing in classical homotopy theory.
  Whereas we say a function $f$ is $n$-connected if all its fibers are $n$-connected, some classical homotopy theorists would call such a function $(n+1)$-connected.
  (This is due to a historical focus on \emph{cofibers} rather than fibers.)
\end{rmk}

We now observe a few closure properties of connected maps.

\begin{lem}
Suppose that $g$ is a retract of a $n$-connected function $f$.  Then $g$ is
$n$-connected.
\end{lem}
\begin{proof}
This is a direct consequence of \autoref{lem:func_retract_to_fiber_retract}.
\end{proof}

\begin{cor}
If $g$ is homotopic to a $n$-connected function $f$, then $g$ is $n$-connected.
\end{cor}

\begin{lem}\label{lem:nconnected_postcomp}
Suppose that $f:A\to B$ is $n$-connected. Then $g:B\to C$ is $n$-connected if and only if $g\circ f$ is
$n$-connected.
\end{lem}

\begin{proof}
Let $c:C$. We have $\hfib{g\circ f}c \simeq \sm{w:\hfib{g}c}\hfib{f}{\proj1 w}$ and, so 
\begin{align*}
\trunc n{\hfib{g\circ f}c}& \simeq \Trunc n{ \sm{w:\hfib{g}c}\hfib{f}{\proj1 w}}\\
& \simeq \Trunc n{\sm{w:\hfib{g}c} \trunc n{\hfib{f}{\proj1 w}}}\\
& \simeq \trunc n{\hfib{g}c}.
\end{align*}
It follows that $\trunc n{\hfib{g}c}$ is contractible if and only if $\trunc n{\hfib{g\circ f}c}$ is
contractible.
\end{proof}

Importantly, $n$-connected functions can be equivalently characterized as those which satisfy an ``induction principle'' with respect to $n$-types.
This idea will lead directly into our proof of the Freudenthal suspension theorem in \autoref{sec:freudenthal}.

\begin{lem}\label{prop:nconnected_tested_by_lv_n_dependent types}
For $f:A\to B$ and $P:B\to\type$, consider the following function:
\begin{equation*}
\lam{s} s\circ f :\left(\prd{b:B} P(b)\right)\to\left(\prd{a:A}P(f(a))\right).
\end{equation*}
For a fixed $f$ and $n\ge -2$, the following are equivalent.
\begin{enumerate}
\item $f$ is $n$-connected.\label{item:conntest1}
\item For every $P:B\to\ntype{n}$, the map $\lam{s} s\circ f$ is an equivalence.\label{item:conntest2}
\item For every $P:B\to\ntype{n}$, the map $\lam{s} s\circ f$ has a section.\label{item:conntest3}
\end{enumerate}
\end{lem}

\begin{proof}
Suppose that $f$ is $n$-connected and let $P:B\to\ntype{n}$. Then we have the equivalences
\begin{align*}
\prd{b:B} P(b) & \simeq \prd{b:B} \trunc n{\hfib{f}b} \to P(b)\\
& \simeq \prd{b:B} \hfib{f}b\to P(b)\\
& \simeq \prd{b:B}{a:A}{p:f(a)= b} P(b)\\
& \simeq \prd{a:A} P(f(a)).
\end{align*}
We omit the proof that this equivalence is indeed given by $\lam{s} s\circ f$.
Thus,~\ref{item:conntest1}$\Rightarrow$\ref{item:conntest2}, and clearly~\ref{item:conntest2}$\Rightarrow$\ref{item:conntest3}.
To show~\ref{item:conntest3}$\Rightarrow$\ref{item:conntest1}, consider the type family
\begin{equation*}
P(b)\defeq \trunc n{\hfib{f}b}.
\end{equation*}
Then~\ref{item:conntest3} yields a function $c:\prd{b:B} \trunc n{\hfib{f}b}$ with
$c(f(a))=\tproj n{\pairr{a,\refl{f(a)}}}$. To show that each $\trunc n{\hfib{f}b}$ is contractible,
we will find a function of type
\begin{equation*}
\prd{b:B}{w:\trunc n{\hfib{f}b}} w= c(b).
\end{equation*}
By \autoref{thm:truncn-ind}, for this it suffices to find a function of type
\begin{equation*}
\prd{b:B}{a:A}{p:f(a)= b} \tproj n{\pairr{a,p}}= c(b).
\end{equation*}
But by rearranging variables and path induction, this is equivalent to the type
\begin{equation*}
\prd{a:A} \tproj n{\pairr{a,\refl{f(a)}}}= c(f(a)).
\end{equation*}
This property holds by our choice of $c(f(a))$. 
\end{proof}

\begin{cor}\label{cor:totrunc-is-connected}
For any $A$, the canonical function $\tproj n-:A\to\trunc n A$ is $n$-connected.
\end{cor}
\begin{proof}
By \autoref{thm:truncn-ind} and the associated uniqueness principle, the condition of \autoref{prop:nconnected_tested_by_lv_n_dependent types} holds.
\end{proof}

For instance, when $n=-1$, \autoref{cor:totrunc-is-connected} says that the map $A\to \brck A$ from a type to its propositional truncation is surjective.

\begin{cor}
A type $A$ is $n$-connected if and only if the map
\begin{equation*}
  \lam{b}{a} b: B \to (A\to B)
\end{equation*}
is an equivalence for every $n$-type $B$.
In other words, ``every map from $A$ to an $n$-type is constant''.
\end{cor}
\begin{proof}
  By \autoref{prop:nconnected_tested_by_lv_n_dependent types} applied to a function with codomain $\unit$.
\end{proof}

\begin{lem}\label{lem:nconnected_to_leveln_to_equiv}
Let $B$ be an $n$-type and let $f:A\to B$ be a function. Then the induced function $g:\trunc n A\to B$ is an
equivalence if and only if $f$ is $n$-connected.
\end{lem}

\begin{proof}
Note that $f$ is homotopic to $g\circ \tproj n-$. By \autoref{cor:totrunc-is-connected}, $\tproj n-$ is $n$-connected, so
by
\autoref{lem:nconnected_postcomp} $f$ is $n$-connected if and only if $g$ is $n$-connected.
But since $g$ is a function between $n$-types, its fibers are also $n$-types.
Thus, $g$ is $n$-connected if and only if it is an equivalence.
\end{proof}

We can also characterize connected pointed types in terms of connectivity of the inclusion of their basepoint.

\begin{lem}\label{thm:connected-pointed}
  Let $A$ be a type and $a_0:\unit\to A$ a basepoint, with $n\ge -1$.
  Then $A$ is $n$-connected if and only if the map $a_0$ is $(n-1)$-connected.
\end{lem}
\begin{proof}
  TODO.
\end{proof}


A useful variation on \autoref{lem:nconnected_postcomp} is:

\begin{lem}\label{lem:nconnected_postcomp_variation}
Let $f:A\to B$ be a function and $P:A\to\type$ and $Q:B\to\type$ be type families. Suppose that $g:\prd{a:A} P(a)\to Q(f(a))$
is a fiberwise $n$-connected family of functions, i.e.\ each $g(a)$ is $n$-connected. Then the function
\begin{align*}
\varphi &:\left(\sm{a:A} P(a)\right)\to\left(\sm{b:B} Q(b)\right)\\
\varphi(a,u) &\defeq \pairr{f(a),g(u)}
\end{align*}
is $n$-connected if and only if $f$ is $n$-connected.
\end{lem}

\begin{proof}
For $b:B$ and $v:Q(b)$ we have
\begin{align*}
\trunc n{\hfib{\varphi}{\pairr{b,v}}} & \simeq \Trunc n{\sm{a:A}{u:P(a)}{p:f(a)= b} \trans{f(p)}{g(u)}= v}\\
& \simeq \Trunc n{\sm{w:\hfib{f}b}{u:P(\proj1(w))} g(u)= \trans{\opp{f(p)}}{v}}\\
& \simeq \Trunc n{\sm{w:\hfib{f}b} \trunc n{\hfib{g(\proj1 w)}{\trans{\opp{f(p)}}{v}}}}\\
& \simeq \trunc n{\hfib{f}b}
\end{align*}
where the transportations along $f(p)$ and $f(p)^{-1}$ are with respect to $Q$.
Therefore, if either is contractible, so is the other.
\end{proof}

In the other direction, we have

\begin{lem}\label{prop:nconn_fiber_to_total}
Let $P,Q:A\to\type$ be dependent types and consider a fiberwise transformation
\begin{equation*}
f:\prd{a:A} P(a)\to Q(a)
\end{equation*}
from $P$ to $Q$. Then the induced map $\total f: \sm{a:A}P(a) \to \sm{a:A} Q(a)$ is $n$-connected if and only if each $f(a)$ is $n$-connected. 
\end{lem}

\begin{proof}
By \autoref{fibwise-fiber-total-fiber-equiv}, we have
$\hfib{\total f}{\pairr{x,v}}\simeq\hfib{f(x)}v$
for each $x:A$ and $v:Q(x)$. Hence $\trunc n{\hfib{\total f}{\pairr{x,v}}}$ is contractible if and only if
$\trunc n{\hfib{f(x)}v}$ is contractible.
\end{proof}


\section{Orthogonal factorization}
\label{sec:image-factorization}

In set theory, the surjections and the injections form a unique factorization system: every function factors essentially uniquely as a surjection followed by an injection.
We have seen that surjections generalize naturally to $n$-connected maps, so it is natural to inquire whether these also participate in a factorization system.
Here is the corresponding generalization of injections.

\begin{defn}
  A function $f:A\to B$ is \define{$n$-truncated} if the fiber $\hfib f b$ is an $n$-type for all $b:B$.
\end{defn}

In particular, $f$ is $(-2)$-truncated if and only if it is an equivalence.
And of course, $A$ is an $n$-type if and only if $A\to\unit$ is $n$-truncated.
Moreover, $n$-truncated maps could equivalently be defined recursively, like $n$-types.

\begin{lem}\label{thm:modal-mono}
  For any $n\ge -2$, a function $f:A\to B$ is $(n+1)$-truncated if and only if for all $x,y:A$, the map $\apfunc{f}:(x=y) \to (f(x)=f(y))$ is $n$-truncated.
  In particular, $f$ is $(-1)$-truncated if and only if it is a monomorphism in the sense of \autoref{sec:mono-surj}.
\end{lem}
\begin{proof}
  Note that for any $(x,p),(y,q):\hfib f b$, we have
  \begin{align*}
    \big((x,p) = (y,q)\big)
    &= \sm{r:x=y} (p = f(r)\ct q)\\
    &= \sm{r:x=y} (\apfunc f (r) = p\ct \opp q)\\
    &= \hfib{\apfunc{f}}{p\ct \opp q}.
  \end{align*}
  Thus, any path space in any fiber of $f$ is a fiber of $\apfunc{f}$.
  On the other hand, choosing $b\defeq f(y)$ and $q\defeq \refl{f(y)}$ we see that any fiber of $\apfunc f$ is a path space in a fiber of $f$.
  The result follows, since $f$ is $(n+1)$-connected if all path spaces of its fibers are $n$-types.
\end{proof}

% The following two corollaries are reformulations of \autoref{prop:nconnected_tested_by_lv_n_dependent types}:

% \begin{cor}
% A function $f:A\to B$ is $n$-connected if and only if for every modal function $g:X\to B$, the function
% \begin{equation*}
% \varphi:\big(\prd{b:B}\hfib{g}b\big)\to\big(\prd{a:A} \hfib{g}{f(a)}\big)
% \end{equation*}
% defined by $\varphi(s):=s\circ f$ is an equivalence.
% \end{cor}

% \begin{cor}
% A function $f:A\to B$ is $n$-connected if and only if for every modal function $g:X\to B$, the function
% \begin{equation*}
% \varphi :\big(\sm{h:B\to X} g\circ h\sim \refl{B}\big)\to\big(\sm{k:A\to X} g\circ k\sim f\big)
% \end{equation*}
% defined by $\varphi(h,H):=\pairr{h\circ f,H\circ f}$ is an equivalence.
% \end{cor}

We can now construct the factorization, in a fairly obvious way.

\begin{defn}\label{def:modal-image}
Let $f:A\to B$ be a function. The \define{$n$-image} of $f$ is defined as
\begin{equation*}
\im_n(f)\defeq \sm{b:B} \trunc n{\hfib{f}b}
\end{equation*}
When $n=-1$, we write simply $\im(f)$.
\end{defn}

\begin{lem}\label{prop:to_image_is_connected}
For any function $f:A\to B$, the canonical function $\tilde{f}:A\to\im_n(f)$ is $n$-connected. 
Consequently, any function factors as an $n$-connected function followed by a modal function.
\end{lem}

\begin{proof}
Note that $A\simeq\sm{b:B}\hfib{f}b$. The function $\tilde{f}$ is the function on total spaces induced by the fiberwise
transformation
\begin{equation*}
\prd{b:B} \hfib{f}b\to\trunc n{\hfib{f}b}.
\end{equation*}
Since this is fiberwise $n$-connected by \autoref{cor:totrunc-is-connected}, the statement follows from
\autoref{prop:nconn_fiber_to_total}.
\end{proof}

In the following lemma we set up some machinery to prove the unique factorization theorem.

\begin{lem}\label{prop:factor_equiv_fiber}
Suppose we have a commutative diagram of functions
\begin{equation*}
\begin{tikzpicture}
\node (A) at (-3em,0) {$A$};
\node (B) at (3em,0) {$B$};
\node (X1) at (0,1.5em) {$X_1$};
\node (X2) at (0,-1.5em) {$X_2$};
\draw[ar] (A) -- node[auto] {$g_1$} (X1);
\draw[ar] (X1) -- node[auto] {$h_1$} (B);
\draw[ar] (A) -- node[auto,swap] {$g_2$} (X2);
\draw[ar] (X2) -- node[auto,swap] {$h_2$} (B);
\end{tikzpicture}
\end{equation*}
with $H:h_1\circ g_1\htpy h_2\circ g_2$, where $g_1$ and $g_2$ are $n$-connected and where $h_1$ and $h_2$ are $n$-truncated.
Then there is an equivalence
\begin{equation*}
E(H,b):\hfib{h_1}b\simeq\hfib{h_2}b
\end{equation*}
for any $b:B$, such that for any $a:A$ we have
\[E(H,h_1(g_1(a)))(\pairr{g_1(a),\refl{h_1(g_1(a))}}) = \pairr{g_2(a),H(a)^{-1}}.\]
\end{lem}

\begin{proof}
Let $b:B$. Then we have the following equivalences:
\begin{align*}
\hfib{h_1}b & \simeq \sm{w:\hfib{h_1}b} \trunc n{ \hfib{g_1}{\proj1 w}}\\
& \simeq \Trunc n{\sm{w:\hfib{h_1}b}\hfib{g_1}{\proj1 w}}\\
& \simeq \trunc n{\hfib{h_1\circ g_1}b}
\end{align*}
and likewise for $h_2$ and $g_2$. Here, the first equivalence holds because $g_1$ is $n$-connected; the second equivalence
holds because $h_1$ is $n$-truncated; and the third equivalence holds by \autoref{lem:hfiber_basics}. Also, since we have a
homotopy $H:h_1\circ g_1\sim h_2\circ g_2$, there is an obvious equivalence $\hfib{h_1\circ g_1}b\simeq\hfib{h_2\circ g_2}b$. Hence we
obtain
\begin{equation*}
\hfib{h_1}b\simeq\hfib{h_2}b
\end{equation*}
for any $b:B$. By analyzing the underlying functions, we get the following representation of what happens to the term
$\pairr{g_1(a),\refl{h_1(g_1(a))}}$ after applying each of the equivalences of which $E$ is composed:
\begin{align*}
\pairr{g_1(a),\refl{h_1(g_1(a))}} & 
    \mapsto \pairr{\pairr{g_1(a),\refl{h_1(g_1(a))}}, \eta( \pairr{a,\refl{g_1(a)}} )}\\
  & \mapsto \eta( \pairr{\pairr{g_1(a),\refl{h_1(g_1(a))}}, \pairr{a,\refl{g_1(a)}} }\\
  & \mapsto \eta( \pairr{a,\refl{h_1(g_1(a))}})\\
  & \mapsto \eta( \pairr{a,H(a)^{-1}})\\
  & \mapsto \eta( \pairr{\pairr{g_2(a),H(a)^{-1}},\pairr{a,\refl{g_2(a)}}}\\
  & \mapsto \pairr{\pairr{g_2(a),H(a)^{-1}}, \eta(\pairr{a,\refl{g_2(a)}}) }\\
  & \mapsto \pairr{g_2(a),H(a)^{-1}}\qedhere
\end{align*}
\end{proof}

The equivalences $E(H,b)$ are such that $E(H^{-1},b)= E(H,b)^{-1}$.

Combining \autoref{prop:to_image_is_connected,prop:factor_equiv_fiber}, we have the following unique factorization result:

\begin{thm}
For each $f:A\to B$, the space $\fact_n(f)$ defined by
\begin{equation*}
\sm{X:\type}(g:A\to X)(h:X\to B),\ (h\circ g\sim f)\times\mathsf{conn}_n(g)\times\mathsf{trunc}_n(h).
\end{equation*}
is contractible. By \autoref{prop:to_image_is_connected} we know that there is the term
\begin{equation*}
\pairr{\im(f),\tilde{f},\proj1,\theta,\varphi,\psi}:\fact_n(f)
\end{equation*}
where $\theta:\proj1\circ\tilde{f}\sim f$ is the canonical homotopy, where $\varphi$ is the proof of
\autoref{prop:to_image_is_connected}, and where $\psi$ is the obvious proof that $\proj1:\im(f)\to B$ has $n$-truncated fibers.
\end{thm}

In the following proof we use the symbols $\leftwhisker$ and $\rightwhisker$ to denote the whisker operations. Recall that if we have paths
$p,p^\prime:x= y$, $s:p= p^\prime$ and $q:y= z$, then left whisker operation provides a path $q\ct p=
q\ct p^\prime$, wich is denoted by $q\leftwhisker s$. Likewise, if $f,f^\prime:X\to Y$ and $g:Y\to X$ are functions and if $H:f\sim
f^\prime$ is a homotopy, then there is a homotopy $g\leftwhisker H:g\circ f\sim g\circ f^\prime$.

\begin{proof}
By \autoref{prop:to_image_is_connected} we know that there is a term of $\fact_n(f)$, hence it is enough to
show that $\fact_n(f)$ is a proposition. Suppose we have two $n$-factorizations
\begin{equation*}
\pairr{X_1,g_1,h_1,H_1,\varphi_1,\psi_1}\qquad\text{and}\qquad\pairr{X_2,g_2,h_2,H_2,\varphi_2,\psi_2}
\end{equation*}
of $f$. Then we have the homotopy $H\defeq H_2^{-1}\circ H_1:h_1\circ g_1\sim h_2\circ g_2$. By the univalence axiom, it suffices to show that
\begin{enumerate}
\item there is an equivalence $e:X_1\simeq X_2$,
\item there is a homotopy $\zeta:\underline{e}\circ g_1\sim g_2$,
% \note{Is it easy enough to see that these terms are the various transports?}
\item there is a homotopy $\eta:h_2\circ\underline{e}\sim h_1$,
\item there is a homotopy $H_1 \circ(h_1\leftwhisker\zeta)^{-1}\circ(\eta\rightwhisker g_2)\sim H_2$.
\end{enumerate}
where $\underline{e}$ is the function underlying the equivalence. We prove these four assertions in that order.
\begin{enumerate}
\item By \autoref{prop:factor_equiv_fiber}, we have a fiberwise equivalence
% \note{It could be a nice exercise for the book to show
% that if $f_1:A_1\to B$ and $f_2:A_2\to B$ have equivalent fibers, then $A_1\simeq A_2$}.
\begin{equation*}
\prd{b:B} \hfib{h_1}b\to\hfib{h_2}b.
\end{equation*}
This induces an equivalence of total spaces, i.e.\ we have
\begin{equation*}
\sm{b:B} \hfib{h_1}b\simeq\sm{b:B}\hfib{h_2}b
\end{equation*}
Of course, we also have the familiar equivalences $X_1\simeq\sm{b:B}\hfib{h_1}b$ and $X_2\simeq\sm{b:B}
\hfib{h_2}b$. This gives us our equivalence $e(H):X_1\simeq X_2$. The reader may verify that the underlying function
$\underline{e}(H)$ of $e(H)$ is defined by
\begin{equation*}
\underline{e}(H,x)\defeq \proj1\underline{E}(H^{-1},h_1(x))(\pairr{x,\idfunc{h_1(x)}})
\end{equation*}
\item By \autoref{prop:factor_equiv_fiber} we have $\underline{e}(H,g_1(a))=g_2(a)$. Thus we get $\zeta(a)\defeq \idfunc{g_2(a)}$. 
\item For every $x:X_1$, we have
\begin{equation*}
\proj2\underline{E}(H^{-1},h_1(x))(\pairr{x,\idfunc{h_1(x)}}):h_2(\underline{e}(H,x))= h_1(x),
\end{equation*}
giving us a homotopy $\eta:h_2\circ \underline{e}\sim h_1$.
\item By \autoref{prop:factor_equiv_fiber} we have $\eta(g_1(a))=H(a)^{-1}$ and by {\it ii.} we have
$h_2(\zeta(a))=\idfunc{h_2(g_2(a))}$. Thus we have
\begin{align*}
(H_1 \circ(h_2\leftwhisker\zeta)^{-1}\circ(\eta\rightwhisker g_1))(a) & = H_1(a)\ct h_2(\zeta(a))^{-1}\ct \eta(g_1(a))\\
& = H_1(a)\ct H(a)^{-1}\\
& = H_2(a).\qedhere
\end{align*}
\end{enumerate}
\end{proof}

% I can't make sense of this, and it doesn't seem necessary
%
% \begin{cor}
% A function $f:A\to B$ is $n$-connected if and only if
% \begin{equation*}
% \prd C\prd{g:\modalfunc(B\to C)} \iscontr\big(\sm{h:\modalfunc(B\to C)}\underline{h}\circ
% f\sim\underline{g}\circ f\big).
% \end{equation*}
% \end{cor}

By standard arguments, this yields the following orthogonality principle.

\begin{thm}
  Let $e:A\to B$ be $n$-connected and $m:C\to D$ be $n$-truncated.
  Then the map
  \[ \varphi: (B\to C) \;\to\; \sm{h:A\to C}{k:B\to D} (m\circ h \htpy k \circ e) \]
  is an equivalence.
\end{thm}
\begin{proof}[Sketch of proof]
  For any $(h,k,H)$ in the codomain, let $h = h_2 \circ h_1$ and $k = k_2 \circ k_1$, where $h_1$ and $k_1$ are $n$-connected and $h_2$ and $k_2$ are $n$-truncated.
  Then $f = (m\circ h_2) \circ h_1$ and $f = k_2 \circ (k_1\circ e)$ are both $n$-factorizations of $m \circ h = k\circ e$.
  Thus, there is a unique equivalence between them.
  It is straightforward (if a bit tedious) to extract from this that $\hfib\varphi{(h,k,H)}$ is contractible.
\end{proof}

We end by showing that images are stable under pullback.

\begin{lem}\label{lem:hfiber_wrt_pullback}
Suppose that the square
\begin{equation*}
\begin{tikzpicture}
\matrix (m) [std] {A & C \\ B & D \\};
\draw[ar] (m-1-1) -- (m-1-2);
\draw[ar] (m-1-2) -- node[right] {$g$} (m-2-2);
\draw[ar] (m-1-1) -- node[left] {$f$} (m-2-1);
\draw[ar] (m-2-1) -- node[below] {$h$} (m-2-2);
\end{tikzpicture}
\end{equation*}
is a pullback square and let $b:B$. Then $\hfib{f}b\simeq\hfib{g}{h(b)}$.
\end{lem}

\begin{proof}
\note{Do we have pasting of pullbacks anywhere?}
This follows from pasting of pullbacks, since the type $X$ in the diagram
\begin{equation*}
\begin{tikzpicture}
\matrix (m) [std] {X & A & C \\ \unit & B & D \\};
\draw[ar] (m-1-1) -- (m-1-2);
\draw[ar] (m-1-2) -- (m-1-3);
\draw[ar] (m-1-3) -- node[right] {$g$} (m-2-3);
\draw[ar] (m-1-2) -- node[left] {$f$} (m-2-2);
\draw[ar] (m-1-1) -- (m-2-1);
\draw[ar] (m-2-1) -- node[below] {$b$} (m-2-2);
\draw[ar] (m-2-2) -- node[below] {$h$} (m-2-3);
\end{tikzpicture}
\end{equation*}
is the pullback of the left square if and only if it is the pullback of the outer rectangle: $\hfib{f}b$ is the pullback of the
square on the left and $\hfib{g}{h(b)}$ is the pullback of the outer rectangle.
\end{proof}

\begin{thm}\label{thm:stable-images}
Consider functions $f:A\to B$, $g:C\to D$ and the diagram
\begin{equation*}
\begin{tikzpicture}
\matrix (m) [std] {A & C \\ \im_n(f) & \im_n(g) \\ B & D \\};
\draw[ar] (m-1-1) -- (m-1-2);
\draw[ar] (m-1-2) -- node[right] {$\tilde{g}_n$} (m-2-2);
\draw[ar] (m-2-2) -- node[right] {$\proj1$} (m-3-2);
\draw[ar] (m-1-1) -- node[left] {$\tilde{f}_n$} (m-2-1);
\draw[ar] (m-2-1) -- (m-2-2);
\draw[ar] (m-2-1) -- node[left] {$\proj1$} (m-3-1);
\draw[ar] (m-3-1) -- node[below] {$h$} (m-3-2);
\end{tikzpicture}
\end{equation*}
Then the outer rectangle is a pullback if and only if the bottom square is a pullback. In either of these equivalent cases, the top square
is also a pullback. Consequently, images are stable under pullbacks.
\end{thm}

\begin{proof}
Suppose first that the outer square is a pullback. Note that we have the equivalences
\begin{align*}
B\times_D\im_n(g) & \jdeq \sm{b:B}{w:\im_n(g)} h(b)=\proj1 w\\
& \simeq \sm{b:B}{d:D}{w:\trunc n{\hfib{g}d}} h(b)= d\\
& \simeq \sm{b:B} \trunc n{\hfib{g}{h(b)}}.\\
& \simeq \sm{b:B} \trunc n{\hfib{f}b} \\
& \equiv \im_n(f).
\end{align*}
In the last equivalence we have used \autoref{lem:hfiber_wrt_pullback}.

Now suppose that the bottom square is a pullback, of which we denote the top arrow by $\psi$. By the pasting lemma for pullbacks, it
suffices to show that the top square is a pullback. We have the equivalences
\begin{align*}
\im(f)\times_{\im(g)} C & \jdeq \sm{w:\im(f)}{c:C} \psi(w)=\tilde{g}_n(c)\\
& \simeq \sm{b:B}{w:\im(g)}{p:h(b)=\proj1 w}{c:C} w=\tilde{g}_n(c)\\
& \simeq \sm{b:B}{c:C} h(b)= g(c)\\
& \simeq \sm{b:B}\hfib{f}b\\
& \simeq A.\qedhere
\end{align*}
\end{proof}


\begin{comment}

\section{Modalities}
\label{sec:modalities}

Since $n$-types form a reflective subuniverse for any $n\ge -2$, the theory of \autoref{sec:pullbacks,sec:pushouts} applies to them.
However, $n$-types have additional properties not shared by other reflective subuniverses, particularly \autoref{thm:ntypes-sigma,thm:truncn-ind}.
In fact, these two properties are equivalent to each other.

\begin{thm}\label{thm:modal-char}
  For a reflective subuniverse \P, the following are logically equivalent.
  \begin{enumerate}
  \item If $A:\P$ and $B:A\to \P$, then $\sm{x:A} B(x)$ is in \P.\label{item:mchr1}
  \item for every $A:\type$, type family $B:\reflect A\to\P$, and function $g:\prd{a:A} B(\project(a))$, there exists $f:\prd{z:\reflect A} B(z)$ such that $f(\project(a)) = g(a)$ for all $a:A$.\label{item:mchr2}
  \end{enumerate}
\end{thm}
\begin{proof}
  Suppose~\ref{item:mchr1}.
  Then in the situation of~\ref{item:mchr2}, the type $\sm{z:\reflect A} B(z)$ lies in $\P$, and we have $g':A\to \sm{z:\reflect A} B(z)$ defined by $g'(a)\defeq (\project(a),g(a))$.
  Thus, we have $\ext(g'):\reflect A \to \sm{z:\reflect A} B(z)$ such that $\ext(g')(\project(a)) = (\project(a),g(a))$.

  Now consider the functions $\proj2 \circ \ext(g') : \reflect A \to \reflect A$ and $\idfunc[\reflect A]$.
  By assumption, these become equal when precomposed with $\project$.
  Thus, by the universal property of $\reflect$, they are equal already, i.e.\ we have $p_z:\proj2(\ext(g')(z)) = z$ for all $z$.
  Now we can define $f(z) \defeq \trans{p_z}{\proj2(\ext(g')(z))}$, and the second component of $\ext(g')(\project(a)) = (\project(a),g(a))$ yields $f(\project(a)) = g(a)$.

  Conversely, suppose~\ref{item:mchr2}, and that $A:\P$ and $B:A\to\P$.
  Let $h$ be the composite
  \[ \reflect(\sm{x:A} B(x)) \xrightarrow{\reflect(\proj1)} \reflect A \xrightarrow{\opp{(\project_A)}} A. \]
  Then for $z:\sm{x:A} B(x)$ we have
  \begin{align*}
    h(\project(z)) &= \opp\project(\reflect(\proj1)(\project(z)))\\
    &= \opp\project(\project(\proj1(z)))\\
    &= \proj1(z).
  \end{align*}
  Denote this path by $p_z$.
  Now if we define $C:\reflect(\sm{x:A} B(x)) \to \type$ by $C(w) \defeq B(h(w))$, we have
  \[ g \defeq \lam{z} \trans{p_z}{\proj2(z)} \;:\; \prd{z:\sm{x:A} B(x)} C(\project(z)). \]
  Thus, the assumption yields $f:\prd{w:\reflect(\sm{x:A}B(x))} C(w)$ such that $f(\project(z)) = g(z)$.
  Together, $h$ and $f$ give a function $k:\reflect(\sm{x:A}B(x)) \to \sm{x:A}B(x)$ defined by $k(w) \defeq (h(w),f(w))$, while $p_z$ and the equality $f(\project(z)) = g(z)$ show that $k$ is a retraction of $\project_{\sm{x:A}B(x)}$.
  Therefore, $\sm{x:A}B(x)$ is in \P.
\end{proof}

Condition~\ref{item:mchr2} is a more familiar type-theoretic condition, which has the flavor of an induction principle.
This flavor is strengthened if we rephrase it in the following way.
Note that any reflective subuniverse can be characterized by the operation $\reflect:\type\to\type$ and the functions $\project_A:A\to \reflect A$, since we have $P(A) = \isequiv(\project_A)$.

\begin{defn}\label{defn:modality}
A \define{modality} is an operation $\modal:\type\to\type$ for which there are
\begin{enumerate}
\item functions $\mreturn^\modal_A:A\to\modal(A)$ for every type $A$.\label{item:modal1}
\item for every $A:\type$ and every type family $B:\modal(A)\to\type$, a function\label{item:modal2}
\begin{equation*}
\ext:\big(\prd{a:A}\modal(B(\mreturn^\modal_A(a)))\big)\to\prd{z:\modal(A)}\modal(B(z)).
\end{equation*}
\item A path $\ext(f)(\mreturn^\modal_A(a)) = f(a)$ for each $f:\prd{a:A}\modal(B(\mreturn^\modal_A(a)))$.\label{item:modal3}
\item For any $z,z':\modal(A)$, the function $\mreturn^\modal_{z=z'} : (z=z') \to \modal(z=z')$ is an equivalence.\label{item:modal4}
\end{enumerate}
We say that $A$ is \define{modal} for $\modal$ if $\mreturn^\modal_A:A\to\modal(A)$ is an equivalence, and we write
\[\modaltype\defeq\sm{X:\type}\ismodal X\]
for the type of modal types.
\end{defn}

Conditions~\ref{item:modal2} and~\ref{item:modal3} are very similar to \autoref{thm:modal-char}\ref{item:mchr2}, but phrased using $\modal B(z)$ rather than assuming $B$ to be valued in $\P$.
This allows us to state the condition purely in terms of the operation $\modal$, rather than requiring the predicate $P:\type\to\prop$ to be given in advance.
(It is not entirely satisfactor, since we still have to refer to $P$ not-so-subtly in clause~\ref{item:modal4}.
We do not know whether~\ref{item:modal4} follows from~\ref{item:modal1}--\ref{item:modal3}.)
However, the stronger-looking property of \autoref{thm:modal-char}\ref{item:mchr2} follows from \autoref{defn:modality}\ref{item:modal2} and~\ref{item:modal3}, since for any $C:\modal A \to \modaltype$ we have $C(z) \simeq \modal C(z)$, and we can pass back across this equivalence.

As with other induction principles, this implies a universal property.

\begin{thm}\label{prop:lv_n_deptype_sec_equiv_by_precomp}
Let $A$ be a type and let $B:\modal(A)\to\modaltype$. Then the function
\begin{equation*}
(-\circ \mreturn^\modal_A) : \Big(\prd{z:\modal(A)}B(z)\Big) \to \Big(\prd{a:A}B(\mreturn^\modal_A(a))\Big)
\end{equation*}
is an equivalence.
\end{thm}
\begin{proof}
By definition, the operation $\ext$ is a right inverse to $(-\circ \mreturn^\modal_A)$.
Thus, we only need to find a homotopy
\begin{equation*}
\prd{z:\modal(A)}s(z)= \ext(s\circ \mreturn^\modal_A)(z)
\end{equation*}
for each $s:\prd{z:\modal(A)}B(z)$, exhibiting it as a left inverse as well.
By assumption, each $B(z)$ is modal, and hence each type $s(z)= R^\modal_X(s\circ \mreturn^\modal_A)(z)$
is also modal.
Thus, it suffices to find a function of type
\begin{equation*}
\prd{a:A}s(\mreturn^\modal_A(a))= \ext(s\circ \mreturn^\modal_A)(\mreturn^\modal_A(a)).
\end{equation*}
which follows straight from the definition of modalities.
\end{proof}

In particular, for every type $A$ and every modal type $B$, we have an equivalence $(\modal A\to B)\simeq (A\to B)$.

\begin{cor}
  For any modality $\modal$, the $\modal$-modal types form a reflective subuniverse satisfying the equivalent conditions of \autoref{thm:modal-char}.
\end{cor}

Thus, modalities can be identified with reflective subuniverses closed under $\Sigma$-types.
The name \emph{modality} comes, of course, from \emph{modal logic}, which studies logic where we can form statements such as ``possibly $A$'' (usually written $\diamond A$) or ``necessarily $A$'' (usually written $\Box A$).
The symbol $\modal$ is somewhat common for a monadic modal operator. % (rather than a specific one such as $\diamond$ or $\Box$).
Under the propositions-as-types principle, of course, a modality corresponds to an operation on types, and \autoref{defn:modality} seems a reasonable candidate for how such an operation should be defined.
More precisely, we should perhaps call these \emph{idempotent, monadic} modalities, but they will be the only kind we consider.



\subsection{The principle of modal choice}\label{sec:ac_truncated}

In this section generalize the principle of unique choice to a lemma we call the principle of modal
choice, or simply $\mathsf{AC}_{\modal}$. We also introduce various degrees of surjectivity and see how they interact with modal choice.
We will derive
the principle of unique choice from the following more general discussion in \autoref{cor:auc}.

\begin{defn}
Suppose $P:A\to\type$ is a dependent type over a type $A$. We write $\{A\mid P\}\defeq\sm{A}{\brck{P}}$ and define:
\begin{align*}
\existsmodal(x:A),\ P(x) & \equiv \modal\sm{A}P\\
\existsmodalunique(P) & \equiv  (\existsmodal(x:A),\ P(x))\times\ismodal(\{A\mid P\}).
\end{align*}
\end{defn}

\begin{lem}\label{lem:modal_Sigma}
 Suppose that $A$ is modal and for each $a:A$, $B (a)$ is modal. Then $\sm{a:A}B(a)$ is modal.
\end{lem}
\begin{proof}
We first define a map $\modal \sm{a:A}B(a)\to \sm{a:A}B(a)$ and then show that it is inverse to $\eta$. Consider $\hat{\proj1}:\modal \sm{a:A}B(a)
\to \modal A$. Because $A$ is modal, this gives us an element $a:A$. Combining this with $\hat{\proj2^a}:\modal \sm{a:A}B(a) \to \modal (B
(a))$ and the modality of $B (a)$, we obtain an element of $\sm{a:A}B(a)$. It is easy to check that this is in fact an inverse.
\end{proof}

\begin{lem}\label{lem:hlevel_sum_destructed}
Let $P:A\to\modaltype$. If $\setof{a:A | \brck{P(a)} }$ is modal, then the total space $\sm{x:A} P(x)$ is modal.
\end{lem}
\begin{proof}
Define $\tilde{P}:\{A\mid P\}\to \type$ by $\tilde{P}(a;p)\defeq P(a)$.
Then $\sm{A}P\simeq \sm{\{A\mid P\}}{\tilde{P}}$. The latter is a sum of modal types and hence itself modal by lemma~\ref{lem:modal_Sigma}.
\end{proof}

\begin{lem}[Iota]\label{lem:iota}
For every dependent type $P:A\to\modaltype$, there is a function
\begin{equation*}
\exists!_\modal(P)\to\sm{x:A} P(x).
\end{equation*}
\end{lem}

\begin{proof}
Since $\{A\mid P\}$ is modal, by \autoref{lem:hlevel_sum_destructed}, 
it follows that $\sm{x:A} P(x)$ is modal, so there is
an equivalence $\existsmodal(x:A),\ P(x)\simeq\sm{x:A}{P(x)}$.
\end{proof}

\begin{thm}[The principle of modal choice]
Let $P:X\to\type$ and $R:\prd{x:X}(P(x)\to\modaltype)$. Then there is a function of type
\begin{equation*}
\prd{x:X} \existsmodalunique(R(x))\to\sum\big(f:\prd{x:X} P(x)\big)\prd{x:X} R(x,f(x)).
\end{equation*}
\end{thm}

\begin{proof}
Suppose that $\prd{x:X} \existsmodalunique(R(x))$. By \autoref{lem:iota}, 
we can find a term of type $\sm{u:P(x)} R(x,u)$
for every $x:X$. The statement now follows from the usual principle of choice.
\end{proof}

For truncation we obtain the principle of unique choice. 
However, the of principle of unique choice is usually stated quite differently:
the type $\ismodal(\{A\mid P\})$ may be replaced by the type
\begin{equation*}
\mathsf{atMostOne}(P) \equiv  \prd{x,y:A} P(x)\to P(y)\to(x= y)
\end{equation*}
However, a term of $\ism{-1}(P)$ can be deduced from a term of $\mathsf{atMostOne}(P)$. This is the content of the following
lemma. In order to prove this, we will use that there is an equivalence
\begin{equation*}
\mathsf{isprop}(P)\simeq \prd{x,y:A} P(x)\to P(y)\to\iscontr(x= y)
\end{equation*}
This is not hard to verify, and a proof relies on the fact that $\pairr{x,u}=\pairr{x^\prime,u^\prime}$ is equivalent to the type
$x= x^\prime$ for any two terms $\pairr{x,u},\pairr{x^\prime,u^\prime}:\sm{x:A}{\brck{P(x)}}$.

% The following lemma will be used later on in the proof that images are strict coequalizers, see
% proposition\ref{prop:images_are_coequalizers}.

\begin{lem}\label{lem:atmostone_to_atlevel}
For any $P:A\to\type$, there is a function of type
\begin{equation*}
\mathsf{atMostOne}(P)\to\ism{-1}(P).
\end{equation*}
\end{lem}

\begin{proof}
We will show that there is a function of type
\begin{equation*}
\mathsf{atMostOne}(P)\to\prd{x,y:A} P(x)\to P(y)\to\iscontr(x= y).
\end{equation*}
Let $H$ be a term of type $\mathsf{atMostOne}(P)$ and let $x,y:A$, $u:P(x)$ and $v:P(y)$. Our goal is to show that $x= y$ is
contractible by showing that for any $p:x= y$ there is a path $p= H(x,y,u,v)$. The idea is that we apply $H$ to the paths $p$ and
$H(x,y,u,v)$ to obtain a path between them. The following general statement is useful to compute the terms $H(p,z,u,w)$:

\begin{lem}
Suppose that $H:\prd{x,y:A} P(x)\to P(y)\to (x= y)$. Then for any $p:x= x^\prime$ and $q:y= y^\prime$ there are paths
\begin{align*}
(p\cdot H(x))(y,u^\prime,v) & = H(x,y,p^{-1}\cdot u^\prime,v)\ct p^{-1} & & [u^\prime:P(x^\prime),\ v:P(y)]\\
(q\cdot H(x,y))(u,v^\prime) & = q\ct H(x,y,u,p^{-1}\cdot v^\prime) & & [u : P(x),\ v^\prime :P(y^\prime)].
\end{align*} 
\end{lem}
The proof of this statement is by path induction. We see from the above that the path $H(p,H(x,y,u,v))$ induces a witness of the
commutativity of the diagram
\begin{equation*}
\begin{tikzpicture}
\matrix (m) [std] {x & y \\ x & y \\};
\draw[patharrow] (m-2-2) -- node[below] {$p^{-1}$} (m-2-1);
\draw[patharrow] (m-2-1) -- node[left]  {$H(x,x,u,u)$} (m-1-1);
\draw[patharrow] (m-1-1) -- node[above] {$H(x,y,u,v)$} (m-1-2);
\draw[patharrow] (m-2-2) -- node[right] {$H(y,y,v,v)$} (m-1-2);
\end{tikzpicture}
\end{equation*}
Therefore, the assertion would follow if there is a function of type
\begin{equation*}
\mathsf{atMostOne}(P)\to\sm{H^\prime:\mathsf{atMostOne}(P)}{\prd{x:X}{u:P(x)}}H^\prime(x,x,u,u)=\idfunc{x}.
\end{equation*}
This is easy, since we can define the function $H^\prime:\prd{x,y:X} P(x)\to P(y)\to(x= y)$ by
\begin{equation*}
H^\prime(x,y,u,v)\equiv H(x,y,u,v)\ct H(x,x,u,u)^{-1}.\qedhere
\end{equation*}
\end{proof}

\begin{defn}
Suppose $P:A\to\type$ is a dependent type over $A$. We define
\begin{equation*}
\exists!(x:A),\ P(x)\equiv \brck{\sm{x:A}P(x)}\times\mathsf{atMostOne}(P).
\end{equation*}
\end{defn}

\begin{cor}[The principle of unique choice]\label{cor:auc}
Let $P:X\to\type$ and $R:\prd{x:X} (P(x)\to\mathsf{hProp})$. Then there is a function
\begin{equation*}
\big(\prd{x:X} \exists !(u:P(x)),\ R(x,u)\big) \to
\sum \big(f:\prd{x:X} P(x)\big) \prd{x:X} R(x,f(x)).
\end{equation*}
\end{cor}


\section{Reflective subuniverses}
\label{subsec:reflective-subuniverses}

So far, in this chapter, we have focused on $n$-types and the $n$-truncation operation.
In fact, $n$-types and their truncation are a special case of a more general notion called a \emph{reflective subuniverse}, and many of
their properties can be proven formally in this generality.

\begin{defn}
  A \textbf{subuniverse} of \type is a predicate $P:\type \to \prop$.
  We write
  \[\P \defeq \sm{A:\type} P(A).\]
\end{defn}

An inhabitant of $\P$ is a pair $(A,s)$ with $A:\type$ and $s:P(A)$, but we will often simply write it as $A$.
Thus $A:\P$ means that $A:\type$ and $P(A)$ holds.
For instance, $P$ could be \isprop, \isset, or more generally $\istype{n}$ for any $n\ge-2$, in which case \P would be \prop, \set, or
$\typele{n}$.
On the other hand, if $P(A)\defeq \unit$, then every type lies in \P, i.e.\ $\P=\type$.

\begin{defn}
  A subuniverse $\P$ of $\type$ is a \textbf{reflective subuniverse} if
  for every $A:\type$ we have a type $\reflect(A):\P$ and a map
  $\project_A:A\to\reflect(A)$ such that for every $A:\type$ and $B:\P$, the following map is an equivalence:
  \[\function{(\reflect(A)\to{}B)}{(A\to{}B)}{f}{f\circ\project_A}.\]
\end{defn}

The notation $\reflect$ for this operation may seem slightly odd, but it will make more sense after \autoref{sec:modalities}.

In particular, reflectivity implies that for every $B:\P$ and $g:A\to{}B$ there is a unique map $\ext(g):\reflect(A)\to{}B$ making the following
diagram commute (up to homotopy, of course).
\[\uppercurveobject{{ }}\lowercurveobject{{ }}\twocellhead{{ }}
\xymatrix{A \ar^{\project_A}[r] \ar_g[rd] \druppertwocell{=} & \reflect(A)
  \ar@{-->}^{\ext(g)}[d] \\
  & B}\]

The universal property means that $\reflect$ is left adjoint to the inclusion
$\iota:\P\to\type$ with $\project_A$ as the unit, because the last $B$ in the
universal property can be replaced by $\iota(B)$ given that $B$ is in $\P$.

One example is truncations, as defined in \autoref{sec:truncations}.
\begin{lem}%[Truncations are reflective]
  $\typele{n}$ is a reflective subuniverse of $\type$, with $\reflect$
  given by $\trunc{n}{-}$.
\end{lem}
\begin{proof}
  Immediate from the universal property of truncations, \autoref{thm:trunc-reflective}.
\end{proof}

\begin{lem}
  Let \P be a subuniverse of \type. The assertion that \P is reflective is \anhprop.
\end{lem}

\begin{proof}
  Let's assume that \P is reflective in two different ways
  $(\reflect,\project,\ext)$ and $(\reflect',\project',\ext')$. We need to
  construct an equivalence between $\reflect(A)$ and $\reflect'(A)$ for every
  $A:\type$, and we need to prove that the following diagram commutes:
  \[\uppercurveobject{{ }}\lowercurveobject{{ }}\twocellhead{{ }}
  \xymatrix{A \ar^{\project_A}[r] \ar_{\project'_A}[rd] \druppertwocell{=} &
    \reflect(A) \ar@{->}^\sim[d] \\
    & \reflect'(A)}\]
  (We can ignore the components $\ext$ and $\ext'$, since they belong to mere propositions---the assertion that precomposing with $\project$
and $\project'$ are equivalences.)

  Now the type $\reflect'(A)$ is in \P, so we can define the map
  \[\ext(\project'_A):\reflect(A)\to\reflect'(A)\]
  which is exactly the map making the previous diagram commute.
  We can also define
  \[\ext'(\project_A):\reflect'(A)\to\reflect(A).\]
  In order to prove that the composite is the identity, we only need to prove
  that $\ext'(\project_A)\circ\ext(\project'_A)\circ\project_A=\project_A$,
  which is the case:
  \[\uppercurveobject{{ }}\lowercurveobject{{ }}\twocellhead{{ }}
  \xymatrix{A \ar^{\project_A}[r] \ar_{\project'_A}[rd]
    \ar@/_5mm/_{\project_A}[rdd] &
    \reflect(A) \ar@{->}^{\ext(\project'_A)}[d] \\
    & \reflect'(A) \ar@{->}^{\ext'(\project_A)}[d] \\
    & \reflect(A)}\]
  The other composite is just as easy.
  Thus, $\ext'(\project_A)$ is a quasi-inverse of $\ext(\project'_A)$, so the latter is an equivalence as desired.
\end{proof}

For the rest of this section, we assume that $\P$ is a reflective subuniverse of
$\type$.

The following lemma says that the counit of the adjunction is an equivalence, as expected for a reflection.
\begin{lem}
  \label{reflectPequiv}
  If $A:\P$, then the map $\project_A:A\to\reflect(A)$ is an equivalence.
\end{lem}
\begin{proof}
  Given that $A$ is in $\P$, we can define $\ext(\idfunc[A]):\reflect(A)\to{}A$.

  Then we have $\ext(\idfunc[A])\circ\project_A=\idfunc[A]:A\to{}A$ by
  definition.  In order to prove that
  $\project_A\circ\ext(\idfunc[A])=\idfunc[\reflect(A)]$, we only need to prove
  that $\project_A\circ\ext(\idfunc[A])\circ\project_A=
  \idfunc[\reflect(A)]\circ\project_A$.
  This is again true:
  \[\xymatrix{
    A \ar^{\project_A}[r] \ar_{\idfunc[A]}[rd] &
    \reflect(A) \ar^>>>{\ext(\idfunc[A])}[d] \ar@/^40pt/^{\idfunc[\reflect(A)]}[dd] \\
    & A \ar_{\project_A}[d] \\
    & \reflect(A)}\]
\end{proof}

Note that the converse is always true: if $\project_A$ is an equivalence, then $A:\P$.
This is easy using univalence and the fact that $\reflect(A):\P$.

The reflector $\reflect$ is a map $\type\to\P$, which (like any map in type theory) is a functor of $\infty$-groupoids.
Using the universal property, we should be able to prove that $\reflect$ is in fact $(\infty,1)$-functorial, i.e.\ preserves composition of
(not necessarily invertible) functions up to all higher homotopies.
But we don't know how to express $(\infty,1)$-functoriality all at once, so we will only prove a few bits of it.

\begin{defn}
  If $f:A\to{}B$, there is a map $\reflect(f):\reflect(A)\to\reflect(B)$ defined
  by
  \begin{equation}
    \reflect(f)\circ\project_A=\project_B\circ{}f\label{eq:project-natural}
  \end{equation}
  (or in other words $\reflect(f)=\ext(\project_B\circ{}f)$).
\end{defn}

\[\uppercurveobject{{ }}\lowercurveobject{{ }}\twocellhead{{ }}
\xymatrix{A \ar^-{\project_A}[r] \ar_-f[d] \drtwocell{=} & \reflect(A)
  \ar@{-->}^-{\reflect(f)}[d]
  \\ B \ar_-{\project_B}[r] & \reflect(B)}\]

Moreover, this operation satisfies the following functoriality conditions:
  \begin{align*}
    \reflect(\idfunc[A])=\idfunc[\reflect(A)]\\
    \reflect(f\circ{}g)=\reflect(f)\circ\reflect(g)
  \end{align*}
  In order to define these equalities, we only need to define them after
  composition by $\project(A)$, because of the universal property.
  The first is defined by
  \[\xymatrix{
    \reflect(\idfunc[A])\circ\project_A \ar@{==}[r] \ar@{=}[d] &
    \idfunc[\reflect(A)]\circ\project_A \ar@{=}[d] \\
    \project_A \ar@{=}[r] & \project_A
  }\]
  The second one is defined by
  \[\xymatrix{
    \reflect(f\circ g)\circ\project_A \ar@{==}[r] \ar@{=}[d] &
    \reflect(f)\circ\reflect(g)\circ\project_A \ar@{=}[d] \\
    \project_C\circ f\circ g \ar@{=}[r] & \reflect(f)\circ\project_B\circ g
  }\]

% \begin{proof}
%   The map $\reflect(f)$ is defined to be the unique map
%   $\reflect(A)\to\reflect(B)$ such that
%   $\reflect(f)\circ\project_A=\project_B\circ{}f$ (using the universal property
%   of $\reflect$), or in other words $\reflect(f)=\ext(\project_B\circ{}f)$.

%   \[\uppercurveobject{{ }}\lowercurveobject{{ }}\twocellhead{{ }}
%   \xymatrix{A \ar^-{\project_A}[r] \ar_-f[d] \drtwocell{=} & \reflect(A)
%     \ar@{-->}^-{\reflect(f)}[d]
%     \\ B \ar_-{\project_B}[r] & \reflect(B)}\]
%   In order to prove that $\reflect(\idfunc[A])=\idfunc[\reflect(A)]$, we only
%   need to prove that
%   $\reflect(\idfunc[A])\circ\project_A=\idfunc[\reflect(A)]\circ\project_A$. But
%   both sides are equal to $\project_A$.

%   Similarly, to prove that $\reflect(f\circ{}g)=\reflect(f)\circ\reflect(g)$ we
%   only need to prove that $\reflect(f\circ{}g)\circ\project_A=
%   \reflect(f)\circ\reflect(g)\circ\project_A$ and both sides are equal to
%   $\project_C\circ{}f\circ{}g$.

%   Note that by definition, the following diagram commutes:

%   \[\xymatrix{
%     \reflect(f\circ g)\circ\project_A \ar@{=}[r] \ar@{=}[d] &
%     \reflect(f)\circ\reflect(g)\circ\project_A \ar@{=}[d] \\
%     \project_C\circ f\circ g \ar@{=}[r] & \reflect(f)\circ\project_B\circ g
%     }\]
% \end{proof}

Note also that the defining equation~\eqref{eq:project-natural} asserts the first level of $(\infty,1)$-naturality for the transformation
$\project$.
This implies a generalization of \autoref{thm:h-level-retracts}.

\begin{lem}\label{thm:reflsubuniv-retract}
  A reflective subuniverse is closed under retractions.
  That is, if $B$ is in $\P$ and $A$ is a retract of $B$, then $A$ is in $\P$.
\end{lem}
\begin{proof}
  By functoriality and naturality, $\project_A$ is a retract of $\project_B$ in the sense of \autoref{defn:retract}.
  But $\project_B$ is an equivalence, hence so is $\project_A$.
  Thus, $A$ is in $\P$.
\end{proof}

We also have the following ``functoriality'' property for the factorizations.

\begin{defn}
  Given types $A,B:\type$ and any $C:\P$, for any functions $f:A\to{}B$ and $i:B\to{}C$ we have
  \[\ext(i\circ{}f)=\ext(i)\circ\reflect(f)\]

  \[\xymatrix{
    A \ar^{\project_A}[r] \ar_f[d] & \reflect(A) \ar^{\reflect(f)}[d]
    \ar@/^15mm/^{\ext(i\circ f)}[dd] \\
    B \ar^{\project_B}[r] \ar_i[rd] & \reflect(B) \ar^{\ext(i)}[d] \\
    & C}\]

  This equality is defined by
  \[\xymatrix{
    \ext(i\circ f)\circ\project_A \ar@{==}[r] \ar@{=}[d] &
    \ext(i)\circ\reflect(f)\circ\project_A \ar@{=}[d] \\
    i\circ f \ar@{=}[r] & \ext(i)\circ\project_B\circ f
  }\]

\end{defn}

% \begin{proof}
%   \[\xymatrix{A \ar^{\project_A}[r] \ar_f[d] & \reflect(A) \ar^{\reflect(f)}[d]
%     \\
%     B \ar^{\project_B}[r] \ar_i[d] & \reflect(B) \ar^{\ext(i)}[ld] \\
%     C &}\]

%   We only have to prove that
%   \[\ext(i\circ{}f)\circ\project_A=\ext(i)\circ\reflect(f)\circ\project_A\]
%   which is the case, because they are both equal to $i\circ{}f$.
% \end{proof}

\section{Localizations}
\label{sec:localizations}

A wide class of reflective subuniverses can be obtained by the process of \emph{localization}.

\begin{defn}
  Let $I:\type$ and $A,B:I\to\type$, with $f:\prd{i:I} A_i \to B_i$.
  A type $Z$ is \define{$f$-local} if for every $i:I$, the induced map
  \[ (-\circ f_i) : (B_i \to Z) \to (A_i \to Z) \]
  is an equivalence.
  Thus, we have
  \[ \islocal f (Z) \defeq \prd{i:I} \isequiv(-\circ f_i). \]
\end{defn}

\noindent
Of course, $\islocal f: \UU\to \prop$ is a subuniverse.

\begin{eg}
  Let $I\defeq\unit$ and $A_{\ttt}\defeq\unit$, while $B_\ttt\defeq \Sn^{n+1}$, with $f_\ttt$ the unique map.
  Then $(A_\ttt \to Z) \simeq Z$, and the fiber of $(-\circ f_\ttt)$ over $z:Z$ is equivalent to $\Omega^{n+1}(Z,z)$.
  Thus, by \autoref{thm:ntype-nloop}, $Z$ is $f$-local exactly when it is an $n$-type.
  (Although \autoref{thm:ntype-nloop} only applies to $n\ge 0$, this statement holds for $n=-1$ as well, recalling that $\Sn^0 = \bool$.)
\end{eg}

\begin{eg}
  Let $p:\nat$ be a prime number and $I\subseteq \nat$ be the set of prime numbers other than $p$.
  For $q:I$, let $A_q \defeq B_q \defeq \Sn^1$, and define $f_q:\Sn^1\to\Sn^1$ by $f_q(\base)\defeq\base$ and $\ap{f_q}{\lloop} \defid \lloop^q$.
  Then $Z$ is $f$-local just when the map $(-)^q : \Omega(Z,z) \to \Omega(Z,z)$ is an equivalence for all $z:Z$ and $q:I$.
  This is closely related to \emph{$p$-locality} in the sense of classical homotopy theory.
\end{eg}

We will show that the subuniverse of $f$-local types is always reflective.
Given $X$, define $\loc f(X)$ to be the higher inductive type generated by:
\begin{itemize}
\item A map $\mreturn_X: X \to \loc f(X)$.
\item For each $i:I$ and $g:A_i\to \loc f(X)$, a function $s_{i}(g):B_i \to \loc f(X)$.
\item For each $i:I$ and $g:A_i\to \loc f(X)$, a function $r_{i}(g):B_i \to \loc f(X)$.
\item For each $i:I$ and $g:A_i\to \loc f(X)$ and $a:A_i$, a path
  \[\alpha_{i}(g,a):\id[\loc f(X)] {s_{i}(g)(f_i(a))}{g(a)}.\]
\item For each $i:I$ and $h:B_i\to \loc f(X)$ and $b:B_i$, a path
  \[\beta_i(h,b):\id[\loc f(X)]{r_{i}(h\circ f_i)(b)}{h(b)}.\]
\end{itemize}

The idea is to start with $X$ and freely force all the functions $(-\circ f_i)$ to be equivalences.

\begin{lem}\label{thm:localization-is-local}
  For any $X$, the type $\loc f(X)$ is $f$-local.
\end{lem}
\begin{proof}
  For any $i$, the operations $s_i$ and $r_i$ are a right and a left inverse to $(-\circ f_i)$, respectively.
  (We use function extensionality, to make the homotopies arising from the last two constructors into paths in function types.)
\end{proof}

Instead of freely adding a left and a right inverse, we could have freely added a quasi-inverse by combining the constructors $s$ and $r$, but this would have been ill-behaved and not given us what we want.
Rather than diving $s$ and $r$, we could also have solved that problem by adding a constructor for the half-adjoint coherence condition, but this would have been a 2-dimensional path, whereas the definition we have given is simpler, as it involves only 1-dimensional paths.

The standard induction principle of $\loc f(X)$ applies to any $C:\loc f(X) \to \type$ with the following:
\begin{itemize}
\item A function $k:\prd{x:X} C(\mreturn(x))$.
\item For each $i,g$ and $g':\prd{a:A_i} C(g(a))$, a function $s'_{i}(g,g'):\prd{b:B_i} C(s_i(g)(b))$.
\item For each $i,g$ and $g':\prd{a:A_i} C(g(a))$, a function $r'_{i}(g,g'):\prd{b:B_i} C(s_i(g)(b))$.
\item For each $i,g,g'$ and $a:A_i$, a dependent path
  \[ \alpha'_i(g,g',a) : \dpath{C}{\alpha_i(g,a)}{s'_i(g,g')(f_i(a))}{g'(a)}. \]
\item For each $i,h$ and $h':\prd{b:B_i} C(h(b))$ and $b:B_i$, a dependent path
  \[ \beta'_i(h,h',b) : \dpath{C}{\beta_i(h,b)}{r'_i(h\circ f_i,h' \circ f_i)(b)}{h'(b)}. \]
\end{itemize}
Under these hypotheses, we can define $\ell:\prd{z:\loc f(X)} C(z)$ such that
\begin{align*}
  \ell(\mreturn(x)) &\defeq k(x)
  % \ell(s_i(g)(b)) &\defeq s'_i(g, \ell \circ g)(b)\\
  % \ell(r_i(g)(b)) &\defeq r'_i(g, \ell \circ g)(b)\\
  % \apd{\ell}{\alpha_i(g,a)} &\defeq \alpha'_i(g,\ell\circ g,a)\\
  % \apd{\ell}{\beta_i(h,b)} &\defeq \beta'_i(h,\ell\circ h,b).
\end{align*}
and four other computation rules.

Note that when $C$ is non-dependent, the latter four hypotheses give exactly (modulo function extensionality) the data in a proof that $(-\circ f_i) : (B_i \to C) \to (A_i\to C)$ is bi-invertible for each $i:I$.
Thus, the recursion principle of $\loc f(X)$ says simply that if $C$ is $f$-local and $k:X\to C$, we have an induced $\ext(k):\loc f(X)\to C$ with $\ext(k)\circ \mreturn \jdeq k$.
As usual, the induction principle allows us to prove essential uniqueness.

\begin{lem}\label{thm:local-eta}
  If $C$ is $f$-local and $\ell,\ell':\loc f(X) \to C$ are functions with $\ell\circ \mreturn = \ell'\circ\mreturn$, then $\ell=\ell'$.
\end{lem}
\begin{proof}
  Define $D:{z:\loc f(X)} \to \type$ by $D(z) \defeq (\ell(z)=\ell'(z))$.
  By assumption, we have $\prd{x:X} D(\mreturn(x))$; it remains to give the other four data of the induction principle.
  Fix $i:I$; since $C$ is $f$-local, the function $(-\circ f_i):(B_i\to C) \to (A_i\to C)$ has a left inverse $\rho$ and a right inverse $\sigma$.

  (TODO\dots)
\end{proof}

\begin{thm}\label{thm:local-reflective}
  The subuniverse of $f$-local types is reflective.
\end{thm}
\begin{proof}
  Just like the proof of \autoref{thm:trunc-reflective}.
\end{proof}

% \begin{thm}\label{thm:local-modal}
%   Suppose $C:\loc f(X) \to \type$ has the property that every $C(z)$ is $f$-local.
%   Then given $k:\prd{x:X} C(\mreturn(x))$, there exists $\ell:\prd{z:\loc f(X)} C(z)$ such that $\ell(\mreturn(x)) \jdeq k(x)$.
% \end{thm}
% \begin{proof}
%   We must specify the four additional data in the standard induction principle.
%   [TODO\dots]
% \end{proof}

% The localization will make all the maps $B a \to
% \mathcal{L}_P(X)$ constant in a free way.
% \begin{defn}\label{defn:localization_as_hit}
% Let $P:B\to\type$ be a dependent type. The localization $\mathcal{L}_P$ with
% respect to $P$ is an operator of type $\type\to\type$, where
% for each $X:\type$, the type $\mathcal{L}_P(X)$ is constructed as a higher
% inductive type. The basic constructors of $\mathcal{L}_P(X)$ are
% \begin{align*}
% i & : X\to\mathcal{L}_P(X)
% \intertext{and terms}
% \alpha(f) & : \mathcal{L}_P(X)\\
% \beta(f) & : \prd{u:P(b)}f(u)=\alpha(f)\\
% \gamma(f) & : \prd{w:\mathcal{L}_P(X)}(H:f\sim\lambda u.w),\ w=\alpha(f)\\
% \delta(f) & : \prd{w:\mathcal{L}_P(X)}(H:f\sim\lambda u.w)(u:P(b)),\ \gamma(f,H)
% \ct H(u)=\beta(f,u)
% \end{align*}
% for every $b:B$ and $f:P(b)\to\mathcal{L}_P(X)$. The induction principle for
% $\mathcal{L}_P(X)$ is that for every dependent type $\Lambda:\mathcal{L}_P(X)
% \to\type$ over $\mathcal{L}_P(X)$, if there are
% \begin{align*}
% I & : \prd{x:X}\Lambda(i(x))
% \intertext{and for every $b:B$, $f:P(b)\to\mathcal{L}_P(X)$ and $F:\prd{u:P(b)}\Lambda(f(u))$}
% \epsilon(F) & : \Lambda(\alpha(f))\\
% \zeta(F) & : \prd{u:P(b)}\beta(f,u)\cdot F(f(u))= \epsilon(F)
% \intertext{and for every $w_0:\mathcal{L}_P(X)$, $w_1:\Lambda(w_0)$, 
% $H_0:f\sim\lambda u.w$ and $H_1:\prd{u:P(b)}H_0(u)\cdot F(f(u))= w_1$
% a path}
% \mreturn(F,H_1) & : \gamma(f,H_0)\cdot w_1= \epsilon(F)
% \end{align*}
% a path $\theta(F,H_1,u)$ witnessing the commutativity of the diagram
% \begin{equation*}
% \begin{tikzpicture}
% \matrix (m) [std] {\beta(f,u)\cdot F(f(u)) & & \varepsilon(F) \\
% \gamma(f,H_0)\cdot H_0(u)\cdot F(f(u)) & & \gamma(f,H_0)\cdot w_1 \\};
% \draw[patharrow] (m-2-3) -- node[below] {$(\gamma(f,H_0)\cdot H_1(u))^{-1}$} (m-2-1);
% \draw[patharrow] (m-2-1) -- node[left]  {$\delta(f,H_0,u)\cdot F(f(u))$} (m-1-1);
% \draw[patharrow] (m-1-1) -- node[above] {$\zeta(f,u)$} (m-1-3);
% \draw[patharrow] (m-2-3) -- node[right] {$\mreturn(F,H_1)$} (m-1-3);
% \end{tikzpicture}
% \end{equation*}
% (note that in this diagram, we have not taken the canonical path
% \begin{equation*}
% \gamma(f,H_0)\cdot (H_0(u)\cdot F(f(u)))=(\gamma(f,H_0)\ct H_0(u))\cdot F(f(u)),
% \end{equation*}
% but it should obviously be there), then there is a section $s:\prd{w:\mathcal{L}P}(X))$
% with the property that there is a term
% \begin{align*}
% \kappa & : \prd{x:X}s(i(x))= I(x)
% \intertext{and for every $b:B$ and $f:P(b)\to\mathcal{L}_P(X)$ there are terms}
% \mu(f) & : s(\alpha(f))= \epsilon(s\circ f)\\
% \nu(f) & : \prd{u:P(b)}\mu(f)\ct s(\beta(f,u))=\zeta(s\circ f,\beta(f))\\
% \xi(f) & : \prd{w:\mathcal{L}_P}(X)(H:f\sim\lambda u.w),\ \mu(f)\ct s(\gamma(f,H))=\mreturn(s\circ f,s\circ H)
% \end{align*}
% and a term $o(f,H,u)$ witnessing the commutativity of the diagram
% \begin{equation}
% \begin{tikzpicture}
% \matrix (m) [std] {\mu(f)\ct s(\beta(f,u))\ct (\delta(f,h,u)\cdot s(f(u)))\ct (\gamma(f,H)\cdot s(H(u)))^{-1} & & \mu(f)\circ
% s(\gamma(f,H)) \\ \zeta(s\circ f,s\circ\beta(f))\ct (\delta(f,h,u)\cdot s(f(u)))\ct (\gamma(f,H)\cdot s(H(u)))^{-1} & &
% \mreturn(f,H)\\};
% \draw[patharrow] (m-2-1) -- node[below] {$\theta(s\circ f,s\circ H,u)$} (m-2-3);
% \draw[patharrow] (m-1-3) -- node[right] {$\xi(f)$} (m-2-3);
% \draw[patharrow] (m-1-1) -- node[right]  {$\nu(f)^{-1}\rightwhisker ((\delta(f,H,u)\cdot s(f(u)))\ct(\gamma(f,H)\cdot s(H(u)))$}(m-2-1);
% \draw[patharrow] (m-1-1) -- node[above,yshift=1ex] {$s(\gamma(f,H))^{-1}\leftwhisker s(\delta(f,H,u))\rightwhisker s(\beta(f,u))$}(m-1-3);
% \end{tikzpicture}
% \end{equation}
% \end{defn}


\section{Limits in reflective subuniverses}
\label{sec:pullbacks}

In \autoref{sec:universal-properties} we observed that certain type forming operations have universal properties in the
$(\infty,1)$-category of types.
For instance, we had an equivalence
\[ \eqvspaced{(X\to A\times B) }{(X\to A)\times (X\to B)}. \]
We can ask for objects in any subuniverse with similar universal properties, for instance:

\begin{defn}
  Let $A,B:\P$, with $\P$ any subuniverse.
  A \textbf{product} of $A$ and $B$ in $\P$ is an object $C:\P$ together with functions $p:C\to A$ and $q:C\to B$ such that for any $X:\P$,
the induced map
  \[ (X\to C) \to (X\to A)\times (X\to B) \]
  is an equivalence.
\end{defn}

Of course, if the product type $A\times B$ lies in $\P$, then it is a product of $A$ and $B$ in $\P$.
As usual in category theory, if the subuniverse is reflective, then this is automatic.

\begin{thm}\label{thm:reflsubuniv-prod}
  If $\P$ is a reflective subuniverse and $A,B:\P$, then $A\times B$ is also in $\P$.
  Therefore, any pair of objects in $\P$ has a product in $\P$.
\end{thm}
\begin{proof}
  Since $A,B:\P$, the projections $\proj1:A\times B\to A$ and $\proj2:A\times B\to B$ factor uniquely through $\reflect(A\times B)$,
yielding $\ext(\proj1)$ and $\ext(\proj2)$.
  But now by the universal property of $A\times B$, we have a unique map $h:\reflect(A\times B) \to A\times B$ such that $\proj1 \circ h =
\ext(\proj1)$ and $\proj2\circ h = \ext(\proj2)$.
  It follows that $h$ is a retraction of $\project_{A\times B}$.
  Thus, $A\times B$ is a retract of $\reflect(A\times B)$, hence by \autoref{thm:reflsubuniv-retract} it is in $\P$.
\end{proof}

A similar argument shows that $\unit$ is in every reflective subuniverse.
Somewhat more interesting is the case of (homotopy) pullbacks, which requires some more definitions to state precisely.

\begin{defn}
  A \emph{cospan} in $\P$ is a diagram of the following form
  \[\xymatrix{& B \ar^g[d] \\ A \ar_f[r] & C}\]
  with $A,B,C:\P$.
\end{defn}

\begin{defn}
  If $\Ddiag$ is a cospan and $D:\P$, a \emph{cone over $\Ddiag$ with
    base $D$} is a triple $(i,j,h)$ such as in the following diagram:
  \[\uppercurveobject{{ }}\lowercurveobject{{ }}\twocellhead{{ }}
  \xymatrix{D \ar^j[r] \ar_i[d] \drtwocell{^h} & B \ar^g[d] \\
    A \ar_f[r] & C
  }\]

  We denote by $\cone{\Ddiag}{D}$ the type of all such cones.
\end{defn}

The map $D\mapsto\cone{\Ddiag}{D}$ is contravariantly $(\infty,1)$-functorial, but we don't know how to express this internally in homotopy type theory.
Thus, we will only prove the bits of functoriality that we need.
First, if $t:E\to{}D$ we have a map
\[\function{\cone{\Ddiag}{D}}{\cone{\Ddiag}{E}}{c}{\composecone{t}{c}}\]
defined by $\composecone{t}{(i,j,h)}=(i\circ{}t,j\circ{}t,h\circ{}t)$.
Second, for any types $D,E,F:\P$, functions $t:E\to{}D$, $u:F\to{}E$ and
$c:\cone{\Ddiag}{D}$ we have
\begin{align*}
\composecone{\idfunc[D]}c &= c\\
\composecone{(t\circ{}u)}c &= \composecone{u}(\composecone{t}c)
\end{align*}
These equations follow easily from the unit laws and associativity of composition of
functions, and from the functoriality of $f\mapsto{}\mapfunc{f}$ which has been proved earlier.

\begin{defn}
  A pair $(D,c)$ where $D:\P$ and $c:\cone{\Ddiag}{D}$ is called a
  \define{pullback} of $\Ddiag$ in \P if for all $E:\P$ the map
  \[\function{(E\to{}D)}{\cone{\Ddiag}{E}}{t}{\composecone{t}{c}}\]
  is an equivalence.
\end{defn}

This definition says that for every $E:\P$ and for every cone over $\Ddiag$
with base $E$, there is an essentially unique map $E\to{}D$ such that the whole
diagram commutes, the proof of commutation being essentially unique too.

As remarked earlier, pullbacks in \type can be constructed directly in the expected way.

\begin{defn}
  Let $\Ddiag$ be a cospan in $\type$. We define the following type
  \[A\times_CB=\setof{(a,b):A\times{}B | f(a) = g(b)}\]

  There is a canonical cone $c_\times=(\pi_1,\pi_2,\pi_3)$ over $\Ddiag$ with
  base $A\times_CB$ given by the following diagram
  \[\uppercurveobject{{ }}\lowercurveobject{{ }}\twocellhead{{ }}
  \xymatrix{A\times_CB \ar^-{\pi_2}[r] \ar_{\pi_1}[d] \drtwocell{^\pi_3\ }
    & B \ar^g[d] \\ A \ar_f[r] & C}\]
  where
  \begin{align*}
    \pi_1((a,b),p)&=a\\
    \pi_2((a,b),p)&=b\\
    \pi_3((a,b),p)&=p\\
  \end{align*}
\end{defn}

\begin{lem}
  Let $\Ddiag$ be a diagram in $\type$. Then $(A\times_CB,c_\times)$ is a
  pullback of $\Ddiag$.
\end{lem}
\begin{proof}
  Given a type $E$ and a cone $c=(i,j,h):\cone{\Ddiag}{E}$ we construct a map
  $\Sn(c):E\to{}A\times_CB$ by
  \[\Sn(c)(x)=((i(x), j(x)), h(x))\]
  and we need to prove that $\composecone{\Sn(c)}{c_\times}=c$ and
  $\Sn(\composecone{t}{c_\times})=t$ for all $c:\cone{\Ddiag}{E}$ and
  $t:E\to{}A\times_CB$ and both are easy computations:
  \begin{align*}
    \composecone{\Sn(c)}{c_\times}
    &=\composecone{\Sn(c)}{(\pi_1,\pi_2,\pi_3)} \\
    &=(\pi_1\circ\Sn(c),\pi_2\circ\Sn(c),
    \pi_3\circ\Sn(c))\\
    &=(i,j,h)\\
    &=c
  \end{align*}
  \begin{align*}
    \Sn(\composecone{t}{c_\times})(x) &=
    \Sn(\pi_1\circ{}t,\pi_2\circ{}t,\pi_3\circ{}t)(x)\\
    &=(\pi_1(t(x)),\pi_2(t(x)),\pi_3(t(x)))\\
    &=t(x)\qedhere
  \end{align*}
\end{proof}

We observe in passing that pullbacks are preserved by function spaces.

\begin{defn}
  If $\Ddiag$ is a cospan and $D:\type$, then we define another
  cospan called $D\to\Ddiag$:
  \[\xymatrix{& (D\to B) \ar^{g\circ-}[d] \\ (D\to A) \ar_{f\circ-}[r] & (D\to
    C)}\]

  Similarly if $\Ddiag$ is a span and $D:\type$, then we have a
  cospan $\Ddiag\to{}D$:
  \[\xymatrix{& (B\to D) \ar^{-\circ{}g}[d] \\ (A\to D) \ar_{-\circ{}f}[r] &
    (C\to D)}\]
\end{defn}

\begin{lem}
  \label{coneispb}
  If $\Ddiag$ is a cospan and $D:\P$, then
  \[\cone{\Ddiag}{D}=(D\to{}A)\times_{(D\to{}C)}(D\to{}B)\]

  If $\Ddiag$ is a span and $D:\P$, then
  \[\cocone{\Ddiag}{D}=(A\to{}D)\times_{(C\to{}D)}(B\to{}D)\]
\end{lem}
\begin{proof}
  In both cases the map from left to right is $(i,j,h)\mapsto(i,j,\funext(h))$
  and the map from right to left is $(i,j,p)\mapsto(i,j,\happly(p))$ and they
  are inverse to each other.
\end{proof}

Finally, we have the analogous result to \autoref{thm:reflsubuniv-prod}.

\begin{thm}\label{thm:reflsubuniv-pb}
  If $A \xrightarrow{f}  C \xleftarrow{g} B$ is a cospan in a reflective $\P$, then $A\times_C B$ is also in $\P$.
\end{thm}
\begin{proof}
  Essentially just like \autoref{thm:reflsubuniv-prod}.
\end{proof}

\begin{cor}\label{thm:reflsubuniv-idtype}
  If $\P$ is reflective, then for any $A:\P$ and $x,y:A$, the identity type $(x=_A y)$ is also in $\P$.
\end{cor}
\begin{proof}
  $(x=_A y)$ is equivalent to the pullback of the two functions $\unit \to A$ and $\unit\to A$ defined by $x$ and $y$.
\end{proof}

There is another interesting generalization of \autoref{thm:reflsubuniv-prod}.
Recall that the dependent function type $\prd{x:A} B(x)$ can also be regarded as the (possibly infinitary) cartesian product of all the
types $B(x)$.

\begin{thm}\label{thm:reflsubunv-forall}
  If $B:A\to\P$ is any family of types in \P, then $\prd{x:A} B(x)$ is also in \P.
\end{thm}
\begin{proof}
  For any $x:A$, consider the function $\mathsf{ev}_x : (\prd{x:A} B(x)) \to B(x)$ defined by $\mathsf{ev}_x(f) \defeq f(x)$.
  Since $B(x)$ lies in $P$, this extends to
  \[ \ext(\mathsf{ev}_x) : \reflect(\prd{x:A} B(x)) \to B(x). \]
  Thus we can define $h:\reflect(\prd{x:A} B(x)) \to \prd{x:A} B(x)$ by $h(z)(x) \defeq \ext(\mathsf{ev}_x)(z)$.
  As before, $h$ is a retraction of $\project_{\prd{x:A} B(x)}$, so that ${\prd{x:A} B(x)}$ is in $\P$.
\end{proof}

In particular, if $B:\P$ and $A$ is any type, then $(A\to B)$ is in \P.
In categorical language, this means that any reflective subuniverse is an \textbf{exponential ideal}.
This, in turn, implies that the reflector preserves finite products.

\begin{cor}\label{cor:trunc_prod}
  For any types $A$ and $B$, the induced map $\reflect(A\times B) \to \reflect(A) \times \reflect(B)$ is an equivalence.
\end{cor}
\begin{proof}
  It suffices to show that $\reflect(A) \times \reflect(B)$ has the same universal property as $\reflect(A\times B)$.
  Thus, let $C:\P$; we have
  \begin{align*}
    (\reflect(A) \times \reflect(B) \to C)
    &= (\reflect(A) \to (\reflect(B) \to C))\\
    &= (\reflect(A) \to (B \to C))\\
    &= (A \to (B \to C))\\
    &= (A \times B \to C)
  \end{align*}
  using the universal properties  of $\reflect(B)$ and $\reflect(A)$, along with the fact that $B\to C$ is in \P since $C$ is.
  It is straightforward to verify that this equivalence is given by composing with $\mreturn_A \times \mreturn_B$, as needed.
\end{proof}

It may seem odd that every reflective subcategory of types is automatically an exponential ideal, with a product-preserving reflector.
However, this is also the case classically in the category of \emph{sets}, for the same reasons.
It's just that this fact is not usually remarked on, since the classical category of sets---in contrast to the category of homotopy
types---does not have many interesting reflective subcategories.


\end{comment}



\sectionNotes

The notion of homotopy $n$-type in classical homotopy theory is quite old.
It was Voevodsky who realized that the notion can be defined recursively in homotopy type theory, starting from contractibility.

The property ``Axiom K'' was so named by Thomas Streicher, as a property of identity types which comes after J, the latter being the traditional name for the eliminator of identity types.
\autoref{thm:hedberg} is due to Hedberg~\cite{hedberg1998coherence}; for more information and generalizations see~\cite{krausgeneralizations}.

The notions of $n$-connected spaces and functions are also classical in homotopy theory.
The importance of the resulting factorization system has been emphasized by recent work in higher topos theory.

In fact, almost all of the theory of $n$-types and $n$-connectedness (for a fixed $n$) can be done much more generally, in terms of an abstract structure that may be called an (idempotent, monadic) \emph{modality}.
This is an operation $\modal:\type\to\type$ which generalizes the $n$-truncation $\trunc n-$, acting as the reflector into a certain subcategory of types, with an induction principle analogous to \autoref{thm:truncn-ind}.
The name comes from \emph{modal logic}, where one studies statements such as ``possibly $A$'' (usually written $\diamond A$) or ``necessarily $A$'' (usually written $\Box A$).
In addition to $n$-truncations, modalities include \emph{subtoposes}, as a categorification of Lawvere--Tierney topologies.

\sectionExercises

\begin{ex}
  \begin{enumerate}
    \item Use \autoref{thm:h-set-refrel-in-paths-sets} to show 
    that if $\brck{A}\to A$ for every type $A$, 
    then every type is a set.
    \item Show that if every surjective function splits, 
    i.e.~if $\prd{b:B}\brck{\hfib{f}{b}}\to\prd{b:B}\hfib{f}{b}$
    for every $f:A\to B$, then every type is a set.
  \end{enumerate}
\end{ex}

\egroup

%%% Local Variables: 
%%% mode: latex
%%% TeX-master: "main"
%%% End: 


%\cleartooddpage[\thispagestyle{empty}] % Needed for correct TOC
\part{Mathematics}
\label{part:mathematics}

%%% Macros that should probably be defined before

% Type of truncated types
\newcommand{\typele}[1]{\ensuremath{\type_{\le #1}}\xspace}
% Natural numbers
\newcommand{\N}{\mathbb{N}}
% Type of truncation levels (integers starting at -2)
\newcommand{\Nt}{\N_{-2}}
% Function definition with domain and codomain
\newcommand{\function}[4]{\left\{\begin{array}{rcl}#1 &
      \longrightarrow & #2 \\ #3 & \longmapsto & #4 \end{array}\right.}
% h-propositions
\newcommand{\anhprop}{a mere proposition\xspace}
\newcommand{\hprops}{mere propositions\xspace}
% Function extensionality
\newcommand{\funext}{\mathsf{funext}}
\newcommand{\happly}{\mathsf{happly}}

%%% Local macros

% Subuniverse
\renewcommand{\P}{\ensuremath{\mathsf{P}}\xspace}
% Cocone
\newcommand{\cocone}[2]{\mathrm{cocone}_{#1}(#2)}
% Diagram
\newcommand{\Ddiag}{\mathscr{D}}
% Pushouts
\newcommand{\inl}{\mathsf{inl}}
\newcommand{\inr}{\mathsf{inr}}
\newcommand{\glue}{\mathsf{glue}}
% Cone
\newcommand{\cone}[2]{\mathrm{cone}_{#1}(#2)}
% Reflector, projection and extension in an arbitrary reflective subuniverse
\newcommand{\reflect}{\mathsf{r}}
\newcommand{\project}{\mathsf{p}}
\newcommand{\ext}{\mathsf{ext}}
% Projection and extension for truncations
\newcommand{\tproj}{\mathsf{proj}}
\newcommand{\extendsmb}{\mathsf{extend}}
\newcommand{\extend}[1]{\extendsmb(#1)}
% Apply a function to a cocone
\newcommand{\composecocone}[2]{#1\circ#2}
\newcommand{\composecone}[2]{#2\circ#1}
% Spheres
\newcommand{\Sn}{\mathbb{S}}

\chapter{Homotopy theory}
\label{cha:homotopy}

In this chapter, we will start developing homotopy theory. Unless otherwise
stated, the notation \type will denote a \emph{fixed} universe and several
occurrences of the word \type will always represent the same universe. The types
\prop, \set, \typele{n} represent the types of propositions, sets and
$n$-truncated types of this fixed universe \type.

\section{Homotopy pushouts}
\label{sec:pushouts}

\subsection{Definition}
\label{sec:push:definition}

Let’s consider \P a subuniverse of \type. More precisely we have a map
$P:\type\to\prop$ and we use the notation $A:\P$ to mean that $A:\type$ and
$P(A)$ holds. For instance \P could be \prop, \set, or more generally
$\typele{n}$ for any $n\in\Nt$.

\begin{defn}
  A \emph{pushout diagram} in $\P$ is 5-uple $\Ddiag=(A,B,C,f,g)$ with
  $A,B,C:\P$ and $f:C\to{}A$ and $g:C\to{}B$.
  \[\Ddiag=\xymatrix{C \ar^g[r] \ar_f[d] & B \\ A & }\]
\end{defn}

\begin{defn}
  Given a pushout diagram $\Ddiag=(A,B,C,f,g)$ and a type $D:\P$, a
  \emph{cocone under $\Ddiag$ with base $D$} is a triple $(i, j, h)$ with
  $i:A\to{}D$, $j:B\to{}D$ and $h : \prod_{c:C}i(f(c))=j(g(c))$
  \[\uppercurveobject{{ }}\lowercurveobject{{ }}\twocellhead{{ }}
  \xymatrix{C \ar^g[r] \ar_f[d] \drtwocell{^h} & B \ar^j[d] \\ A \ar_i[r] & D
  }\]

  We denote by $\cocone{\Ddiag}{D}$ the type of all such cocones.
\end{defn}

The map $D\mapsto\cocone{\Ddiag}{D}$ is $(\infty,1)$-functorial, but we don’t
know how to express this internally to homotopy type theory, so we will only
prove the bits of functoriality that we need.

\begin{defn}
  Given $D,E:\P$ and a map $t:D\to{}E$, there is a map
  \[\function{\cocone{\Ddiag}{D}}{\cocone{\Ddiag}{E}}{c}{\composecocone{t}c}\]
  defined by:
  \[\composecocone{t}(i,j,h)=(t\circ{}i,t\circ{}j,\mapfunc{t}\circ{}h)\]

  \[\uppercurveobject{{ }}\lowercurveobject{{ }}\twocellhead{{ }}
  \xymatrix{C \ar^g[r] \ar_f[d] \drtwocell{^h} & B \ar_j[d]
    \ar@/_/^{t\circ{}j}[rdd] & \\
    A \ar^i[r] \ar@/^/_{t\circ{}i}[rrd] & D \ar[rd]|<<<<t & \\
    & & E }\]
\end{defn}

\begin{lem}
  For any types $D,E,F:\P$, functions $t:D\to{}E$, $u:E\to{}F$ and
  $c:\cocone{\Ddiag}{D}$ we have
  \[\composecocone{\idfunc[D]}c = c\]
  \[\composecocone{(u\circ{}t)}c=\composecocone{u}(\composecocone{t}c)\]
\end{lem}
\begin{proof}
  This follows easily from the unit laws and associativity of composition of
  functions and from the functoriality of $f\mapsto{}\mapfunc{f}$ which has been proved
  earlier.
\end{proof}

We can now define the notion of pushout.

\begin{defn}
  Given a pushout diagram $\Ddiag$, a type $D:\P$ and a cocone
  $c:\cocone{\Ddiag}{D}$, the pair $(D,c)$ is said to be a \emph{pushout}
  of $\Ddiag$ in $\P$ if for every $E:\P$, the map
  \[\function{(D\to{}E)}{\cocone{\Ddiag}{E}}{t}{\composecocone{t}c}\]
  is an equivalence.
\end{defn}

This definition says that for every $E:\P$ and for every cocone under $\Ddiag$
with base $E$, there is an essentially unique map $D\to{}E$ such that the whole
diagram commutes, the proof of commutation being essentially unique too.

We can now prove some properties of pushouts. First, pushouts are unique when
they exist.

\begin{lem}
  If $(D,c)$ and $(D',c')$ are two pushouts of $\Ddiag$ in $\P$, then
  $(D,c)=(D',c')$.
\end{lem}
\begin{proof}
  We first prove that the two types $D$ and $D'$ are equivalent.

  Using the universal property of $D$ with $D'$, the following map is an
  equivalence
  \[\function{(D\to{}D')}{\cocone{\Ddiag}{D'}}{t}{\composecocone{t}c}\]

  In particular there is a function $f:D\to{}D'$ satisfying $\composecocone{f}c=c'$. In the
  same way there is a function $g:D'\to{}D$ such that $\composecocone{g}c'=c$.

  In order to prove that $g\circ{}f=\idfunc[D]$ we use the universal property of
  $D$ for $D$, which says that the following map is an equivalence:
  \[\function{(D\to{}D)}{\cocone{\Ddiag}{D}}{t}{\composecocone{t}c}\]

  Using the functoriality of $t\mapsto{}\composecocone{t}c$ we see that
  \begin{align*}
    \composecocone{(g\circ{}f)}c &= \composecocone{g}(\composecocone{f}c) \\
    &= \composecocone{g}c' \\
    &= c \\
    &= \composecocone{\idfunc[D]}c
  \end{align*}
  hence
  $g\circ{}f=\idfunc[D]$ because equivalences are injective. The same argument
  with $D'$ instead of $D$ shows that $f\circ{}g=\idfunc[D']$.

  Hence $D$ and $D'$ are equal, and the fact that $(D,c)=(D',c')$ follows from
  the fact that the equivalence between $D$ and $D'$ we just defined sends $c$
  to $c'$.
\end{proof}

\begin{cor}
  The type of pushouts of $\Ddiag$ in $\P$ is \anhprop. In particular if
  pushouts merely exist then they actually exist.
\end{cor}

\subsection{Pushouts in \type}
\label{sec:push:pushoutsintype}

We will now study the special case where $\P$ is $\type$ itself and construct
explicit pushouts using higher inductive types.

We consider a pushout diagram

\[\Ddiag=\xymatrix{C \ar^g[r] \ar_f[d] & B \\ A & }\]

where $A,B,C:\type$.

\begin{defn}
  We define a type $A\sqcup^CB:\type$ by the following higher inductive
  definition:

  \begin{align*}
    A\sqcup^CB\defeq&\ |\ \inl:A\to{}A\sqcup^CB \\
    &\ |\ \inr : B\to{}A\sqcup^CB \\
    &\ |\ \glue : \prod_{c:C}(\inl(f(c))=\inr(g(c)))
  \end{align*}

  In other words, $A\sqcup^CB$ is the disjoint union of $A$ and $B$ where for
  every $c:C$ we add a witness that $f(c)$ and $g(c)$ are equal.

  If $D$ is another type, we can define a map $s:A\sqcup^CB\to{}D$ by defining
  \begin{align*}
    (a:A)\qquad& s(\inl(a)):D\\
    (b:B)\qquad& s(\inr(b)):D\\
    (c:C)\qquad& \mapfunc{s}(\glue(c)):s(\inl(f(c)))=s(\inr(g(c)))
  \end{align*}

  And if $s,s':A\sqcup^CB\to{}D$ are two maps such that
  \begin{align*}
    s(\inl(a))=s'(\inl(a))\\
    s(\inr(b))=s'(\inr(b))\\
    \mapfunc{s}(\glue(c))=\mapfunc{s'}(\glue(c))
  \end{align*}

  for every $a,b,c$, then $s=s'$.
\end{defn}

The type $A\sqcup^CB$ is also called the \emph{pushout} of $\mathscr{D}$ because
of the following lemma:

\begin{lem}
  The type $A\sqcup^CB$ together with the cocone $c_{\sqcup}=(\inl,\inr,\glue)$
  is a pushout of $\mathscr{D}$ in $\type$.

  \[\uppercurveobject{{ }}\lowercurveobject{{ }}\twocellhead{{ }}
  \xymatrix{C \ar^g[r] \ar_f[d] \drtwocell{^\glue\ \ } & B \ar^\inr[d] \\
    A \ar_-\inl[r] & A\sqcup^CB }\]
\end{lem}
\begin{proof}
  Let’s consider an arbitrary type $E:\type$. We need to prove that the map
  \[\function{(A\sqcup^CB\to{}E)}{\cocone{\Ddiag}{E}}
  {t}{\composecocone{t}c_\sqcup}\]
  is an equivalence.

  Given a $c=(i,j,h):\cocone{\mathscr{D}}{E}$, we need to construct a
  map $\mathsf{s}(c)$ from $A\sqcup^CB$ to $E$.

  \[\uppercurveobject{{ }}\lowercurveobject{{ }}\twocellhead{{ }}
  \xymatrix{C \ar^g[r] \ar_f[d] \drtwocell{^h} & B \ar^{j}[d] \\
    A \ar_-{i}[r] & E }\]

 The map $\mathsf{s}(c)$ is defined in the following way
  \begin{align*}
    \mathsf{s}(c)(\inl(a))&=i(a)\\
    \mathsf{s}(c)(\inr(b))&=j(b)\\
    \mapfunc{\mathsf{s}(c)}(\glue(x))&=h(x)\\
  \end{align*}

We have defined a map
\[\function{\cocone{\Ddiag}{E}}{(A\sqcup^BC\to{}E)}{c}{\mathsf{s}(c)}\]
and we need to prove that this map is an inverse to
$t\mapsto{}\composecocone{t}c_\sqcup$.

On the one hand, if $c=(i,j,h):\cocone{\Ddiag}{E}$, we have
\begin{align*}
  \composecocone{\mathsf{s}(c)}c_\sqcup & =
  (\mathsf{s}(c)\circ\inl,\mathsf{s}(c)\circ\inr,
  \mapfunc{\mathsf{s}(c)}\circ\glue) \\
  & = (\lambda{}a^A.\mathsf{s}(c)(\inl(a)),\lambda{}b^B.\mathsf{s}(c)(\inr(b)),
  \lambda{}x^C.\mapfunc{\mathsf{s}(c)}(\glue(x))) \\
  & = (\lambda{}a^A.i(a),\lambda{}b^B.j(b),
  \lambda{}x^C.h(x)) \\
  & = (i, j, h) \\
  & = c
\end{align*}

On the other hand, if $t:A\sqcup^BC\to{}E$, we want to prove that
$\mathsf{s}(\composecocone{t}c_\sqcup)=t$.

For $a:A$, we have
\[\mathsf{s}(\composecocone{t}c_\sqcup)(\inl(a))=t(\inl(a))\]
because the first component of $\composecocone{t}c_\sqcup$ is $t\circ\inl$. In
the same way, for $b:B$ we have
\[\mathsf{s}(\composecocone{t}c_\sqcup)(\inr(b))=t(\inr(b))\]
and for $x:C$ we have
\[\mapfunc{\mathsf{s}(\composecocone{t}c_\sqcup)}(\glue(x))
=\mapfunc{t}(\glue(x))\]
hence $\mathsf{s}(\composecocone{t}c_\sqcup)=t$.

This proves that $c\mapsto\mathsf{s}(c)$ is an inverse to
$t\mapsto{}\composecocone{t}c_\sqcup$ which proves that
$(A\sqcup^BC,c_\sqcup)$ is a pushout of $\Ddiag$ in $\type$.
\end{proof}

\begin{defn}
  If $A:\type$, the type $\unit\sqcup^A\unit$ (with the two obvious maps
  $A\to\unit$) is called the \emph{suspension} of $A$.
\end{defn}

\section{Homotopy pullbacks}
\label{sec:pullbacks}

Most of the previous section can be dualised to give a definition of
pullbacks. More precisely we have the following

\begin{defn}
  A \emph{pullback diagram} in $\P$ is a diagram of the following form
  \[\xymatrix{& B \ar^g[d] \\ A \ar_f[r] & C}\]
  with $A,B,C:\P$
\end{defn}

\begin{defn}
  If $\Ddiag$ is a pullback diagram and $D:\P$, a \emph{cone over $\Ddiag$ with
    base $D$} is a triple $(i,j,h)$ such as in the following diagram:
  \[\uppercurveobject{{ }}\lowercurveobject{{ }}\twocellhead{{ }}
  \xymatrix{D \ar^j[r] \ar_i[d] \drtwocell{^h} & B \ar^g[d] \\
    A \ar_f[r] & C
  }\]

  We denote by $\cone{\Ddiag}{D}$ the type of all such cones.
\end{defn}

The map $D\mapsto\cone{\Ddiag}{D}$ is contravariant in $D$, if $t:E\to{}D$ we
have a map
\[\function{\cone{\Ddiag}{D}}{\cone{\Ddiag}{E}}{c}{\composecone{t}{c}}\]
defined by $\composecone{t}{(i,j,h)}=(i\circ{}t,j\circ{}t,h\circ{}t)$ and this
map is (contravariantly) functorial.

\begin{defn}
  A pair $(D,c)$ where $D:\P$ and $c:\cone{\Ddiag}{D}$ is called a
  \emph{pullback} of $\Ddiag$ in \P if for all $E:\P$ the map
  \[\function{(E\to{}D)}{\cone{\Ddiag}{E}}{t}{\composecone{t}{c}}\]
  is an equivalence.
\end{defn}

The construction of pullbacks in $\type$ is easier than the construction of
pushouts, we don’t need higher inductive types.

\begin{defn}
  Let $\Ddiag$ be a diagram in $\type$. We define the following type
  \[A\times_CB=\{(a,b):A\times{}B\ |\ f(a) = g(b)\}\]

  There is a canonical cone $c_\times=(\pi_1,\pi_2,\pi_3)$ over $\Ddiag$ with
  base $A\times_CB$ given by the following diagram
  \[\uppercurveobject{{ }}\lowercurveobject{{ }}\twocellhead{{ }}
  \xymatrix{A\times_CB \ar^-{\pi_2}[r] \ar_{\pi_1}[d] \drtwocell{^\pi_3\ }
    & B \ar^g[d] \\ A \ar_f[r] & C}\]
  where
  \begin{align*}
    \pi_1((a,b),p)&=a\\
    \pi_2((a,b),p)&=b\\
    \pi_3((a,b),p)&=p\\
  \end{align*}
\end{defn}

\begin{lem}
  Let $\Ddiag$ be a diagram in $\type$. Then $(A\times_CB,c_\times)$ is a
  pullback of $\Ddiag$.
\end{lem}
\begin{proof}
  Given a type $E$ and a cone $c=(i,j,h):\cone{\Ddiag}{E}$ we construct a map
  $\mathsf{s}(c):E\to{}A\times_CB$ by
  \[\mathsf{s}(c)(x)=((i(x), j(x)), h(x))\]
  and we need to prove that $\composecone{\mathsf{s}(c)}{c_\times}=c$ and
  $\mathsf{s}(\composecone{t}{c_\times})=t$ for all $c:\cone{\Ddiag}{E}$ and
  $t:E\to{}A\times_CB$ and both are easy computations:

  \begin{align*}
    \composecone{\mathsf{s}(c)}{c_\times}
    &=\composecone{\mathsf{s}(c)}{(\pi_1,\pi_2,\pi_3)} \\
    &=(\pi_1\circ\mathsf{s}(c),\pi_2\circ\mathsf{s}(c),
    \pi_3\circ\mathsf{s}(c))\\
    &=(i,j,h)\\
    &=c
  \end{align*}

  \begin{align*}
    \mathsf{s}(\composecone{t}{c_\times})(x) &=
    \mathsf{s}(\pi_1\circ{}t,\pi_2\circ{}t,\pi_3\circ{}t)(x)\\
    &=(\pi_1(t(x)),\pi_2(t(x)),\pi_3(t(x)))\\
    &=t(x)
  \end{align*}
\end{proof}

\begin{defn}
  If $\Ddiag$ is a pullback diagram and $D:\type$, then we define another
  pullback diagram called $D\to\Ddiag$:
  \[\xymatrix{& (D\to B) \ar^{g\circ-}[d] \\ (D\to A) \ar_{f\circ-}[r] & (D\to
    C)}\]

  Similarly if $\Ddiag$ is a pushout diagram and $D:\type$, then we have a
  pullback diagram $\Ddiag\to{}D$:
  \[\xymatrix{& (B\to D) \ar^{-\circ{}g}[d] \\ (A\to D) \ar_{-\circ{}f}[r] &
    (C\to D)}\]
\end{defn}

\begin{lem}
  \label{coneispb}
  If $\Ddiag$ is a pullback diagram and $D:\P$, then
  \[\cone{\Ddiag}{D}=(D\to{}A)\times_{(D\to{}C)}(D\to{}B)\]

  If $\Ddiag$ is a pushout diagram and $D:\P$, then
  \[\cocone{\Ddiag}{D}=(A\to{}D)\times_{(C\to{}D)}(B\to{}D)\]
\end{lem}
\begin{proof}
  In both cases the map from left to right is $(i,j,h)\mapsto(i,j,\funext(h))$
  and the map from right to left is $(i,j,p)\mapsto(i,j,\happly(p))$ and they
  are inverse to each other.
\end{proof}

\section{Reflective subuniverses of $\type$}

\begin{defn}
  A subuniverse $\P$ of $\type$ is said to be a \emph{reflective subuniverse} of
  $\type$ if for every $A:\type$ we have a type $\reflect(A):\P$ and a map
  $\project_A:A\to\reflect(A)$ satisfying the following universal property:

  For every $A:\type$ and $B:\P$, the following map is an equivalence
  \[\function{(\reflect(A)\to{}B)}{(A\to{}B)}{f}{f\circ\project_A}\]

  In other words, for every $g:A\to{}B$ in the following diagram there is a
  unique map $\ext(g):\reflect(A)\to{}B$ making the diagram commute.

  \[\uppercurveobject{{ }}\lowercurveobject{{ }}\twocellhead{{ }}
  \xymatrix{A \ar^{\project_A}[r] \ar_g[rd] \druppertwocell{=} & \reflect(A)
    \ar@{-->}^{\ext(g)}[d] \\
    & B}\]
\end{defn}

% If $\P$ is a reflective subuniverse of $\type$, then we can construct pushouts
% in $\P$ by reflecting the corresponding pushout in $\type$.

The reflector $\reflect$ is a map $\type\to\P$ and using the universal property
we should be able to prove that $\reflect$ is $(\infty,1)$-functorial. But as
before we don’t know how to express this property, so we will only express a few
bits of it.

\begin{defn}
  If $f:A\to{}B$, there is a map $\reflect(f):\reflect(A)\to\reflect(B)$ and
  this operation satisfy the following functoriality conditions:
  \begin{align*}
    \reflect(\idfunc[A])=\idfunc[\reflect(A)]\\
    \reflect(f\circ{}g)=\reflect(f)\circ\reflect(g)
  \end{align*}
\end{defn}

\begin{proof}
  The map $\reflect(f)$ is defined to be the unique map
  $\reflect(A)\to\reflect(B)$ such that
  $\reflect(f)\circ\project_A=\project_B\circ{}f$ (using the universal property
  of $\reflect$), or in other words $\reflect(f)=\ext(\project_B\circ{}f)$.

  \[\uppercurveobject{{ }}\lowercurveobject{{ }}\twocellhead{{ }}
  \xymatrix{A \ar^-{\project_A}[r] \ar_-f[d] \drtwocell{=} & \reflect(A)
    \ar@{-->}^-{\reflect(f)}[d]
    \\ B \ar_-{\project_B}[r] & \reflect(B)}\]
  In order to prove that $\reflect(\idfunc[A])=\idfunc[\reflect(A)]$, we only
  need to prove that
  $\reflect(\idfunc[A])\circ\project_A=\idfunc[\reflect(A)]\circ\project_A$. But
  both sides are equal to $\project_A$.

  Similarly, to prove that $\reflect(f\circ{}g)=\reflect(f)\circ\reflect(g)$ we
  only need to prove that $\reflect(f\circ{}g)\circ\project_A=
  \reflect(f)\circ\reflect(g)\circ\project_A$ and both sides are equal to
  $\project_C\circ{}f\circ{}g$.
\end{proof}

\begin{lem}
  \label{reflectPequiv}
  If $A:\P$, then the map $\project_A:A\to\reflect(A)$ is an equivalence.

  Moreover, if $f:A\to{}B$ with $B:\P$, then
  $\project_B^{-1}\circ\reflect(f)\circ\project_A = f$.
\end{lem}
\begin{proof}
  Given that $A$ is in $\P$, we can define $\ext(\idfunc[A]):\reflect(A)\to{}A$.

  Then we have $\ext(\idfunc[A])\circ\project_A=\idfunc[A]:A\to{}A$ by
  definition, and in order to prove that
  $\project_A\circ\ext(\idfunc[A])=\idfunc[\reflect(A)]$ we only need to prove
  that $\project_A\circ\ext(\idfunc[A])\circ\project_A=
  \idfunc[\reflect(A)]\circ\project_A$ which is true.

  For the second point is just a consequence of the fact that
  $\reflect(f)\circ\project_A=\project_B\circ{}f$.
\end{proof}

\begin{lem}
  For every $A,B,C:\type$, $f:A\to{}B$ and $i:B\to{}C$ we have
  \[\ext(i\circ{}f)=\ext(i)\circ\reflect(f)\]
\end{lem}

\begin{proof}
  \[\xymatrix{A \ar^{\project_A}[r] \ar_f[d] & \reflect(A) \ar^{\reflect(f)}[d]
    \\
    B \ar^{\project_B}[r] \ar_i[d] & \reflect(B) \ar^{\ext(i)}[ld] \\
    C &}\]

  We only have to prove that
  \[\ext(i\circ{}f)\circ\project_A=\ext(i)\circ\reflect(f)\circ\project_A\]
  which is the case because they are both equal to $i\circ{}f$.
\end{proof}

\begin{defn}
  Let
  \[\Ddiag=\xymatrix{C \ar^g[r] \ar_f[d] & B \\ A & }\]
  be a pushout diagram in $\type$. We note $\reflect(\Ddiag)$ the following
  pushout diagram in $\P$:
  \[\reflect(\Ddiag)=\xymatrix{\reflect(C) \ar^{\reflect(g)}[r]
    \ar_{\reflect(f)}[d] & \reflect(B) \\ \reflect(A) & }\]

  We can also reflect cocones. If $D:\P$ and $c=(i,j,h):\cocone{\Ddiag}{D}$ we
  define
  \[\reflect(c)=(\reflect(i),\reflect(j),\reflect(h)):
  \cocone{\reflect(\Ddiag)}{\reflect(D)}\]
  where
  \[\reflect(h):\prod_{c:\reflect(C)}\reflect(i)(\reflect(f)(c))=\reflect(j)(\reflect(g)(c))\]
  is defined in the following way:

  We have \[h:\prod_{c:C}i(f(c))=j(g(c))\]
  hence
  \[\funext(h):i\circ{}f=j\circ{}g\]
  We can apply $\reflect$ and we get
  \[\mapfunc{\reflect}(\funext(h)):\reflect(i\circ{}f)=\reflect(j\circ{}g)\]
  Now we can compose with the fact that $\reflect$ commutes with composition and
  we get the following (the proofs that $\reflect$ commutes with composition are
  not written in order to keep terms readable, and we will see that they don’t
  get in the way):
  \[\mapfunc{\reflect}(\funext(h)):
  \reflect(i)\circ\reflect(f)=\reflect(j)\circ\reflect(g)\]
  and then
  \[\reflect(h)\defeq\happly(\mapfunc{\reflect}(\funext(h))):
  \prod_{c:\reflect(C)}\reflect(i)(\reflect(f)(c))=\reflect(j)(\reflect(g)(c))\]
\end{defn}

\begin{lem}
  \label{reflectcommutespushout}
  Let $\Ddiag$ be a pushout diagram in $\type$ and $(D,c)$ a pushout of $\Ddiag$
  in $\type$. Then $(\reflect(D),\reflect(c))$ is a pushout of
  $\reflect(\Ddiag)$ in $\P$.
\end{lem}

\begin{proof}
  Let $E:\P$ and let’s consider the following diagram:

  \[\xymatrix{ (\reflect(D)\to E)
    \ar^{t\mapsto{}\composecocone{t}{\reflect(c)}}[r] \ar_{f_1}^\sim[d]
    &
    \cocone{\reflect(\Ddiag)}{E}\\
    (D\to E) \ar_{f_2}^\sim[d] &
    (\reflect(A)\to{}E)\times_{(\reflect(C)\to{}E)}(\reflect(B)\to{}E)
    \ar_{f_5}^\sim[u] \\
    \cocone{\Ddiag}{E}\ar_-{f_3}^-\sim[r] & (A\to{}E)\times_{(C\to{}E)}(B\to{}E)
    \ar_{f_4}^\sim[u] }\]

  We need to prove that the top arrow is an equivalence, we will do this by
  proving that it’s a composite of five maps which are already known to be
  equivalences.

  The first map comes from the universal property of $\reflect$, we have
  \[f_1(t)=t\circ\project_D\]

  The second map comes from the universal property of $(D,c)$, as a pushout of
  $\Ddiag$ in \type, we have
  \[f_2(u)=(u\circ{}i,u\circ{}j,\mapfunc{u}\circ{}h)\]

  The third map comes from \autoref{coneispb}, we have
  \[f_3(i,j,h)=(i,j,\funext(h))\]

  The fourth map comes from the universal property of $\reflect$ applied three
  times and the fact that pullbacks are invariant under equivalence (everything
  is invariant under equivalence anyway), we have
  \[f_4(i,j,p)=(\ext(i),\ext(j),\mapfunc{\ext}(p))\]

  Note that $\mapfunc{\ext}(p)$ has type $\ext(i\circ{}f)=\ext(j\circ{}g)$
  instead of $\ext(i)\circ\reflect(f)=\ext(j)\circ\reflect(g)$ but we
  (implicitely) concatenate with the proofs that
  $\ext(i\circ{}f)=\ext(i)\circ\reflect(f)$ and the same for $j$ and $g$. Again,
  we will see later that these proof do not get in the way.

  The fifth map comes from \autoref{coneispb}, we have
  \[f_5(a,b,q)=(a,b,\happly(q))\]

  Putting everything together we get
  \begin{align*}
    f_5(f_4(f_3(f_2(f_1(t))))) &= f_5(f_4(f_3(f_2(t\circ\project_D)))) \\
    &= f_5(f_4(f_3(t\circ\project_D\circ{}i,t\circ\project_D\circ{}j,
    \mapfunc{(t\circ\project_D)}\circ{}h))) \\
    &=f_5(f_4(t\circ\project_D\circ{}i,t\circ\project_D\circ{}j,
    \funext(\mapfunc{(t\circ\project_D)}\circ{}h))) \\
    &=f_5(\ext(t\circ\project_D\circ{}i),\ext(t\circ\project_D\circ{}j),
    \mapfunc{\ext}(\funext(\mapfunc{(t\circ\project_D)}\circ{}h))) \\
    &=f_5(\ext(t\circ\project_D)\circ\reflect(i),
    \ext(t\circ\project_D)\circ\reflect(j),
    \mapfunc{\ext}(\funext(\mapfunc{(t\circ\project_D)}\circ{}h))) \\
    &=f_5(t\circ\reflect(i),
    t\circ\reflect(j),
    \mapfunc{\ext}(\funext(\mapfunc{(t\circ\project_D)}\circ{}h))) \\
    &=(t\circ\reflect(i),t\circ\reflect(j),
    \happly(\mapfunc{\ext}(\funext(\mapfunc{(t\circ\project_D)}\circ{}h))))
  \end{align*}

  In order to prove that the diagram commutes, we now only need to prove that
  \begin{align*}
    \happly(\mapfunc{\ext}(\funext(\mapfunc{(t\circ\project_D)}\circ{}h))) &=
    \mapfunc{t}\circ\reflect(h)
  \end{align*}
  This is an equality in the type
  \[(c:\reflect(C))\to{}t(\reflect(i)(\reflect(f)(c)))=
  t(\reflect(j)(\reflect(g)(c)))\]
  This type is equal to
  \[t\circ\reflect(i)\circ\reflect(f) = t\circ\reflect(j)\circ\reflect(g)\]
  via $\funext$ so we only need to prove
  \begin{align*}
    \mapfunc{\ext}(\funext(\mapfunc{(t\circ\project_D)}\circ{}h)) &=
    \funext(\mapfunc{t}\circ\reflect(h)) \\
    &: t\circ\reflect(i)\circ\reflect(f) =
    t\circ\reflect(j)\circ\reflect(g)
  \end{align*}

  Note that for any appropriatedly typed $t$ and $p$ we have

  \[\funext(\mapfunc{t}\circ\happly(p))=\map{(\lambda{}u.\,t\circ{}u)}p\]

  (by induction on $p$)

  Hence the previous equality becomes the following, using the definition of
  $\reflect(h)$ and expanding $h$ to $\happly(\funext(h))$:

  \[\map{\ext}{\map{(\lambda{}u.\,t\circ\project_D\circ{}u)}{\funext(h)}}=
  \map{(\lambda{}u.\,t\circ{}u)}{\map{\reflect}{\funext(h)}}\]

  This is the point where we need to check that the implicit equalities around
  $\mapfunc{\ext}$ and $\mapfunc{\reflect}$ cancel (TODO, this seems true but
  awfully complicated)

  After that, using functoriality of $f\mapsto\mapfunc{f}$ it will be enough to
  prove the following:

  \[\ext\circ(\lambda{}u.\,t\circ\project_D\circ{}u)=
  (\lambda{}u.\,t\circ{}u)\circ\reflect\]

  We apply function extensionality, so for every $u:C\to{}D$ we have to prove

  \[\ext(t\circ\project_D\circ{}u) = t\circ\reflect(u) : \reflect(C)\to E\]

  But we have
  $\ext(t\circ\project_D\circ{}u)=\ext(t\circ\project_D)\circ\reflect(u)=t\circ\reflect(u)$
  hence the equality holds.

  This proves that the diagram commutes, hence the map
  $t\mapsto{}\composecocone{t}\reflect(c)$ is an equivalence which proves that
  the reflector commutes with pushouts.
\end{proof}

\begin{cor}
  Every pushout diagram $\Ddiag$ in $\P$ has a pushout in $\P$.
\end{cor}

\begin{proof}
  According to \autoref{reflectPequiv}, the diagram $\reflect(\Ddiag)$ is
  equivalent to $\Ddiag$. But we just proved that $\reflect(\Ddiag)$ has a
  pushout, namely the reflection of the pushout in \type of $\Ddiag$, hence
  $\Ddiag$ has a pushout in \P.
\end{proof}

\section{Truncations}

Given an integer $n\ge-2$, we recall that $\typele{n}$ is the type of all
$n$-truncated types (in the universe $\type$). For instance $\typele{-2}$ is
contractible (there is only one $(-2)$-truncated type, the point), $\typele{-1}$
is also known as $\prop$, and $\typele{0}$ is also known as $\set$.

Given a type $A:\type$ and an integer $n\ge-2$ we define an $n$-truncated type
$\tau_nA:\typele{n}$, called the \emph{$n$-truncation} of $A$, by a
\emph{truncated inductive type}:
\begin{align*}
  \tau_nA&:\typele{n}\\
  &\coloneqq\ |\ \tproj_n:A\to\tau_nA
\end{align*}

This means that we have
\[\tau_nA:\typele{n}\]
\[\tproj_n:A\to\tau_nA\]
and for every dependent type $P$ over $\tau_nA$ and $d:\prod_{a:A}P(\tproj_na)$,
we can define a section $\extend{d}$ of $P$ by $\extend{d}(\tproj_n(a))=d(a)$.

The nondependent elimination rule says that if $E$ is $n$-truncated, then every
map $f:A\to{}E$ can be extended to a map $\extend{f}:\tau_nA\to{}E$ defined by
$\extend{f}(\tproj_n(a))=f(a)$.

The nondependent $\eta$-rule says that if two maps $g,g':\tau_nA\to{}E$ are
such that $g(\tproj_n(a))=g'(\tproj_n(a))$ for every $a:A$, then $g(x)=g'(x)$
for all $x:\tau_nA$ (and then $g=g'$ using function extensionality).

\begin{lem}
  [Universal property of truncations]

  Let $n\ge-2$, $A:\type$ and $B:\typele{n}$. The following map is an
  equivalence:
  \[\function{(\tau_nA\to{}B)}{A\to{}B}{g}{g\circ\tproj_n}\]

  In particular, $\typele{n}$ is a reflective subuniverse of $\type$.
\end{lem}

\begin{proof}
  Any $f:A\to{}B$ can be extended to a map $\extend{f}:\tau_nA\to{}B$ (because
  $B$ is assumed to be $n$-truncated).

  The map $\extend{f}\circ\tproj_n$ is equal to $f$ because for every $a:A$ we
  have $\extend{f}(\tproj_n(a))=f(a)$ by definition and the map
  $\extend{g\circ\tproj_n}$ is equal to $g$ because they both send $\tproj_n(a)$
  to $g(\tproj_n(a))$.
\end{proof}

\begin{defn}
  A type $A$ is called \emph{$n$-connected} if $\tau_nA$ is contractible.
\end{defn}

\begin{lem}
  \label{connectedtotruncated}
  Let $n\ge-2$, $A:\type$ and $B:\typele{n}$ and assume moreover that $A$ is
  $n$-connected.

  Then every map from $A$ to $B$ is constant in the sense that the constant map
  operation
  \[\function{B}{(A\to{}B)}{b}{\lambda{}x^A.b}\]
  is an equivalence.
\end{lem}
\begin{proof}
  This is just an application of the universal property. Note that
  \[(\tau_nA\to{}B)=(\unit\to{}B)=B\]
\end{proof}

\begin{lem}
  Let $k,n\ge-2$ with $k\le{}n$ and $A:\type$. Then
  $\tau_k(\tau_nA)=\tau_kA$.
\end{lem}
\begin{proof}
  We define two maps $f:\tau_k(\tau_nA)\to\tau_kA$ and
  $g:\tau_kA\to\tau_k(\tau_nA)$ in the following way:

  \[f(\tproj_k(\tproj_n(a)))=\tproj_k(a)\]
  \[g(\tproj_k(a))=\tproj_k(\tproj_n(a))\]

  The map $f$ is well-defined because $\tau_kA$ is $k$-truncated and also
  $n$-truncated (because $k\le{}n$), and the map $g$ is well-defined because
  $\tau_k(\tau_nA)$ is $k$-truncated.

  The composition $f\circ{}g:\tau_kA\to\tau_kA$ satisfy
  $(f\circ{}g)(\tproj_k(a))=\tproj_k(a)$ hence $f\circ{}g=\idfunc[\tau_kA]$, and
  we also have $g\circ{}f=\idfunc[\tau_k(\tau_nA)]$ in the same way.
\end{proof}

\begin{lem}
  We have $\tau_n(\unit)=\unit$.
\end{lem}
\begin{proof}
  Indeed, $\unit$ is $n$-truncated for every $n$ hence $\tau_n(\unit)=\unit$ by
  \autoref{reflectPequiv}.
\end{proof}

\section{Connectedness of suspensions}

The aim of this section is to prove that the operation of suspension increases
connectedness.

\begin{thm}
  Let $n:\Nt$ and $A:\type$.

  If $A$ is $n$-connected then the suspension of $A$ is $(n+1)$-connected.
\end{thm}

\begin{proof}
  By definition, the suspension of $A$ is $\unit\sqcup^A\unit$, so we need to
  prove that the following type is contractible:

  \[\tau_{n+1}(\unit\sqcup^A\unit)\]

  By \autoref{reflectcommutespushout} we know that
  $\tau_{n+1}(\unit\sqcup^A\unit)$ is a pushout in $\typele{n+1}$ of the diagram
  \[\xymatrix{\tau_{n+1}(A) \ar[d] \ar[r] & \tau_{n+1}(\unit) \\
    \tau_{n+1}(\unit) & }\]

  Given that $\tau_{n+1}(\unit)=\unit$, the type
  $\tau_{n+1}(\unit\sqcup^A\unit)$ is also a pushout of the following diagram in
  $\typele{n+1}$ (because both diagrams are equal)
  \[\Ddiag=\xymatrix{\tau_{n+1}(A) \ar[d] \ar[r] & \unit \\
    \unit & }\]

  We will now prove that $\unit$ is also a pushout of $\Ddiag$ in
  $\typele{n+1}$.

  \bigskip

  Let $E$ be an $(n+1)$-truncated type, we need to prove that the following map
  is an equivalence
  \[\function{(\unit\to{}E)}{\cocone{\Ddiag}{E}}{y}
  {(y,y,\lambda{}u^{\tau_{n+1}A}.\refl{y(\ttt)})}\]

  where we recall that $\cocone{\Ddiag}{E}$ is the type
  \[\sum_{f:\unit\to{}E}\sum_{g:\unit\to{}E}(\tau_{n+1}A\to
  (f(\ttt)=_E{}g(\ttt)))\]

  The map $\function{(\unit\to{}E)}{E}{f}{f(\ttt)}$ is an equivalence, hence
  we also have
  \[\cocone{\Ddiag}{E}=\sum_{x:E}\sum_{y:E}(\tau_{n+1}A\to(x=_Ey))\]

  Now $A$ is $n$-connected hence so is $\tau_{n+1}A$ because
  $\tau_n(\tau_{n+1}A)=\tau_nA=\unit$, and $(x=_Ey)$ is $n$-truncated because
  $E$ is $(n+1)$-connected. Hence by \autoref{connectedtotruncated} the
  following map is an equivalence
  \[\function{(x=_Ey)}{(\tau_{n+1}A\to(x=_Ey))}{p}{\lambda{}z.p}\]

  Hence we have
  \[\cocone{\Ddiag}{E}=\sum_{x:E}\sum_{y:E}(x=_Ey)\]

  But the following map is an equivalence
  \[\function{E}{\sum_{x:E}\sum_{y:E}(x=_Ey)}{x}{(x,x,\refl{x})}\]

  Hence
  \[\cocone{\Ddiag}{E}=E\]

  Finally we get an equivalence
  \[(\unit\to{}E)\simeq\cocone{\Ddiag}{E}\]

  We can now unfold the definitions in order to get the explicit expression of
  this map, and we see easily that this is exactly the map we had at the
  beginning.

  \bigskip

  Hence we proved that $\unit$ in a pushout of $\Ddiag$ in $\typele{n+1}$. Using
  uniqueness of pushouts we get that $\tau_{n+1}(\unit\sqcup^A\unit)=\unit$
  which proves that the suspension of $A$ is $(n+1)$-connected.
\end{proof}

\begin{cor}
  For all $n:\N$, the sphere $\Sn^n$ is $(n-1)$-connected.
\end{cor}

\begin{proof}
  We prove this by induction on $n$.

  For $n=0$ we have to prove that $\Sn^0$ is inhabited which is clear.

  Let $n:\N$ such that $\Sn^n$ is $(n-1)$-connected. By definition $\Sn^{n+1}$
  is the suspension of $\Sn^n$ hence by the previous lemma $\Sn^{n+1}$ is
  $n$-connected.
\end{proof}

\section{Homotopy groups}

\begin{defn}
  A \emph{pointed type} $(A,a)$ is a type $A:\type$ together with a point $a:A$.
\end{defn}

\begin{defn}
  Given $(A,a)$ a pointed type, we define the \emph{loop space} of $(A,a)$ which
  is the following pointed type:
  \[\Omega(A,a)=((a=_Aa),\refl{A}(a))\]

  For $n:\N$, the \emph{iterated loop space} of a pointed type $(A,a)$ is
  defined by:
  \begin{align*}
    \Omega^0(A,a)&=(A,a)\\
    \Omega^{n+1}(A,a)&=\Omega^n(\Omega(A,a))
  \end{align*}
\end{defn}

\begin{defn}
  Given $n:\N$ and $(A,a)$ a pointed type, we define the homotopy groups of $A$
  at $a$ by
  \[\pi_n(A,a)=\tau_0(\Omega^n(A,a))\]
\end{defn}

\begin{lem}
  Let $(A,a)$ be a pointed type and $n:\N$.

  Then if $n>0$ the set $\pi_n(A,a)$ has a group structure and if $n>1$ it is
  even an abelian group.
\end{lem}

\begin{proof}
  TODO
\end{proof}

We can now say something about homotopy groups of $n$-truncated and
$n$-connected types.

\begin{lem}
  If $A$ is $n$-truncated and $a:A$, then we have
  \[\forall k>n,\ \pi_k(A,a)=\unit\]
\end{lem}

\begin{proof}
  Indeed, taking the loop space of an $n$-truncated type gives an
  $(n-1)$-truncated type, hence $\Omega^k(A,a)$ is $(n-k)$-truncated and we have
  $(n-k)\le-1$ so $\Omega^k(A,a)$ is \anhprop. But $\Omega^k(A,a)$ is inhabited,
  so it is actually contractible and
  $\pi_k(A,a)=\tau_0(\Omega^k(A,a))=\tau_0(\unit)=\unit$.
\end{proof}

In order to compute the homotopy groups of $n$-connected types we will need the
following lemma about commutation of truncation and path spaces.

\begin{lem}
  Let $n\ge-2$, $A:\type$ and $x,y:A$. There is a canonical map
  \[f:\tau_n(x=_Ay)\to\tproj_{n+1}x=_{\tau_{n+1}A}\tproj_{n+1}y\]
  defined by
  \[f(\tproj_n(\refl{A}(x)))=\refl{\tau_{n+1}A}(\tproj_{n+1}(x))\]
  This definition is correct because $\tau_{n+1}A$ is $(n+1)$-truncated hence
  the right hand side of the arrow is $n$-truncated.

  The lemma says that this map is an equivalence.
\end{lem}

\begin{proof}
  We cannot directly define an inverse to $f$ because there is no way to induct
  on an equality between $\tproj_{n+1}x$ and $\tproj_{n+1}y$. We will instead
  generalize the type of $f$ in order to have general elements of the type
  $\tau_{n+1}A$ instead of $\tproj_{n+1}x$ and $\tproj_{n+1}y$.

  Let $P:\tau_{n+1}A\to\tau_{n+1}A\to\typele{n}$ defined by
  \[P(\tproj_{n+1}(x),\tproj_{n+1}(y))=\tau_n(x=_Ay)\]

  This definition is correct because $\tau_n(x=_Ay)$ is $n$-truncated and
  $\typele{n}$ is $(n+1)$-truncated (here we are using the univalence axiom in a
  very nontrivial way).

  We can now generalize the result we want to prove. For every $u,v:\tau_{n+1}A$
  there is a map
  \[f:P(u,v)\to{}u=_{\tau_{n+1}A}v\]
  defined for $u=\tproj_{n+1}x$ and $v=\tproj_{n+1}y$ by
  \[f(\tproj_n(\refl{A}(x)))=\refl{\tau_{n+1}A}(\tproj_{n+1}(x))\]

  The type of $f$ being $n$-truncated, we are allowed to define it only for $u$
  and $v$ of this form, and then it’s just the same definition as before.

  But now we can define an inverse map
  \[g:u=_{\tau_{n+1}A}v\to{}P(u,v)\]
  by
  \[g(\refl{\tau_{n+1}A}(\tproj_{n+1}(x))) = \tproj_n(\refl{A}(x))\]

  because $u$ and $v$ are general elements of $\tau_{n+1}A$, and the output type
  is $n$-truncated so we can moreover assume that $u$ and $v$ of the form
  $\tproj_{n+1}(x)$.

  It is clear that $g$ and $f$ are inverse to each other, which prove that they
  are both equivalences. The result stated in the lemma is then the special case
  where $u=\tproj_{n+1}x$ and $v=\tproj_{n+1}y$.
\end{proof}

\begin{cor}
  Let $n\ge-2$ and $(A,a)$ be a pointed type. Then
  \[\tau_n(\Omega(A,a))=\Omega(\tau_{n+1}(A,a))\]
\end{cor}
\begin{proof}
  This is a special case of the previous lemma where $x=y=a$.
\end{proof}

\begin{lem}
  If $A$ is $n$-connected and $a:A$, then we have
  \[\forall k\le{}n,\ \pi_k(A,a)=\unit\]
\end{lem}

\begin{proof}
  We have the following sequence of equalities:

  \begin{align*}
    \pi_k(A,a) &= \tau_0(\Omega^k(A,a)) \\
    &= \Omega^k(\tau_k(A,a)) \\
    &= \Omega^k(\tau_k(\tau_n(A,a))) \\
    &= \Omega^k(\tau_k(\unit)) \\
    &= \Omega^k(\unit) \\
    &= \unit
  \end{align*}

  The third line uses the fact that $k\le{}n$ in order to use that
  $\tau_k\circ\tau_n=\tau_k$ and the fourth line uses the fact that $A$ is
  $n$-connected.
\end{proof}

%%% Local Variables:
%%% mode: latex
%%% TeX-master: "main"
%%% End:


\chapter{Category theory}
\label{cha:category-theory}

Of the branches of mathematics, category theory is one which perhaps fits the least comfortably in set theoretic foundations.
One problem is that most of category theory is invariant under weaker notions of ``sameness'' than equality, such as isomorphism in a category or equivalence of categories, in a way which set theory fails to capture.
But this is the same sort of problem that the univalence axiom solves for types, by identifying equality with equivalence.
Thus, in univalent foundations it makes sense to consider a notion of ``category'' in which equality of objects is identified with isomorphism in a similar way.

Ignoring size issues, in set-based mathematics a category consists of a \emph{set} $A_0$ of objects and, for each $x,y\in A_0$, a \emph{set} $\hom_A(x,y)$ of morphisms.
Under univalent foundations, a ``naive'' definition of category would simply mimic this with a \emph{type} of objects and \emph{types} of morphisms.
If we allowed these types to contain arbitrary higher homotopy, then we ought to impose higher coherence conditions, leading to some notion of $(\infty,1)$-category,
\index{.infinity1-category@$(\infty,1)$-category}%
but at present our goal is more modest.
We consider only 1-categories, and therefore we restrict the types $\hom_A(x,y)$ to be sets, i.e.\ 0-types.
If we impose no further conditions, we will call this notion a \emph{precategory}.

If we add the requirement that the type $A_0$ of objects is a set, then we end up with a definition that behaves much like the traditional set-theoretic one.
Following Toby Bartels, we call this notion a \emph{strict category}.
\index{strict!category}%
But we can also require a generalized version of the univalence axiom, identifying $(x=_{A_0} y)$ with the type $\mathsf{iso}(x,y)$ of isomorphisms from $x$ to $y$.
Since we regard this as usually the ``correct'' definition, we will call it simply a \emph{category}.

A good example of the difference between the three notions of category is provided by the statement ``every fully faithful and essentially surjective functor is an equivalence of categories'', which in classical set-based category theory is equivalent to the axiom of choice.
\index{mathematics!classical}%
\index{axiom!of choice}%
\index{classical!category theory}%
\begin{enumerate}
\item For strict categories, this is still equivalent to the axiom of choice.
\item For precategories, there is no consistent axiom of choice which can make it true.
\item For categories, it is provable \emph{without} any axiom of choice.
\end{enumerate}
We will prove the latter statement in this chapter, as well as other pleasant properties of categories, e.g.\ that equivalent categories are equal (as elements of the type of categories).
We will also describe a universal way of ``saturating'' a precategory $A$ into a category $\widehat A$, which we call its \emph{Rezk completion},
\index{completion!Rezk}%
although it could also reasonably be called the \emph{stack completion} (see the Notes).

The Rezk completion also sheds further light on the notion of equivalence of categories.
For instance, the functor $A \to \widehat{A}$ is always fully faithful and essentially surjective, hence a ``weak equivalence''.
It follows that a precategory is a category exactly when it ``sees'' all fully faithful and essentially surjective functors as equivalences; thus our notion of ``category'' is already inherent in the notion of ``fully faithful and essentially surjective functor''.

We assume the reader has some basic familiarity with classical category theory.\index{classical!category theory}
Recall that whenever we write \type it denotes some universe of types, but perhaps a different one at different times; everything we say remains true for any consistent choice of universe levels\index{universe level}.
We will use the basic notions of homotopy type theory from \cref{cha:typetheory,cha:basics} and the propositional truncation from \cref{cha:logic}, but not much else from \cref{part:foundations}, except that our second construction of the Rezk completion will use a higher inductive type.


\section{Categories and precategories}
\label{sec:cats}

In classical mathematics, there are many equivalent definitions of a category.
In our case, since we have dependent types, it is natural to choose the arrows to be a type family indexed by the objects.
This matches the way hom-types are always used in category theory: we never even consider comparing two arrows unless we know their domains and codomains agree.
Furthermore, it seems clear that for a theory of 1-categories, the hom-types should all be sets.
This leads us to the following definition.

\begin{defn}\label{ct:precategory}
  A \define{precategory}
  \indexdef{precategory}
  $A$ consists of the following.
  \begin{enumerate}
  \item A type $A_0$ of \define{objects}.%
    \indexdef{object!in a (pre)category}
    We write $a:A$ for $a:A_0$.
  \item For each $a,b:A$, a set $\hom_A(a,b)$ of \define{arrows} or \define{morphisms}.%
    \indexsee{arrow}{morphism}%
    \indexdef{morphism!in a (pre)category}%
    \indexdef{hom-set}%
  \item For each $a:A$, a morphism $1_a:\hom_A(a,a)$.%
    \indexdef{identity!morphism in a (pre)category}
  \item For each $a,b,c:A$, a function%
    \indexdef{composition!of morphisms in a (pre)category}
    \[  \hom_A(b,c) \to \hom_A(a,b) \to \hom_A(a,c) \]
    denoted infix by $g\mapsto f\mapsto g\circ f$, or sometimes simply by $gf$.
  \item For each $a,b:A$ and $f:\hom_A(a,b)$, we have $\id f {1_b\circ f}$ and $\id f {f\circ 1_a}$.
  \item For each $a,b,c,d:A$ and
    \begin{equation*}
      f:\hom_A(a,b), \qquad
      g:\hom_A(b,c), \qquad
      h:\hom_A(c,d),
    \end{equation*}
    we have $\id {h\circ (g\circ f)}{(h\circ g)\circ f}$.
  \end{enumerate}
\end{defn}

The problem with the notion of precategory is that for objects $a,b:A$, we have two possibly-different notions of ``sameness''.
On the one hand, we have the type $(\id[A_0]{a}{b})$.
But on the other hand, there is the standard categorical notion of \emph{isomorphism}.

\begin{defn}\label{ct:isomorphism}
  A morphism $f:\hom_A(a,b)$ is an \define{isomorphism}
  \indexdef{isomorphism!in a (pre)category}%
  if there is a morphism $g:\hom_A(b,a)$ such that $\id{g\circ f}{1_a}$ and $\id{f\circ g}{1_b}$.
  We write $a\cong b$ for the type of such isomorphisms.
\end{defn}

\begin{lem}\label{ct:isoprop}
  For any $f:\hom_A(a,b)$, the type ``$f$ is an isomorphism'' is a mere proposition.
  Therefore, for any $a,b:A$ the type $a\cong b$ is a set.
\end{lem}
\begin{proof}
  Suppose given $g:\hom_A(b,a)$ and $\eta:(\id{1_a}{g\circ f})$ and $\epsilon:(\id{f\circ g}{1_b})$, and similarly $g'$, $\eta'$, and $\epsilon'$.
We must show $\id{(g,\eta,\epsilon)}{(g',\eta',\epsilon')}$.
  But since all hom-sets are sets, their identity types are mere propositions, so it suffices to show $\id g {g'}$.
  For this we have
  \[g' = 1_a\circ g' = (g\circ f)\circ g' = g\circ (f\circ g') = g\circ 1_b = g\]
  using $\eta$ and $\epsilon'$.
\end{proof}

\symlabel{ct:inv}
\index{inverse!in a (pre)category}%
If $f:a\cong b$, then we write $\inv f$ for its inverse, which by \cref{ct:isoprop} is uniquely determined.

The only relationship between these two notions of sameness that we have in a precategory is the following.

\begin{lem}[\textsf{idtoiso}]\label{ct:idtoiso}
  If $A$ is a precategory and $a,b:A$, then
  \[(\id a b)\to (a \cong b).\]
\end{lem}
\begin{proof}
  By induction on identity, we may assume $a$ and $b$ are the same.
  But then we have $1_a:\hom_A(a,a)$, which is clearly an isomorphism.
\end{proof}

Evidently, this situation is analogous to the issue that motivated us to introduce the univalence axiom.
In fact, we have the following:

\begin{eg}\label{ct:precatset}
  \index{set}%
  There is a precategory \uset, whose type of objects is \set, and with $\hom_{\uset}(A,B) \defeq (A\to B)$.
  The identity morphisms are identity functions and the composition is function composition.
  For this precategory, \cref{ct:idtoiso} is equal to (the restriction to sets of) the map $\idtoeqv$ from \cref{sec:compute-universe}.

  Of course, to be more precise we should call this category $\uset_\UU$, since its objects are only the \emph{small sets}
  \index{small!set}%
  relative to a universe \UU.
\end{eg}

Thus, it is natural to make the following definition.

\begin{defn}\label{ct:category}
  A \define{category}
  \indexdef{category}
  is a precategory such that for all $a,b:A$, the function $\idtoiso_{a,b}$ from \cref{ct:idtoiso} is an equivalence.
\end{defn}

In particular, in a category, if $a\cong b$, then $a=b$.

\begin{eg}\label{ct:eg:set}
  \index{univalence axiom}%
  The univalence axiom implies immediately that \uset is a category.
  One can also show, using univalence, that any precategory of set-level structures such as groups, rings, topological spaces, etc.\ is a category; see \cref{sec:sip}.
\end{eg}

We also note the following.

\begin{lem}\label{ct:obj-1type}
  In a category, the type of objects is a 1-type.
\end{lem}
\begin{proof}
  It suffices to show that for any $a,b:A$, the type $\id a b$ is a set.
  But $\id a b$ is equivalent to $a \cong b$, which is a set.
\end{proof}

\symlabel{isotoid}
We write $\isotoid$ for the inverse $(a\cong b) \to (\id a b)$ of the map $\idtoiso$ from \cref{ct:idtoiso}.
The following relationship between the two is important.

\begin{lem}\label{ct:idtoiso-trans}
  For $p:\id a a'$ and $q:\id b b'$ and $f:\hom_A(a,b)$, we have
  \begin{equation}\label{ct:idtoisocompute}
    \id{\trans{(p,q)}{f}}
    {\idtoiso(q)\circ f \circ \inv{\idtoiso(p)}}.
  \end{equation}
\end{lem}
\begin{proof}
  By induction, we may assume $p$ and $q$ are $\refl a$ and $\refl b$ respectively.
Then the left-hand side of~\eqref{ct:idtoisocompute} is simply $f$.
  But by definition, $\idtoiso(\refl a)$ is $1_a$, and $\idtoiso(\refl b)$ is $1_b$, so the right-hand side of~\eqref{ct:idtoisocompute} is $1_b\circ f\circ 1_a$, which is equal to $f$.
\end{proof}

Similarly, we can show
\begin{gather}
  \id{\idtoiso(\rev p)}{\inv {(\idtoiso(p))}}\\
  \id{\idtoiso(p\ct q)}{\idtoiso(q)\circ \idtoiso(p)}\\
  \id{\isotoid(f\circ e)}{\isotoid(e)\ct \isotoid(f)}
\end{gather}
and so on.

\begin{eg}\label{ct:orders}
  A precategory in which each set $\hom_A(a,b)$ is a mere proposition is equivalently a type $A_0$ equipped with a mere relation ``$\le$'' that is reflexive ($a\le a$) and transitive (if $a\le b$ and $b\le c$, then $a\le c$).
  We call this a \define{preorder}.
  \indexdef{preorder}

  In a preorder, a witness $f: a\le b$ is an isomorphism just when there exists some witness $g: b\le a$.
  Thus, $a\cong b$ is the mere proposition that $a\le b$ and $b\le a$.
  Therefore, a preorder $A$ is a category just when (1) each type $a=b$ is a mere proposition, and (2) for any $a,b:A_0$ there exists a function $(a\cong b) \to (a=b)$.
  In other words, $A_0$ must be a set, and $\le$ must be antisymmetric\index{relation!antisymmetric} (if $a\le b$ and $b\le a$, then $a=b$).
  We call this a \define{(partial) order} or a \define{poset}.
  \indexdef{partial order}%
  \indexdef{poset}%
\end{eg}

\begin{eg}\label{ct:gaunt}
  If $A$ is a category, then $A_0$ is a set if and only if for any $a,b:A_0$, the type $a\cong b$ is a mere proposition.
  This is equivalent to saying that every isomorphism in $A$ is an identity; thus it is rather stronger than the classical\index{mathematics!classical} notion of ``skeletal'' category.
  Categories of this sort are sometimes called \define{gaunt}~\cite{bsp12infncats}.
  \indexdef{category!gaunt}%
  \indexdef{gaunt category}%
  \index{skeletal category}%
  \index{category!skeletal}%
  There is not really any notion of ``skeletality'' for our categories, unless one considers \cref{ct:category} itself to be such.
\end{eg}

\begin{eg}\label{ct:discrete}
  For any 1-type $X$, there is a category with $X$ as its type of objects and with $\hom(x,y) \defeq (x=y)$.
  If $X$ is a set, we call this the \define{discrete}
  \indexdef{category!discrete}%
  \indexdef{discrete!category}%
  category on $X$.
  In general, we call it a \define{groupoid}
  \indexdef{groupoid}
  (see \cref{ct:groupoids}).
\end{eg}

\begin{eg}\label{ct:fundgpd}
  For \emph{any} type $X$, there is a precategory with $X$ as its type of objects and with $\hom(x,y) \defeq \pizero{x=y}$.
  The composition operation
  \[ \pizero{y=z} \to \pizero{x=y} \to \pizero{x=z} \]
  is defined by induction on truncation from concatenation $(y=z)\to(x=y)\to(x=z)$.
  We call this the \define{fundamental pregroupoid}
  \indexdef{fundamental!pregroupoid}%
  \indexsee{pregroupoid, fundamental}{fundamental pregroupoid}%
  of $X$.
  (In fact, we have met it already in \cref{sec:van-kampen}; see also \cref{ex:rezk-vankampen}.)
\end{eg}

\begin{eg}\label{ct:hoprecat}
  There is a precategory whose type of objects is \type and with $\hom(X,Y) \defeq \pizero{X\to Y}$, and composition defined by induction on truncation from ordinary composition $(Y\to Z) \to (X\to Y) \to (X\to Z)$.
  We call this the \define{homotopy precategory of types}.
  \indexdef{precategory!of types}%
  \index{homotopy!category of types@(pre)category of types}%
\end{eg}

\begin{eg}\label{ct:rel}
  Let \urel be the following precategory:
  \begin{itemize}
  \item Its objects are sets.
  \item $\hom_{\urel}(X,Y) = X\to Y\to \prop$.
  \item For a set $X$, we have $1_X(x,x') \defeq (x=x')$.
  \item For $R:\hom_{\urel}(X,Y)$ and $S:\hom_{\urel}(Y,Z)$, their composite is defined by
    \[ (S\circ R)(x,z) \defeq \Brck{\sm{y:Y} R(x,y) \times S(y,z)}.\]
  \end{itemize}
  Suppose $R:\hom_{\urel}(X,Y)$ is an isomorphism, with inverse $S$.
  We observe the following.
  \begin{enumerate}
  \item If $R(x,y)$ and $S(y',x)$, then $(R\circ S)(y',y)$, and hence $y'=y$.
    Similarly, if $R(x,y)$ and $S(y,x')$, then $x=x'$.\label{item:rel1}
  \item For any $x$, we have $x=x$, hence $(S\circ R)(x,x)$.
    Thus, there merely exists a $y:Y$ such that $R(x,y)$ and $S(y,x)$.\label{item:rel2}
  \item Suppose $R(x,y)$.
    By~\ref{item:rel2}, there merely exists a $y'$ with $R(x,y')$ and $S(y',x)$.
    But then by~\ref{item:rel1}, merely $y'=y$, and hence $y'=y$ since $Y$ is a set.
    Therefore, by transporting $S(y',x)$ along this equality, we have $S(y,x)$.
    In conclusion, $R(x,y)\to S(y,x)$.
    Similarly, $S(y,x) \to R(x,y)$.\label{item:rel3}
  \item If $R(x,y)$ and $R(x,y')$, then by~\ref{item:rel3}, $S(y',x)$, so that by~\ref{item:rel1}, $y=y'$.
    Thus, for any $x$ there is at most one $y$ such that $R(x,y)$.
    And by~\ref{item:rel2}, there merely exists such a $y$, hence there exists such a $y$.
  \end{enumerate}
  In conclusion, if $R:\hom_{\urel}(X,Y)$ is an isomorphism, then for each $x:X$ there is exactly one $y:Y$ such that $R(x,y)$, and dually.
  Thus, there is a function $f:X\to Y$ sending each $x$ to this $y$, which is an equivalence; hence $X=Y$.
  With a little more work, we conclude that \urel is a category.
\end{eg}

We might now restrict ourselves to considering categories rather than precategories.
Instead, we will develop many concepts for precategories as well as categories, in order to emphasize how much better-behaved categories are, as compared both to precategories and to ordinary categories in classical\index{mathematics!classical} mathematics.

We will also see in \crefrange{sec:strict-categories}{sec:dagger-categories} that in slightly more exotic contexts, there are uses for certain kinds of precategories other than categories, each of which ``fixes'' the equality of objects in different ways.
This emphasizes the ``pre''-ness of precategories: they are the raw material out of which multiple important categorical structures can be defined.


\section{Functors and transformations}
\label{sec:transfors}

The following definitions are fairly obvious, and need no modification.

\begin{defn}\label{ct:functor}
  Let $A$ and $B$ be precategories.
  A \define{functor}
  \indexdef{functor}%
  $F:A\to B$ consists of
  \begin{enumerate}
  \item A function $F_0:A_0\to B_0$, generally also denoted $F$.
  \item For each $a,b:A$, a function $F_{a,b}:\hom_A(a,b) \to \hom_B(Fa,Fb)$, generally also denoted $F$.
  \item For each $a:A$, we have $\id{F(1_a)}{1_{Fa}}$.
  \item For each $a,b,c:A$ and $f:\hom_A(a,b)$ and $g:\hom_B(b,c)$, we have
    \[\id{F(g\circ f)}{Fg\circ Ff}.\]
  \end{enumerate}
\end{defn}

Note that by induction on identity, a functor also preserves \idtoiso.

\begin{defn}\label{ct:nattrans}
  For functors $F,G:A\to B$, a \define{natural transformation}
  \indexdef{natural!transformation}%
  \indexsee{transformation!natural}{natural transformation}%
  $\gamma:F\to G$ consists of
  \begin{enumerate}
  \item For each $a:A$, a morphism $\gamma_a:\hom_B(Fa,Ga)$ (the ``components'').
  \item For each $a,b:A$ and $f:\hom_A(a,b)$, we have $\id{Gf\circ \gamma_a}{\gamma_b\circ Ff}$ (the ``naturality axiom'').
  \end{enumerate}
\end{defn}

Since each type $\hom_B(Fa,Gb)$ is a set, its identity type is a mere proposition.
Thus, the naturality axiom is a mere proposition, so identity of natural transformations is determined by identity of their components.
In particular, for any $F$ and $G$, the type of natural transformations from $F$ to $G$ is again a set.

Similarly, identity of functors is determined by identity of the functions $A_0\to B_0$ and (transported along this) of the corresponding functions on hom-sets.

\begin{defn}\label{ct:functor-precat}
  \indexdef{precategory!of functors}%
  For precategories $A,B$, there is a precategory $B^A$ defined by
  \begin{itemize}
  \item $(B^A)_0$ is the type of functors from $A$ to $B$.
  \item $\hom_{B^A}(F,G)$ is the type of natural transformations from $F$ to $G$.
  \end{itemize}
\end{defn}
\begin{proof}
  We define $(1_F)_a\defeq 1_{Fa}$.
  Naturality follows by the unit axioms of a precategory.
  For $\gamma:F\to G$ and $\delta:G\to H$, we define $(\delta\circ\gamma)_a\defeq \delta_a\circ \gamma_a$.
  Naturality follows by associativity.
  Similarly, the unit and associativity laws for $B^A$ follow from those for $B$.
\end{proof}

\begin{lem}\label{ct:natiso}
  \index{natural!isomorphism}%
  \index{isomorphism!natural}%
  A natural transformation $\gamma:F\to G$ is an isomorphism in $B^A$ if and only if each $\gamma_a$ is an isomorphism in $B$.
\end{lem}
\begin{proof}
  If $\gamma$ is an isomorphism, then we have $\delta:G\to F$ that is its inverse.
  By definition of composition in $B^A$, $(\delta\gamma)_a\jdeq \delta_a\gamma_a$ and similarly.
  Thus, $\id{\delta\gamma}{1_F}$ and $\id{\gamma\delta}{1_G}$ imply $\id{\delta_a\gamma_a}{1_{Fa}}$ and $\id{\gamma_a\delta_a}{1_{Ga}}$, so $\gamma_a$ is an isomorphism.

  Conversely, suppose each $\gamma_a$ is an isomorphism, with inverse called $\delta_a$, say.
We define a natural transformation $\delta:G\to F$ with components $\delta_a$; for the naturality axiom we have
  \[ Ff\circ \delta_a = \delta_b\circ \gamma_b\circ Ff \circ \delta_a = \delta_b\circ Gf\circ \gamma_a\circ \delta_a = \delta_b\circ Gf. \]
  Now since composition and identity of natural transformations is determined on their components, we have $\id{\gamma\delta}{1_G}$ and $\id{\delta\gamma}{1_F}$.
\end{proof}

The following result is fundamental.

\begin{thm}\label{ct:functor-cat}
  \indexdef{category!of functors}%
  \indexdef{functor!category of}%
  If $A$ is a precategory and $B$ is a category, then $B^A$ is a category.
\end{thm}
\begin{proof}
  Let $F,G:A\to B$; we must show that $\idtoiso:(\id{F}{G}) \to (F\cong G)$ is an equivalence.

  To give an inverse to it, suppose $\gamma:F\cong G$ is a natural isomorphism.
  Then for any $a:A$, we have an isomorphism $\gamma_a:Fa \cong Ga$, hence an identity $\isotoid(\gamma_a):\id{Fa}{Ga}$.
  By function extensionality, we have an identity $\bar{\gamma}:\id[(A_0\to B_0)]{F_0}{G_0}$.

  Now since the last two axioms of a functor are mere propositions, to show that $\id{F}{G}$ it will suffice to show that for any $a,b:A$, the functions
  \begin{align*}
    F_{a,b}&:\hom_A(a,b) \to \hom_B(Fa,Fb)\mathrlap{\qquad\text{and}}\\
    G_{a,b}&:\hom_A(a,b) \to \hom_B(Ga,Gb)
  \end{align*}
  become equal when transported along $\bar\gamma$.
  By computation for function extensionality, when applied to $a$, $\bar\gamma$ becomes equal to $\isotoid(\gamma_a)$.
  But by \cref{ct:idtoiso-trans}, transporting $Ff:\hom_B(Fa,Fb)$ along $\isotoid(\gamma_a)$ and $\isotoid(\gamma_b)$ is equal to the composite $\gamma_b\circ Ff\circ \inv{(\gamma_a)}$, which by naturality of $\gamma$ is equal to $Gf$.

  This completes the definition of a function $(F\cong G) \to (\id F G)$.
  Now consider the composite
  \[ (\id F G) \to (F\cong G) \to (\id F G). \]
  Since hom-sets are sets, their identity types are mere propositions, so to show that two identities $p,q:\id F G$ are equal, it suffices to show that $\id[\id{F_0}{G_0}]{p}{q}$.
  But in the definition of $\bar\gamma$, if $\gamma$ were of the form $\idtoiso(p)$, then $\gamma_a$ would be equal to $\idtoiso(p_a)$ (this can easily be proved by induction on $p$).
  Thus, $\isotoid(\gamma_a)$ would be equal to $p_a$, and so by function extensionality we would have $\id{\bar\gamma}{p}$, which is what we need.

  Finally, consider the composite
  \[(F\cong G)\to  (\id F G) \to (F\cong G). \]
  Since identity of natural transformations can be tested componentwise, it suffices to show that for each $a$ we have $\id{\idtoiso(\bar\gamma)_a}{\gamma_a}$.
  But as observed above, we have $\id{\idtoiso(\bar\gamma)_a}{\idtoiso((\bar\gamma)_a)}$, while $\id{(\bar\gamma)_a}{\isotoid(\gamma_a)}$ by computation for function extensionality.
  Since $\isotoid$ and $\idtoiso$ are inverses, we have $\id{\idtoiso(\bar\gamma)_a}{\gamma_a}$ as desired.
\end{proof}

In particular, naturally isomorphic functors between categories (as opposed to precategories) are equal.

\mentalpause

We now define all the usual ways to compose functors and natural transformations.

\begin{defn}\label{ct:functor-composition}
  For functors $F:A\to B$ and $G:B\to C$, their composite $G\circ F:A\to C$ is given by
  \begin{itemize}
  \item The composite $(G_0\circ F_0) : A_0 \to C_0$
  \item For each $a,b:A$, the composite
    \[(G_{Fa,Fb}\circ F_{a,b}):\hom_A(a,b) \to \hom_C(GFa,GFb).\]
  \end{itemize}
  It is easy to check the axioms.
\end{defn}

\begin{defn}\label{ct:whisker}
  For functors $F:A\to B$ and $G,H:B\to C$ and a natural transformation $\gamma:G\to H$, the composite $(\gamma F):GF\to HF$ is given by
  \begin{itemize}
  \item For each $a:A$, the component $\gamma_{Fa}$.
  \end{itemize}
  Naturality is easy to check.
  Similarly, for $\gamma$ as above and $K:C\to D$, the composite $(K\gamma):KG\to KH$ is given by
  \begin{itemize}
  \item For each $b:B$, the component $K(\gamma_b)$.
  \end{itemize}
\end{defn}

\begin{lem}\label{ct:interchange}
  \index{interchange law}%
  For functors $F,G:A\to B$ and $H,K:B\to C$ and natural transformations $\gamma:F\to G$ and $\delta:H\to K$, we have
  \[\id{(\delta G)(H\gamma)}{(K\gamma)(\delta F)}.\]
\end{lem}
\begin{proof}
  It suffices to check componentwise: at $a:A$ we have
  \begin{align*}
    ((\delta G)(H\gamma))_a
    &\jdeq (\delta G)_{a}(H\gamma)_a\\
    &\jdeq \delta_{Ga}\circ H(\gamma_a)\\
    &= K(\gamma_a) \circ \delta_{Fa} \tag{by naturality of $\delta$}\\
    &\jdeq (K \gamma)_a\circ (\delta F)_a\\
    &\jdeq ((K \gamma)(\delta F))_a.\qedhere
  \end{align*}
\end{proof}

\index{horizontal composition!of natural transformations}%
\index{classical!category theory}%
Classically, one defines the ``horizontal composite'' of $\gamma:F\to G$ and $\delta:H\to K$ to be the common value of ${(\delta G)(H\gamma)}$ and ${(K\gamma)(\delta F)}$.
We will refrain from doing this, because while equal, these two transformations are not \emph{definitionally} equal.
This also has the consequence that we can use the symbol $\circ$ (or juxtaposition) for all kinds of composition unambiguously: there is only one way to compose two natural transformations (as opposed to composing a natural transformation with a functor on either side).

\begin{lem}\label{ct:functor-assoc}
  \index{associativity!of functor composition}
  Composition of functors is associative: $\id{H(GF)}{(HG)F}$.
\end{lem}
\begin{proof}
  Since composition of functions is associative, this follows immediately for the actions on objects and on homs.
  And since hom-sets are sets, the rest of the data is automatic.
\end{proof}

The equality in \cref{ct:functor-assoc} is likewise not definitional.
(Composition of functions is definitionally associative, but the axioms that go into a functor must also be composed, and this breaks definitional associativity.)  For this reason, we need also to know about \emph{coherence}\index{coherence} for associativity.

\begin{lem}\label{ct:pentagon}
  \index{associativity!of functor composition!coherence of}%
  \cref{ct:functor-assoc} is coherent, i.e.\ the following pentagon\index{pentagon, Mac Lane} of equalities commutes:
  \[ \xymatrix{ & K(H(GF)) \ar@{=}[dl] \ar@{=}[dr]\\
    (KH)(GF) \ar@{=}[d] && K((HG)F) \ar@{=}[d]\\
    ((KH)G)F && (K(HG))F \ar@{=}[ll] }
  \]
\end{lem}
\begin{proof}
  As in \cref{ct:functor-assoc}, this is evident for the actions on objects, and the rest is automatic.
\end{proof}

We will henceforth abuse notation by writing $H\circ G\circ F$ or $HGF$ for either $H(GF)$ or $(HG)F$, transporting along \cref{ct:functor-assoc} whenever necessary.
We have a similar coherence result for units.

\begin{lem}\label{ct:units}
  For a functor $F:A\to B$, we have equalities $\id{(1_B\circ F)}{F}$ and $\id{(F\circ 1_A)}{F}$, such that given also $G:B\to C$, the following triangle of equalities commutes.
  \[ \xymatrix{
    G\circ (1_B \circ F) \ar@{=}[rr] \ar@{=}[dr] &&
    (G\circ 1_B)\circ F \ar@{=}[dl] \\
    & G \circ F.}
  \]
\end{lem}

See \cref{ct:pre2cat,ct:2cat} for further development of these ideas.


\section{Adjunctions}
\label{sec:adjunctions}

The definition of adjoint functors is straightforward; the main interesting aspect arises from proof-relevance.

\begin{defn}\label{ct:adjoints}
  A functor $F:A\to B$ is a \define{left adjoint}
  \indexdef{left!adjoint}%
  \indexdef{adjoint!functor}%
  \indexdef{right!adjoint}%
  \indexdef{adjoint!functor}%
  \index{functor!adjoint}%
  if there exists
  \begin{itemize}
  \item A functor $G:B\to A$.
  \item A natural transformation $\eta:1_A \to GF$ (the \define{unit}\indexdef{unit!of an adjunction}).
  \item A natural transformation $\epsilon:FG\to 1_B$ (the \define{counit}\indexdef{counit of an adjunction}).
  \item $\id{(\epsilon F)(F\eta)}{1_F}$.
  \item $\id{(G\epsilon)(\eta G)}{1_G}$.
  \end{itemize}
\end{defn}

The last two equations are called the \define{triangle identities}\indexdef{triangle!identity} or \define{zigzag identities}\indexdef{zigzag identity}.
\indexdef{identity!triangle}\indexdef{identity!zigzag}
We leave it to the reader to define right adjoints analogously.

\begin{lem}\label{ct:adjprop}
  If $A$ is a category (but $B$ may be only a precategory), then the type ``$F$ is a left adjoint'' is a mere proposition.
\end{lem}
\begin{proof}
  Suppose we are given $(G,\eta,\epsilon)$ with the triangle identities and also $(G',\eta',\epsilon')$.
  Define $\gamma:G\to G'$ to be $(G'\epsilon)(\eta' G)$, and $\delta:G'\to G$ to be $(G\epsilon')(\eta G')$.
  Then
  \begin{align*}
    \delta\gamma &=
    (G\epsilon')(\eta G')(G'\epsilon)(\eta'G)\\
    &= (G\epsilon')(G F G'\epsilon)(\eta G' F G)(\eta'G)\\
    &= (G\epsilon)(G\epsilon'FG)(G F \eta' G)(\eta G)\\
    &= (G\epsilon)(\eta G)\\
    &= 1_G
  \end{align*}
  using \cref{ct:interchange} and the triangle identities.
  Similarly, we show $\id{\gamma\delta}{1_{G'}}$, so $\gamma$ is a natural isomorphism $G\cong G'$.
  By \cref{ct:functor-cat}, we have an identity $\id G {G'}$.

  Now we need to know that when $\eta$ and $\epsilon$ are transported along this identity, they become equal to $\eta'$ and $\epsilon'$.
  By \cref{ct:idtoiso-trans}, this transport is given by composing with $\gamma$ or $\delta$ as appropriate.
  For $\eta$, this yields
  \begin{equation*}
    (G'\epsilon F)(\eta'GF)\eta
    = (G'\epsilon F)(G'F\eta)\eta'
    = \eta'
  \end{equation*}
  using \cref{ct:interchange} and the triangle identity.
  The case of $\epsilon$ is similar.
  Finally, the triangle identities transport correctly automatically, since hom-sets are sets.
\end{proof}

In \cref{sec:yoneda} we will give another proof of \cref{ct:adjprop}.


\section{Equivalences}
\label{sec:equivalences}

It is usual in category theory to define an \emph{equivalence of categories} to be a functor $F:A\to B$ such that there exists a functor $G:B\to A$ and natural isomorphisms $F G \cong 1_B$ and $G F \cong 1_A$.
Unlike the property of being an adjunction, however, this would not be a mere proposition without truncating it, for the same reasons that the type of quasi-inverses is ill-behaved (see \cref{sec:quasi-inverses}).
And as in \cref{sec:hae}, we can avoid this by using the usual notion of \emph{adjoint} equivalence.
\indexdef{adjoint!equivalence!of (pre)categories}

\begin{defn}\label{ct:equiv}
  A functor $F:A\to B$ is an \define{equivalence of (pre)categories}
  \indexdef{equivalence!of (pre)categories}%
  \indexdef{category!equivalence of}%
  \indexdef{precategory!equivalence of}%
  \index{functor!equivalence}%
  if it is a left adjoint for which $\eta$ and $\epsilon$ are isomorphisms.
  We write $\cteqv A B$ for the type of equivalences of categories from $A$ to $B$.
\end{defn}

By \cref{ct:adjprop,ct:isoprop}, if $A$ is a category, then the type ``$F$ is an equivalence of precategories'' is a mere proposition.

\begin{lem}\label{ct:adjointification}
  If for $F:A\to B$ there exists $G:B\to A$ and isomorphisms $GF\cong 1_A$ and $FG\cong 1_B$, then $F$ is an equivalence of precategories.
\end{lem}
\begin{proof}
  Just like the proof of \cref{thm:equiv-iso-adj} for equivalences of types.
\end{proof}

\begin{defn}\label{ct:full-faithful}
  We say a functor $F:A\to B$ is \define{faithful}
  \indexdef{functor!faithful}%
  \index{faithful functor}%a
  if for all $a,b:A$, the function
  \[F_{a,b}:\hom_A(a,b) \to \hom_B(Fa,Fb)\]
  is injective, and \define{full}
  \indexdef{functor!full}%
  \indexdef{full functor}%
  if for all $a,b:A$ this function is surjective.
  If it is both (hence each $F_{a,b}$ is an equivalence) we say $F$ is \define{fully faithful}.
  \indexdef{functor!fully faithful}%
  \indexdef{fully faithful functor}%
\end{defn}

\begin{defn}\label{ct:split-essentially-surjective}
  We say a functor $F:A\to B$ is \define{split essentially surjective}
  \indexdef{functor!split essentially surjective}%
  \indexdef{split!essentially surjective functor}%
  if for all $b:B$ there exists an $a:A$ such that $Fa\cong b$.
\end{defn}

\begin{lem}\label{ct:ffeso}
  For any precategories $A$ and $B$ and functor $F:A\to B$, the following types are equivalent.
  \begin{enumerate}
  \item $F$ is an equivalence of precategories.\label{item:ct:ffeso1}
  \item $F$ is fully faithful and split essentially surjective.\label{item:ct:ffeso2}
  \end{enumerate}
\end{lem}
\begin{proof}
  Suppose $F$ is an equivalence of precategories, with $G,\eta,\epsilon$ specified.
  Then we have the function
  \begin{align*}
      \hom_B(Fa,Fb) &\to \hom_A(a,b),\\
      g &\mapsto \inv{\eta_b}\circ G(g)\circ \eta_a.
  \end{align*}
  For $f:\hom_A(a,b)$, we have
  \[ \inv{\eta_{b}}\circ G(F(f))\circ \eta_{a}  =
  \inv{\eta_{b}} \circ \eta_{b} \circ f=
  f
  \]
  while for $g:\hom_B(Fa,Fb)$ we have
  \begin{align*}
    F(\inv{\eta_b} \circ G(g)\circ\eta_a)
    &= F(\inv{\eta_b})\circ F(G(g))\circ F(\eta_a)\\
    &= \epsilon_{Fb}\circ F(G(g))\circ F(\eta_a)\\
    &= g\circ\epsilon_{Fa}\circ F(\eta_a)\\
    &= g
  \end{align*}
  using naturality of $\epsilon$, and the triangle identities twice.
  Thus, $F_{a,b}$ is an equivalence, so $F$ is fully faithful.
  Finally, for any $b:B$, we have $Gb:A$ and $\epsilon_b:FGb\cong b$.

  On the other hand, suppose $F$ is fully faithful and split essentially surjective.
  Define $G_0:B_0\to A_0$ by sending $b:B$ to the $a:A$ given by the specified essential splitting, and write $\epsilon_b$ for the likewise specified isomorphism $FGb\cong b$.

  Now for any $g:\hom_B(b,b')$, define $G(g):\hom_A(Gb,Gb')$ to be the unique morphism such that $\id{F(G(g))}{\inv{(\epsilon_{b'})}\circ g \circ \epsilon_b }$ (which exists since $F$ is fully faithful).
  Finally, for $a:A$ define $\eta_a:\hom_A(a,GFa)$ to be the unique morphism such that $\id{F\eta_a}{\inv{\epsilon_{Fa}}}$.
  It is easy to verify that $G$ is a functor and that $(G,\eta,\epsilon)$ exhibit $F$ as an equivalence of precategories.

  Now consider the composite~\ref{item:ct:ffeso1}$\to$\ref{item:ct:ffeso2}$\to$\ref{item:ct:ffeso1}.
  We clearly recover the same function $G_0:B_0 \to A_0$.
  For the action of $G$ on hom-sets, we must show that for $g:\hom_B(b,b')$, $G(g)$ is the (necessarily unique) morphism such that $F(G(g)) = \inv{(\epsilon_{b'})}\circ g \circ \epsilon_b$.
  But this equation holds by the assumed naturality of $\epsilon$.
  We also clearly recover $\epsilon$, while $\eta$ is uniquely characterized by $\id{F\eta_a}{\inv{\epsilon_{Fa}}}$ (which is one of the triangle identities assumed to hold in the structure of an equivalence of precategories).
  Thus, this composite is equal to the identity.

  Finally, consider the other composite~\ref{item:ct:ffeso2}$\to$\ref{item:ct:ffeso1}$\to$\ref{item:ct:ffeso2}.
  Since being fully faithful is a mere proposition, it suffices to observe that we recover, for each $b:B$, the same $a:A$ and isomorphism $F a \cong b$.
  But this is clear, since we used this function and isomorphism to define $G_0$ and $\epsilon$ in~\ref{item:ct:ffeso1}, which in turn are precisely what we used to recover~\ref{item:ct:ffeso2} again.
  Thus, the composites in both directions are equal to identities, hence we have an equivalence \eqv{\text{\ref{item:ct:ffeso1}}}{\text{\ref{item:ct:ffeso2}}}.
\end{proof}

However, if $B$ is not a category, then neither type in \cref{ct:ffeso} may necessarily be a mere proposition.
This suggests considering as well the following notions.

\begin{defn}\label{ct:essentially-surjective}
  A functor $F:A\to B$ is \define{essentially surjective}
  \indexdef{functor!essentially surjective}%
  \indexdef{essentially surjective functor}%
  if for all $b:B$, there \emph{merely} exists an $a:A$ such that $Fa\cong b$.
  We say $F$ is a \define{weak equivalence}
  \indexsee{equivalence!of (pre)categories!weak}{weak equivalence}%
  \indexdef{weak equivalence!of precategories}%
  \indexsee{functor!weak equivalence}{weak equivalence}%
  if it is fully faithful and essentially surjective.
\end{defn}

Being a weak equivalence is \emph{always} a mere proposition.
For categories, however, there is no difference between equivalences and weak ones.

\index{acceptance}
\begin{lem}\label{ct:catweq}
  If $F:A\to B$ is fully faithful and $A$ is a category, then for any $b:B$ the type $\sm{a:A} (Fa\cong b)$ is a mere proposition.
  Hence a functor between categories is an equivalence if and only if it is a weak equivalence.
\end{lem}
\begin{proof}
  Suppose given $(a,f)$ and $(a',f')$ in $\sm{a:A} (Fa\cong b)$.
  Then $\inv{f'}\circ f$ is an isomorphism $Fa \cong Fa'$.
  Since $F$ is fully faithful, we have $g:a\cong a'$ with $Fg = \inv{f'}\circ f$.
  And since $A$ is a category, we have $p:a=a'$ with $\idtoiso(p)=g$.
  Now $Fg = \inv{f'}\circ f$ implies $\trans{(\map{(F_0)}{p})}{f} = f'$, hence (by the characterization of equalities in dependent pair types) $(a,f)=(a',f')$.

  Thus, for fully faithful functors whose domain is a category, essential surjectivity is equivalent to split essential surjectivity, and so being a weak equivalence is equivalent to being an equivalence.
\end{proof}

This is an important advantage of our category theory over set-based approaches.
With a purely set-based definition of category, the statement ``every fully faithful and essentially surjective functor is an equivalence of categories'' is equivalent to the axiom of choice \choice{}.
Here we have it for free, as a category-theoretic version of the principle of unique choice (\cref{sec:unique-choice}).
(In fact, this property characterizes categories among precategories; see \cref{sec:rezk}.)

On the other hand, the following characterization of equivalences of categories is perhaps even more useful.

\begin{defn}\label{ct:isocat}
  A functor $F:A\to B$ is an \define{isomorphism of (pre)cat\-ego\-ries}
  \indexdef{isomorphism!of (pre)categories}%
  \indexdef{category!isomorphism of}%
  \indexdef{precategory!isomorphism of}%
  if $F$ is fully faithful and $F_0:A_0\to B_0$ is an equivalence of types.
\end{defn}

This definition is an exception to our general rule (see \cref{sec:basics-equivalences}) of only using the word ``isomorphism'' for sets and set-like objects.
However, it does carry an appropriate connotation here, because for general precategories, isomorphism is stronger than equivalence.

Note that being an isomorphism of precategories is always a mere property.
Let $A\cong B$ denote the type of isomorphisms of (pre)categories from $A$ to $B$.

\begin{lem}\label{ct:isoprecat}
  For precategories $A$ and $B$ and $F:A\to B$, the following are equivalent.
  \begin{enumerate}
  \item $F$ is an isomorphism of precategories.\label{item:ct:ipc1}
  \item There exist $G:B\to A$ and $\eta:1_A = GF$ and $\epsilon:FG=1_B$ such that\label{item:ct:ipc2}
    \begin{equation}
      \apfunc{(\lam{H} F H)}({\eta}) = \apfunc{(\lam{K} K F)}({\opp\epsilon}).\label{eq:ct:isoprecattri}
    \end{equation}
  \item There merely exist $G:B\to A$ and $\eta:1_A = GF$ and $\epsilon:FG=1_B$.\label{item:ct:ipc3}
  \end{enumerate}
\end{lem}

Note that if $B_0$ is not a 1-type, then~\eqref{eq:ct:isoprecattri} may not be a mere proposition.

\begin{proof}
  First note that since hom-sets are sets, equalities between equalities of functors are uniquely determined by their object-parts.
  Thus, by function extensionality,~\eqref{eq:ct:isoprecattri} is equivalent to
  \begin{equation}
    \map{(F_0)}{\eta_0}_a = \opp{(\epsilon_0)}_{F_0 a}.\label{eq:ct:ipctri}
  \end{equation}
  for all $a:A_0$.
  Note that this is precisely the triangle identity for $G_0$, $\eta_0$, and $\epsilon_0$ to be a proof that $F_0$ is a half adjoint equivalence of types.

  Now suppose~\ref{item:ct:ipc1}.
  Let $G_0:B_0 \to A_0$ be the inverse of $F_0$, with $\eta_0: \idfunc[A_0] = G_0 F_0$ and $\epsilon_0:F_0G_0 = \idfunc[B_0]$ satisfying the triangle identity, which is precisely~\eqref{eq:ct:ipctri}.
  Now define $G_{b,b'}:\hom_B(b,b') \to \hom_A(G_0b,G_0b')$ by
  \[ G_{b,b'}(g) \defeq
  \inv{(F_{G_0b,G_0b'})}\Big(\idtoiso(\opp{(\epsilon_0)}_{b'}) \circ g \circ \idtoiso((\epsilon_0)_b)\Big)
  \]
  (using the assumption that $F$ is fully faithful).
  Since \idtoiso takes inverses to inverses and concatenation to composition, and $F$ is a functor, it follows that $G$ is a functor.

  By definition, we have $(GF)_0 \jdeq G_0 F_0$, which is equal to $\idfunc[A_0]$ by $\eta_0$.
  To obtain $1_A = GF$, we need to show that when transported along $\eta_0$, the identity function of $\hom_A(a,a')$ becomes equal to the composite $G_{Fa,Fa'} \circ F_{a,a'}$.
  In other words, for any $f:\hom_A(a,a')$ we must have
  \begin{multline*}
    \idtoiso((\eta_0)_{a'}) \circ f \circ \idtoiso(\opp{(\eta_0)}_a)\\
    = \inv{(F_{GFa,GFa'})}\Big(\idtoiso(\opp{(\epsilon_0)}_{Fa'})
    \circ F_{a,a'}(f) \circ \idtoiso((\epsilon_0)_{Fa})\Big).
  \end{multline*}
  But this is equivalent to
  \begin{multline*}
    (F_{GFa,GFa'})\Big(\idtoiso((\eta_0)_{a'}) \circ f \circ \idtoiso(\opp{(\eta_0)}_a)\Big)\\
    = \idtoiso(\opp{(\epsilon_0)}_{Fa'})
    \circ F_{a,a'}(f) \circ \idtoiso((\epsilon_0)_{Fa}).
  \end{multline*}
  which follows from functoriality of $F$, the fact that $F$ preserves \idtoiso, and~\eqref{eq:ct:ipctri}.
  Thus we have $\eta:1_A = GF$.

  On the other side, we have $(FG)_0\jdeq F_0 G_0$, which is equal to $\idfunc[B_0]$ by $\epsilon_0$.
  To obtain $FG=1_B$, we need to show that when transported along $\epsilon_0$, the identity function of $\hom_B(b,b')$ becomes equal to the composite $F_{Gb,Gb'} \circ G_{b,b'}$.
  That is, for any $g:\hom_B(b,b')$ we must have
  \begin{multline*}
    F_{Gb,Gb'}\Big(\inv{(F_{Gb,Gb'})}\Big(\idtoiso(\opp{(\epsilon_0)}_{b'}) \circ g \circ \idtoiso((\epsilon_0)_b)\Big)\Big)\\
    = \idtoiso((\opp{\epsilon_0})_{b'}) \circ g \circ \idtoiso((\epsilon_0)_b).
  \end{multline*}
  But this is just the fact that $\inv{(F_{Gb,Gb'})}$ is the inverse of $F_{Gb,Gb'}$.
  And we have remarked that~\eqref{eq:ct:isoprecattri} is equivalent to~\eqref{eq:ct:ipctri}, so~\ref{item:ct:ipc2} holds.

  Conversely, suppose given~\ref{item:ct:ipc2}; then the object-parts of $G$, $\eta$, and $\epsilon$ together with~\eqref{eq:ct:ipctri} show that $F_0$ is an equivalence of types.
  And for $a,a':A_0$, we define $\overline{G}_{a,a'}: \hom_B(Fa,Fa') \to \hom_A(a,a')$ by
  \begin{equation}
    \overline{G}_{a,a'}(g) \defeq \idtoiso(\opp{\eta})_{a'} \circ G(g) \circ \idtoiso(\eta)_a.\label{eq:ct:gbar}
  \end{equation}
  By naturality of $\idtoiso(\eta)$, for any $f:\hom_A(a,a')$ we have
  \begin{align*}
    \overline{G}_{a,a'}(F_{a,a'}(f))
    &= \idtoiso(\opp{\eta})_{a'} \circ G(F(f)) \circ \idtoiso(\eta)_a\\
    &= \idtoiso(\opp{\eta})_{a'} \circ \idtoiso(\eta)_{a'} \circ f \\
    &= f.
  \end{align*}
  On the other hand, for $g:\hom_B(Fa,Fa')$ we have
  \begin{align*}
    F_{a,a'}(\overline{G}_{a,a'}(g))
    &= F(\idtoiso(\opp{\eta})_{a'}) \circ F(G(g)) \circ F(\idtoiso(\eta)_a)\\
    &= \idtoiso(\epsilon)_{Fa'}
    \circ F(G(g))
    \circ \idtoiso(\opp{\epsilon})_{Fa}\\
    &= \idtoiso(\epsilon)_{Fa'}
    \circ \idtoiso(\opp{\epsilon})_{Fa'}
    \circ g\\
    &= g.
  \end{align*}
  (There are lemmas needed here regarding the compatibility of \idtoiso and whiskering, which we leave it to the reader to state and prove.)
  Thus, $F_{a,a'}$ is an equivalence, so $F$ is fully faithful; i.e.~\ref{item:ct:ipc1} holds.

  Now the composite~\ref{item:ct:ipc1}$\to$\ref{item:ct:ipc2}$\to$\ref{item:ct:ipc1} is equal to the identity since~\ref{item:ct:ipc1} is a mere proposition.
  On the other side, tracing through the above constructions we see that the composite~\ref{item:ct:ipc2}$\to$\ref{item:ct:ipc1}$\to$\ref{item:ct:ipc2} essentially preserves the object-parts $G_0$, $\eta_0$, $\epsilon_0$, and the object-part of~\eqref{eq:ct:isoprecattri}.
  And in the latter three cases, the object-part is all there is, since hom-sets are sets.

  Thus, it suffices to show that we recover the action of $G$ on hom-sets.
  In other words, we must show that if $g:\hom_B(b,b')$, then
  \[ G_{b,b'}(g) =
  \overline{G}_{G_0b,G_0b'}\Big(\idtoiso(\opp{(\epsilon_0)}_{b'}) \circ g \circ \idtoiso((\epsilon_0)_b)\Big)
  \]
  where $\overline{G}$ is defined by~\eqref{eq:ct:gbar}.
  However, this follows from functoriality of $G$ and the \emph{other} triangle identity, which we have seen in \cref{cha:equivalences} is equivalent to~\eqref{eq:ct:ipctri}.

  Now since~\ref{item:ct:ipc1} is a mere proposition, so is~\ref{item:ct:ipc2}, so it suffices to show they are logically equivalent to~\ref{item:ct:ipc3}.
  Of course,~\ref{item:ct:ipc2}$\to$\ref{item:ct:ipc3}, so let us assume~\ref{item:ct:ipc3}.
  Since~\ref{item:ct:ipc1} is a mere proposition, we may assume given $G$, $\eta$, and $\epsilon$.
  Then $G_0$ along with $\eta$ and $\epsilon$ imply that $F_0$ is an equivalence.
  Moreover, we also have natural isomorphisms $\idtoiso(\eta):1_A\cong GF$ and $\idtoiso(\epsilon):FG\cong 1_B$, so by \cref{ct:adjointification}, $F$ is an equivalence of precategories, and in particular fully faithful.
\end{proof}

From \cref{ct:isoprecat}\ref{item:ct:ipc2} and $\idtoiso$ in functor categories, we conclude immediately that any isomorphism of precategories is an equivalence.
For precategories, the converse can fail.

\begin{eg}\label{ct:chaotic}
  Let $X$ be a type and $x_0:X$ an element, and let $X_{\mathrm{ch}}$ denote the \emph{chaotic}\indexdef{chaotic precategory} or \emph{indiscrete}\indexdef{indiscrete precategory} precategory on $X$.
  By definition, we have $(X_{\mathrm{ch}})_0\defeq X$, and $\hom_{X_{\mathrm{ch}}}(x,x') \defeq \unit$ for all $x,x'$.
  Then the unique functor $X_{\mathrm{ch}}\to \unit$ is an equivalence of precategories, but not an isomorphism unless $X$ is contractible.

  This example also shows that a precategory can be equivalent to a category without itself being a category.
  Of course, if a precategory is \emph{isomorphic} to a category, then it must itself be a category.
\end{eg}

However, for categories, the two notions coincide.

\begin{lem}\label{ct:eqv-levelwise}
  For categories $A$ and $B$, a functor $F:A\to B$ is an equivalence of categories if and only if it is an isomorphism of categories.
\end{lem}
\begin{proof}
  Since both are mere properties, it suffices to show they are logically equivalent.
  So first suppose $F$ is an equivalence of categories, with $(G,\eta,\epsilon)$ given.
  We have already seen that $F$ is fully faithful.
  By \cref{ct:functor-cat}, the natural isomorphisms $\eta$ and $\epsilon$ yield identities $\id{1_A}{GF}$ and $\id{FG}{1_B}$, hence in particular identities $\id{\idfunc[A]}{G_0\circ F_0}$ and $\id{F_0\circ G_0}{\idfunc[B]}$.
Thus, $F_0$ is an equivalence of types.

  Conversely, suppose $F$ is fully faithful and $F_0$ is an equivalence of types, with inverse $G_0$, say.
  Then for each $b:B$ we have $G_0 b:A$ and an identity $\id{FGb}{b}$, hence an isomorphism $FGb\cong b$.
  Thus, by \cref{ct:ffeso}, $F$ is an equivalence of categories.
\end{proof}

Of course, there is yet a third notion of sameness for (pre)categories: equality.
However, the univalence axiom implies that it coincides with isomorphism.

\begin{lem}\label{ct:cat-eq-iso}
  If $A$ and $B$ are precategories, then the function
  \[(\id A B) \to (A\cong B)\]
  (defined by induction from the identity functor) is an equivalence of types.
\end{lem}
\begin{proof}
  As usual for dependent sum types, to give an element of $\id A B$ is equivalent to giving
  \begin{itemize}
  \item an identity $P_0:\id{A_0}{B_0}$,
  \item for each $a,b:A_0$, an identity
    \[P_{a,b}:\id{\hom_A(a,b)}{\hom_B(\trans {P_0} a,\trans {P_0} b)},\]
  \item identities $\id{\trans {(P_{a,a})} {1_a}}{1_{\trans {P_0} a}}$ and
    \narrowequation{\id{\trans {(P_{a,c})} {gf}}{\trans {(P_{b,c})} g \circ \trans {(P_{a,b})} f}.}
  \end{itemize}
  (Again, we use the fact that the identity types of hom-sets are mere propositions.)
  However, by univalence, this is equivalent to giving
  \begin{itemize}
  \item an equivalence of types $F_0:\eqv{A_0}{B_0}$,
  \item for each $a,b:A_0$, an equivalence of types
    \[F_{a,b}:\eqv{\hom_A(a,b)}{\hom_B(F_0 (a),F_0 (b))},\]
  \item and identities $\id{F_{a,a}(1_a)}{1_{F_0 (a)}}$ and $\id{F_{a,c}(gf)}{F_{b,c} (g)\circ F_{a,b} (f)}$.
  \end{itemize}
  But this consists exactly of a functor $F:A\to B$ that is an isomorphism of categories.
  And by induction on identity, this equivalence $\eqv{(\id A B)}{(A\cong B)}$ is equal to the one obtained by induction.
\end{proof}

Thus, for categories, equality also coincides with equivalence.
We can interpret this as saying that categories, functors, and natural transformations form, not just a pre-2-category, but a 2-category (see \cref{ct:pre2cat}).

\begin{thm}\label{ct:cat-2cat}
  If $A$ and $B$ are categories, then the function
  \[(\id A B) \to (\cteqv A B)\]
  (defined by induction from the identity functor) is an equivalence of types.
\end{thm}
\begin{proof}
  By \cref{ct:cat-eq-iso,ct:eqv-levelwise}.
\end{proof}

As a consequence, the type of categories is a 2-type.
For since $\cteqv A B$ is a subtype of the type of functors from $A$ to $B$, which are the objects of a category, it is a 1-type; hence the identity types $\id A B$ are also 1-types.


\section{The Yoneda lemma}
\label{sec:yoneda}
\index{Yoneda!lemma|(}

Recall that we have a category \uset whose objects are sets and whose morphisms are functions.
We now show that every precategory has a \uset-valued hom-functor.
First we need to define opposites and products of (pre)categories.

\begin{defn}\label{ct:opposite-category}
  For a precategory $A$, its \define{opposite}
  \indexdef{opposite of a (pre)category}%
  \indexdef{precategory!opposite}%
  \indexdef{category!opposite}%
  $A\op$ is a precategory with the same type of objects, with $\hom_{A\op}(a,b) \defeq \hom_A(b,a)$, and with identities and composition inherited from $A$.
\end{defn}

\begin{defn}\label{ct:prod-cat}
  For precategories $A$ and $B$, their \define{product}
  \index{precategory!product of}%
  \index{category!product of}%
  \index{product!of (pre)categories}%
  $A\times B$ is a precategory with $(A\times B)_0 \defeq A_0 \times B_0$ and
  \[\hom_{A\times B}((a,b),(a',b')) \defeq \hom_A(a,a') \times \hom_B(b,b').\]
  Identities are defined by $1_{(a,b)}\defeq (1_a,1_b)$ and composition by
  \narrowequation{(g,g')(f,f') \defeq ((gf),(g'f')).}
\end{defn}

\begin{lem}\label{ct:functorexpadj}
  For precategories $A,B,C$, the following types are equivalent.
  \begin{enumerate}
  \item Functors $A\times B\to C$.
  \item Functors $A\to C^B$.
  \end{enumerate}
\end{lem}
\begin{proof}
  Given $F:A\times B\to C$, for any $a:A$ we obviously have a functor $F_a : B\to C$.
  This gives a function $A_0 \to (C^B)_0$.
  Next, for any $f:\hom_A(a,a')$, we have for any $b:B$ the morphism $F_{(a,b),(a',b)}(f,1_b):F_a(b) \to F_{a'}(b)$.
  These are the components of a natural transformation $F_a \to F_{a'}$.
  Functoriality in $a$ is easy to check, so we have a functor $\hat{F}:A\to C^B$.

  Conversely, suppose given $G:A\to C^B$.
  Then for any $a:A$ and $b:B$ we have the object $G(a)(b):C$, giving a function $A_0 \times B_0 \to C_0$.
  And for $f:\hom_A(a,a')$ and $g:\hom_B(b,b')$, we have the morphism
  \begin{equation*}
     G(a')_{b,b'}(g)\circ G_{a,a'}(f)_b = G_{a,a'}(f)_{b'} \circ  G(a)_{b,b'}(g)
  \end{equation*}
  in $\hom_C(G(a)(b), G(a')(b'))$.
  Functoriality is again easy to check, so we have a functor $\check{G}:A\times B \to C$.

  Finally, it is also clear that these operations are inverses.
\end{proof}

Now for any precategory $A$, we have a hom-functor
\indexdef{hom-functor}%
\[\hom_A : A\op \times A \to \uset.\]
It takes a pair $(a,b): (A\op)_0 \times A_0 \jdeq A_0 \times A_0$ to the set $\hom_A(a,b)$.
For a morphism $(f,f') : \hom_{A\op\times A}((a,b),(a',b'))$, by definition we have $f:\hom_A(a',a)$ and $f':\hom_A(b,b')$, so we can define
\begin{align*}
  (\hom_A)_{(a,b),(a',b')}(f,f')
  &\defeq (g \mapsto (f'gf))\\
  &: \hom_A(a,b) \to \hom_A(a',b').
\end{align*}
Functoriality is easy to check.

By \cref{ct:functorexpadj}, therefore, we have an induced functor $\y:A\to \uset^{A\op}$, which we call the \define{Yoneda embedding}.
\indexdef{Yoneda!embedding}%
\indexdef{embedding!Yoneda}%

\begin{thm}[The Yoneda lemma]\label{ct:yoneda}
  \indexdef{Yoneda!lemma}
  For any precategory $A$, any $a:A$, and any functor $F:\uset^{A\op}$, we have an isomorphism
  \begin{equation}\label{eq:yoneda}
    \hom_{\uset^{A\op}}(\y a, F) \cong Fa.
  \end{equation}
  Moreover, this is natural in both $a$ and $F$.
\end{thm}
\begin{proof}
  Given a natural transformation $\alpha:\y a \to F$, we can consider the component $\alpha_a : \y a(a) \to F a$.
  Since $\y a(a)\jdeq \hom_A(a,a)$, we have $1_a : \y a(a)$, so that $\alpha_a(1_a) : F a$.
  This gives a function $(\alpha \mapsto \alpha_a(1_a))$ from left to right in~\eqref{eq:yoneda}.

  In the other direction, given $x:F a$, we define $\alpha:\y a \to F$ by
  \[\alpha_{a'}(f) \defeq F_{a',a}(f)(x). \]
  Naturality is easy to check, so this gives a function from right to left in~\eqref{eq:yoneda}.

  To show that these are inverses, first suppose given $x:F a$.
  Then with $\alpha$ defined as above, we have $\alpha_a(1_a) = F_{a,a}(1_a)(x) = 1_{F a}(x) = x$.
  On the other hand, if we suppose given $\alpha:\y a \to F$ and define $x$ as above, then for any $f:\hom_A(a',a)$ we have
  \begin{align*}
    \alpha_{a'}(f)
    &= \alpha_{a'} (\y a_{a',a}(f))\\
    &= (\alpha_{a'}\circ \y a_{a',a}(f))(1_a)\\
    &= (F_{a',a}(f)\circ \alpha_a)(1_a)\\
    &= F_{a',a}(f)(\alpha_a(1_a))\\
    &= F_{a',a}(f)(x).
  \end{align*}
  Thus, both composites are equal to identities.
  We leave the proof of naturality to the reader.
\end{proof}

\begin{cor}\label{ct:yoneda-embedding}
  The Yoneda embedding $\y :A\to \uset^{A\op}$ is fully faithful.
\end{cor}
\begin{proof}
  By \cref{ct:yoneda}, we have
  \[ \hom_{\uset^{A\op}}(\y a, \y b) \cong \y b(a) \jdeq \hom_A(a,b). \]
  It is easy to check that this isomorphism is in fact the action of \y on hom-sets.
\end{proof}

\begin{cor}\label{ct:yoneda-mono}
  If $A$ is a category, then $\y_0 : A_0 \to (\uset^{A\op})_0$ is an embedding.
  In particular, if $\y a = \y b$, then $a=b$.
\end{cor}
\begin{proof}
  By \cref{ct:yoneda-embedding}, \y induces an isomorphism on sets of isomorphisms.
  But as $A$ and $\uset^{A\op}$ are categories and \y is a functor, this is equivalently an isomorphism on identity types, which is the definition of being an embedding.
\end{proof}

\begin{defn}\label{ct:representable}
  A functor $F:\uset^{A\op}$ is said to be \define{representable}
  \indexdef{functor!representable}%
  \indexdef{representable functor}%
  if there exists $a:A$ and an isomorphism $\y a \cong F$.
\end{defn}

\begin{thm}\label{ct:representable-prop}
  If $A$ is a category, then the type ``$F$ is representable'' is a mere proposition.
\end{thm}
\begin{proof}
  By definition ``$F$ is representable'' is just the fiber of $\y_0$ over $F$.
  Since $\y_0$ is an embedding by \cref{ct:yoneda-mono}, this fiber is a mere proposition.
\end{proof}

In particular, in a category, any two representations of the same functor are equal.
We can use this to give a different proof of \cref{ct:adjprop}.
First we give a characterization of adjunctions in terms of representability.

\begin{lem}\label{ct:adj-repr}
  For any precategories $A$ and $B$ and a functor $F:A\to B$, the following types are equivalent.
  \begin{enumerate}
  \item $F$ is a left adjoint\index{adjoint!functor}.\label{item:ct:ar1}
  \item For each $b:B$, the functor $(a \mapsto \hom_B(Fa,b))$ from $A\op$ to \uset is representable\index{representable functor}.\label{item:ct:ar2}
  \end{enumerate}
\end{lem}
\begin{proof}
  An element of the type~\ref{item:ct:ar2} consists of a function $G_0:B_0 \to A_0$ together with, for every $a:A$ and $b:B$ an isomorphism
  \[ \gamma_{a,b}:\hom_B(Fa,b) \cong \hom_A(a,G_0 b) \]
  such that $\gamma_{a,b}(g \circ Ff) = \gamma_{a',b}(g)\circ f$ for $f:\hom_{A}(a,a')$.

  Given this, for $a:A$ we define $\eta_a \defeq \gamma_{a,Fa}(1_{Fa})$, and for $b:B$ we define $\epsilon_b \defeq \inv{(\gamma_{Gb,b})}(1_{Gb})$.
  Now for $g:\hom_B(b,b')$ we define
  \[ G_{b,b'}(g) \defeq \gamma_{G b, b'}(g \circ \epsilon_b) \]
  The verifications that $G$ is a functor and $\eta$ and $\epsilon$ are natural transformations satisfying the triangle identities are exactly as in the classical case, and as they are all mere propositions we will not care about their values.
  Thus, we have a function~\ref{item:ct:ar2}$\to$\ref{item:ct:ar1}.

  In the other direction, if $F$ is a left adjoint, we of course have $G_0$ specified, and we can take $\gamma_{a,b}$ to be the composite
  \[ \hom_B(Fa,b)
  \xrightarrow{G_{Fa,b}} \hom_A(GFa,Gb)
  \xrightarrow{(\blank\circ \eta_a)} \hom_A(a,Gb).
  \]
  This is clearly natural since $\eta$ is, and it has an inverse given by
  \[ \hom_A(a,Gb)
  \xrightarrow{F_{a,Gb}} \hom_B(Fa,FGb)
  \xrightarrow{(\epsilon_b \circ \blank )} \hom_A(Fa,b)
  \]
  (by the triangle identities).
  Thus we also have~\ref{item:ct:ar1}$\to$~\ref{item:ct:ar2}.

  For the composite~\ref{item:ct:ar2}$\to$\ref{item:ct:ar1}$\to$~\ref{item:ct:ar2}, clearly the function $G_0$ is preserved, so it suffices to check that we get back $\gamma$.
  But the new $\gamma$ is defined to take $f:\hom_B(Fa,b)$ to
  \begin{align*}
    G(f) \circ \eta_a
    &\jdeq \gamma_{G Fa, b}(f \circ \epsilon_{Fa}) \circ \eta_a\\
    &= \gamma_{G Fa, b}(f \circ \epsilon_{Fa} \circ F\eta_a)\\
    &= \gamma_{G Fa, b}(f)
  \end{align*}
  so it agrees with the old one.

  Finally, for~\ref{item:ct:ar1}$\to$\ref{item:ct:ar2}$\to$~\ref{item:ct:ar1}, we certainly get back the functor $G$ on objects.
  The new $G_{b,b'}:\hom_B(b,b') \to \hom_A(Gb,Gb')$ is defined to take $g$ to
  \begin{align*}
    \gamma_{G b, b'}(g \circ \epsilon_b)
    &\jdeq G(g \circ \epsilon_b) \circ \eta_{Gb}\\
    &= G(g) \circ G\epsilon_b \circ \eta_{Gb}\\
    &= G(g)
  \end{align*}
  so it agrees with the old one.
  The new $\eta_a$ is defined to be $\gamma_{a,Fa}(1_{Fa}) \jdeq G(1_{Fa}) \circ \eta_a$, so it equals the old $\eta_a$.
  And finally, the new $\epsilon_b$ is defined to be $\inv{(\gamma_{Gb,b})}(1_{Gb}) \jdeq \epsilon_b \circ F(1_{Gb})$, which also equals the old $\epsilon_b$.
\end{proof}

\begin{cor}\label{ct:adjprop2}[\cref{ct:adjprop}]
  If $A$ is a category and $F:A\to B$, then the type ``$F$ is a left adjoint'' is a mere proposition.
\end{cor}
\begin{proof}
  By \cref{ct:representable-prop}, if $A$ is a category then the type in \cref{ct:adj-repr}\ref{item:ct:ar2} is a mere proposition.
\end{proof}
\index{Yoneda!lemma|)}


\section{Strict categories}
\label{sec:strict-categories}

\index{bargaining|(}%

\begin{defn}\label{ct:strict-category}
  A \define{strict category}
  \indexdef{category!strict}%
  \indexdef{strict!category}%
  is a precategory whose type of objects is a set.
\end{defn}

In accordance with the mathematical red herring principle,\index{red herring principle} a strict category is not necessarily a category.
In fact, a category is a strict category precisely when it is gaunt (\cref{ct:gaunt}).
\index{gaunt category}%
\index{category!gaunt}%
Most of the time, category theory is about categories, not strict ones, but sometimes one wants to consider strict categories.
The main advantage of this is that strict categories have a stricter notion of ``sameness'' than equivalence, namely isomorphism (or equivalently, by \cref{ct:cat-eq-iso}, equality).

Here is one origin of strict categories.

\begin{eg}\label{ct:mono-cat}
  Let $A$ be a precategory and $x:A$ an object.
  Then there is a precategory $\mathsf{mono}(A,x)$ as follows:
  \index{monomorphism}
  \indexsee{mono}{monomorphism}
  \indexsee{monic}{monomorphism}
  \begin{itemize}
  \item Its objects consist of an object $y:A$ and a monomorphism $m:\hom_A(y,x)$.
    (As usual, $m:\hom_A(y,x)$ is a \define{monomorphism} (or is \define{monic}) if $(m\circ f = m\circ g) \Rightarrow (f=g)$.)
  \item Its morphisms from $(y,m)$ to $(z,n)$ are arbitrary morphisms from $y$ to $z$ in $A$ (not necessarily respecting $m$ and $n$).
  \end{itemize}
  An equality $(y,m)=(z,n)$ of objects in $\mathsf{mono}(A,x)$ consists of an equality $p:y=z$ and an equality $\trans{p}{m}=n$, which by \cref{ct:idtoiso-trans} is equivalently an equality $m=n\circ \idtoiso(p)$.
  Since hom-sets are sets, the type of such equalities is a mere proposition.
  But since $m$ and $n$ are monomorphisms, the type of morphisms $f$ such that $m = n\circ f$ is also a mere proposition.
  Thus, if $A$ is a category, then $(y,m)=(z,n)$ is a mere proposition, and hence $\mathsf{mono}(A,x)$ is a strict category.
\end{eg}

This example can be dualized, and generalized in various ways.
Here is an interesting application of strict categories.

\begin{eg}\label{ct:galois}
  Let $E/F$ be a finite Galois extension
  \index{Galois!extension}%
  of fields, and $G$ its Galois group.
  \index{Galois!group}%
  Then there is a strict category whose objects are intermediate fields $F\subseteq K\subseteq E$, and whose morphisms are field homomorphisms\index{homomorphism!field} which fix $F$ pointwise (but need not commute with the inclusions into $E$).
  There is another strict category whose objects are subgroups $H\subseteq G$, and whose morphisms are morphisms of $G$-sets $G/H \to G/K$.
  The fundamental theorem of Galois theory
  \index{fundamental!theorem of Galois theory}%
  says that these two precategories are isomorphic (not merely equivalent).
\end{eg}

\index{bargaining|)}%

\section{\texorpdfstring{$\dagger$}{†}-categories}
\label{sec:dagger-categories}

It is also worth mentioning a useful kind of precategory whose type of objects is not a set, but which is not a category either.

\begin{defn}\label{ct:dagger-precategory}
  A \define{$\dagger$-precategory}
  \indexdef{.dagger-precategory@$\dagger$-precategory}%
  \indexdef{precategory!.dagger-@$\dagger$-}%
  is a precategory $A$ together with the following.
  \begin{enumerate}
  \item For each $x,y:A$, a function $\dgr{(-)}:\hom_A(x,y) \to \hom_A(y,x)$.
  \item For all $x:A$, we have $\dgr{(1_x)} = 1_x$.
  \item For all $f,g$ we have $\dgr{(g\circ f)} = \dgr f \circ \dgr g$.
  \item For all $f$ we have $\dgr{(\dgr f)} = f$.
  \end{enumerate}
\end{defn}

\begin{defn}\label{ct:unitary}
  A morphism $f:\hom_A(x,y)$ in a $\dagger$-precategory is \define{unitary}
  \indexdef{.dagger-precategory@$\dagger$-precategory!unitary morphism in}%
  \indexdef{unitary morphism}%
  \indexdef{morphism!unitary}%
  \indexdef{isomorphism!unitary}%
  if $\dgr f \circ f = 1_x$ and $f\circ \dgr f = 1_y$.
\end{defn}

Of course, every unitary morphism is an isomorphism, and being unitary is a mere proposition.
Thus for each $x,y:A$ we have a set of unitary isomorphisms from $x$ to $y$, which we denote $(x\unitaryiso y)$.

\begin{lem}\label{ct:idtounitary}
  If $p:(x=y)$, then $\idtoiso(p)$ is unitary.
\end{lem}
\begin{proof}
  By induction, we may assume $p$ is $\refl x$.
  But then $\dgr{(1_x)} \circ 1_x = 1_x\circ 1_x = 1_x$ and similarly.
\end{proof}

\begin{defn}\label{ct:dagger-category}
  A \define{$\dagger$-category}
  \indexdef{.dagger-category@$\dagger$-category}%
  is a $\dagger$-precategory such that for all $x,y:A$, the function
  \[ (x=y) \to (x \unitaryiso y) \]
  from \cref{ct:idtounitary} is an equivalence.
\end{defn}

\begin{eg}\label{ct:rel-dagger-cat}
  The category \urel from \cref{ct:rel} becomes a $\dagger$-pre\-cat\-e\-go\-ry if we define $(\dgr R)(y,x) \defeq R(x,y)$.
  The proof that \urel is a category actually shows that every isomorphism is unitary; hence \urel is also a $\dagger$-category.
\end{eg}

\begin{eg}\label{ct:groupoid-dagger-cat}
  Any groupoid becomes a $\dagger$-category if we define $\dgr f \defeq \inv{f}$.
\end{eg}

\begin{eg}\label{ct:hilb}
  Let \uhilb be the following precategory.
  \begin{itemize}
  \item Its objects are finite-dimensional \index{finite!-dimensional vector space} vector spaces\index{vector!space} equipped with an inner product $\langle \blank,\blank\rangle$.
  \item Its morphisms are arbitrary linear maps.
    \index{function!linear}%
    \indexsee{linear map}{function, linear}%
  \end{itemize}
  By standard linear algebra, any linear map $f:V\to W$ between finite
  dimensional inner product spaces has a uniquely defined adjoint\index{adjoint!linear map} $\dgr f:W\to V$, characterized by $\langle f v,w\rangle = \langle v,\dgr f w\rangle$.
  In this way, \uhilb becomes a $\dagger$-precategory.
  Moreover, a linear isomorphism is unitary precisely when it is an \define{isometry},
  \indexdef{isometry}%
  i.e.\ $\langle fv,fw\rangle = \langle v,w\rangle$.
  It follows from this that \uhilb is a $\dagger$-category, though it is not a category (not every linear isomorphism is unitary).
\end{eg}

There has been a good deal of general theory developed for $\dagger$-cat\-e\-gor\-ies under classical\index{mathematics!classical}\index{classical!category theory} foundations.
It was observed early on that the unitary isomorphisms, not arbitrary isomorphisms, are the correct notion of ``sameness'' for objects of a $\dagger$-category, which has caused some consternation among category theorists.
Homotopy type theory resolves this issue by identifying $\dagger$-categories, like strict categories, as simply a different kind of precategory.


\section{The structure identity principle}
\label{sec:sip}
 \index{structure!identity principle|(}

The \emph{structure identity principle} is an informal principle
that expresses that isomorphic structures are identical.  We aim to
prove a general abstract result which can be applied to a wide family
of notions of structure, where structures may be many-sorted or even
dependently-sorted, infinitary, or even higher order.

The simplest kind of single-sorted structure consists of a type with
no additional structure.  The univalence axiom expresses the structure identity principle for that
notion of structure in a strong form: for types $A,B$, the
canonical function $(A=B)\to (\eqv A B)$ is an equivalence.

We start with a precategory $X$.  In our application to
single-sorted first order structures, $X$ will be the category %\uset%
of $\bbU$-small sets, where $\bbU$ is a univalent type universe.

\begin{defn}\label{ct:sig}
  A \define{notion of structure}
  \indexdef{structure!notion of}%
  $(P,H)$ over $X$ consists of the following.
  \begin{enumerate}
  \item A type family $P:X_0 \to \type$.
    For each $x:X_0$ the elements of $Px$ are called \define{$(P,H)$-structures}
    \indexsee{PH-structure@$(P,H)$-structure}{structure}%
    \indexdef{structure!PH@$(P,H)$-}%
    on $x$.
  \item For $x,y:X_0$ and $\alpha:Px$, $\;\beta:Py$, to each $f:\hom_X(x,y)$ a mere proposition
  \[ H_{\alpha\beta}(f).\]
    If $H_{\alpha\beta}(f)$ is true, we say that $f$ is a \define{$(P,H)$-homomorphism}
    \indexdef{homomorphism!of structures}%
    \indexdef{structure!homomorphism of}%
    from $\alpha$ to $\beta$.
  \item For $x:X_0$ and $\alpha:Px$, we have $H_{\alpha\alpha}(1_x)$.\label{item:sigid}
  \item For $x,y,z:X_0$ and $\alpha:Px$, $\;\beta:Py$, $\;\gamma:Pz$,
if $f:\hom_X(x,y)$, we have\label{item:sigcmp}
  \[ H_{\alpha\beta}(f)\to H_{\beta\gamma}(g)\to H_{\alpha\gamma}(g\circ   f).\]
   \end{enumerate}
  When $(P,H)$ is a notion of structure, for $\alpha,\beta:Px$ we define
  \[ (\alpha\leq_x\beta) \defeq H_{\alpha\beta}(1_x).\]
  By~\ref{item:sigid} and~\ref{item:sigcmp}, this is a preorder (\cref{ct:orders}) with $Px$ its type of objects.
  We say that $(P,H)$ is a \define{standard notion of structure}
  \indexdef{structure!standard notion of}%
  if this preorder is in fact a partial order, for all $x:X$.
\end{defn}

Note that for a standard notion of structure, each type $Px$ must actually be a set.
We now define, for any notion of structure $(P,H)$, a \define{precategory of $(P,H)$-structures},
\indexdef{precategory!of PH-structures@of $(P,H)$-structures}%
\indexdef{structure!precategory of PH@precategory of $(P,H)$-}%
$A = \mathsf{Str}_{(P,H)}(X)$.
\begin{itemize}
\item The type of objects of $A$ is the type $A_0 \defeq \sm{x:X} Px$.
  If $a\jdeq (x,\alpha):A_0$, we may write $|a| \defeq x$.
\item For $(x,\alpha):A_0$ and $(y,\beta):A_0$, we define
  \[\hom_A((x,\alpha),(y,\beta)) \defeq \setof{ f:x \to y | H_{\alpha\beta}(f)}.\]
\end{itemize}
The composition and identities are inherited from $X$; conditions~\ref{item:sigid} and \ref{item:sigcmp} ensure that these lift to $A$.

\begin{thm}[Structure identity principle]\label{thm:sip}
  \indexdef{structure!identity principle}%
  If $X$ is a category and $(P,H)$ is a standard notion of structure over $X$, then the precategory $\mathsf{Str}_{(P,H)}(X)$ is a category.
\end{thm}
\begin{proof}
  By the definition of equality in dependent pair types, to give an equality $(x,\alpha)=(y,\beta)$ consists of
  \begin{itemize}
  \item An equality $p:x=y$, and
  \item An equality $\trans{p}{\alpha}=\beta$.
  \end{itemize}
  Since $P$ is set-valued, the latter is a mere proposition.
  On the other hand, it is easy to see that an isomorphism $(x,\alpha)\cong (y,\beta)$ in $\mathsf{Str}_{(P,H)}(X)$ consists of
  \begin{itemize}
  \item An isomorphism $f:x\cong y$ in $X$, such that
  \item $H_{\alpha\beta}(f)$ and $H_{\beta\alpha}(\inv f)$.
  \end{itemize}
  Of course, the second of these is also a mere proposition.
  And since $X$ is a category, the function $(x=y) \to (x\cong y)$ is an equivalence.
  Thus, it will suffice to show that for any $p:x=y$ and for any $(\alpha:Px)$, $(\beta:Py)$, we have $\trans{p}{\alpha}=\beta$ if and only if both  $H_{\alpha\beta}(\idtoiso (p))$ and $H_{\beta\alpha}(\inv{\idtoiso(p)})$.

  The ``only if'' direction is just the existence of the function $\idtoiso$ for the category $\mathsf{Str}_{(P,H)}(X)$.
  For the ``if'' direction, by induction on $p$ we may assume that $y\jdeq x$ and $p\jdeq\refl x$.
  However, in this case $\idtoiso (p)\jdeq 1_x$ and therefore $\inv{\idtoiso(p)}=1_x$.
  Thus, $\alpha\leq_x \beta$ and $\beta\leq_x \alpha$, which implies $\alpha=\beta$ since $(P,H)$ is a standard notion of structure.
\end{proof}

As an example, this methodology gives an alternative way to express the proof of \cref{ct:functor-cat}.

\begin{eg}\label{ct:sip-functor-cat}
  Let $A$ be a precategory and $B$ a category.
  There is a precategory $B^{A_0}$ whose objects are functions $A_0 \to B_0$, and whose set of morphisms from $F_0:A_0 \to B_0$ to $G_0:A_0 \to B_0$ is $\prd{a:A_0} \hom_B(F_0 a, G_0 a)$.
  Composition and identities are inherited directly from those in $B$.
  It is easy to show that $\gamma:\hom_{B^{A_0}}(F_0, G_0)$ is an isomorphism exactly when each component $\gamma_a$ is an isomorphism, so that we have $\eqv{(F_0 \cong G_0)}{\prd{a:A_0} (F_0 a \cong G_0 a)}$.
  Moreover, the map $\idtoiso : (F_0 = G_0) \to (F_0 \cong G_0)$ of $B^{A_0}$ is equal to the composite
  \[ (F_0 = G_0) \longrightarrow \prd{a:A_0} (F_0 a  = G_0 a) \longrightarrow \prd{a:A_0} (F_0 a \cong G_0 a) \longrightarrow (F_0 \cong G_0) \]
  in which the first map is an equivalence by function extensionality, the second because it is a dependent product of equivalences (since $B$ is a category), and the third as remarked above.
  Thus, $B^{A_0}$ is a category.

  Now we define a notion of structure on $B^{A_0}$ for which $P(F_0)$ is the type of operations $F:\prd{a,a':A_0} \hom_A(a,a') \to \hom_B(F_0 a,F_0 a')$ which extend $F_0$ to a functor (i.e.\ preserve composition and identities).
  This is a set since each $\hom_B(\blank,\blank)$ is so.
  Given such $F$ and $G$, we define $\gamma:\hom_{B^{A_0}}(F_0, G_0)$ to be a homomorphism if it forms a natural transformation.\index{natural!transformation}
  In \cref{ct:functor-precat} we essentially verified that this is a notion of structure.
  Moreover, if $F$ and $F'$ are both structures on $F_0$ and the identity is a natural transformation from $F$ to $F'$, then for any $f:\hom_A(a,a')$ we have $F'f = F'f \circ 1_{F_0 a} = 1_{F_0 a}\circ F f = F f$.
  Applying function extensionality, we conclude $F = F'$.
  Thus, we have a \emph{standard} notion of structure, and so by \cref{thm:sip}, the precategory $B^A$ is a category.
\end{eg}

As another example, we consider categories of structures for a first-order signature.
We define a \define{first-order signature},
\indexdef{first-order!signature}%
\indexdef{signature!first-order}%
$\Omega$, to consist of sets $\Omega_0$ and $\Omega_1$ of function symbols, $\omega:\Omega_0$, and relation symbols, $\omega:\Omega_1$, each having an arity\index{arity} $|\omega|$ that is a set.
An \define{$\Omega$-structure}
\indexdef{structure!Omega@$\Omega$-}%
\indexsee{omega-structure@$\Omega$-structure}{structure}%
$a$ consists of a set $|a|$ together with an assignment of an $|\omega|$-ary function $\omega^a:|a|^{|\omega|}\to |a|$ on $|a|$ to each function symbol, $\omega$, and an assignment of an $|\omega|$-ary relation $\omega^a$ on $|a|$, assigning a mere proposition $\omega^ax$ to each $x:|a|^{|\omega|}$, to each relation symbol.
And given $\Omega$-structures $a,b$, a function $f:|a|\to |b|$ is a \define{homomorphism $a\to b$}
\indexdef{homomorphism!of Omega-structures@of $\Omega$-structures}%
\indexdef{structure!homomorphism of Omega@homomorphism of $\Omega$-}%
if it preserves the structure; i.e.\ if for each symbol $\omega$ of the signature and each $x:|a|^{|\omega|}$,
\begin{enumerate}
\item $f(\omega^ax) = \omega^b(f\circ x)$ if $\omega:\Omega_0$, and
\item $\omega^ax\to\omega^b(f\circ x)$ if $\omega:\Omega_1$.
\end{enumerate}
Note that each $x:|a|^{|\omega|}$ is a function $x:|\omega|\to |a|$ so that $f\circ x : b^\omega$.

Now we assume given a (univalent) universe $\bbU$ and a $\bbU$-small signature $\Omega$; i.e. $|\Omega|$ is a $\bbU$-small set and, for each $\omega:|\Omega|$, the set $|\omega|$ is $\bbU$-small.
Then we have the category $\uset_\bbU$ of $\bbU$-small sets.  We want to define the precategory of $\bbU$-small $\Omega$-structures over $\uset_\bbU$ and use \cref{thm:sip} to show that it is a category.

We use the first order signature $\Omega$ to give us a standard notion of structure $(P,H)$ over $\uset_\bbU$.

\begin{defn}\label{defn:fo-notion-of-structure}
\mbox{}
\begin{enumerate}
\item For each $\bbU$-small set $x$ define
  \[ Px \defeq P_0x\times P_1x.\]
  Here
  %
  \begin{align*}
    P_0x &\defeq \prd{\omega:\Omega_0} x^{|\omega|}\to x, \mbox{ and } \\
    P_1x &\defeq \prd{\omega:\Omega_1} x^{|\omega|}\to \propU,
  \end{align*}
\item For $\bbU$-small sets $x,y$ and
  $\alpha:P^\omega x,\;\beta:P^\omega y,\; f:x\to y$, define
  \[ H_{\alpha\beta}(f) \defeq H_{0,\alpha\beta}(f)\wedge H_{1,\alpha\beta}(f).\]
  Here
  \begin{align*}
    H_{0,\alpha\beta}(f) &\defeq
    \fall{\omega:\Omega_0}{u:x^{|\omega|}} f(\alpha u)=\;\beta(f\circ u),
    \mbox{ and }\\
    H_{1,\alpha\beta}(f) &\defeq
    \fall{\omega:\Omega_1}{u:x^{|\omega|}} \alpha u\to\beta(f\circ u).
  \end{align*}
\end{enumerate}
\end{defn}

It is now routine to check that $(P,H)$ is a standard notion of structure over $\uset_\bbU$ and hence we may use \cref{thm:sip} to get that the precategory $Str_{(P,H)}(\uset_\bbU)$ is a category.  It only remains to observe that this is essentially the same as the precategory of $\bbU$-small $\Omega$-structures over $\uset_\bbU$.
 \index{structure!identity principle|)}


\section{The Rezk completion}
\label{sec:rezk}

In this section we will give a universal way to replace a precategory by a category.
In fact, we will give two.
Both rely on the fact that ``categories see weak equivalences as equivalences''.

To prove this, we begin with a couple of lemmas which are completely standard category theory, phrased carefully so as to make sure we are using the eliminator for $\truncf{-1}$ correctly.
One would have to be similarly careful in classical\index{mathematics!classical}\index{classical!category theory} category theory if one wanted to avoid the axiom of choice: any time we want to define a function, we need to characterize its values uniquely somehow.

\begin{lem}\label{ct:esosurj-postcomp-faithful}
  If $A,B,C$ are precategories and $H:A\to B$ is an essentially surjective functor, then $(\blank\circ H):C^B \to C^A$ is faithful.
\end{lem}
\begin{proof}
  Let $F,G:B\to C$, and $\gamma,\delta:F\to G$ be such that $\gamma H = \delta H$; we must show $\gamma=\delta$.
  Thus let $b:B$; we want to show $\gamma_b=\delta_b$.
  This is a mere proposition, so since $H$ is essentially surjective, we may assume given an $a:A$ and an isomorphism $f:Ha\cong b$.
  But now we have
  \[ \gamma_b = G(f) \circ \gamma_{Ha} \circ F(\inv{f})
  = G(f) \circ \delta_{Ha} \circ F(\inv{f})
  = \delta_b.\qedhere
  \]
\end{proof}

\begin{lem}\label{ct:esofull-precomp-ff}
  If $A,B,C$ are precategories and $H:A\to B$ is essentially surjective and full, then $(\blank\circ H):C^B \to C^A$ is fully faithful.
\end{lem}
\begin{proof}
  It remains to show fullness.
  Thus, let $F,G:B\to C$ and $\gamma:FH \to GH$.
  We claim that for any $b:B$, the type
  \begin{equation}\label{eq:fullprop}
    \sm{g:\hom_C(Fb,Gb)} \prd{a:A}{f:Ha\cong b} (\gamma_a =  \inv{Gf}\circ g\circ Ff)
  \end{equation}
  is contractible.
  Since contractibility is a mere property, and $H$ is essentially surjective, we may assume given $a_0:A$ and $h:Ha_0\cong b$.

  Now take $g\defeq Gh \circ \gamma_{a_0} \circ \inv{Fh}$.
  Then given any other $a:A$ and $f:Ha\cong b$, we must show $\gamma_a =  \inv{Gf}\circ g\circ Ff$.
  Since $H$ is full, there merely exists a morphism $k:\hom_A(a,a_0)$ such that $Hk = \inv{h}\circ f$.
  And since our goal is a mere proposition, we may assume given some such $k$.
  Then we have
  \begin{align*}
    \gamma_a &= \inv{GHk}\circ \gamma_{a_0} \circ FHk\\
    &= \inv{Gf} \circ Gh \circ \gamma_{a_0} \circ \inv{Fh} \circ Ff\\
    &= \inv{Gf}\circ g\circ Ff.
  \end{align*}
  Thus,~\eqref{eq:fullprop} is inhabited.
  It remains to show it is a mere proposition.
  Let $g,g':\hom_C(Fb, Gb)$ be such that for all $a:A$ and $f:Ha\cong b$, we have both $(\gamma_a =  \inv{Gf}\circ g\circ Ff)$ and $(\gamma_a =  \inv{Gf}\circ g'\circ Ff)$.
  The dependent product types are mere propositions, so all we have to prove is $g=g'$.
  But this is a mere proposition, so we may assume $a_0:A$ and $h:Ha_0\cong b$, in which case we have
  \[ g = Gh \circ \gamma_{a_0} \circ \inv{Fh} = g'.\]
  %
  This proves that~\eqref{eq:fullprop} is contractible for all $b:B$.
  Now we define $\delta:F\to G$ by taking $\delta_b$ to be the unique $g$ in~\eqref{eq:fullprop} for that $b$.
  To see that this is natural, suppose given $f:\hom_B(b,b')$; we must show $Gf \circ \delta_b = \delta_{b'}\circ Ff$.
  As before, we may assume $a:A$ and $h:Ha\cong b$, and likewise $a':A$ and $h':Ha'\cong b'$.
  Since $H$ is full as well as essentially surjective, we may also assume $k:\hom_A(a,a')$ with $Hk = \inv{h'}\circ f\circ h$.

  Since $\gamma$ is natural, $GHk\circ \gamma_a = \gamma_{a'} \circ FHk$.
  Using the definition of $\delta$, we have
  \begin{align*}
    Gf \circ \delta_b
    &= Gf \circ Gh \circ \gamma_a \circ \inv{Fh}\\
    &= Gh' \circ GHk\circ \gamma_a \circ \inv{Fh}\\
    &= Gh' \circ \gamma_{a'} \circ FHk \circ \inv{Fh}\\
    &= Gh' \circ \gamma_{a'} \circ \inv{Fh'} \circ Ff\\
    &= \delta_{b'} \circ Ff.
  \end{align*}
  Thus, $\delta$ is natural.
  Finally, for any $a:A$, applying the definition of $\delta_{Ha}$ to $a$ and $1_a$, we obtain $\gamma_a = \delta_{Ha}$.
  Hence, $\delta \circ H = \gamma$.
\end{proof}

The rest of the theorem follows almost exactly the same lines, with the category-ness of $C$ inserted in one crucial step, which we have italicized below for emphasis.
This is the point at which we are trying to define a function into \emph{objects} without using choice, and so we must be careful about what it means for an object to be ``uniquely specified''.
In classical\index{mathematics!classical}\index{classical!category theory} category theory, all one can say is that this object is specified up to unique isomorphism, but in set-theoretic foundations this is not a sufficient amount of uniqueness to give us a function without invoking \choice{}.
In univalent foundations, however, if $C$ is a category, then isomorphism is equality, and we have the appropriate sort of uniqueness (namely, living in a contractible space).

\index{weak equivalence!of precategories|(}%

\begin{thm}\label{ct:cat-weq-eq}
  If $A,B$ are precategories, $C$ is a category, and $H:A\to B$ is a weak equivalence, then $(\blank\circ H):C^B \to C^A$ is an isomorphism.
\end{thm}
\begin{proof}
  By \cref{ct:functor-cat}, $C^B$ and $C^A$ are categories.
  Thus, by \cref{ct:eqv-levelwise} it will suffice to show that $(\blank\circ H)$ is an equivalence.
  But since we know from the preceding two lemmas that it is fully faithful, by \cref{ct:catweq} it will suffice to show that it is essentially surjective.
  Thus, suppose $F:A\to C$; we want there to merely exist a $G:B\to C$ such that $GH\cong F$.

  For each $b:B$, let $X_b$ be the type whose elements consist of:
  \begin{enumerate}
  \item An element $c:C$; and
  \item For each $a:A$ and $h:Ha\cong b$, an isomorphism $k_{a,h}:Fa\cong c$; such that\label{item:eqvprop2}
  \item For each $(a,h)$ and $(a',h')$ as in~\ref{item:eqvprop2} and each $f:\hom_A(a,a')$ such that $h'\circ Hf = h$, we have $k_{a',h'}\circ Ff = k_{a,h}$.\label{item:eqvprop3}
  \end{enumerate}
  We claim that for any $b:B$, the type $X_b$ is contractible.
  As this is a mere proposition, we may assume given $a_0:A$ and $h_0:Ha_0 \cong b$.
  Let $c^0\defeq Fa_0$.
  Next, given $a:A$ and $h:Ha\cong b$, since $H$ is fully faithful there is a unique isomorphism $g_{a,h}:a\to a_0$ with $Hg_{a,h} = \inv{h_0}\circ h$; define $k^0_{a,h} \defeq Fg_{a,h}$.
  Finally, if $h'\circ Hf = h$, then $\inv{h_0}\circ h'\circ Hf = \inv{h_0}\circ h$, hence $g_{a',h'} \circ f = g_{a,h}$ and thus $k^0_{a',h'}\circ Ff = k^0_{a,h}$.
  Therefore, $X_b$ is inhabited.

  Now suppose given another $(c^1,k^1): X_b$.
  Then $k^1_{a_0,h_0}:c^0 \jdeq Fa_0 \cong c^1$.
  \emph{Since $C$ is a category, we have $p:c^0=c^1$ with $\idtoiso(p) = k^1_{a_0,h_0}$.}
  And for any $a:A$ and $h:Ha\cong b$, by~\ref{item:eqvprop3} for $(c^1,k^1)$ with $f\defeq g_{a,h}$, we have
  \[k^1_{a,h} = k^1_{a_0,h_0} \circ k^0_{a,h} = \trans{p}{k^0_{a,h}}\]
  This gives the requisite data for an equality $(c^0,k^0)=(c^1,k^1)$, completing the proof that $X_b$ is contractible.

  Now since $X_b$ is contractible for each $b$, the type $\prd{b:B} X_b$ is also contractible.
  In particular, it is inhabited, so we have a function assigning to each $b:B$ a $c$ and a $k$.
  Define $G_0(b)$ to be this $c$; this gives a function $G_0 :B_0 \to C_0$.

  Next we need to define the action of $G$ on morphisms.
  For each $b,b':B$ and $f:\hom_B(b,b')$, let $Y_f$ be the type whose elements consist of:
  \begin{enumerate}[resume]
  \item A morphism $g:\hom_C(Gb,Gb')$, such that
  \item For each $a:A$ and $h:Ha\cong b$, and each $a':A$ and $h':Ha'\cong b'$, and any $\ell:\hom_A(a,a')$, we have\label{item:eqvprop5}
    \[ (h' \circ H\ell = f \circ h)
    \to
    (k_{a',h'} \circ F\ell = g\circ k_{a,h}). \]
  \end{enumerate}
  We claim that for any $b,b'$ and $f$, the type $Y_f$ is contractible.
  As this is a mere proposition, we may assume given $a_0:A$ and $h_0:Ha_0\cong b$, and each $a'_0:A$ and $h'_0:Ha'_0\cong b'$.
  Then since $H$ is fully faithful, there is a unique $\ell_0:\hom_A(a_0,a_0')$ such that $h'_0 \circ H\ell_0 = f \circ h_0$.
  Define $g_0 \defeq k_{a_0',h_0'} \circ F \ell_0 \circ \inv{(k_{a_0,h_0})}$.

  Now for any $a,h,a',h'$, and $\ell$ such that $(h' \circ H\ell = f \circ h)$, we have $\inv{h}\circ h_0:Ha_0\cong Ha$, hence there is a unique $m:a_0\cong a$ with $Hm = \inv{h}\circ h_0$ and hence $h\circ Hm = h_0$.
  Similarly, we have a unique $m':a_0'\cong a'$ with $h'\circ Hm' = h_0'$.
  Now by~\ref{item:eqvprop3}, we have $k_{a,h}\circ Fm = k_{a_0,h_0}$ and $k_{a',h'}\circ Fm' = k_{a_0',h_0'}$.
  We also have
  \begin{align*}
    Hm' \circ H\ell_0
    &= \inv{(h')} \circ h_0' \circ H\ell_0\\
    &= \inv{(h')} \circ f \circ h_0\\
    &= \inv{(h')} \circ f \circ h \circ \inv{h} \circ h_0\\
    &= H\ell \circ Hm
  \end{align*}
  and hence $m'\circ \ell_0 = \ell\circ m$ since $H$ is fully faithful.
  Finally, we can compute
  \begin{align*}
    g_0 \circ k_{a,h}
    &= k_{a_0',h_0'} \circ F \ell_0 \circ \inv{(k_{a_0,h_0})} \circ k_{a,h}\\
    &= k_{a_0',h_0'} \circ F \ell_0 \circ \inv{Fm}\\
    &= k_{a_0',h_0'} \circ \inv{(Fm')} \circ F\ell\\
    &= k_{a',h'}\circ F\ell.
  \end{align*}
  This completes the proof that $Y_f$ is inhabited.
  To show it is contractible, since hom-sets are sets, it suffices to take another $g_1:\hom_C(Gb,Gb')$ satisfying~\ref{item:eqvprop5} and show $g_0=g_1$.
  However, we still have our specified $a_0,h_0,a_0',h_0',\ell_0$ around, and~\ref{item:eqvprop5} implies both $g_0$ and $g_1$ must be equal to $k_{a_0',h_0'} \circ F \ell_0 \circ \inv{(k_{a_0,h_0})}$.

  This completes the proof that $Y_f$ is contractible for each $b,b':B$ and $f:\hom_B(b,b')$.
  Therefore, there is a function assigning to each such $f$ its unique inhabitant; denote this function $G_{b,b'}:\hom_B(b,b') \to \hom_C(Gb,Gb')$.
  The proof that $G$ is a functor is straightforward; in each case we can choose $a,h$ and apply~\ref{item:eqvprop5}.

  Finally, for any $a_0:A$, defining $c\defeq Fa_0$ and $k_{a,h}\defeq F g$, where $g:\hom_A(a,a_0)$ is the unique isomorphism with $Hg = h$, gives an element of $X_{Ha_0}$.
  Thus, it is equal to the specified one; hence $GHa=Fa$.
  Similarly, for $f:\hom_A(a_0,a_0')$ we can define an element of $Y_{Hf}$ by transporting along these equalities, which must therefore be equal to the specified one.
  Hence, we have $GH=F$, and thus $GH\cong F$ as desired.
\end{proof}

\index{universal!property!of Rezk completion}%
Therefore, if a precategory $A$ admits a weak equivalence functor $A\to \widehat{A}$, then that is its ``reflection'' into categories: any functor from $A$ into a category will factor essentially uniquely through $\widehat{A}$.
We now give two constructions of such a weak equivalence.

\indexsee{Rezk completion}{completion, Rezk}%
\index{completion!Rezk|(defstyle}%

\begin{thm}\label{thm:rezk-completion}
  For any precategory $A$, there is a category $\widehat A$ and a weak equivalence $A\to\widehat{A}$.
\end{thm}

\begin{proof}[First proof]
  Let $\widehat{A}_0 \defeq \setof{ F:\uset^{A\op} | \exis{a:A} (\y a \cong F)}$, with hom-sets inherited from $\uset^{A\op}$.
  Then the inclusion $\widehat{A} \to \uset^{A\op}$ is fully faithful and an embedding on objects.
  Since $\uset^{A\op}$ is a category (by \cref{ct:functor-cat}, since \uset is so by univalence), $\widehat A$ is also a category.

  Let $A\to\widehat A$ be the Yoneda embedding.
  This is fully faithful by \cref{ct:yoneda-embedding}, and essentially surjective by definition of $\widehat{A}_0$.
  Thus it is a weak equivalence.
\end{proof}

This proof is very slick, but it has the drawback that it increases universe level.
If $A$ is a category in a universe \bbU, then in this proof \uset must be at least as large as $\uset_\bbU$.
Then $\uset_\bbU$ and $(\uset_\bbU)^{A\op}$ are not themselves categories in \bbU, but only in a higher universe, and \emph{a priori} the same is true of $\widehat A$.
One could imagine a resizing axiom that could deal with this, but it is also possible to give a direct construction using higher inductive types.

\begin{proof}[Second proof]
  We define a higher inductive type $\widehat A_0$ with the following constructors:
  \begin{itemize}
  \item A function $i:A_0 \to \widehat A_0$.
  \item For each $a,b:A$ and $e:a\cong b$, an equality $je:\id{ia}{ib}$.
  \item For each $a:A$, an equality $\id{j(1_a)}{\refl{ia}}$.
  \item For each $(a,b,c:A)$, $(f:a\cong b)$, and $(g:b\cong c)$, an equality $\id{j(g \circ f)}{j(f)\ct j(g)}$.
  \item 1-truncation: for all $x,y:\widehat A_0$ and $p,q:\id x y$ and $r,s:\id p q$, an equality $\id r s$.
  \end{itemize}
  Note that for any $a,b:A$ and $p:\id a b$, we have $\id{j(\idtoiso(p))}{\map i p}$.
  This follows by path induction on $p$ and the third constructor.

  The type $\widehat A_0$ will be the type of objects of $\widehat A$; we now build all the rest of the structure.
  (The following proof is of the sort that can benefit a lot from the help of a computer proof assistant:\index{proof!assistant} it is wide and shallow with many short cases to consider, and a large part of the work consists of writing down what needs to be checked.)

  \mentalpause

  \emph{Step 1:} We define a family $\hom_{\widehat A}:\widehat A_0\to \widehat A_0 \to \set$ by double induction on $\widehat A_0$.
  Since \set is a 1-type, we can ignore the 1-truncation constructor.
  When $x$ and $y$ are of the form $ia$ and $ib$, we take $\hom_{\widehat A}(ia,ib) \defeq \hom_A(a,b)$.
  It remains to consider all the other possible pairs of constructors.

  Let us keep $x=ia$ fixed at first.
  If $y$ varies along the identity $je:\id{ib}{ib'}$, for some $e:b\cong b'$, we require an identity $\id{\hom_A(a,b)}{\hom_A(a,b')}$.
  By univalence, it suffices to give an equivalence $\eqv{\hom_A(a,b)}{\hom_A(a,b')}$.
  We take this to be the function $(e\circ \blank ):\hom_A(a,b)\to \hom_A(a,b')$.
  To see that this is an equivalence, we give its inverse as $(\inv e\circ \blank )$, with witnesses to inversion coming from the fact that $\inv e$ is the inverse of $e$ in $A$.

  As $y$ varies along the identity $\id{j(1_b)}{\refl{ib}}$, we require an identity $\id{(1_b\circ \blank )}{\refl{\hom_A(a,b)}}$; this follows from the identity axiom $\id{1_b\circ g}{g}$ of a precategory.
  Similarly, as $y$ varies along the identity $\id{j(g\circ f)}{j(f)\ct j(g)}$, we require an identity $\id{((g\circ f)\circ \blank )}{(g\circ (f\circ \blank ))}$, which follows from associativity.
  % Finally, as $y$ varies along the 1-truncation constructor, we need only to observe that \set is 1-truncated.

  Now we consider the other constructors for $x$.
  Say that $x$ varies along the identity $j(e):\id{ia}{ia'}$, for some $e:a \cong a'$; we again must deal with all the constructors for $y$.
  If $y$ is $ib$, then we require an identity $\id{\hom_A(a,b)}{\hom_A(a',b)}$.
  By univalence, this may come from an equivalence, and for this we can use $(\blank\circ \inv e)$, with inverse $(\blank\circ e)$.

  Still with $x$ varying along $j(e)$, suppose now that $y$ also varies along $j(f)$ for some $f:b\cong b'$.
  Then we need to know that the two concatenated identities
  \begin{gather*}
    \hom_A(a,b) = \hom_A(a',b) = \hom_A(a',b') \mathrlap{\qquad\text{and}}\\
    \hom_A(a,b) = \hom_A(a,b') = \hom_A(a',b')
  \end{gather*}
  are identical.
  This follows from associativity: $(f\circ \blank)\circ \inv e = f\circ (\blank\circ \inv e)$.
  The other two constructors for $y$ are trivial, since they are 2-fold equalities in sets.

  For the next two constructors of $x$, all but the first constructor for $y$ is likewise trivial.
  When $x$ varies along $j(1_a)=\refl{ia}$ and $y$ is $ib$, we use the identity axiom again.
  Similarly, when $x$ varies along $\id{j(g\circ f)}{j(f)\ct j(g)}$, we use associativity again.
  This completes the construction of $\hom_{\widehat A}:\widehat A_0 \to \widehat A_0 \to \set$.

  \mentalpause

  \emph{Step 2:} We give the precategory structure on $\widehat A$, always by induction on $\widehat A_0$.
  % The reader is probably getting bored at this point, so we skip the details.
  We are now eliminating into sets (the hom-sets of $\widehat A$), so all but the first two constructors are trivial to deal with.

  For identities, if $x$ is $ia$ then we have $\hom_{\widehat A}(x,x) \jdeq \hom_A(a,a)$ and we define $1_x \defeq 1_{ia}$.
  If $x$ varies along $je$ for $e:a\cong a'$, we must show that $\transfib{x\mapsto \hom_{\widehat A}(x,x)}{je}{1_{ia}} = 1_{ia'}$.
  But by definition of $\hom_{\widehat A}$, transporting along $je$ is given by composing with $e$ and $\inv e$, and we have $e\circ 1_{ia} \circ \inv{e} = 1_{ia'}$.

  For composition, if $x,y,z$ are $ia,ib,ic$ respectively, then $\hom_{\widehat A}$ reduces to $\hom_A$ and we can define composition in $\widehat A$ to be composition in $A$.
  And when $x$, $y$, or $z$ varies along $je$, then we verify the following equalities:
  \begin{align*}
    e \circ (g\circ f) &= (e\circ g) \circ f,\\
    g\circ f &= (g\circ \inv e) \circ (e\circ f),\\
    (g\circ f) \circ \inv e &= g \circ (f\circ \inv e).
  \end{align*}
  Finally, the associativity and unitality axioms are mere propositions, so all constructors except the first are trivial.
  But in that case, we have the corresponding axioms in $A$.

  \mentalpause

  \emph{Step 3}: We show that $\widehat A$ is a category.
  That is, we must show that for all $x,y:\widehat A$, the function $\idtoiso:(x=y) \to (x\cong y)$ is an equivalence.
  First we define, for all $x,y:\widehat A$, a function $k_{x,y}:(x\cong y) \to (x=y)$ by induction.
  As before, since our goal is a set, it suffices to deal with the first two constructors.

  When $x$ and $y$ are $ia$ and $ib$ respectively, we have $\hom_{\widehat A}(ia,ib)\jdeq \hom_A(a,b)$, with composition and identities inherited as well, so that $(ia\cong ib)$ is equivalent to $(a\cong b)$.
  But now we have the constructor $j:(a\cong b) \to (ia=ib)$.

  Next, if $y$ varies along $j(e)$ for some $e:b\cong b'$, we must show that for $f:a\cong b$ we have $j(\trans{j(e)}{f}) = j(f) \ct j(e)$.
  But by definition of $\hom_{\widehat A}$ on equalities, transporting along $j(e)$ is equivalent to post-composing with $e$, so this equality follows from the last constructor of $\widehat A_0$.
  The remaining case when $x$ varies along $j(e)$ for $e:a\cong a'$ is similar.
  This completes the definition of $k:\prd{x,y:\widehat A_0} (x\cong y) \to (x=y)$.

  Now one thing we must show is that if $p:x=y$, then $k(\idtoiso(p))=p$.
  By induction on $p$, we may assume it is $\refl x$, and hence $\idtoiso(p)\jdeq 1_x$.
  Now we argue by induction on $x:\widehat A_0$, and since our goal is a mere proposition (since $\widehat A_0$ is a 1-type), all constructors except the first are trivial.
  But if $x$ is $ia$, then $k(1_{ia}) \jdeq j(1_a)$, which is equal to $\refl{ia}$ by the third constructor of $\widehat A_0$.

  To complete the proof that $\widehat A$ is a category, we must show that if $f:x\cong y$, then $\idtoiso(k(f))=f$.
  By induction we may assume that $x$ and $y$ are $ia$ and $ib$ respectively, in which case $f$ must arise from an isomorphism $g:a\cong b$ and we have $k(f)\jdeq j(g)$.
  However, for any $p$ we have $\idtoiso(p) = \trans{p}{1}$, so in particular $\idtoiso (j(g)) = \trans{j(g)}{1_{ia}}$.
  And by definition of $\hom_{\widehat A}$ on equalities, this is given by composing $1_{ia}$ with the equivalence $g$, hence is equal to $g$.

  \index{encode-decode method}%
  Note the similarity of this step to the encode-decode method\index{encode-decode method} used in \cref{sec:compute-coprod,sec:compute-nat,cha:homotopy}.
  Once again we are characterizing the identity types of a higher inductive type (here, $\widehat A_0$) by defining recursively a family of codes (here, $(x,y)\mapsto (x\cong y)$) and encoding and decoding functions by induction on $\widehat A_0$ and on paths.

  \mentalpause

  \emph{Step 4}: We define a weak equivalence $I:A \to \widehat A$.
  We take $I_0 \defeq i : A_0 \to \widehat A_0$, and by construction of $\hom_{\widehat A}$ we have functions $I_{a,b}:\hom_A(a,b) \to \hom_{\widehat A}(Ia,Ib)$ forming a functor $I:A \to \widehat A$.
  This functor is fully faithful by construction, so it remains to show it is essentially surjective.
  That is, for all $x:\widehat A$ we want there to merely exist an $a:A$ such that $Ia\cong x$.
  As always, we argue by induction on $x$, and since the goal is a mere proposition, all but the first constructor are trivial.
  But if $x$ is $ia$, then of course we have $a:A$ and $Ia\jdeq ia$, hence $Ia \cong ia$.
  (Note that if we were trying to prove $I$ to be \emph{split} essentially surjective, we would be stuck, because we know nothing about equalities in $A_0$ and thus have no way to deal with any further constructors.)
\end{proof}

We call the construction $A\mapsto \widehat A$ the \define{Rezk completion},
although there is also an argument (coming from higher topos semantics)
\index{.infinity1-topos@$(\infty,1)$-topos}%
for calling it the \define{stack completion}.
\index{stack}%
\index{completion!Rezk|)}%

We have seen that most precategories arising in practice are categories, since they are constructed from \uset, which is a category by the univalence axiom.
However, there are a few cases in which the Rezk completion is necessary to obtain a category.

\begin{eg}\label{ct:rezk-fundgpd-trunc1}
  Recall from \cref{ct:fundgpd} that for any type $X$ there is a pregroupoid with $X$ as its type of objects and $\hom(x,y) \defeq \pizero{x=y}$.
  \indexdef{fundamental!groupoid}%
  \index{fundamental!pregroupoid}%
  \indexsee{groupoid!fundamental}{fundamental group\-oid}%
  Its Rezk completion is the \emph{fundamental groupoid} of $X$.
  Recalling that groupoids are equivalent to 1-types, it is not hard to identify this groupoid with $\trunc1X$.
\end{eg}

\begin{eg}\label{ct:hocat}
  Recall from \cref{ct:hoprecat} that there is a precategory whose type of objects is \type and with $\hom(X,Y) \defeq \pizero{X\to Y}$.
  Its Rezk completion may be called the \define{homotopy category of types}.
  \index{category!of types}%
  \index{homotopy!category of types@(pre)category of types}%
  Its type of objects can be identified with $\trunc1\type$ (see \cref{ct:ex:hocat}).
\end{eg}

The Rezk completion also allows us to show that the notion of ``category'' is determined by the notion of ``weak equivalence of precategories''.
Thus, insofar as the latter is inevitable, so is the former.

\begin{thm}\label{ct:weq-iso-precat-cat}
  A precategory $C$ is a category if and only if for every weak equivalence of precategories $H:A\to B$, the induced functor $(\blank\circ H):C^B \to C^A$ is an isomorphism of precategories.
\end{thm}
\begin{proof}
  ``Only if'' is \cref{ct:cat-weq-eq}.
  In the other direction, let $H$ be $I:A\to\widehat A$.
  Then since $(\blank\circ I)_0$ is an equivalence, there exists $R:\widehat A\to A$ such that $RI=1_A$.
  Hence $IRI=I$, but again since $(\blank\circ I)_0$ is an equivalence, this implies $IR =1_{\widehat A}$.
  By \cref{ct:isoprecat}\ref{item:ct:ipc3}, $I$ is an isomorphism of precategories.
  But then since $\widehat A$ is a category, so is $A$.
\end{proof}

\index{weak equivalence!of precategories|)}%


\newpage

\sectionNotes

The original definition of categories, of course, was in set-theoretic foundations, so that the collection of objects of a category formed a set (or, for large categories, a class).
Over time, it became clear that all ``category-theoretic'' properties of objects were invariant under isomorphism, and that equality of objects in a category was not usually a very useful notion.
Numerous authors~\cite{blanc:eqv-log,freyd:invar-eqv,makkai:folds,makkai:comparing} discovered that a dependently typed logic enabled formulating the definition of category without invoking any notion of equality for objects, and that the statements provable in this logic are precisely the ``category-theoretic'' ones that are invariant under isomorphism.
\index{evil}%

Although most of category theory appears to be invariant under isomorphism of objects and under equivalence of categories, there are some interesting exceptions, which have led to philosophical discussions about what it means to be ``category-theoretic''.
For instance, \cref{ct:galois} was brought up by Peter May on the categories mailing list in May 2010, as a case where it matters that two categories (defined as usual in set theory) are isomorphic rather than only equivalent.
The case of $\dagger$-categories was also somewhat confounding to those advocating an isomorphism-invariant version of category theory, since the ``correct'' notion of sameness between objects of a $\dagger$-category is not ordinary isomorphism but \emph{unitary} isomorphism.
\index{isomorphism!invariance under}%

Categories satisfying the ``saturation'' or ``univalence'' principle as in \cref{ct:category} were first considered by Hofmann and Streicher~\cite{hs:gpd-typethy}.
The condition then occurred independently to Voevodsky, Shulman, and perhaps others around the same time several years later, and was formalized by Ahrens and Kapulkin~\cite{aks:rezk}.
This framework puts all the above examples in a unified context: some precategories are categories, others are strict categories, and so on.
A general theorem that ``isomorphism implies equality'' for a large class of algebraic structures (assuming the univalence axiom) was proven by Coquand and Danielsson; the formulation of the structure identity principle in \cref{sec:sip} is due to Aczel.

Independently of philosophical considerations about category theory, Rezk~\cite{rezk01css} discovered that when defining a notion of $(\infty,1)$-cat\-e\-go\-ry,
\index{.infinity1-category@$(\infty,1)$-category}%
it was very convenient to use not merely a \emph{set} of objects with spaces of morphisms between them, but a \emph{space} of objects incorporating all the equivalences and homotopies between them.
This yields a very well-behaved sort of model for $(\infty,1)$-categories as particular simplicial spaces, which Rezk called \emph{complete Segal spaces}.
\index{complete!Segal space}%
\index{Segal!space}%
One especially good aspect of this model is the analogue of \cref{ct:eqv-levelwise}: a map of complete Segal spaces is an equivalence just when it is a levelwise equivalence of simplicial spaces.

When interpreted in Voevodsky's simplicial\index{simplicial!sets} set model of univalent foundations, our precategories are similar to a truncated analogue of Rezk's ``Segal spaces'', while our categories correspond to his ``complete Segal spaces''.
\index{Segal!category}%
Strict categories correspond instead to (a weakened and truncated version of) what are called ``Segal categories''.
It is known that Segal categories and complete Segal spaces are equivalent models for $(\infty,1)$-categories (see e.g.~\cite{bergner:infty-one}), so that in the simplicial set model, categories and strict categories yield ``equivalent'' category theories---although as we have seen, the former still have many advantages.
However, in the more general categorical semantics of a higher topos,
\index{.infinity1-topos@$(\infty,1)$-topos}%
a strict category corresponds to an internal category (in the traditional sense) in the corresponding 1-topos\index{topos} of sheaves, while a category corresponds to a \emph{stack}.
\index{stack}%
The latter are generally a more appropriate sort of ``category'' relative to a topos.

In Rezk's context, what we have called the ``Rezk completion'' corresponds to fibrant replacement
\index{fibrant replacement}
in the model category for complete Segal spaces.
Since this is built using a transfinite induction argument, it most closely matches our second construction as a higher inductive type.
However, in higher topos models of homotopy type theory, the Rezk completion corresponds to \emph{stack completion},\index{completion!stack}\index{stack!completion} which can be constructed either with a transfinite induction~\cite{jt:strong-stacks} or using a Yoneda embedding \cite{bunge:stacks-morita-internal}.


\sectionExercises

\begin{ex}\label{ex:slice-precategory}
  For a precategory $A$ and $a:A$, define the \define{slice precategory} $A/a$.
  \indexsee{precategory!slice}{category, slice}%
  \indexsee{slice (pre)category}{category, slice}%
  Show that if $A$ is a category, so is $A/a$.
  \indexdef{category!slice}%
\end{ex}

\begin{ex}\label{ex:set-slice-over-equiv-functor-category}
  For any set $X$, prove that the slice category $\uset/X$ is equivalent to the functor category $\uset^X$, where in the latter case we regard $X$ as a discrete category.
\end{ex}

\begin{ex}\label{ex:functor-equiv-right-adjoint}
  \index{adjoint!functor}%
  \index{adjoint!equivalence}%
  Prove that a functor is an equivalence of categories if and only if it is a \emph{right} adjoint whose unit and counit are isomorphisms.
\end{ex}

\begin{ex}\label{ct:pre2cat}
  Define the notion of \define{pre-2-category}.
  \indexdef{pre-2-category}%
  Show that precategories, functors, and natural transformations as defined in \cref{sec:transfors} form a pre-2-category.
  Similarly, define a \define{pre-bicategory}
  \indexdef{pre-bicategory}%
  by replacing the equalities (such as those in \cref{ct:functor-assoc,ct:units}) with natural isomorphisms satisfying analogous coherence conditions.
  Define a function from pre-2-categories to pre-bicategories, and show that it becomes an equivalence when restricted and corestricted to those whose hom-pre\-cat\-egories are categories.
\end{ex}

\begin{ex}\label{ct:2cat}
  Define a \define{2-category}
  \indexdef{2-category}%
  to be a pre-2-category satisfying a condition analogous to that of \cref{ct:category}.
  Verify that the pre-2-category of categories \ucat is a 2-category.
  How much of this chapter can be done internally to an arbitrary 2-category?
\end{ex}

\begin{ex}\label{ct:groupoids}
  Define a 2-category whose objects are 1-types, whose morphisms are functions, and whose 2-morphisms are homotopies.
  Prove that it is equivalent, in an appropriate sense, to the full sub-2-category of \ucat spanned by the \emph{groupoids} (categories in which every arrow is an isomorphism).
\end{ex}

\begin{ex}\label{ex:2strict-cat}
  \index{strict!category}%
  Recall that a \emph{strict category} is a precategory whose type of objects is a set.
  Prove that the pre-2-category of strict categories is equivalent to the following pre-2-category.
  \begin{itemize}
  \item Its objects are categories $A$ equipped with a surjection
    % \footnote{Recall that a function $f:X\to Y$ is a \emph{surjection} if for every $y:Y$, there \emph{merely exists} an $x:X$ such that $f(x)=y$.  This is to be distinguished from a \emph{split surjection}, which has the property that for every $y:Y$ there \emph{exists} an $x:X$ such that $f(x)=y$.}
    $p_A:A_0'\to A_0$, where $A_0'$ is a set.
  \item Its morphisms are functors $F:A\to B$ equipped with a function $F_0':A_0' \to B_0'$ such that $p_B \circ F_0' = F_0 \circ p_A$.
  \item Its 2-morphisms are simply natural transformations.
  \end{itemize}
\end{ex}

\begin{ex}\label{ex:pre2dagger-cat}
  Define the pre-2-category of $\dagger$-categories, which has $\dagger$-struc\-tures on its hom-pre\-cat\-egories.
  Show that two $\dagger$-categories are equal precisely when they are ``unitarily equivalent'' in a suitable sense.
\end{ex}

\begin{ex}\label{ct:ex:hocat}
  Prove that a function $X\to Y$ is an equivalence if and only if its image in the homotopy category of \cref{ct:hocat} is an isomorphism.
  Show that the type of objects of this category is $\trunc1\type$.
\end{ex}

\begin{ex}\label{ex:dagger-rezk}
  Construct the $\dagger$-Rezk completion of a $\dagger$-precategory into a $\dagger$-category, and give it an appropriate universal property.
\end{ex}

\begin{ex}\label{ex:rezk-vankampen}
  \index{van Kampen theorem}%
  \index{theorem!van Kampen}%
  \index{fundamental!groupoid}%
  \index{fundamental!pregroupoid}%
  Using fundamental (pre)groupoids from \cref{ct:fundgpd,ct:rezk-fundgpd-trunc1} and the Rezk completion from \cref{sec:rezk}, give a different proof of van Kampen's theorem (\cref{sec:van-kampen}).
\end{ex}

\begin{ex}\label{ex:stack}
  Let $X$ and $Y$ be sets and $p:Y\to X$ a surjection.
  \begin{enumerate}
  \item Define, for any precategory $A$, the category $\mathrm{Desc}(A,p)$ of \define{descent data}
    \indexdef{descent data}%
    in $A$ relative to $p$.
  \item Show that any precategory $A$ is a \define{prestack}
    \indexdef{prestack}%
    for $p$, i.e.\ the canonical functor $A^X \to \mathrm{Desc}(A,p)$ is fully faithful.
  \item Show that if $A$ is a category, then it is a \define{stack}
    \indexdef{stack}%
    for $p$, i.e.\ $A^X \to \mathrm{Desc}(A,p)$ is an equivalence.
  \item Show that the statement ``every strict category is a stack for every surjection of sets'' is equivalent to the axiom of choice.
    \index{axiom!of choice}%
    \index{strict!category}%
  \end{enumerate}
\end{ex}

% Local Variables:
% TeX-master: "hott-online"
% End:


\chapter{Set theory}
\label{cha:set-math}

\index{set|(}%

Our conception of sets as types with particularly simple homotopical character, cf.\
\cref{sec:basics-sets}, is quite different from the sets of Zermelo--Fraenkel\index{set theory!Zermelo--Fraenkel} set theory, which form a
cumulative hierarchy with an intricate nested membership structure.
For many mathematical purposes, the homotopy-the\-o\-ret\-ic sets are just as good as
the Zermelo--Fraenkel ones, but there are important differences.

We begin this chapter in \cref{sec:piw-pretopos} by showing that the category $\uset$ has (most of) the usual properties of the category of sets.
\index{mathematics!constructive}%
\index{mathematics!predicative}%
In constructive, predicative, univalent foundations, it is a ``$\Pi\mathsf{W}$-pretopos''; whereas if we assume propositional resizing
\index{propositional!resizing}%
(\cref{subsec:prop-subsets}) it is an elementary topos,\index{topos} and if we assume \LEM{} and \choice{} then it is a model of Lawvere's \emph{Elementary Theory of the Category of Sets}\index{Lawvere}.
\index{Elementary Theory of the Category of Sets}%
This is sufficient to ensure that the sets in homotopy type theory behave like sets as used by most mathematicians outside of set theory.

In the rest of the chapter, we investigate some subjects that traditionally belong to ``set theory''.
In \cref{sec:cardinals,sec:ordinals,sec:wellorderings} we study cardinal and ordinal numbers.
These are traditionally defined in set theory using the global membership relation, but we will see that the univalence axiom enables an equally convenient, more ``structural'' approach.

Finally, in \cref{sec:cumulative-hierarchy} we consider the possibility of constructing \emph{inside} of homotopy type theory a cumulative hierarchy of sets, equipped with a binary membership relation akin to that of Zermelo--Fraenkel set theory.
This combines higher inductive types with ideas from the field of algebraic set theory.
\index{algebraic set theory}%
\index{set theory!algebraic}%

In this chapter we will often use the traditional logical notation described in \cref{subsec:prop-trunc}.
In addition to the basic theory of \cref{cha:basics,cha:logic}, we use higher inductive types for colimits and quotients as in \cref{sec:colimits,sec:set-quotients}, as well as some of the theory of truncation from \cref{cha:hlevels}, particularly the factorization system of \cref{sec:image-factorization} in the case $n=-1$.
In \cref{sec:ordinals} we use an inductive family (\cref{sec:generalizations}) to describe well-foundedness, and in \cref{sec:cumulative-hierarchy} we use a more complicated higher inductive type to present the cumulative hierarchy.


%\section{\texorpdfstring{$\set$}{Set} is a \texorpdfstring{$\Pi$}{Π}W-pretopos}
\section{The category of sets}
\label{sec:piw-pretopos}

Recall that in \cref{cha:category-theory} we defined the category \uset to consist of all $0$-types (in some universe \UU) and maps between them, and observed that it is a category (not just a precategory).
We consider successively the levels of structure which \uset possesses.

\subsection{Limits and colimits}
\label{subsec:limits-sets}

\index{limit!of sets}%
\index{colimit!of sets}%

Since sets are closed under products, the universal property of products in \cref{thm:prod-ump} shows immediately that \uset has finite products.
In fact, infinite products follow just as easily from the equivalence
\[ \Parens{X\to \prd{a:A} B(a)} \eqvsym \Parens{\prd{a:A} (X\to B(a))}.\]
And we saw in \cref{ex:pullback}\index{pullback} that the pullback of $f:A\to C$ and $g:B\to C$ can be defined as $\sm{a:A}{b:B} f(a)=g(b)$; this is a set if $A,B,C$ are and inherits the correct universal property.
Thus, \uset is a \emph{complete} category in the obvious sense.
\index{category!complete}%
\index{complete!category}%

Since sets are closed under $+$ and contain \emptyt, \uset has finite coproducts.
Similarly, since $\sm{a:A}B(a)$ is a set whenever $A$ and each $B(a)$ are, it yields a coproduct of the family $B$ in \uset.
Finally, we showed in \cref{sec:pushouts} that pushouts exist in $n$-types, which includes \uset in particular.
Thus, \uset is also \emph{cocomplete}.
\index{category!cocomplete}%
\index{cocomplete category}%

\subsection{Images}
\label{sec:image}

%We will show that $\uset$ is a $\Pi$W-pretopos.
Next, we show that $\uset$ is a \define{regular category}, i.e.:
\indexdef{category!regular}%
\indexdef{regular!category}%
%
\begin{enumerate}
\item $\uset$ is finitely complete.\label{item:reg1}
\item The kernel pair $\proj1,\proj2: (\sm{x,y:A} f(x)= f(y)) \to A$ of any
  function $f : A \to B$ has a coequalizer.\label{item:reg2}
  \indexdef{kernel!pair}
\item Pullbacks of regular epimorphisms are again regular epimorphisms.\label{item:reg3}
\end{enumerate}
%
Recall that a \define{regular epimorphism}
\indexdef{epimorphism!regular}%
\indexdef{regular!epimorphism}%
is a morphism that is the coequalizer of \emph{some} pair of maps.
Thus in~\ref{item:reg3} the pullback of a coequalizer is required to again be a coequalizer, but not necessarily of the pulled-back pair.

\index{set-coequalizer}%
\index{image}
The obvious candidate for the coequalizer of the kernel pair of $f:A\to B$ is the \emph{image} of $f$, as defined in \cref{sec:image-factorization}.
Recall that we defined $\im(f)\defeq \sm{b:B} \brck{\hfib f b}$, with functions
$\tilde{f}:A\to\im(f)$ and $i_f:\im(f)\to B$ defined by
\begin{align*}
  \tilde{f} & \defeq \lam{a} \Pairr{f(a),\,\bproj{\pairr{a,\refl{f(a)}}}}\\
i_f & \defeq \proj1
\end{align*}
fitting into a diagram:
\begin{equation*}
  \xymatrix{
    **[l]{\sm{x,y:A} f(x)= f(y)}
    \ar@<0.25em>[r]^{\proj1}
    \ar@<-0.25em>[r]_{\proj2}
    &
    {A}
    \ar[r]^(0.4){\tilde{f}}
    \ar[rd]_{f}
    &
    {\im(f)}
    \ar@{..>}[d]^{i_f}
    \\ & &
    B
  }
\end{equation*}

Recall that a function $f:A\to B$ is called \emph{surjective} if
\index{function!surjective}%
\narrowequation{\fall{b:B}\brck{\hfib f b},}
or equivalently $\fall{b:B} \exis{a:A} f(a)=b$.
We have also said that a function $f:A\to B$ between sets is called \emph{injective} if
\index{function!injective}%
$\fall{a,a':A} (f(a) = f(a')) \Rightarrow (a=a')$, or equivalently if each of its fibers is a mere proposition.
Since these are the $(-1)$-connected and $(-1)$-truncated maps in the sense of \cref{cha:hlevels}, the general theory there implies that $\tilde f$ above is surjective and $i_f$ is injective, and that this factorization is stable under pullback.

We now identify surjectivity and injectivity with the appropriate cat\-e\-go\-ry-theoretic notions.
First we observe that categorical monomorphisms and epimorphisms have a slightly stronger equivalent formulation.

\begin{lem}\label{thm:mono}
  For a morphism $f:\hom_A(a,b)$ in a category $A$, the following are equivalent.
  \begin{enumerate}
  \item $f$ is a \define{monomorphism}:
    \indexdef{monomorphism}%
    for all $x:A$ and ${g,h:\hom_A(x,a)}$, if $f\circ g = f\circ h$ then $g=h$.\label{item:mono1}
  \item (If $A$ has pullbacks) the diagonal map $a\to a\times_b a$ is an isomorphism.\label{item:mono4}
  \item For all $x:A$ and $k:\hom_A(x,b)$, the type $\sm{h:\hom_A(x,a)} (k = f\circ h)$ is a mere proposition.\label{item:mono2}
  \item For all $x:A$ and ${g:\hom_A(x,a)}$, the type $\sm{h:\hom_A(x,a)} (f\circ g = f\circ h)$ is contractible.\label{item:mono3}
  \end{enumerate}
\end{lem}
\begin{proof}
  The equivalence of conditions~\ref{item:mono1} and~\ref{item:mono4} is standard category theory.
  Now consider the function $(f\circ \blank ):\hom_A(x,a) \to \hom_A(x,b)$ between sets.
  Condition~\ref{item:mono1} says that it is injective, while~\ref{item:mono2} says that its fibers are mere propositions; hence they are equivalent.
  And~\ref{item:mono2} implies~\ref{item:mono3} by taking $k\defeq f\circ g$ and recalling that an inhabited mere proposition is contractible.
  Finally,~\ref{item:mono3} implies~\ref{item:mono1} since if $p:f\circ g= f\circ h$, then $(g,\refl{})$ and $(h,p)$ both inhabit the type in~\ref{item:mono3}, hence are equal and so $g=h$.
\end{proof}

\begin{lem}
  A function $f:A\to B$ between sets is injective if and only if it is a monomorphism\index{monomorphism} in \uset.
\end{lem}
\begin{proof}
  Left to the reader.
\end{proof}

Of course, an \define{epimorphism}
\indexdef{epimorphism}
\indexsee{epi}{epimorphism}
is a monomorphism in the opposite category.
We now show that in \uset, the epimorphisms are precisely the surjections, and also precisely the coequalizers (regular epimorphisms).

The coequalizer of a pair of maps $f,g:A\to B$ in $\uset$ is defined as the 0-truncation of a general (homotopy) coequalizer.
For clarity, we may call this the \define{set-coequalizer}.
\indexdef{set-coequalizer}%
\indexsee{coequalizer!of sets}{set-coequalizer}%
It is convenient to express its universal property as follows.

\begin{lem}
\index{universal!property!of set-coequalizer}%
Let $f,g:A\to B$ be functions between sets $A$ and $B$. The
{set-co}equalizer $c_{f,g}:B\to Q$ has the property that, for any set $C$ and any $h:B\to C$ with $h\circ f = h\circ g$, the type
\begin{equation*}
\sm{k:Q\to C} (k\circ c_{f,g} = h)
\end{equation*}
is contractible.
\end{lem}

\begin{lem}\label{epis-surj}
For any function $f:A\to B$ between sets, the following are equivalent:
\begin{enumerate}
\item $f$ is an epimorphism.
\item Consider the pushout diagram
\begin{equation*}
  \xymatrix{
    {A}
    \ar[r]^{f}
    \ar[d]
    &
    {B}
    \ar[d]^{\iota}
    \\
    {\unit}
    \ar[r]_{t}
    &
    {C_f}
  }
\end{equation*}
in $\uset$ defining the mapping cone\index{cone!of a function}. Then the type $C_f$ is contractible.
\item $f$ is surjective.
\end{enumerate}
\end{lem}

\begin{proof}
Let $f:A\to B$ be a function between sets, and suppose it to be an epimorphism; we show $C_f$ is contractible.
The constructor $\unit\to C_f$ of $C_f$ gives us an element $t:C_f$.
We have to show that
\begin{equation*}
\prd{x:C_f} x= t.
\end{equation*}
Note that $x= t$ is a mere proposition, hence we can use induction on $C_f$.
Of course when $x$ is $t$ we have $\refl{t}:t=t$, so it suffices to find
\begin{align*}
I_0 & : \prd{b:B} \iota(b)= t\\
I_1 & : \prd{a:A} \opp{\alpha_1(a)} \ct I_0(f(a))=\refl{t}.
\end{align*}
where $\iota:B\to C_f$ and $\alpha_1:\prd{a:A} \iota(f(a))= t$ are the other constructors
of $C_f$. Note that $\alpha_1$ is a homotopy from $\iota\circ f$ to
$\mathsf{const}_t\circ f$, so we find the elements
\begin{equation*}
\pairr{\iota,\refl{\iota\circ f}},\pairr{\mathsf{const}_t,\alpha_1}:
\sm{h:B\to C_f} \iota\circ f \htpy h\circ f.
\end{equation*}
By the dual of \cref{thm:mono}\ref{item:mono3} (and function extensionality), there is a path
\begin{equation*}
\gamma:\pairr{\iota,\refl{\iota\circ f}}=\pairr{\mathsf{const}_t,\alpha_1}.
\end{equation*}
Hence, we may define $I_0(b)\defeq \happly(\projpath1(\gamma),b):\iota(b)=t$.
We also have
\[\projpath2(\gamma) : \trans{\projpath1(\gamma)}{\refl{\iota\circ f}} = \alpha_1. \]
This transport involves precomposition with $f$, which commutes with $\happly$.
Thus, from transport in path types we obtain $I_0(f(a)) = \alpha_1(a)$ for any $a:A$, which gives us $I_1$.

Now suppose $C_f$ is contractible; we show $f$ is surjective.
We first construct a type family $P:C_f\to\prop$ by recursion on $C_f$, which is valid since \prop is a set.
On the point constructors, we define
\begin{align*}
P(t) & \defeq \unit\\
P(\iota(b)) & \defeq \brck{\hfiber{f}b}.
\end{align*}
To complete the construction of $P$, it remains to give a path
\narrowequation{\brck{\hfiber{f}{f(a)}} =_\prop \unit}
for all $a:A$.
However, $\brck{\hfiber{f}{f(a)}}$ is inhabited by $(f(a),\refl{f(a)})$.
Since it is a mere proposition, this means it is contractible --- and thus equivalent, hence equal, to \unit.
This completes the definition of $P$.
Now, since $C_f$ is assumed to be contractible, it follows that $P(x)$ is equivalent to $P(t)$ for any $x:C_f$.
In particular, $P(\iota(b))\jdeq \brck{\hfiber{f}b}$ is equivalent to $P(t)\jdeq \unit$ for each $b:B$, and hence contractible.
Thus, $f$ is surjective.

Finally, suppose $f:A\to B$ to be surjective, and consider a set $C$ and two functions
$g,h:B\to C$ with the property that $g\circ f = h\circ f$. Since $f$
is assumed to be surjective, for all $b:B$ the type $\brck{\hfib f b}$ is contractible.
Thus we have the following equivalences:
\begin{align*}
\prd{b:B} (g(b)= h(b))
& \eqvsym \prd{b:B} \Parens{\brck{\hfib f b} \to (g(b)= h(b))}\\
& \eqvsym \prd{b:B} \Parens{\hfib f b \to (g(b)= h(b))}\\
& \eqvsym \prd{b:B}{a:A}{p:f(a)= b} g(b)= h(b)\\
& \eqvsym \prd{a:A} g(f(a))= h(f(a))
\end{align*}
using on the second line the fact that $g(b)=h(b)$ is a mere proposition, since $C$ is a set.
But by assumption, there is an element of the latter type.
\end{proof}

% \begin{rem}
% The above theorem is not true when we replace $\set$ by $\type$
% (replacing it also in the definition of $\mathsf{epi}$ and $\mathsf{epi}'$).
% However, we do
% get the implications $\textit{ii.}\Rightarrow\textit{iii.}\Rightarrow
% \textit{iv.}$
% \end{rem}

\begin{thm}\label{thm:set_regular}\label{lem:images_are_coequalizers}
The category $\uset$ is regular. Moreover, surjective functions between sets are regular epimorphisms.
\end{thm}

\begin{proof}
It is a standard lemma in category theory that a category is regular as soon as it admits finite limits and a pullback-stable orthogonal
factorization system\index{orthogonal factorization system} $(\mathcal{E},\mathcal{M})$ with $\mathcal{M}$ the monomorphisms, in which case $\mathcal{E}$ consists automatically of
the regular epimorphisms.
(See e.g.~\cite[A1.3.4]{elephant}.)
The existence of the factorization system was proved in \cref{thm:orth-fact}.
\end{proof}

\begin{lem}\label{lem:pb_of_coeq_is_coeq}
Pullbacks of regular epis in \uset are regular epis.
\end{lem}
\begin{proof}
  We showed in \cref{thm:stable-images} that pullbacks of $n$-connected functions are $n$-connected.
  By \cref{lem:images_are_coequalizers}, it suffices to apply this when $n=-1$.
\end{proof}

\indexdef{image!of a subset}
One of the consequences of \uset being a regular category is that we have an ``image'' operation on subsets.
That is, given $f:A\to B$, any subset $P:\power A$ (i.e.\ a predicate $P:A\to \prop$) has an \define{image} which is a subset of $B$.
This can be defined directly as $\setof{ y:B | \exis{x:A} f(x)=y \land P(x)}$, or indirectly as the image (in the previous sense) of the composite function
\[ \setof{ x:A | P(x) } \to A \xrightarrow{f} B.\]
\symlabel{subset-image}
We will also sometimes use the common notation $\setof{f(x) | P(x)}$ for the image of $P$.


\subsection{Quotients}\label{subsec:quotients}

\index{set-quotient|(}%
Now that we know that $\uset$ is regular, to show that $\uset$ is exact, we need to show that every
equivalence relation is effective.
\index{effective!equivalence relation|(}%
\index{relation!effective equivalence|(}%
In other words, given an equivalence
relation $R:A\to A\to\prop$, there is a coequalizer $c_R$ of the pair
$\proj1,\proj2:\sm{x,y:A} R(x,y)\to A$ and, moreover, the $\proj1$ and $\proj2$
form the kernel\index{kernel!pair} pair of $c_R$.

We have already seen, in \cref{sec:set-quotients}, two general ways to construct the quotient of a set by an equivalence relation $R:A\to A\to\prop$.
The first can be described as the set-coequalizer of the two projections
\[\proj1,\proj2:\Parens{\sm{x,y:A} R(x,y)} \to A.\]
The important property of such a quotient is the following.

\begin{defn}
  A relation $R:A\to A\to\prop$ is said to be \define{effective}
  \indexdef{effective!relation}
  \indexdef{effective!equivalence relation}%
  \indexdef{relation!effective equivalence}%
  if the square
\begin{equation*}
  \xymatrix{
    {\sm{x,y:A} R (x,y)}
    \ar[r]^(0.7){\proj1}
    \ar[d]_{\proj2}
    &
    {A}
    \ar[d]^{c_R}
    \\
    {A}
    \ar[r]_{c_R}
    &
    {A/R}
    }
\end{equation*}
is a pullback.
\end{defn}

Since the standard pullback of $c_R$ and itself is $\sm{x,y:A} (c_R(x)=c_R(y))$, by \cref{thm:total-fiber-equiv} this is equivalent to asking that the canonical transformation $\prd{x,y:A} R(x,y) \to (c_R(x)=c_R(y))$ be a fiberwise equivalence.

\begin{lem}\label{lem:sets_exact}
Suppose $\pairr{A,R}$ is an equivalence relation. Then there is an equivalence
\begin{equation*}
(c_R(x)= c_R(y))\eqvsym R(x,y)
\end{equation*}
for any $x,y:A$. In other words, equivalence relations are effective.
\end{lem}

\begin{proof}
We begin by extending $R$ to a relation $\widetilde{R}:A/R\to A/R\to\prop$, which we will then show is equivalent
to the identity type on $A/R$. We define $\widetilde{R}$ by double induction on
$A/R$ (note that $\prop$ is a set by univalence for mere propositions). We
define $\widetilde{R}(c_R(x),c_R(y)) \defeq R(x,y)$. For $r:R(x,x')$ and $s:R(y,y')$,
the transitivity and symmetry
of $R$ gives an equivalence from $R(x,y)$ to $R(x',y')$. This completes the
definition of $\widetilde{R}$.

It remains to show that $\widetilde{R}(w,w')\eqvsym (w= w')$ for every $w,w':A/R$.
The direction $(w=w')\to \widetilde{R}(w,w')$ follows by transport once we show that $\widetilde{R}$ is reflexive, which is an easy induction.
The other direction $\widetilde{R}(w,w')\to (w= w')$ is a mere proposition, so since $c_R:A\to A/R$ is surjective, it suffices to assume that $w$ and $w'$ are of the form $c_R(x)$ and $c_R(y)$.
But in this case, we have the canonical map $\widetilde{R}(c_R(x),c_R(y)) \defeq R(x,y) \to (c_R(x)=c_R(y))$.
(Note again the appearance of the encode-decode method.\index{encode-decode method})
\end{proof}

The second construction of quotients is as the set of equivalence classes of $R$ (a subset
of its power set\index{power set}):
\[ A\sslash R \defeq \setof{ P:A\to\prop | P \text{ is an equivalence class of } R} \]
This requires propositional resizing\index{propositional resizing}\index{impredicative!quotient}\index{resizing} in order to remain in the same universe as $A$ and $R$.

Note that if we regard $R$ as a function from $A$ to $A\to \prop$, then $A\sslash R$ is equivalent to $\im(R)$, as constructed in \cref{sec:image}.
Now in \cref{lem:images_are_coequalizers} we have shown that images are
coequalizers. In particular, we immediately get the coequalizer diagram
\begin{equation*}
  \xymatrix{
    **[l]{\sm{x,y:A} R (x)= R (y)}
    \ar@<0.25em>[r]^{\proj1}
    \ar@<-0.25em>[r]_{\proj2}
    &
    {A}
    \ar[r]
    &
    {A \sslash R.}
  }
\end{equation*}
We can use this to give an alternative proof that any equivalence relation is effective and that the two definitions of quotients agree.

\begin{thm}\label{prop:kernels_are_effective}
For any function $f:A\to B$ between any two sets,
the relation $\ker(f):A\to A\to\prop$ given by
$\ker(f,x,y)\defeq (f(x)= f(y))$ is effective.
\end{thm}

\begin{proof}
We will use that $\im(f)$ is the coequalizer of $\proj1,\proj2:
(\sm{x,y:A} f(x)= f(y))\to A$.
%we get this equivalence from~\cref{prop:images_are_coequalizers}
Note that the kernel pair of the function
\[c_f\defeq\lam{a} \Parens{f(a),\brck{\pairr{a,\refl{f(a)}}}}
: A \to \im(f)
\]
consists of the two projections
\begin{equation*}
\proj1,\proj2:\Parens{\sm{x,y:A} c_f(x)= c_f(y)}\to A.
\end{equation*}
For any $x,y:A$, we have equivalences
\begin{align*}
  (c_f(x)= c_f(y))
  & \eqvsym \Parens{\sm{p:f(x)= f(y)} \trans{p}{\brck{\pairr{x,\refl{f(x)}}}} =\brck{\pairr{y,\refl{f(x)}}}}\\
  & \eqvsym (f(x)= f(y)),
\end{align*}
where the last equivalence holds because
$\brck{\hfiber{f}b}$ is a mere proposition for any $b:B$.
Therefore, we get that
\begin{equation*}
\Parens{\sm{x,y:A} c_f(x)= c_f(y)}\eqvsym \Parens{\sm{x,y:A} f(x)= f(y)}
\end{equation*}
and hence we may conclude that $\ker f$ is an effective relation
for any function $f$.
\end{proof}

\begin{thm}
Equivalence relations are effective and there is an equivalence $A/R \eqvsym A\sslash  R $.
\end{thm}

\begin{proof}
We need to analyze the coequalizer diagram
\begin{equation*}
  \xymatrix{
    **[l]{\sm{x,y:A} R (x)= R (y)}
    \ar@<0.25em>[r]^{\proj1}
    \ar@<-0.25em>[r]_{\proj2}
    &
    {A}
    \ar[r]
    &
    {A \sslash R}
  }
\end{equation*}
By the univalence axiom, the type $R(x) = R(y)$ is equivalent to the type of homotopies from $R(x)$ to $R(y)$, which is
equivalent to
\narrowequation{\prd{z:A} R (x,z)\eqvsym R (y,z).}
Since $R$ is an equivalence relation, the latter space is equivalent to $R(x,y)$. To
summarize, we get that $(R(x) = R(y)) \eqvsym R(x,y)$, so $R $ is effective since it is equivalent to an effective relation. Also,
the diagram
\begin{equation*}
  \xymatrix{
    **[l]{\sm{x,y:A} R(x, y)}
    \ar@<0.25em>[r]^{\proj1}
    \ar@<-0.25em>[r]_{\proj2}
    &
    {A}
    \ar[r]
    &
    {A \sslash R.}
  }
\end{equation*}
is a coequalizer diagram. Since coequalizers are unique up to equivalence, it follows that $A/R \eqvsym A\sslash  R $.
\end{proof}

We finish this section by mentioning a possible third construction of the quotient of a set $A$ by an equivalence relation $R$.
Consider the precategory with objects $A$ and hom-sets $R$; the type of objects of the Rezk completion
\index{completion!Rezk}%
(see \cref{sec:rezk}) of this precategory will then be the
quotient. The reader is invited to check the details.

\index{effective!equivalence relation|)}%
\index{relation!effective equivalence|)}%
\index{set-quotient|)}%

\subsection{\texorpdfstring{$\uset$}{Set} is a \texorpdfstring{$\Pi\mathsf{W}$}{ΠW}-pretopos}
\label{subsec:piw}

\index{structural!set theory|(}%

The notion of a \emph{$\Pi\mathsf{W}$-pretopos}
\index{PiW-pretopos@$\Pi\mathsf{W}$-pretopos}%
\indexsee{pretopos}{$\Pi\mathsf{W}$-pretopos}
--- that is, a locally cartesian closed category
\index{locally cartesian closed category}%
\index{category!locally cartesian closed}%
with disjoint finite coproducts, effective equivalence relations, and initial algebras for polynomial endofunctors --- is intended as a ``predicative''
\index{mathematics!predicative}%
notion of topos, i.e.\ a category of ``predicative sets'', which can serve the purpose for constructive mathematics
\index{mathematics!constructive}%
that the usual category of sets does for classical
\index{mathematics!classical}%
mathematics.

Typically, in constructive type theory, one resorts to an external construction of ``setoids'' --- an exact completion --- to obtain a category with such closure properties.
\index{setoid}\index{completion!exact}%
  In particular, the well-behaved quotients are required for many constructions in mathematics that usually involve (non-constructive) power sets.  It is noteworthy that univalent foundations provides these constructions \emph{internally} (via higher inductive types), without requiring such external constructions.  This represents a powerful advantage of our approach, as we shall see in subsequent examples.

\begin{thm}
  \index{PiW-pretopos@$\Pi\mathsf{W}$-pretopos}
  The category $\uset$ is a $\Pi\mathsf{W}$-pretopos.
\end{thm}
\begin{proof}
  We have an initial object
  \index{initial!set}%
  $\emptyt$ and finite, disjoint sums $A+B$.  These are stable under pullback, simply because pullback has a right adjoint\index{adjoint!functor}.  Indeed, $\uset$ is locally cartesian closed, since for any map $f:A\to B$ between sets, the ``fibrant replacement'' \index{fibrant replacement} $\sm{a:A}f(a)=b$ is equivalent to $A$ (over $B$), and we have dependent function types for the replacement.
We've just shown that $\uset$ is regular (\cref{thm:set_regular}) and that quotients are effective (\cref{lem:sets_exact}). We thus have a locally cartesian closed pretopos. Finally, since the $n$-types are closed under the formation of $W$-types by \cref{ex:ntypes-closed-under-wtypes}, and by \cref{thm:w-hinit} $W$-types are initial algebras for polynomial endofunctors, we see that $\uset$ is a $\Pi\mathsf{W}$-pretopos.
\end{proof}


\index{topos|(}
One naturally wonders what, if anything, prevents $\uset$ from being an (elementary) topos?
In addition to the structure already mentioned, a topos has a
\emph{subobject classifier}:
\indexdef{subobject classifier}%
\index{classifier!subobject}%
\index{power set}%
a pointed object classifying (equivalence classes of) monomorphisms\index{monomorphism}.  (In fact, in the presence of a subobject
classifier, things become somewhat simpler: one merely needs cartesian closure in order to get the colimits.)
In homotopy type theory, univalence implies that the type $\prop \defeq \sm{X:\UU}\isprop(X)$ does classify monomorphisms (by an argument similar to \cref{sec:object-classification}), but in general it is as large as the ambient universe $\UU$.
Thus, it is a ``set'' in the sense of being a $0$-type, but it is not ``small'' in the sense of being an object of $\UU$, hence not an object of the category \uset.
However, if we assume an appropriate form of propositional resizing (see \cref{subsec:prop-subsets}), then we can find a small version of $\prop$, so that \uset becomes an elementary topos.

\begin{thm}\label{thm:settopos}
  \index{propositional!resizing}%
  If there is a type $\Omega:\UU$ of all mere propositions, then the category $\uset_\UU$ is an elementary topos.
\end{thm}
\index{topos|)}

A sufficient condition for this is the law of excluded middle, in the ``mere-propositional'' form that we have called \LEM{}; for then we have $\prop = \bool$, which \emph{is} small, and which then also classifies all mere propositions.
Moreover, in topos theory a well-known sufficient condition for \LEM{} is the axiom of choice, which is of course often assumed as an axiom in classical\index{mathematics!classical} set theory.
In the next section, we briefly investigate the relation between these conditions in our setting.

\index{structural!set theory|)}%

%%%%%%%%%%%%%%%%%%%%%%%%%%%%%%%%%%%%%%%%%%%%%%%%
\subsection{The axiom of choice implies excluded middle}
\label{subsec:emacinsets}

% In this section we prove a classic result that the axiom of choice implies excluded
% middle.

We begin with the following lemma.

\begin{lem}\label{prop:trunc_of_prop_is_set}
If $A$ is a mere proposition then its suspension $\susp(A)$ is a set,
and $A$ is equivalent to $\id[\susp(A)]{\north}{\south}$.
\end{lem}

\begin{proof}
To show that $\susp(A)$ is a set, we define a
family $P:\susp(A)\to\susp(A)\to\type$ with the
property that $P(x,y)$ is a mere proposition for each $x,y:\susp(A)$,
and which is equivalent to its identity type $\idtypevar{\susp(A)}$.
%
We make the following definitions:
\begin{align*}
P(\north,\north) & \defeq \unit &
P(\south,\north) & \defeq A\\
P(\north,\south) & \defeq A &
P(\south,\south) & \defeq \unit.
\end{align*}
We have to check that the definition preserves paths.
Given any $a : A$, there is a meridian $\merid(a) : \north = \south$,
so we should also have
%
\begin{equation*}
  P(\north, \north) = P(\north, \south) = P(\south, \north) = P(\south, \south).
\end{equation*}
%
But since $A$ is inhabited by $a$, it is equivalent to $\unit$, so we have
%
\begin{equation*}
  P(\north, \north) \eqvsym P(\north, \south) \eqvsym P(\south, \north) \eqvsym P(\south, \south).
\end{equation*}
%
The univalence axiom turns these into the desired equalities. Also, $P(x,y)$ is a mere
proposition for all $x, y : \susp(A)$, which is proved by induction on $x$ and $y$, and
using the fact that being a mere proposition is a mere proposition.

Note that $P$ is a reflexive relation.
Therefore we may apply \cref{thm:h-set-refrel-in-paths-sets}, so it suffices to
construct $\tau : \prd{x,y:\susp(A)}P(x,y)\to(x=y)$. We do this by a double induction.
When $x$ is $\north$, we define $\tau(\north)$ by
%
\begin{equation*}
  \tau(\north,\north,u) \defeq \refl{\north}
  \qquad\text{and}\qquad
  \tau(\north,\south,a) \defeq \merid(a).
\end{equation*}
%
If $A$ is inhabited by $a$ then $\merid(a) : \north = \south$ so we also need
\narrowequation{
  \trans{\merid(a)}{\tau(\north, \north)} = \tau(\north, \south).
}
This we get by function extensionality using the fact that, for all $x : A$,
%
\begin{multline*}
  \trans{\merid(a)}{\tau(\north,\north,x)} =
  \tau(\north,\north,x) \ct \opp{\merid(a)} \jdeq \\
  \refl{\north} \ct \merid(a) =
  \merid(a) =
  \merid(x) \jdeq
  \tau(\north, \south, x).
\end{multline*}
In a symmetric fashion we may define $\tau(\south)$ by
%
\begin{equation*}
  \tau(\south,\north, a) \defeq \opp{\merid(a)}
  \qquad\text{and}\qquad
  \tau(\south,\south, u) \defeq \refl{\south}.
\end{equation*}
%
To complete the construction of $\tau$, we need to check $\trans{\merid(a)}{\tau(\north)} = \tau(\south)$,
given any $a : A$. The verification proceeds much along the same lines by induction on the
second argument of $\tau$.

Thus, by \cref{thm:h-set-refrel-in-paths-sets} we have that $\susp(A)$ is a set and that $P(x,y) \eqvsym (\id{x}{y})$ for all $x,y:\susp(A)$.
Taking $x\defeq \north$ and $y\defeq \south$ yields $\eqv{A}{(\id[\susp(A)]\north\south)}$ as desired.
\end{proof}

\begin{thm}[Diaconescu]\label{thm:1surj_to_surj_to_pem}
  \index{axiom!of choice}%
  \index{excluded middle}%
  \index{Diaconescu's theorem}\index{theorem!Diaconescu's}%
  The axiom of choice implies the law of excluded middle.
\end{thm}

\begin{proof}
We use the equivalent form of choice given in \cref{thm:ac-epis-split}.
Consider a mere proposition $A$.
The function $f:\bool\to\susp(A)$ defined by
$f(\bfalse) \defeq \north$ and $f(\btrue) \defeq \south$
is surjective.
Indeed, we have
$\pairr{\bfalse,\refl{\north}} : \hfiber{f}{\north}$
and $\pairr{\btrue,\refl{\south}} :\hfiber{f}{\south}$.
Since $\bbrck{\hfiber{f}{x}}$ is a mere proposition, by induction the claimed surjectivity follows.

By \cref{prop:trunc_of_prop_is_set} the suspension $\susp(A)$
is a set, so by the axiom of choice there merely exists a
section $g: \susp(A) \to \bool$ of $f$.
As equality on $\bool$ is decidable we get
\begin{equation*}
 (g(f(\bfalse))= g(f(\btrue))) +
 \lnot (g(f(\bfalse))= g(f(\btrue))),
\end{equation*}
and, since $g$ is a section of $f$, hence injective,
\begin{equation*}
(f(\bfalse) = f(\btrue)) +
\lnot (f(\bfalse) = f(\btrue)).
\end{equation*}
Finally, since $(f(\bfalse)=f(\btrue)) = (\north=\south) = A$ by \cref{prop:trunc_of_prop_is_set}, we have $A+\neg A$.
\end{proof}

% This conclusion needs only \LEM{}, see \cref{ex:lemnm}.

% \begin{cor}\label{cor:ACtoLEM0}
%   If the axiom of choice \choice{} holds then $\brck{A + \neg A}$ for every set $A$.
% \end{cor}

% \begin{proof}
%   There is a surjection
%   \[
%   A + \neg A \epi \brck{A} + \brck{\neg A} \epi
%   \brck{(\brck{A} + \brck{\neg A})} = \brck{A} \vee \brck{\neg A} = \brck{A} \vee \neg \brck{A} = \unit,
%   \]
%   %
%   where in the last step excluded middle is available as a consequence of the axiom of choice.
%   Again by the axiom of choice there merely exists a section of the surjection, but this
%   is none other than an inhabitant of $A + \neg A$. Therefore $\brck{A+\neg A}$.
% \end{proof}

\index{denial}
\begin{thm}\label{thm:ETCS}
  \index{Elementary Theory of the Category of Sets}%
  \index{category!well-pointed}%
  If the axiom of choice holds then the category $\uset$ is a well-pointed boolean\index{topos!boolean}\index{boolean!topos} elementary topos\index{topos} with choice.
\end{thm}

\begin{proof}
  Since \choice{} implies \LEM{}, we have a boolean elementary topos with choice by \cref{thm:settopos} and the remark following it.  We leave the proof of well-pointedness as
an exercise for the reader (\cref{ex:well-pointed}).
\end{proof}

\begin{rmk}
  The conditions on a category mentioned in the theorem are known as Lawvere's\index{Lawvere}
  axioms for the \emph{Elementary Theory of the Category of Sets}~\cite{lawvere:etcs-long}.
\end{rmk}

\section{Cardinal numbers}
\label{sec:cardinals}

\begin{defn}\label{defn:card}
  The \define{type of cardinal numbers}
  \indexdef{type!of cardinal numbers}%
  \indexdef{cardinal number}%
  \indexsee{number!cardinal}{cardinal number}%
  is the 0-truncation of the type \set of sets:
  \[ \card \defeq \pizero{\set} \]
  Thus, a \define{cardinal number}, or \define{cardinal}, is an inhabitant of $\card\jdeq \pizero\set$.
  Technically, of course, there is a separate type $\card_\UU$ associated to each universe \type.
\end{defn}

%\begin{rmk}

  % , but with these conventions we can state theorems beginning with ``for all cardinal numbers\dots''\ and give them exactly the same sort of meaning as those beginning ``for all types\dots''.
%\end{rmk}

As usual for truncations, if $A$ is a set, then $\cd{A}$ denotes its image under the canonical projection $\set \to \trunc0\set \jdeq \card$; we call $\cd{A}$ the \define{cardinality}\indexdef{cardinality} of $A$.
By definition, \card is a set.
It also inherits the structure of a semiring from \set.

\begin{defn}
  The operation of \define{cardinal addition}
  \indexdef{addition!of cardinal numbers}%
  \index{cardinal number!addition of}%
  \[ (\blank+\blank) : \card \to \card \to \card \]
  is defined by induction on truncation:
  \[ \cd{A} + \cd{B} \defeq \cd{A+B}.\]
\end{defn}
\begin{proof}
  Since $\card\to\card$ is a set, to define $(\alpha+\blank):\card\to\card$ for all $\alpha:\card$, by induction it suffices to assume that $\alpha$ is $\cd{A}$ for some $A:\set$.
  Now we want to define $(\cd{A}+\blank) :\card\to\card$, i.e.\ we want to define $\cd{A}+\beta :\card$ for all $\beta:\card$.
  However, since $\card$ is a set, by induction it suffices to assume that $\beta$ is $\cd{B}$ for some $B:\set$.
  But now we can define $\cd{A}+\cd{B}$ to be $\cd{A+B}$.
\end{proof}

\begin{defn}
  Similarly, the operation of \define{cardinal multiplication}
  \indexdef{multiplication!of cardinal numbers}%
  \index{cardinal number!multiplication of}%
  \[ (\blank\cdot\blank) : \card \to \card \to \card \]
  is defined by induction on truncation:
  \[ \cd{A} \cdot \cd{B} \defeq \cd{A\times B} \]
\end{defn}

\begin{lem}\label{card:semiring}
  \card is a commutative semiring\index{semiring}, i.e.\ for $\alpha,\beta,\gamma:\card$ we have the following.
  \begin{align*}
    (\alpha+\beta)+\gamma &= \alpha+(\beta+\gamma)\\
    \alpha+0 &= \alpha\\
    \alpha + \beta &= \beta + \alpha\\
    (\alpha \cdot \beta) \cdot \gamma &= \alpha \cdot (\beta\cdot\gamma)\\
    \alpha \cdot 1 &= \alpha\\
    \alpha\cdot\beta &= \beta\cdot\alpha\\
    \alpha\cdot(\beta+\gamma) &= \alpha\cdot\beta + \alpha\cdot\gamma
  \end{align*}
  where $0 \defeq \cd{\emptyt}$ and $1\defeq\cd{\unit}$.
\end{lem}
\begin{proof}
  We prove the commutativity of multiplication, $\alpha\cdot\beta = \beta\cdot\alpha$; the others are exactly analogous.
  Since \card is a set, the type $\alpha\cdot\beta = \beta\cdot\alpha$ is a mere proposition, and in particular a set.
  Thus, by induction it suffices to assume $\alpha$ and $\beta$ are of the form $\cd{A}$ and $\cd{B}$ respectively, for some $A,B:\set$.
  Now $\cd{A}\cdot \cd{B} \jdeq \cd{A\times B}$ and $\cd{B}\times\cd{A} \jdeq \cd{B\times A}$, so it suffices to show $A\times B = B\times A$.
  Finally, by univalence, it suffices to give an equivalence $A\times B \eqvsym B\times A$.
  But this is easy: take $(a,b) \mapsto (b,a)$ and its obvious inverse.
\end{proof}

\begin{defn}
  The operation of \define{cardinal exponentiation} is also defined by induction on truncation:
  \indexdef{exponentiation, of cardinal numbers}%
  \index{cardinal number!exponentiation of}%
  \[ \cd{A}^{\cd{B}} \defeq \cd{B\to A}. \]
\end{defn}

\begin{lem}\label{card:exp}
  For $\alpha,\beta,\gamma:\card$ we have
  \begin{align*}
    \alpha^0 &= 1\\
    1^\alpha &= 1\\
    \alpha^1 &= \alpha\\
    \alpha^{\beta+\gamma} &= \alpha^\beta \cdot \alpha^\gamma\\
    \alpha^{\beta\cdot \gamma} &= (\alpha^{\beta})^\gamma\\
    (\alpha\cdot\beta)^\gamma &= \alpha^\gamma \cdot \beta^\gamma
  \end{align*}
\end{lem}
\begin{proof}
  Exactly like \cref{card:semiring}.
\end{proof}

\begin{defn}
  The relation of \define{cardinal inequality}
  \index{order!non-strict}%
  \index{cardinal number!inequality of}%
  \[ (\blank\le\blank) : \card\to\card\to\prop \]
  is defined by induction on truncation:
  \symlabel{inj}
  \[ \cd{A} \le \cd{B} \defeq \brck{\inj(A,B)} \]
  where $\inj(A,B)$ is the type of injections from $A$ to $B$.
  \index{function!injective}%
  In other words, $\cd{A} \le \cd{B}$ means that there merely exists an injection from $A$ to $B$.
\end{defn}

\begin{lem}
  Cardinal inequality is a preorder, i.e.\ for $\alpha,\beta:\card$ we have
  \index{preorder!of cardinal numbers}%
  \begin{gather*}
    \alpha \le \alpha\\
    (\alpha \le \beta) \to (\beta\le\gamma) \to (\alpha\le\gamma)
  \end{gather*}
\end{lem}
\begin{proof}
  As before, by induction on truncation.
  For instance, since $(\alpha \le \beta) \to (\beta\le\gamma) \to (\alpha\le\gamma)$ is a mere proposition, by induction on 0-truncation we may assume $\alpha$, $\beta$, and $\gamma$ are $\cd{A}$, $\cd{B}$, and $\cd{C}$ respectively.
  Now since $\cd{A} \le \cd{C}$ is a mere proposition, by induction on $(-1)$-truncation we may assume given injections $f:A\to B$ and $g:B\to C$.
  But then $g\circ f$ is an injection from $A$ to $C$, so $\cd{A} \le \cd{C}$ holds.
  Reflexivity is even easier.
\end{proof}

We may likewise show that cardinal inequality is compatible with the semiring operations.

\begin{lem}\label{thm:injsurj}
  \index{function!injective}%
  \index{function!surjective}%
  Consider the following statements:
  \begin{enumerate}
  \item There is an injection $A\to B$.\label{item:cle-inj}
  \item There is a surjection $B\to A$.\label{item:cle-surj}
  \end{enumerate}
  Then, assuming excluded middle:
  \index{excluded middle}%
  \index{axiom!of choice}%
  \begin{itemize}
  \item Given $a_0:A$, we have~\ref{item:cle-inj}$\to$\ref{item:cle-surj}.
  \item Therefore, if $A$ is merely inhabited, we have~\ref{item:cle-inj} $\to$ merely \ref{item:cle-surj}.
  \item Assuming the axiom of choice, we have~\ref{item:cle-surj} $\to$ merely \ref{item:cle-inj}.
  \end{itemize}
\end{lem}
\begin{proof}
  If $f:A\to B$ is an injection, define $g:B\to A$ at $b:B$ as follows.
  Since $f$ is injective, the fiber of $f$ at $b$ is a mere proposition.
  Therefore, by excluded middle, either there is an $a:A$ with $f(a)=b$, or not.
  In the first case, define $g(b)\defeq a$; otherwise set $g(b)\defeq a_0$.
  Then for any $a:A$, we have $a = g(f(a))$, so $g$ is surjective.

  The second statement follows from this by induction on truncation.
  For the third, if $g:B\to A$ is surjective, then by the axiom of choice, there merely exists a function $f:A\to B$ with $g(f(a)) = a$ for all $a$.
  But then $f$ must be injective.
\end{proof}

\begin{thm}[Schroeder--Bernstein]
  \index{theorem!Schroeder--Bernstein}%
  \index{Schroeder--Bernstein theorem}%
  Assuming excluded middle, for sets $A$ and $B$ we have
  \[ \inj(A,B) \to \inj(B,A) \to (A\cong B) \]
\end{thm}
\begin{proof}
  The usual ``back-and-forth'' argument applies without significant changes.
  Note that it actually constructs an isomorphism $A\cong B$ (assuming excluded middle so that we can decide whether a given element belongs to a cycle, an infinite chain, a chain beginning in $A$, or a chain beginning in $B$).
\end{proof}

\begin{cor}
  Assuming excluded middle, cardinal inequality is a partial order, i.e.\ for $\alpha,\beta:\card$ we have
  \[ (\alpha\le\beta) \to (\beta\le\alpha) \to (\alpha=\beta). \]
\end{cor}
\begin{proof}
  Since $\alpha=\beta$ is a mere proposition, by induction on truncation we may assume $\alpha$ and $\beta$ are $\cd{A}$ and $\cd{B}$, respectively, and that we have injections $f:A\to B$ and $g:B\to A$.
  But then the Schroeder--Bernstein theorem gives an isomorphism $A\eqvsym B$, hence an equality $\cd{A}=\cd{B}$.
\end{proof}

Finally, we can reproduce Cantor's theorem, showing that for every cardinal there is a greater one.

\begin{thm}[Cantor]
  \index{Cantor's theorem}%
  \index{theorem!Cantor's}%
  For $A:\set$, there is no surjection $A \to (A\to \bool)$.
\end{thm}
\begin{proof}
  Suppose $f:A \to (A\to \bool)$ is any function, and define $g:A\to \bool$ by $g(a) \defeq \neg f(a)(a)$.
  If $g = f(a_0)$, then $g(a_0) = f(a_0)(a_0)$ but $g(a_0) = \neg f(a_0)(a_0)$, a contradiction.
  Thus, $f$ is not surjective.
\end{proof}

\begin{cor}
  Assuming excluded middle, for any $\alpha:\card$, there is a cardinal $\beta$ such that $\alpha\le\beta$ and $\alpha\neq\beta$.
\end{cor}
\begin{proof}
  Let $\beta = 2^\alpha$.
  Now we want to show a mere proposition, so by induction we may assume $\alpha$ is $\cd{A}$, so that $\beta\jdeq \cd{A\to \bool}$.
  Using excluded middle, we have a function $f:A\to (A\to \bool)$ defined by
  \[f(a)(a') \defeq
  \begin{cases}
    \btrue &\quad a=a'\\
    \bfalse &\quad a\neq a'.
  \end{cases}
  \]
  And if $f(a)=f(a')$, then $f(a')(a) = f(a)(a) = \btrue$, so $a=a'$; hence $f$ is injective.
  Thus, $\alpha \jdeq \cd{A} \le \cd{A\to \bool} \jdeq 2^\alpha$.

  On the other hand, if $2^\alpha \le \alpha$, then we would have an injection $(A\to\bool)\to A$.
  By \cref{thm:injsurj}, since we have $(\lam{x} \bfalse):A\to \bool$ and excluded middle, there would then be a surjection $A \to (A\to \bool)$, contradicting Cantor's theorem.
\end{proof}

\section{Ordinal numbers}
\label{sec:ordinals}

\index{ordinal|(}%

\begin{defn}\label{defn:accessibility}
  Let $A$ be a set and
  \[(\blank<\blank):A\to A\to \prop\]
  a binary relation on $A$.
  We define by induction what it means for an element $a:A$ to be \define{accessible}
  \indexdef{accessibility}%
  \indexsee{accessible}{accessibility}%
  by $<$:
  \begin{itemize}
  \item If $b$ is accessible for every $b<a$, then $a$ is accessible.
  \end{itemize}
  We write $\acc(a)$ to mean that $a$ is accessible.
\end{defn}

It may seem that such an inductive definition can never get off the ground, but of course if $a$ has the property that there are \emph{no} $b$ such that $b<a$, then $a$ is vacuously accessible.

Note that this is an inductive definition of a family of types, like the type of vectors considered in \cref{sec:generalizations}.
More precisely, it has one constructor, say $\acc_<$, with type
\[ \acc_< : \prd{a:A} \Parens{\prd{b:A} (b<a) \to \acc(b)} \to \acc(a). \]
\index{induction principle!for accessibility}%
The induction principle for $\acc$ says that for any $P:\prd{a:A} \acc(a) \to \type$, if we have
\[f:\prd{a:A}{h:\prd{b:A} (b<a) \to \acc(b)}
\Parens{\prd{b:A}{l:b<a} P(b,h(b,l))} \to
P(a,\acc_<(a,h)),
\]
then we have $g:\prd{a:A}{c:\acc(a)} P(a,c)$ defined by induction, with
\[g(a,\acc_<(a,h)) \jdeq f(a,\,h,\,\lam{b}{l} g(b,h(b,l))).\]
This is a mouthful, but generally we apply it only in the simpler case where $P:A\to\type$ depends only on $A$.
In this case the second and third arguments of $f$ may be combined, so that what we have to prove is
\[f:\prd{a:A} \Parens{\prd{b:A} (b<a) \to \acc(b) \times P(b)}
\to P(a).
\]
That is, we assume every $b<a$ is accessible and $g(b):P(b)$ is defined, and from these define $g(a):P(a)$.

The omission of the second argument of $P$ is justified by the following lemma, whose proof is the only place where we use the more general form of the induction principle.

\begin{lem}
  Accessibility\index{accessibility} is a mere property.
\end{lem}
\begin{proof}
  We must show that for any $a:A$ and $s_1,s_2:\acc(a)$ we have $s_1=s_2$.
  We prove this by induction on $s_1$, with
  \[P_1(a,s_1) \defeq \prd{s_2:\acc(a)} (s_1=s_2). \]
  Thus, we must show that for any $a:A$ and ${h_1:\prd{b:A} (b<a) \to \acc(b)}$ and
  \[ k_1:{\prd{b:A}{l:b<a}{t:\acc(b)} h_1(b,l) = t},\]
  we have $\acc_<(a,h) = s_2$ for any $s_2:\acc(a)$.
  We regard this statement as $\prd{a:A}{s_2:\acc(a)} P_2(a,s_2)$, where
  \[P_2(a,s_2) \defeq
  \prd{h_1 : \cdots } %{h_1:\prd{b:A} (b<a) \to \acc(b)}
  {k_1 : \cdots} % \Parens{\prd{b:A}{l:b<a}{t:\acc(b)} h_1(b,l) = t} \to
  (\acc_<(a,h_1) = s_2);
  \]
  thus we may prove it by induction on $s_2$.
  Therefore, we assume $h_2 : \prd{b:A} (b<a) \to \acc(b)$, and $k_2$ with a monstrous but irrelevant type,
  % \begin{narrowmultline*}
  %   k_2:\prd{b:A}{l:b<a}
  %   \prd{h_1:\prd{b':A} (b'<b) \to \acc(b')}
  %   \narrowbreak
  %   \Parens{\prd{b':A}{l':b'<b}{t':\acc(b')} h_1(b',l') = t'} \to
  %   (\acc_<(b,h_1) = h_2(b,l)).
  % \end{narrowmultline*}
  and must show that for any $h_1$ and $k_1$ with types as above,
  we have $\acc_<(a,h_1) = \acc_<(a,h_2)$.
  By function extensionality, it suffices to show $h_1(b,l) = h_2(b,l)$ for all $b:A$ and $l:b<a$.
  This follows from $k_1$.
\end{proof}

\begin{defn}
  A binary relation $<$ on a set $A$ is \define{well-founded}
  \indexdef{relation!well-founded}%
  \indexdef{well-founded!relation}%
  if every element of $A$ is accessible.
\end{defn}

The point of well-foundedness is that for $P:A\to \type$, we can use the induction principle of $\acc$ to conclude $\prd{a:A} \acc(a) \to P(a)$, and then apply well-foundedness to conclude $\prd{a:A} P(a)$.
In other words, if from $\fall{b:A} (b<a) \to P(b)$ we can prove $P(a)$, then $\fall{a:A} P(a)$.
This is called \define{well-founded induction}\indexdef{well-founded!induction}.

\begin{lem}
  Well-foundedness is a mere property.
\end{lem}
\begin{proof}
  Well-foundedness of $<$ is the type $\prd{a:A} \acc(a)$, which is a mere proposition since each $\acc(a)$ is.
\end{proof}

\begin{eg}\label{thm:nat-wf}
  Perhaps the most familiar well-founded relation is the usual strict ordering on \nat.
  To show that this is well-founded, we must show that $n$ is accessible for each $n:\nat$.
  \index{strong!induction}%
  This is just the usual proof of ``strong induction'' from ordinary induction on \nat.

  Specifically, we prove by induction on $n:\nat$ that $k$ is accessible for all $k\le n$.
  The base case is just that $0$ is accessible, which is vacuously true since nothing is strictly less than $0$.
  For the inductive step, we assume that $k$ is accessible for all $k\le n$, which is to say for all $k<n+1$; hence by definition $n+1$ is also accessible.

  A different relation on \nat which is also well-founded is obtained by setting only $n < \suc(n)$ for all $n:\nat$.
  Well-foundedness of this relation is almost exactly the ordinary induction principle of \nat.
\end{eg}

\begin{eg}\label{thm:wtype-wf}
  Let $A:\set$ and $B : A \to \set$ be a family of sets.
  Recall from \cref{sec:w-types} that the $W$-type $\wtype{a:A} B(a)$ is inductively generated by the single constructor
  \begin{itemize}
  \item $\supp : \prd{a:A} (B(a) \to \wtype{x:A} B(x)) \to \wtype{x:A} B(x)$
  \end{itemize}
  We define the relation $<$ on $\wtype{x:A} B(x)$ by recursion on its second argument:
  \begin{itemize}
  \item For any $a:A$ and $f:B(a) \to \wtype{x:A} B(x)$, we define $w<\supp(a,f)$ to mean that there merely exists a $b:B(a)$ such that $w = f(b)$.
  \end{itemize}
  Now we prove that every $w:\wtype{x:A} B(x)$ is accessible for this relation, using the usual induction principle for $\wtype{x:A}B(x)$.
  This means we assume given $a:A$ and $f:B(a) \to \wtype{x:A} B(x)$, and also a lifting $f' : \prd{b:B(a)} \acc(f(b))$.
  But then by definition of $<$, we have $\acc(w)$ for all $w<\supp(a,f)$; hence $\supp(a,f)$ is accessible.
\end{eg}

Well-foundedness allows us to define functions by recursion and prove statements by induction, such as for instance the following.
Recall from \cref{subsec:prop-subsets} that $\power B$ denotes the \emph{power set}\index{power set} $\power B \defeq (B\to\prop)$.

\begin{lem}\label{thm:wfrec}
  Suppose $B$ is a set and we have a function
  \[ g : \power B \to B \]
  Then if $<$ is a well-founded relation on $A$, there is a function $f:A\to B$ such that for all $a:A$ we have
  \begin{equation*}
    f(a) = g\Big(\setof{ f(a') | a'<a }\Big).
  \end{equation*}
\end{lem}
\noindent
(We are using the notation for images of subsets from \cref{sec:image}.)
\begin{proof}
  We first define, for every $a:A$ and $s:\acc(a)$, an element $\bar f(a,s):B$.
  By induction, it suffices to assume that $s$ is a function assigning to each $a'<a$ a witness $s(a'):\acc(a')$, and that moreover for each such $a'$ we have an element $\bar f(a',s(a')):B$.
  In this case, we define
  \begin{equation*}
    \bar f(a,s) \defeq g\Big(\setof{ \bar f(a',s(a')) | a'<a }\Big).
  \end{equation*}

  Now since $<$ is well-founded, we have a function $w:\prd{a:A} \acc(a)$.
  Thus, we can define $f(a)\defeq \bar f (a,w(a))$.
\end{proof}

In classical\index{mathematics!classical} logic, well-foundedness has a more well-known reformulation.
In the following, we say that a subset $B: \power A$ is \define{nonempty}
\indexdef{nonempty subset}
if it is unequal to the empty subset $(\lam{x}\bot) : \power X$.
We leave it to the reader to verify that assuming excluded middle, this is equivalent to mere inhabitation, i.e.\ to the condition $\exis{x:A} x\in B$.

\begin{lem}\label{thm:wfmin}
  \index{excluded middle}%
  Assuming excluded middle, $<$ is well-founded if and only if every nonempty subset $B: \power A$ merely has a minimal element.
\end{lem}
\begin{proof}
  Suppose first $<$ is well-founded, and suppose $B\subseteq A$ is a subset with no minimal element.
  That is, for any $a:A$ with $a\in B$, there merely exists a $b:A$ with $b<a$ and $b\in B$.

  We claim that for any $a:A$ and $s:\acc(a)$, we have $a\notin B$.
  By induction, we may assume $s$ is a function assigning to each $a'<a$ a proof $s(a'):\acc(a)$, and that moreover for each such $a'$ we have $a'\notin B$.
  If $a\in B$, then by assumption, there would merely exist a $b<a$ with $b\in B$, which contradicts this assumption.
  Thus, $a\notin B$; this completes the induction.
  Since $<$ is well-founded, we have $a\notin B$ for all $a:A$, i.e. $B$ is empty.

  Now suppose each nonempty subset merely has a minimal element.
  Let $B = \setof{ a:A | \neg \acc(a) }$.
  Then if $B$ is nonempty, it merely has a minimal element.
  Thus there merely exists an $a:A$ with $a\in B$ such that for all $b<a$, we have $\acc(b)$.
  But then by definition (and induction on truncation), $a$ is merely accessible, and hence accessible, contradicting $a\in B$.
  Thus, $B$ is empty, so $<$ is well-founded.
\end{proof}

\begin{defn}
  A well-founded relation $<$ on a set $A$ is \define{extensional}
  \indexdef{relation!extensional}%
  \indexdef{extensional!relation}%
  if for any $a,b:A$, we have
  \[ \Parens{\fall{c:A} (c<a) \Leftrightarrow (c<b)} \to (a=b). \]
\end{defn}

Note that since $A$ is a set, extensionality is a mere proposition.
This notion of ``extensionality'' is unrelated to function extensionality, and also unrelated to the extensionality of identity types.
\index{axiom!of extensionality}%
Rather, it is a ``local'' counterpart of the axiom of extensionality in classical set theory.

\begin{thm}
  The type of extensional well-founded relations is a set.
\end{thm}
\begin{proof}
  By the univalence axiom, it suffices to show that if $(A,<)$ is extensional and well-founded and $f:(A,<) \cong (A,<)$, then $f=\idfunc[A]$.
  \index{automorphism!of extensional well-founded relations}%
  We prove by induction on $<$ that $f(a)=a$ for all $a:A$.
  The inductive hypothesis is that for all $a'<a$, we have $f(a')=a'$.

  Now since $A$ is extensional, to conclude $f(a)=a$ it is sufficient to show
  \[\fall{c:A}(c<f(a)) \Leftrightarrow (c<a).\]
  However, since $f$ is an automorphism, we have $(c<a) \Leftrightarrow (f(c)<f(a))$.
  But $c<a$ implies $f(c)=c$ by the inductive hypothesis, so $(c<a) \to (c<f(a))$.
  On the other hand, if $c<f(a)$, then $f^{-1}(c)<a$, and so $c = f(f^{-1}(c)) = f^{-1}(c)$ by the inductive hypothesis again; thus $c<a$.
  Therefore, we have $(c<a) \Leftrightarrow (c<f(a))$ for any $c:A$, so $f(a)=a$.
\end{proof}

\begin{defn}\label{def:simulation}
  If $(A,<)$ and $(B,<)$ are extensional and well-founded, a \define{simulation}
  \indexdef{simulation}%
  \indexsee{function!simulation}{simulation}%
  is a function $f:A\to B$ such that
  \begin{enumerate}
  \item if $a<a'$, then $f(a)<f(a')$, and\label{item:sim1}
  \item for all $a:A$ and $b:B$, if $b<f(a)$, then there merely exists an $a'<a$ with $f(a')=b$.\label{item:sim2}
  \end{enumerate}
\end{defn}

\begin{lem}
  Any simulation is injective.
\end{lem}
\begin{proof}
  We prove by double well-founded induction that for any $a,b:A$, if $f(a)=f(b)$ then $a=b$.
  The inductive hypothesis for $a:A$ says that for any $a'<a$, and any $b:B$, if $f(a')=f(b)$ then $a=b$.
  The inner inductive hypothesis for $b:A$ says that for any $b'<b$, if $f(a)=f(b')$ then $a=b'$.

  Suppose $f(a)=f(b)$; we must show $a=b$.
  By extensionality, it suffices to show that for any $c:A$ we have $(c<a)\Leftrightarrow (c<b)$.
  If $c<a$, then $f(c)<f(a)$ by \cref{def:simulation}\ref{item:sim1}.
  Hence $f(c)<f(b)$, so by \cref{def:simulation}\ref{item:sim2} there merely exists $c':A$ with $c'<b$ and $f(c)=f(c')$.
  By the inductive hypothesis for $a$, we have $c=c'$, hence $c<b$.
  The dual argument is symmetrical.
\end{proof}

In particular, this implies that in \cref{def:simulation}\ref{item:sim2} the word ``merely'' could be omitted without change of sense.

\begin{cor}
  If $f:A\to B$ is a simulation, then for all $a:A$ and $b:B$, if $b<f(a)$, there \emph{purely} exists an $a'<a$ with $f(a')=b$.
\end{cor}
\begin{proof}
  Since $f$ is injective, $\sm{a:A} (f(a)=b)$ is a mere proposition.
\end{proof}

We say that a subset $C :\power B$ is an \define{initial segment}
\indexdef{initial!segment}%
\indexsee{segment, initial}{initial segment}%
if $c\in C$ and $b<c$ imply $b\in C$.
The image of a simulation must be an initial segment, while the inclusion of any initial segment is a simulation.
Thus, by univalence, every simulation $A\to B$ is \emph{equal} to the inclusion of some initial segment of $B$.

\begin{thm}
  For a set $A$, let $P(A)$ be the type of extensional well-founded relations on $A$.
  If $\mathord{<_A} : P(A)$ and $\mathord{<_B} : P(B)$ and $f:A\to B$, let $H_{\mathord{<_A}\mathord{<_B}}(f)$ be the mere proposition that $f$ is a simulation.
  Then $(P,H)$ is a standard notion of structure over \uset in the sense of \cref{sec:sip}.
\end{thm}
\begin{proof}
  We leave it to the reader to verify that identities are simulations, and that composites of simulations are simulations.
  Thus, we have a notion of structure.
  For standardness, we must show that if $<$ and $\prec$ are two extensional well-founded relations on $A$, and $\idfunc[A]$ is a simulation in both directions, then $<$ and $\prec$ are equal.
  Since extensionality and well-foundedness are mere propositions, for this it suffices to have $\fall{a,b:A} (a<b) \Leftrightarrow (a\prec b)$.
  But this follows from \cref{def:simulation}\ref{item:sim1} for $\idfunc[A]$.
\end{proof}

\begin{cor}\label{thm:wfcat}
  There is a category whose objects are sets equipped with extensional well-founded relations, and whose morphisms are simulations.
\end{cor}

In fact, this category is a poset.

\begin{lem}
  For extensional and well-founded $(A,<)$ and $(B,<)$, there is at most one simulation $f:A\to B$.
\end{lem}
\begin{proof}
  Suppose $f,g:A\to B$ are simulations.
  Since being a simulation is a mere property, it suffices to show $\fall{a:A}(f(a)=g(a))$.
  By induction on $<$, we may suppose $f(a')=g(a')$ for all $a'<a$.
  And by extensionality of $B$, to have $f(a)=g(a)$ it suffices to have $\fall{b:B}(b<f(a)) \Leftrightarrow (b<g(a))$.

  But since $f$ is a simulation, if $b<f(a)$, then we have $a'<a$ with $f(a')=b$.
  By the inductive hypothesis, we have also $g(a')=b$, hence $b<g(a)$.
  The dual argument is symmetrical.
\end{proof}

Thus, if $A$ and $B$ are equipped with extensional and well-founded relations, we may write $A\le B$ to mean there exists a simulation $f:A\to B$.
\cref{thm:wfcat} implies that if $A\le B$ and $B\le A$, then $A=B$.

\begin{defn}
  An \define{ordinal}
  \indexdef{ordinal}%
  \indexsee{number!ordinal}{ordinal}%
  is a set $A$ with an extensional well-founded relation which is \emph{transitive}, i.e.\ satisfies $\fall{a,b,c:A}(a<b)\to (b<c) \to (a<c)$.
\end{defn}

\begin{eg}
  Of course, the usual strict order on \nat is transitive.
  It is easily seen to be extensional as well; thus it is an ordinal.
  As usual, we denote this ordinal by $\omega$.
\end{eg}

\symlabel{ord}
Let \ord denote the type of ordinals.
By the previous results, \ord is a set and has a natural partial order.
We now show that \ord also admits a well-founded relation.

\symlabel{initial-segment}
If $A$ is an ordinal and $a:A$, let $\ordsl A a \defeq \setof{ b:A | b<a}$ denote the initial segment.
\index{initial!segment}%
Note that if $\ordsl A a = \ordsl A b$ as ordinals, then that isomorphism must respect their inclusions into $A$ (since simulations form a poset), and hence they are equal as subsets of $A$.
Therefore, since $A$ is extensional, $a=b$.
Thus the function $a\mapsto \ordsl A a$ is an injection $A\to \ord$.

\begin{defn}
  For ordinals $A$ and $B$, a simulation $f:A\to B$ is said to be \define{bounded}
  \indexdef{simulation!bounded}%
  \indexdef{bounded!simulation}%
  if there exists $b:B$ such that $A = \ordsl B b$.
\end{defn}

The remarks above imply that such a $b$ is unique when it exists, so that boundedness is a mere property.

We write $A<B$ if there exists a bounded simulation from $A$ to $B$.
Since simulations are unique, $A<B$ is also a mere proposition.

\begin{thm}\label{thm:ordord}
  $(\ord,<)$ is an ordinal.
\end{thm}

\noindent
More precisely, this theorem says that the type $\ord_{\UU_i}$ of ordinals in one universe\index{universe level} is itself an ordinal in the next higher universe, i.e.\ $(\ord_{\UU_i},<):\ord_{\UU_{i+1}}$.

\begin{proof}
  Let $A$ be an ordinal; we first show that $\ordsl A a$ is accessible (in \ord) for all $a:A$.
  By well-founded induction on $A$, suppose $\ordsl A b$ is accessible for all $b<a$.
  By definition of accessibility, we must show that $B$ is accessible in \ord for all $B<\ordsl A a$.
  However, if $B<\ordsl A a$ then there is some $b<a$ such that $B = \ordsl{(\ordsl A a)}{b} = \ordsl A b$, which is accessible by the inductive hypothesis.
  Thus, $\ordsl A a$ is accessible for all $a:A$.

  Now to show that $A$ is accessible in \ord, by definition we must show $B$ is accessible for all $B<A$.
  But as before, $B<A$ means $B=\ordsl A a$ for some $a:A$, which is accessible as we just proved.
  Thus, \ord is well-founded.

  For extensionality, suppose $A$ and $B$ are ordinals such that
  \narrowequation{\prd{C:\ord} (C<A) \Leftrightarrow (C<B).}
  Then for every $a:A$, since $\ordsl A a<A$, we have $\ordsl A a<B$, hence there is $b:B$ with $\ordsl A a = \ordsl B b$.
  Define $f:A\to B$ to take each $a$ to the corresponding $b$; it is straightforward to verify that $f$ is an isomorphism.
  Thus $A\cong B$, hence $A=B$ by univalence.

  Finally, it is easy to see that $<$ is transitive.
\end{proof}

Treating \ord as an ordinal is often very convenient, but it has its pitfalls as well.
For instance, consider the following lemma, where we pay attention to how universes are used.

\begin{lem}\label{thm:ordsucc}
  Let \bbU be a universe.
  For any $A:\ord_\bbU$, there is a $B:\ord_\bbU$ such that $A<B$.
\end{lem}
\begin{proof}
  Let $B=A+\unit$, with the element $\ttt:\unit$ being greater than all elements of $A$.
  Then $B$ is an ordinal and it is easy to see that $A\cong \ordsl B \ttt$.
\end{proof}

This lemma illustrates a potential pitfall of the ``typically ambiguous''\index{typical ambiguity} style of using \UU to denote an arbitrary, unspecified universe.
Consider the following alternative proof of it.

\begin{proof}[Another putative proof of \cref{thm:ordsucc}]
  Note that $C<A$ if and only if $C=\ordsl A a$ for some $a:A$.
  This gives an isomorphism $A \cong \ordsl \ord A$, so that $A<\ord$.
  Thus we may take $B\defeq\ord$.
\end{proof}

The second proof would be valid if we had stated \cref{thm:ordsucc} in a typically ambiguous style.
But the resulting lemma would be less useful, because the second proof would constrain the second ``\ord'' in the lemma statement to refer to a higher universe level than the first one.
The first proof allows both universes to be the same.

Similar remarks apply to the next lemma, which could be proved in a less useful way by observing that $A\le \ord$ for any $A:\ord$.

\begin{lem}\label{thm:ordunion}
  Let \bbU be a universe.
  For any $X:\type$ and $F:X\to \ord_\bbU$, there exists $B:\ord_\bbU$ such that $Fx\le B$ for all $x:X$.
\end{lem}
\begin{proof}
  Let $B$ be the quotient of the equivalence relation $\eqr$ on $\sm{x:X} Fx$ defined as follows:
  \[ (x,y) \eqr (x',y')
  \;\defeq\;
  \Big(\ordsl{(Fx)}{y} \cong \ordsl{(Fx')}{y'}\Big).
  \]
  Define $(x,y)<(x',y')$ if $\ordsl{(Fx)}{y} < \ordsl{(Fx')}{y'}$.
  This clearly descends to the quotient, and can be seen to make $B$ into an ordinal.
  Moreover, for each $x:X$ the induced map $Fx\to B$ is a simulation.
\end{proof}



\section{Classical well-orderings}
\label{sec:wellorderings}

\index{denial|(}%
We now show the equivalence of our ordinals with the more familiar classical\index{mathematics!classical} well-orderings.

\begin{lem}
  \index{excluded middle}%
  Assuming excluded middle, every ordinal is trichotomous:
  \index{trichotomy of ordinals}%
  \index{ordinal!trichotomy of}%
  \[ \fall{a,b:A} (a<b) \vee (a=b) \vee (b<a). \]
\end{lem}
\begin{proof}
  By induction on $a$, we may assume that for every $a'<a$ and every $b':A$, we have $(a'<b') \vee (a'=b') \vee (b'<a')$.
  Now by induction on $b$, we may assume that for every $b'<b$, we have $(a<b') \vee (a=b') \vee (b'<a)$.

  By excluded middle, either there merely exists a $b'<b$ such that $a<b'$, or there merely exists a $b'<b$ such that $a=b'$, or for every $b'<b$ we have $b'<a$.
  In the first case, merely $a<b$ by transitivity, hence $a<b$ as it is a mere proposition.
  Similarly, in the second case, $a<b$ by transport.
  Thus, suppose $\fall{b':A}(b'<b)\to (b'<a)$.

  Now analogously, either there merely exists $a'<a$ such that $b<a'$, or there merely exists $a'<a$ such that $a'=b$, or for every $a'<a$ we have $a'<b$.
  In the first and second cases, $b<a$, so we may suppose $\fall{a':A}(a'<a)\to (a'<b)$.
  However, by extensionality, our two suppositions now imply $a=b$.
\end{proof}

\begin{lem}
  A well-founded relation contains no cycles, i.e.\
  \[ \fall{n:\mathbb{N}}{a:\mathbb{N}_n\to A} \neg\Big((a_0<a_1) \wedge \dots \wedge (a_{n-1}<a_n)\wedge (a_n<a_0)\Big). \]
\end{lem}
\begin{proof}
  We prove by induction on $a:A$ that there is no cycle containing $a$.
  Thus, suppose by induction that for all $a'<a$, there is no cycle containing $a'$.
  But in any cycle containing $a$, there is some element less than $a$ and contained in the same cycle.
\end{proof}

\indexdef{relation!irreflexive}%
\index{irreflexivity!of well-founded relation}%
In particular, a well-founded relation must be \define{irreflexive}, i.e.\ $\neg(a<a)$ for all $a$.

\begin{thm}\label{thm:wellorder}
  Assuming excluded middle, $(A,<)$ is an ordinal if and only if every nonempty subset $B\subseteq A$ has a least element.
\end{thm}
\begin{proof}
  If $A$ is an ordinal, then by \cref{thm:wfmin} every nonempty subset merely has a minimal element.
  But trichotomy implies that any minimal element is a least element.
  Moreover, least elements are unique when they exist, so merely having one is as good as having one.

  Conversely, if every nonempty subset has a least element, then by \cref{thm:wfmin}, $A$ is well-founded.
  We also have trichotomy, since for any $a,b$ the subset
  $ \setof{a,b} \defeq \setof{x:A | x=a \lor x=b} $
  merely has a least element, which must be either $a$ or $b$.
  This implies transitivity, since if $a<b$ and $b<c$, then either $a=c$ or $c<a$ would produce a cycle.
  Similarly, it implies extensionality, for if $\fall{c:A}(c<a)\Leftrightarrow (c<b)$, then $a<b$ implies (letting $c$ be $a$) that $a<a$, which is a cycle, and similarly if $b<a$; hence $a=b$.
\end{proof}

In classical\index{mathematics!classical} mathematics, the characterization of \cref{thm:wellorder} is taken as the definition of a \emph{well-ordering}, with the \emph{ordinals} being a canonical set of representatives of isomorphism classes for well-orderings.
In our context, the structure identity principle means that there is no need to look for such representatives: any well-ordering is as good as any other.

We now move on to consider consequences of the axiom of choice.
For any set $X$, let $\powerp X$ denote the type of merely inhabited subsets of $X$:
\symlabel{inhabited-powerset}
\[ \powerp X \defeq \setof{ Y : \power X | \exis{x:X} x\in Y}. \]
Assuming excluded middle, this is equivalently the type of \emph{nonempty}\index{nonempty subset} subsets of $X$, and we have $\power X \eqvsym (\powerp X) + \unit$.

\begin{thm}\label{thm:wop}
  \index{axiom!of choice}%
  \index{excluded middle}%
  Assuming excluded middle, the following are equivalent.
  \begin{enumerate}
  \item For every set $X$, there merely exists a function
    $ f: \powerp X \to X $
    such that $f(Y)\in Y$ for all $Y:\power X$.\label{item:wop1}
  \item Every set merely admits the structure of an ordinal.\label{item:wop2}
  \end{enumerate}
\end{thm}

\noindent
Of course,~\ref{item:wop1} is a standard classical\index{mathematics!classical} version of the axiom of choice; see \cref{ex:choice-function}.

\begin{proof}
  One direction is easy: suppose~\ref{item:wop2}.
  Since we aim to prove the mere proposition~\ref{item:wop1}, we may assume $A$ is an ordinal.
  But then we can define $f(B)$ to be the least element of $B$.

  Now suppose~\ref{item:wop1}.
  As before, since~\ref{item:wop2} is a mere proposition, we may assume given such an $f$.
  We extend $f$ to a function
  \[ \bar f:\power X \eqvsym (\powerp X) + \unit \longrightarrow X+\unit
  \]
  in the obvious way.
  Now for any ordinal $A$, we can define $g_A:A\to X+\unit$ by well-founded recursion:
  \[ g_A(a) \defeq
    \bar f\Big(X \setminus \setof{ g_A(b) | \strut (b<a) \wedge (g_A(b) \in X) }\Big)
  \]
  (regarding $X$ as a subset of $X+\unit$ in the obvious way).

  Let $A'\defeq \setof{a:A | g_A(a) \in X}$ be the preimage of $X\subseteq X+\unit$; then we claim the restriction $g_A':A' \to X$ is injective.
  For if $a,a':A$ with $a\neq a'$, then by trichotomy and without loss of generality, we may assume $a'<a$.
  Thus $g_A(a') \in \setof{ g_A(b) | b<a }$, so since $f(Y)\in Y$ for all $Y$ we have $g_A(a) \neq g_A(a')$.

  Moreover, $A'$ is an initial segment of $A$.
  For $g_A(a)$ lies in \unit if and only if $\setof{g_A(b)|b<a} = X$, and if this holds then it also holds for any $a'>a$.
  Thus, $A'$ is itself an ordinal.

  Finally, since \ord is an ordinal, we can take $A\defeq\ord$.
  Let $X'$ be the image of $g_\ord':\ord' \to X$; then the inverse of $g_\ord'$ yields an injection $H:X'\to \ord$.
  By \cref{thm:ordunion}, there is an ordinal $C$ such that $Hx\le C$ for all $x:X'$.
  Then by \cref{thm:ordsucc}, there is a further ordinal $D$ such that $C<D$, hence $Hx<D$ for all $x:X'$.
  Now we have
  \begin{align*}
    g_{\ord}(D) &= \bar f\Big( X \setminus \setof{ g_\ord(B) | \rule{0pt}{1em} B<D \wedge (g_\ord(B) \in X)} \Big)\\
    &=\bar f\Big( X \setminus \setof{ g_\ord(B) | \rule{0pt}{1em} g_\ord(B) \in X} \Big)
  \end{align*}
  since if $B:\ord$ and $(g_\ord(B) \in X)$, then $B = Hx$ for some $x:X'$, hence $B<D$.
  Now if
  \[\setof{ g_\ord(B) | \rule{0pt}{1em} g_\ord(B) \in X}\]
  is not all of $X$, then $g_\ord(D)$ would lie in $X$ but not in this subset, which would be a contradiction since $D$ is itself a potential value for $B$.
  So this set must be all of $X$, and hence $g_\ord'$ is surjective as well as injective.
  Thus, we can transport the ordinal structure on $\ord'$ to $X$.
\end{proof}

\begin{rmk}
  If we had given the wrong proof of \cref{thm:ordsucc} or \cref{thm:ordunion}, then the resulting proof of \cref{thm:wop} would be invalid: there would be no way to consistently assign universe levels\index{universe level}.
  As it is, we require propositional resizing (which follows from \LEM{}) to ensure that $X'$ lives in the same universe as $X$ (up to equivalence).
\end{rmk}

\begin{cor}
  Assuming the axiom of choice, the function $\ord\to\set$ (which forgets the order structure) is a surjection.
\end{cor}

Note that \ord is a set, while \set is a 1-type.
In general, there is no reason for a 1-type to admit any surjective function from a set.
Even the axiom of choice does not appear to imply that \emph{every} 1-type does so (although see \cref{ex:acnm-surjset}), but it readily implies that this is so for 1-types constructed out of \set, such as the types of objects of categories of structures as in \cref{sec:sip}.
The following corollary also applies to such categories.

\begin{cor}
  \index{weak equivalence!of precategories}%
  Assuming \choice{}, \uset admits a weak equivalence functor from a strict category.
\end{cor}
\begin{proof}
  Let $X_0\defeq \ord$, and for $A,B:X_0$ let $\hom_X(A,B) \defeq (A\to B)$.
  Then $X$ is a strict category, since \ord is a set, and the above surjection $X_0 \to \set$ extends to a weak equivalence functor $X\to \uset$.
\end{proof}

Now recall from \cref{sec:cardinals} that we have a further surjection $\cd{\blank}:\set\to\card$, and hence a composite surjection $\ord\to\card$ which sends each ordinal to its cardinality.

\begin{thm}
  Assuming \choice{}, the surjection $\ord\to\card$ has a section.
\end{thm}
\begin{proof}
  There is an easy and wrong proof of this: since \ord and \card are both sets, \choice{} implies that any surjection between them \emph{merely} has a section.
  However, we actually have a canonical \emph{specified} section: because \ord is an ordinal, every nonempty subset of it has a uniquely specified least element.
  Thus, we can map each cardinal to the least element in the corresponding fiber.
\end{proof}

It is traditional in set theory to identify cardinals with their image in \ord: the least ordinal having that cardinality.

It follows that \card also canonically admits the structure of an ordinal: in fact, one isomorphic to \ord.
Specifically, we define by well-founded recursion a function $\aleph:\ord\to\ord$, such that $\aleph(A)$ is the least ordinal having cardinality greater than $\aleph({\ordsl A a})$ for all $a:A$.
Then (assuming \choice{}) the image of $\aleph$ is exactly the image of \card.

\index{denial|)}%

\index{ordinal|)}%

\section{The cumulative hierarchy}
\label{sec:cumulative-hierarchy}

\index{bargaining|(}%
We can define a cumulative hierarchy $V$ of all sets in a given universe $\UU$ as a higher inductive type, in such a way that $V$ is again a set (in a larger universe $\UU'$), equipped with a binary ``membership'' relation $x\in y$ which satisfies the usual laws of set theory.

\begin{defn}\label{defn:V}
  The \define{cumulative hierarchy}
  \indexdef{cumulative!hierarchy, set-theoretic}%
  \indexdef{hierarchy!cumulative, set-theoretic}%
  $V$ relative to a type universe $\UU$ is the
  higher inductive type generated by the following constructors.
  %
  \begin{enumerate}
  \item For every $A : \UU$ and $f : A \to V$, there is an element $\vset(A, f)$ : V.
  \item For all $A, B : \UU$, $f : A \to V$ and $g : B \to V$ such that
    %
    \begin{narrowmultline} \label{eq:V-path}
      \big(\fall{a:A} \exis{b:B} \id[V]{f(a)}{g(b)}\big) \land \narrowbreak
      \big(\fall{b:B} \exis{a:A} \id[V]{f(a)}{g(b)}\big)
    \end{narrowmultline}
    %
    there is a path $\id[V]{\vset(A,f)}{\vset(B,g)}$.
  \item The 0-truncation constructor: for all $x,y:V$ and $p,q:x=y$, we have $p=q$.
  \end{enumerate}
\end{defn}

In set-theoretic language, $\vset(A,f)$ can be understood as the set (in the sense of classical set theory) that is the image of $A$ under $f$, i.e.\ $\setof{ f(a) | a \in A }$.
However, we will avoid this notation, since it would clash with our notation for subtypes (but see~\eqref{eq:class-notation} and \cref{def:TypeOfElements} below).

The hierarchy $V$ is
bootstrapped from the empty map $\rec\emptyt(V) : \emptyt \to V$, which gives the empty set as $\emptyset = \vset(\emptyt,\rec\emptyt(V))$.
Then the singleton $\{\emptyset\}$ enters $V$ through $\unit \to V$, defined as $\ttt \mapsto \emptyset$, and so
on. The type $V$ lives in the same universe as the base universe $\UU$.

The second constructor of $V$ has a form unlike any we have seen before: it involves not only paths in $V$ (which in \cref{sec:hittruncations} we claimed were slightly fishy) but truncations of sums of them.
It certainly does not fit the general scheme described in \cref{sec:naturality}, and thus it may not be obvious what its induction principle should be.
Fortunately, like our first definition of the 0-truncation in \cref{sec:hittruncations}, it can be re-expressed using auxiliary higher inductive types.
We leave it to the reader to work out the details (see \cref{ex:cumhierhit}).

\index{induction principle!for cumulative hierarchy}%
At the end of the day, the induction principle for $V$ (written in pattern matching language) says that given $P:V\to \set$, in order to construct $h:\prd{x:V} P(x)$, it suffices to give the following.
\begin{enumerate}
\item For any $f:A\to V$, construct $h(\vset(A,f))$, assuming as given $h(f(a))$ for all $a:A$.
\item Verify that if $f : A \to V$ and $g : B \to V$ satisfy~\eqref{eq:V-path}, then $h(\vset(A,f)) = h(\vset(B,g))$, assuming inductively that $h(f(a)) = h(g(b))$ whenever $f(a)=g(b)$.
\end{enumerate}
The second clause checks that the map being defined must respect the paths introduced in \eqref{eq:V-path}.
As usual when we state higher induction principles using pattern matching, it may seem tautologous, but is not.
The point is that ``$h(f(a))$'' is essentially a formal symbol which we cannot peek inside of, which $h(\vset(A,f))$ must be defined in terms of. Thus, in the second clause, we assume equality of these formal symbols when appropriate, and verify that the elements resulting from the construction of the first clause are also equal.
Of course, if $P$ is a family of mere propositions, then the second clause is automatic.

Observe that, by induction, for each $v:V$ there merely exist $A:\UU$ and $f:A\to V$ such that $v=\vset(A,f)$.
Thus, it is reasonable to try to define the \define{membership relation}
\indexdef{membership, for cumulative hierarchy}%
$x\in v$ on $V$ by setting:
%
% Note: "membership" rather than "elementhood", because "element" is taken.
%
\symlabel{V-membership}
\begin{equation*}
  (x \in \vset(A,f)) \defeq (\exis{a : A} x = f(a)).
\end{equation*}
%
To see that the definition is valid, we must use the recursion principle of $V$.  Thus, suppose we have a path $\vset(A, f) = \vset(B, g)$
constructed through~\eqref{eq:V-path}. If $x \in \vset(A,f)$ then there merely is $a : A$ such
that $x = f(a)$, but by~\eqref{eq:V-path} there merely is $b : B$ such that $f(a) = g(b)$, hence
$x = g(b)$ and $x \in \vset(B,f)$. The converse is symmetric.

The \define{subset relation}
\indexdef{subset!relation on the cumulative hierarchy}%
$x\subseteq y$ is defined on $V$ as usual by
%
\begin{equation*}
  (x \subseteq y) \defeq \fall{z : V} z \in x \Rightarrow z \in y.
\end{equation*}

A \define{class}
\indexdef{class}%
may be taken to be a mere predicate on~$V$. We can say that a class $C : V \to \prop$ is a
\define{$V$-set}
\indexdef{set!in the cumulative hierarchy}%
if there merely exists $v\in V$ such that
%
\begin{equation*}
  \fall{x : V} C(x) \Leftrightarrow x \in v.
\end{equation*}
We may also use the conventional notation for classes, which matches our standard notation for subtypes:
\begin{equation}
  \setof{ x | C(x) } \defeq \lam{x}C(x).\label{eq:class-notation}
\end{equation}
%
A class $C: V\to \prop$ will be called \define{$\UU$-small}
\indexdef{class!small}%
\indexdef{small!class}%
if all of its values $C(x)$ lie in $\UU$, specifically $C: V\to \prop_{\UU}$.
Since $V$ lives in the same universe $\UU'$ as does the base universe $\UU$ from which it is built, the same is true for the identity types $v=_V w$ for any $v,w:V$. To obtain a well-behaved theory in the absence of propositional resizing,
\index{propositional!resizing}%
\index{resizing}%
therefore, it will be convenient to have a $\UU$-small ``resizing'' of the identity relation, which we can define by induction as follows.

\begin{defn}\label{def:bisimulation}
  Define the \define{bisimulation}
  \indexdef{bisimulation}%
  relation
  %
  \begin{equation*}
    \mathord\bisim : V \times V \longrightarrow \prop_{\UU}
  \end{equation*}
  %
  by double induction over $V$, where for $\vset(A,f)$ and $\vset(B,g)$ we let:
  \begin{narrowmultline*}
    \vset(A,f)  \bisim \vset(B,g) \defeq \narrowbreak
    \big(\fall{a:A}\exis{b:B} f(a)  \bisim g(b)\big) \land
    \big(\fall{b:B}\exis{a:A} f(a) \bisim g(b)\big).
  \end{narrowmultline*}
\end{defn}
%
To verify that the definition is correct, we just need to check that it respects paths $\vset(A, f) = \vset(B, g)$ constructed through~\eqref{eq:V-path}, but this is obvious, and that $\prop_{\UU}$ is a set, which it is.  Note that $u \bisim v$ is in $\propU$ by construction.

\begin{lem}\label{lem:BisimEqualsId}
For any $u,v:V$ we have $(u=_V v) = (u \bisim v)$.
\end{lem}

\begin{proof}
An easy induction shows that $\bisim$ is reflexive, so by transport we have $(u=_V v)\to (u \bisim v)$.
Thus, it remains to show that $(u \bisim v)\to (u=_V v)$.
By induction on $u$ and $v$, we may assume they are $\vset(A,f)$ and $\vset(B,g)$ respectively.
(We can ignore the path-constructors of $V$, since $(u \bisim v)\to (u=_V v)$ is a mere proposition.)
Then by definition, $\vset(A,f)\bisim\vset(B,g)$ implies $(\fall{a:A}\exis{b:B}f(a)  \bisim g(b))$ and conversely.
But the inductive hypothesis then tells us that $(\fall{a:A}\exis{b:B}f(a) = g(b))$ and conversely.
So by the path-con\-struc\-tor for $V$ we have $\vset(A,f) =_V \vset(B,g)$.
\end{proof}

One might think that we could omit the 0-truncation constructor of $V$ and \emph{prove} that $V$ is 0-truncated by applying \cref{thm:h-set-refrel-in-paths-sets} to the bisimulation.
However, in the proof of \cref{lem:BisimEqualsId} we used the fact that $V$ is 0-truncated, to conclude that $(u \bisim v)\to (u=_V v)$ is a mere proposition so that in the induction it suffices to assume $u$ and $v$ are $\vset(A,f)$ and $\vset(B,g)$.

Now we can use the resized identity relation to get the following useful principle.

\begin{lem}\label{lem:MonicSetPresent}
For every $u:V$ there is a given $A_u:\UU$ and monic $m_u: A_u \mono V$ such that $u = \vset(A_u, m_u)$.
\end{lem}

\begin{proof}
  Take any presentation $u = \vset(A,f)$ and factor $f:A\to V$ as a surjection followed by an injection:
  %
  \begin{equation*}
    f = m_u\circ e_u : A \epi A_u \mono V.
  \end{equation*}
  %
  Clearly $u = \vset(A_u, m_u)$ if only $A_u$ is still in $\UU$, which holds if the kernel of $e_u : A \epi A_u$ is in $\UU$.  But the kernel of $e_u : A \epi A_u$ is the pullback along $f : A\to V$ of the identity on $V$, which we just showed to be $\UU$-small, up to equivalence.  Now, this construction of the pair $(A_u, m_u)$ with $m_u :A_u \mono V$ and $u = \vset(A_u, m_u)$ from $u:V$ is unique up to equivalence over $V$, and hence up to identity by univalence.  Thus by the principle of unique choice \eqref{cor:UC} there is a map $c : V\to\sm{A:\UU}(A\to V)$ such that $c(u) = (A_u, m_u)$, with $m_u :A_u \mono V$ and $u = \vset(c(u))$, as claimed.
\end{proof}

\begin{defn}\label{def:TypeOfElements}
  For $u:V$, the just constructed monic presentation $m_u: A_u \mono V$ such that $u = \vset(A_u, m_u)$ may be called the \define{type of members}
  \indexdef{type!of members}%
  of $u$ and denoted $m_u : [u] \mono V$, or even $[u] \mono V$.  We can think of $[u]$ as the ``subclass of $V$ consisting of members of $u$''.
\end{defn}

\begin{thm}\label{thm:VisCST}
  \index{axiom!of set theory, for the cumulative hierarchy}%
  The following hold for $(V, {\in})$:
  %
  \begin{enumerate}
  \item \emph{extensionality:}
    %
    \begin{equation*}
      \fall{x, y : V} x \subseteq y \land y \subseteq x \Leftrightarrow x = y.
    \end{equation*}
    %
     \item \emph{empty set:} for all $x:V$, we have $\neg (x\in \emptyset)$.
    %
    \item \emph{pairing:} for all $u, v:V$, the class $u\cup v \defeq \setof{ x | x = u \vee x = v}$ is a $V$-set.
      %
    \item \emph{infinity:}\index{axiom!of infinity}  there is a $v:V$ with $\emptyset\in v$ and $x\in v$ implies $x\cup \{x\}\in v$.
    %
  \item \emph{union:} for all $v:V$, the class $\cup v\defeq \setof{ x | \exis{u:V} x \in u \in v}$ is a $V$-set.
    %
    \item \emph{function set:} for all $u, v:V$, the class $v^u \defeq \setof{ x | x : u\to v}$ is a $V$-set.%
      \footnote{Here $x:u\to v$ means that $x$ is an appropriate set of ordered pairs, according to the usual way of encoding functions in set theory.}
    %
   \item \emph{$\in$-induction:} if $C : V \to \prop$ is a class such that $C(a)$ holds whenever $C(x)$ for all $x\in a$, then $C(v)$ for all $v:V$.
   %
     \item \emph{replacement:}\index{axiom!of replacement} given any $r : V \to V$ and $a : V$, the class
       %
       \begin{equation*}
         \setof{ x | \exis{y : V} y \in a \land x = r(y)}
       \end{equation*}
       %
       is a $V$-set.
  %
   \item \emph{separation:}\index{axiom!of separation}  given any $a : V$ and $\UU$-small $C : V \to \propU$, the class
     %
     \begin{equation*}
       \setof{ x | x \in a \land C(x)}
     \end{equation*}
     %
     is a $V$-set.
  \end{enumerate}
\end{thm}


\begin{proof}[Sketch of proof]
  \mbox{}
  %
  \begin{enumerate}
  \item Extensionality: if $\vset(A,f) \subseteq \vset(B, g)$ then $f(a) \in \vset(B, g)$
    for every $a : A$, therefore for every $a : A$ there merely exists $b : B$ such that
    $f(a) = g(b)$. The assumption $\vset(B, g) \subseteq \vset(A, f)$ gives the other half
    of~\eqref{eq:V-path}, therefore $\vset(A,f) = \vset(B,g)$.

  \item Empty set: suppose $x\in \emptyset = \vset(\emptyt,\rec\emptyt(V))$.  Then $\exis{a:\emptyt}x=\, \rec\emptyt(V,a)$, which is absurd.

  \item Pairing: given $u$ and $v$, let $w=\vset(\bool,\rec\bool(V,u,v))$.
    \index{pair!unordered}

  \item Infinity: take $w = \vset(\nat,I)$, where $I: \nat \to V$ is given by the recursion $I(0) \defeq \emptyset$ and $I(n+1) \defeq I(n)\cup \{I(n)\}$.

  \item Union: Take any $v:V$ and any presentation $f :A\to V$ with $v=\vset(A,f)$.  Then let $\tilde{A} \defeq \sm{a:A}[fa]$, where $m_{fa} : [fa] \mono V$ is the type of members from \cref{def:TypeOfElements}.  $\tilde{A}$ is plainly $\UU$-small, and we have $\cup v \defeq \vset(\tilde{A}, \lam{x} m_{f(\proj1(x))}(\proj2(x)))$.

  \item Function set: given $u, v:V$, take the types of elements $[u] \mono V$ and $[u] \mono V$, and the function type $[u]\to [v]$.  We want to define a map
  \[
 r: ([u]\to [v])\ \longrightarrow\ V
  \]
   with ``$r(f) = \setof{ \pairr{x, f(x)} | x : [u] }$'', but in order for this to make sense we must first define the ordered pair $\pairr{x, y}$, and then we take the map $r': x \mapsto \pairr{x, f(x)}$, and then we can put $r(f)\defeq \vset([u], r')$.  But the ordered pair can be defined in terms of unordered pairing as usual.

  \item $\in$-induction: let $C : V \to \prop$ be a class such that $C(a)$ holds whenever $C(x)$ for all $x\in a$, and take any $v=\vset(B,g)$.  To show that $C(v)$ by induction, assume that $C(g(b))$ for all $b:B$.  For every $x\in v$ there merely exists some $b:B$ with $x = g(b)$, and so $C(x)$.  Thus $C(v)$.

  \item Replacement: the statement ``$C$ is a $V$-set'' is a mere proposition, so we may
    proceed by induction as follows. Supposing $x$ is $\vset(A, f)$, we claim that $w
    \defeq \vset(A, r \circ f)$ is the set we are looking for.  If $C(y)$ then there merely exists
    $z : V$ and $a : A$ such that $z = f(a)$ and $y = r(z)$, therefore $y \in w$.
    Conversely, if $y \in w$ then there merely exists $a : A$ such that $y = r(f(a))$, so
    if we take $z \defeq f(a)$ we see that $C(y)$ holds.

  \item Let us say that a class $C: V\to\prop$ is \define{separable}
    \indexdef{class!separable}%
    \indexdef{separable class}%
    if for any $a:V$ the class
  %
  \symlabel{class-intersection}
  \begin{equation*}
    a \cap C \defeq\setof{x | x\in a \wedge C(x)}
  \end{equation*}
  %
  is a $V$-set.
We need to show that any $\UU$-small  $C: V \to \propU$ is separable. Indeed, given $a=\vset(A,f)$, let $A' = \exis{x:A}C(fx)$, and take $f' = f\circ i$, where $i : A' \to A$ is the obvious inclusion.  Then we can take $a' = \vset(A',f')$ and we have $x\in a\wedge C(x) \Leftrightarrow x\in a'$ as claimed.  We needed the assumption that $C$ lands in $\UU$ in order for $A' = \exis{x:A}C(fx)$ to be in $\UU$.\qedhere
\end{enumerate}
\end{proof}

It is also convenient to have a strictly syntactic criterion of separability, so that one can read off from the expression for a class that it produces a $V$-set.  One such familiar condition is being ``$\Delta_0$'', which means that the expression is built up from equality $x=_V y$ and membership $x\in y$, using only mere-propositional connectives $\neg$, $\land$, $\lor$, $\Rightarrow$ and quantifiers $\forall$, $\exists$ over particular sets, i.e.\ of the form $\exists(x\in a)$ and $\forall(y\in b)$ (these are called \define{bounded} quantifiers\index{bounded!quantifier}\index{quantifier!bounded}).\indexdef{separation!.Delta0@$\Delta_0$}%

\begin{cor}\label{cor:Delta0sep}
If the class $C: V \to \prop$ is $\Delta_0$ in the above sense, then it is separable.
\end{cor}
\index{axiom!of $\Delta_0$-separation}%

\begin{proof}
Recall that we have a $\UU$-small resizing $x \bisim y$ of identity $x = y$. Since $x\in y$ is defined in terms of $x=y$, we also have a $\UU$-small resizing of membership
%
\symlabel{resized-membership}
\begin{equation*}
  x\bin\vset(A,f) \defeq \exis{a:A} x \bisim f(a).
\end{equation*}
%
Now, let $\Phi$ be a $\Delta_0$ expression for $C$, so that as classes $\Phi = C$ (strictly speaking, we should distinguish expressions from their meanings, but we will blur the difference). Let $\widetilde{\Phi}$ be the result of replacing all occurrences of $=$ and $\in$ by their resized equivalents $\bisim$ and $\bin$.  Clearly then $\widetilde{\Phi}$ also expresses $C$, in the sense that for all $x:V$, $\widetilde{\Phi}(x) \Leftrightarrow C(x)$, and hence $\widetilde{\Phi}=C$ by univalence.  It now suffices to show that $\widetilde{\Phi}$ is $\UU$-small, for then it will be separable by the theorem.

We show that  $\widetilde{\Phi}$ is $\UU$-small by induction on the construction of the expression.  The base cases are $x \bisim y$ and $x\bin y$, which have already been resized into $\UU$.  It is also clear that $\UU$ is closed under the mere-propositional operations (and $(-1)$-truncation), so it just remains to check the bounded quantifiers $\exists(x\in a)$ and $\forall(y\in b)$.  By definition,
\begin{align*}
\exists(x\in a) P(x) &\defeq \Brck {\sm{x:V}(x\bin a \land P(x))},\\
\forall(y\in b) P(x) &\defeq  \prd{x:V}(x\bin a \to P(x)).
\end{align*}
Let us consider $\brck {\sm{x:V}(x\bin a \land P(x))}$.  Although the body $(x\bin a \land P(x))$ is $\UU$-small since $P(x)$ is so by the inductive hypothesis, the quantification over $V$ need not stay inside $\UU$.  However, in the present case we can replace this with a quantification over the type $[a]\mono V$ of members of $a$, and easily show that
\begin{equation*}
  \sm{x:V}(x\bin a \land P(x)) = \sm{x:[a]} P(x).
\end{equation*}
The right-hand side does remain in $\UU$, since both $[a]$ and $P(x)$ are in $\UU$.  The case of $\prd{x:V}(x\bin a \to P(x))$ is analogous, using $\prd{x:V}(x\bin a \to P(x)) = \prd{x:[a]}P(x)$.
\end{proof}

We have shown that in type theory with a universe $\UU$, the cumulative hierarchy $V$ is a model of a ``constructive set theory''
\index{constructive!set theory}%
with many of the standard axioms.
However, as far as we know, it lacks the \emph{strong collection}
\index{axiom!strong collection}%
\index{collection!strong}%
\index{strong!collection}%
and \emph{subset collection}
\index{axiom!subset collection}%
\index{collection!subset}%
\index{subset!collection}%
axioms which are included in \CZF{}~\cite{AczelCZF}.
In the usual interpretation of this set theory into type theory, these two axioms are consequences of the setoid-like definition of equality; while in other constructed models of set theory, strong collection may hold for other reasons.
We do not know whether either of these axioms holds in our model $(V,\in)$, but it seems unlikely.
Since $V$ is a higher inductive type \emph{inside} the system, rather than being an \emph{external} construction, it is not surprising that it differs in some ways from prior interpretations.

Finally, consider the result of adding the axiom of choice for sets to our type theory, in the form  $\choice{}$ from \cref{subsec:emacinsets} above.  This has the consequence that $\LEM{}$ then also holds, by \cref{thm:1surj_to_surj_to_pem}, and so $\set$ is a topos\index{topos} with subobject classifier $\bool$, by \cref{thm:settopos}.  In this case, we have $\prop = \bool:\UU$, and so \emph{all classes are separable}.
Thus we have shown:

\begin{lem}\label{lem:fullsep}
  In type theory with $\choice{}$, the law of \define{(full) separation}
  \indexdef{separation!full}%
  holds for $V$: given \emph{any} class $C : V \to \prop$ and $a : V$, the class $a \cap C$ is a $V$-set.
\end{lem}

\begin{thm}\label{thm:zfc}
In type theory with $\choice{}$ and a universe $\UU$, the cumulative hierarchy $V$ is a model of Zermelo--Fraenkel\index{set theory!Zermelo--Fraenkel} set theory with choice, ZFC.
\end{thm}

\begin{proof}
We have all the axioms listed in \cref{thm:VisCST}, plus full separation, so we just need to show that there are power sets\index{power set} $\power a:V$ for all $a:V$.  But since we have $\LEM{}$ these are simply function types $\power a = (a\to\bool)$.  Thus $V$ is a model of Zermelo--Fraenkel set theory ZF. We leave the verification of the set-theoretic axiom of choice from $\choice{}$ as an easy exercise.
\end{proof}

\index{bargaining|)}%

%%%%%%%%%%%%%%%%%%%%%%%%%%%%%%%%%%%%%%%%%%%%%%%%%%%%%%%%%%%%%%%%%%%%%%
\sectionNotes

The basic properties one expects of the category of sets date back to the early days of elementary topos theory.
The \emph{Elementary theory of the category of sets} referred to in \cref{subsec:emacinsets} was introduced by Lawvere\index{Lawvere} in
\cite{lawvere:etcs-long}, as a category-theoretic axiomatization of set theory.
\index{Elementary Theory of the Category of Sets}%
The notion of $\Pi W$-pretopos, regarded as a predicative version of an elementary topos, was introduced in~\cite{MoerdijkPalmgren2002}; see also~\cite{palmgren:cetcs}.

The treatment of the category of sets in \cref{sec:piw-pretopos} roughly follows that in~\cite{RijkeSpitters}.
The fact that epimorphisms are surjective (\cref{epis-surj}) is well known in classical mathematics, but is not as trivial as it may seem to prove \emph{predicatively}.
\index{mathematics!predicative}%
The proof in~\cite{Mines/R/R:1988} uses the power set operation (which is impredicative), although it can also be seen as a predicative proof of the weaker statement that a map in a universe $\UU_i$ is surjective if it is an epimorphism in the next universe $\UU_{i+1}$.
A predicative proof for setoids was given by Wilander~\cite{Wilander2010}.
Our proof is similar to Wilander's, but avoids setoids by using pushouts and univalence.

The implication in \cref{thm:1surj_to_surj_to_pem} from $\choice{}$ to $\LEM{}$ is an adaptation to homotopy type
theory of a theorem from topos theory due to Diaconescu~\cite{Diaconescu}; it was posed as a problem already by Bishop~\cite[Problem~2]{Bishop1967}.

For the intuitionistic theory of ordinal numbers, see~\cite{taylor:ordinals} and also \cite{JoyalMoerdijk1995}.
Definitions of well-foundedness in type theory by an induction principle, including the inductive predicate of accessibility\index{accessibility}, were studied in~\cite{Huet80,Paulson86,Nordstrom88}, although the idea dates back to Gentzen's proof of the consistency\index{consistency!of arithmetic} of arithmetic~\cite{Gentzen36}.

The idea of algebraic set theory, which informs our development in \cref{sec:cumulative-hierarchy} of the cumulative hierarchy, is due to~\cite{JoyalMoerdijk1995}, but it derives from earlier work by~\cite{AczelCZF}.
\index{algebraic set theory}%
\index{set theory!algebraic}%


%%%%%%%%%%%%%%%%%%%%%%%%%%%%%%%%%%%%%%%%%%%%%%%%%%%%%%%%%%%%%%%%%%%%%%
\sectionExercises

\begin{ex}\label{ex:utype-ct}
  Following the pattern of $\uset$, we would like to make a category $\utype$ of all types and maps between them (in a given universe $\UU$).  In order for this to be a category in the sense of \cref{sec:cats}, however, we must first declare $\hom(X,Y) \defeq \pizero{X\to Y}$, with composition defined by induction on truncation from ordinary composition $(Y\to Z) \to (X\to Y) \to (X\to Z)$.  This was defined as the \emph{homotopy precategory of types} in \cref{ct:hoprecat}.  It is still not a category, however, but only a precategory (its type of objects $\UU$ is not even a $0$-type).  It becomes a category by Rezk completion
  \index{completion!Rezk}%
  (see \cref{ct:hocat}), and its type of objects can be identified with $\trunc1\type$ by \cref{ct:ex:hocat}.  Show that the resulting category $\utype$, unlike $\uset$, is not a pretopos.
\end{ex}

\begin{ex}\label{ex:surjections-have-sections-impl-ac}
  Show that if every surjection has a section in the category $\uset$, then the axiom of choice holds.
\end{ex}

\begin{ex}\label{ex:well-pointed}
  Show that with $\LEM{}$, the category $\uset$ is well-pointed,
  \indexdef{category!well-pointed}%
  in the sense that the following statement holds: for any $f, g : A\to B$, if $f \neq g$ then there is a function $a : 1\to A$ such that $f(a) \neq g(a)$.
  Show that the slice category
  \index{category!slice}%
  $\uset/\bool$ consisting of functions $A\to \bool$ and commutative triangles does not have this property.
  (Hint: the terminal object in $\uset/\bool$ is the identity function $\bool \to \bool$, so in this category, there are objects $X$ that have no elements $1\to X$.)
\end{ex}

\begin{ex}\label{ex:add-ordinals}
  \index{addition!of ordinal numbers}%
  Prove that if $(A,<_A)$ and $(B,<_B)$ are well-founded, extensional, or ordinals, then so is $A+B$, with $<$ defined by
  \begin{align*}
    (a<a') &\defeq (a<_A a') & \text{for }& a,a':A\\
    (b<b') &\defeq (b<_B b') & \text{for }& b,b':B\\
    (a<b) &\defeq \unit      & \text{for }& (a:A),(b:B)\\
    (b<a) &\defeq \emptyt    & \text{for }& (a:A),(b:B).
  \end{align*}
\end{ex}
% \begin{proof}
%   We first prove by induction on $<_A$ that every element of $A$ is accessible in $A+B$.
%   This is easy since the only elements less than $a:A$ in $A+B$ are also in $A$.
%   We then prove by induction on $<_B$ that every element of $B$ is accessible in $A+B$.
%   This is easy since we have already proven that every element of $A$ is accessible.
% \end{proof}

\begin{ex}\label{ex:multiply-ordinals}
  \index{multiplication!of ordinal numbers}%
  Prove that if $(A,<_A)$ and $(B,<_B)$ are well-founded, extensional, or ordinals, then so is $A\times B$, with $<$ defined by
  \[ ((a,b) <(a',b')) \defeq (a<_A a') \vee ((a=a') \wedge (b<_B b')). \]
\end{ex}
% \begin{proof}
%   We prove by induction on $<_A$ that for every $a:A$, every element of the form $(a,b)$ is accessible in $A\times B$.
%   The inductive hypothesis is that for all $a'<_A a$, every pair $(a',b)$ is accessible.
%   Inside this induction, we prove by induction on $<_B$ that for every $b:B$, the element $(a,b)$ is accessible.
%   The nested inductive hypothesis is that for every $b'<_B b$, the element $(a,b')$ is accessible.
%   But now, if $(a',b')< (a,b)$, then either $a<_A a'$ in which case $(a',b')$ is accessible by the first inductive hypothesis, or $a=a'$ and $b'<_B b$, in which case $(a,b')$ is accessible by the second inductive hypothesis.
%   Thus, by definition of accessibility, $(a,b)$ is accessible.
%   This completes both inductions.
% \end{proof}

\begin{ex}\label{ex:algebraic-ordinals}
  Define the usual algebraic operations on ordinals, and prove that they satisfy the usual properties.
\end{ex}

\begin{ex}\label{ex:prop-ord}
  Note that $\bool$ is an ordinal, under the obvious relation $<$ such that $\bfalse<\btrue$ only.
  \begin{enumerate}
  \item Define a relation $<$ on $\prop$ which makes it into an ordinal.
  \item Show that $\id[\ord]\bool\prop$ if and only if \LEM{} holds.
  \end{enumerate}
\end{ex}

\begin{ex}\label{ex:ninf-ord}
  Recall that we denote \nat by $\omega$ when regarding it as an ordinal; thus we have also the ordinal $\omega+1$.
  On the other hand, let us define
  \[ \nat_\infty \defeq \setof{a:\nat\to\bool | \fall{n:\nat} (a_n \le a_{\suc(n)}) } \]
  where $\le$ denotes the obvious partial order on $\bool$, with $\bfalse\le\btrue$.
  \begin{enumerate}
  \item Define a relation $<$ on $\nat_\infty$ which makes it into an ordinal.
  \item Show that $\id[\ord]{\omega+1}{\nat_\infty}$ if and only if the limited principle of omniscience~\eqref{eq:lpo} holds.%
    \index{limited principle of omniscience}%
  \end{enumerate}
\end{ex}

\begin{ex}\label{ex:well-founded-extensional-simulation}
  Show that if $(A,<)$ is well-founded and extensional and $A:\UU$, then there is a simulation $A\to V$, where $(V,\in)$ is the cumulative hierarchy from \cref{sec:cumulative-hierarchy} built from the universe~\UU.
\end{ex}

\begin{ex}\label{ex:choice-function}
  Show that \cref{thm:wop}\ref{item:wop1} is equivalent to the axiom of choice~\eqref{eq:ac}.
\end{ex}

\begin{ex}\label{ex:cumhierhit}
  Given types $A$ and $B$, define a \define{bitotal relation}
  \indexsee{bitotal relation}{relation, bitotal}%
  \indexdef{relation!bitotal}%
  to be $R:A\to B\to \prop$ such that
  \[ \Big(\fall{a:A}\exis{b:B} R(a,b) \Big) \land \Big(\fall{b:B}\exis{a:A} R(a,b) \Big). \]
  For such $A,B,R$, let $A\sqcup^R B$ be the higher inductive type generated by
  \begin{itemize}
  \item $i:A\to A\sqcup^R B$
  \item $j:B\to A\sqcup^R B$
  \item For each $a:A$ and $b:B$ such that $R(a,b)$, a path $i(a)=j(b)$.
  \end{itemize}
  Show that the cumulative hierarchy $V$ can be defined by the following more straightforward list of constructors, and that the resulting induction principle is the one given in \cref{sec:cumulative-hierarchy}.
  \begin{itemize}
  \item For every $A : \UU$ and $f : A \to V$, there is an element $\vset(A, f) : V$.
  \item For any $A,B:\UU$ and bitotal relation
    \index{relation!bitotal}%
    $R:A\to B\to \prop$, and any map $h:A\sqcup^R B \to V$, there is a path $\id{\vset(A,h\circ i)}{\vset(B,h\circ j)}$.
  \item The 0-truncation constructor.
  \end{itemize}
\end{ex}

\begin{ex}\label{ex:strong-collection}
  In \CZF, the \define{axiom of strong collection}
  \indexdef{axiom!strong collection}%
  \indexdef{collection!strong}%
  \indexdef{strong!collection}%
  has the form:
   \begin{multline*}
   \fall{x\in v}\exis{y} R(x,y) \Rightarrow \\
   \exis{w}\big[\big(\fall{x\in v}\exis{y\in w}R(x,y)\big)\land \big(\fall{y\in w}\exis{x\in v}R(x,y) \big)\big]
   \end{multline*}
   Does it hold in the cumulative hierarchy $V$?  (We do not know the answer to this.)
\end{ex}

\begin{ex}\label{ex:choice-cumulative-hierarchy-choice}
Verify that, if we assume $\choice{}$, then the cumulative hierarchy $V$ satisfies the usual set-theoretic axiom of choice, which may be stated in the form:
  \[
   \fall{x\in V} \fall{y\in x}\exis{z\in V} z\in y \Rightarrow  \exis{c\in(\cup x)^x}\fall{y\in x} c(y)\in y
   \]
\end{ex}

\index{set|)}%

% Local Variables:
% TeX-master: "hott-online"
% End:


\chapter{Real numbers}
\label{cha:real-numbers}

\index{real numbers|(}%
Any foundation of mathematics worthy of its name must eventually address the construction of real numbers as understood by mathematical analysis, namely as a complete archimedean ordered field.
\index{ordered field}%
There are two notions of completeness. The one by Cauchy requires that the reals be closed under limits of Cauchy sequences\index{Cauchy!sequence}, while the stronger one by Dedekind requires closure under Dedekind cuts.\index{cut!Dedekind}
These lead to two ways of constructing reals, which we study in \autoref{sec:dedekind-reals} and \autoref{sec:cauchy-reals}, respectively. In \autoref{RD-final-field,RC-initial-Cauchy-complete} we characterize the two constructions in terms of universal properties: the Dedekind reals are the final archimedean ordered field, and the Cauchy reals the initial Cauchy complete archimedean ordered field.

In traditional constructive mathematics,
\index{mathematics!constructive}%
real numbers always seem to require certain compromises. For example, the Dedekind reals work better with power sets or some other form of impredicativity, while Cauchy reals work well in the presence of countable choice.
\index{axiom!of choice!countable}%
However, we give a new construction of the Cauchy reals as a higher inductive-inductive type that seems to be a third possibility, which requires neither power sets nor countable choice.

In~\autoref{sec:comp-cauchy-dedek} we compare the two constructions of reals. The Cauchy reals are included in the Dedekind reals. They coincide if excluded middle or countable choice holds, but in general the inclusion might be proper.

In~\autoref{sec:compactness-interval} we consider three notions of compactness of the closed interval~$[0,1]$. We first show that $[0,1]$ is metrically compact\indexdef{metrically compact}\indexdef{compactness!metric} in the sense that it is complete and totally bounded, and that uniformly continuous maps on metrically compact spaces behave as expected. In contrast, the Bolzano--Weierstra\ss{} property that every sequence has a convergent subsequence implies the limited principle of omniscience, which is an instance of excluded middle. Finally, we discuss Heine-Borel compactness. A naive formulation of the finite subcover property does not work, but a proof relevant notion of inductive covers does.
This section is basically standard constructive analysis.

The development of real numbers and analysis in homotopy type theory can be easily made compatible with classical mathematics. By assuming excluded middle and the axiom of choice we get standard classical analysis:\index{mathematics!classical}\index{classical!analysis} the Dedekind and Cauchy reals coincide, foundational questions about the impredicative nature of the Dedekind reals disappear, and the interval is as compact as it could be.

We close the chapter by constructing Conway's surreals as a higher inductive-inductive type in \autoref{sec:surreals};
the construction is more natural in univalent type theory than in  classical set theory.

In addition to the basic theory of \autoref{cha:basics,cha:logic}, as noted above we use ``higher inductive-inductive types'' for the Cauchy reals and the surreals: these combine the ideas of \autoref{cha:hits} with the notion of inductive-inductive type mentioned in \autoref{sec:generalizations}.
We will also frequently use the traditional logical notation described in \autoref{subsec:prop-trunc}, and the fact (proven in \autoref{sec:piw-pretopos}) that our ``sets'' behave the way we would expect.

Note that the total space of the universal cover of the circle, which
in \autoref{subsec:pi1s1-homotopy-theory} played a role similar to ``the real numbers'' in
classical algebraic topology, is \emph{not} the type of reals we are looking for. That
type is contractible, and thus equivalent to the singleton type, so it cannot be equipped
with a non-trivial algebraic structure.



\section{The field of rational numbers}
\label{sec:field-rati-numb}

\indexdef{rational numbers}%
\indexsee{number!rational}{rational numbers}%
We first construct the rational numbers \Q, as the reals can then be seen as a completion
of~\Q. An expert will point out that \Q could be replaced by any approximate field,
\indexdef{field!approximate}%
i.e., a subring of \Q in which arbitrarily precise approximate inverses
\index{inverse!approximate}%
exist. An example is the
ring of dyadic rationals,
\index{rational numbers!dyadic}%
which are those of the form $n/2^k$.
If we were implementing constructive mathematics on a computer,
an approximate field would be more suitable, but we leave such finesse for those
who care about the digits of~$\pi$.

We constructed the integers \Z in \autoref{sec:set-quotients} as a quotient of $\N\times
\N$, and observed that this quotient is generated by an idempotent. In
\autoref{sec:free-algebras} we saw that \Z is the free group on \unit; we could similarly
show that it is the free commutative ring\index{ring} on \emptyt. The field of rationals \Q is
constructed along the same lines as well, namely as the quotient
%
\[ \Q \defeq (\Z \times \N)/{\approx} \]
%
where
\[ (u,a) \approx (v,b) \defeq (u (b + 1) = v (a + 1)). \]
%
In other words, a pair $(u, a)$ represents the rational number $u / (1 + a)$. There can be
no division by zero because we cunningly added one to the denominator~$a$. Here too we
have a canonical choice of representatives, namely fractions in lowest terms. Thus we may
apply \autoref{lem:quotient-when-canonical-representatives} to obtain a set \Q, which
again has a decidable equality.
\index{decidable!equality}%

We do not bother to write down the arithmetical operations on \Q as we trust our readers
know how to compute with fractions even in the case when one is added to the denominator.
Let us just record the conclusion that there is an entirely unproblematic construction of
the ordered field of rational numbers \Q, with a decidable equality and decidable order.
It can also be characterized as the initial ordered field.
\index{initial!ordered field}%

\symlabel{positive-rationals}
\indexdef{rational numbers!positive}%
\indexdef{positive!rational numbers}%
Let $\Qp = \setof{ q : \Q | q > 0 }$ be the type of positive rational numbers.

\section{Dedekind reals}
\label{sec:dedekind-reals}

\index{real numbers!Dedekind|(}%
Let us first recall the basic idea of Dedekind's construction. We use two-sided Dedekind
cuts, as opposed to an often used one-sided version, because the symmetry makes
constructions more elegant, and it works constructively as well as classically.
\index{mathematics!constructive}%
A \emph{Dedekind cut}\index{cut!Dedekind} consists of a pair $(L, U)$ of subsets $L, U \subseteq \Q$, called the
\emph{lower} and \emph{upper cut} respectively, which are:
%
\begin{enumerate}
\item \emph{inhabited:} there are $q \in L$ and $r \in U$,
\item \emph{rounded:} $q \in L \Leftrightarrow \exis {r \in \Q} q < r \land r \in L$
  and $r \in U \Leftrightarrow \exis {q \in \Q} q \in U \land q < r$,
  \index{rounded!Dedekind cut}
\item \emph{disjoint:} $\lnot (q \in L \land q \in U)$, and
\item \emph{located:} $q < r \Rightarrow q \in L \lor r \in U$.
  \index{locatedness}%
\end{enumerate}
%
Reading the roundedness condition from left to right tells us that cuts are \emph{open},
\index{open!cut}%
and from right to left that they are \emph{lower}, respectively \emph{upper}, sets. The
locatedness condition states that there is no large gap between $L$ and $U$. Because cuts
are always open, they never include the ``point in between'', even when it is rational. A
typical Dedekind cut looks like this:
%
\begin{center}
  \begin{tikzpicture}[x=\textwidth]
    \draw[<-),line width=0.75pt] (0,0) -- (0.297,0) node[anchor=south east]{$L\ $};
    \draw[(->,line width=0.75pt] (0.300, 0) node[anchor=south west]{$\ U$} -- (0.9, 0) ;
  \end{tikzpicture}
\end{center}
%
We might naively translate the informal definition into type theory by saying that a cut
is a pair of maps $L, U : \Q \to \prop$. But we saw in \autoref{subsec:prop-subsets} that
$\prop$ is an ambiguous\index{typical ambiguity} notation for $\prop_{\UU_i}$ where~$\UU_i$ is a universe. Once we
use a particular $\UU_i$ to define cuts, the type of reals will reside in the next
universe $\UU_{i+1}$, a property of reals two levels higher in $\UU_{i+2}$, a property of
subsets of reals in $\UU_{i+3}$, etc. In principle we should be able to keep track of the
universe levels\index{universe level}, especially with the help of a proof assistant, but doing so here would
just burden us with bureaucracy that we prefer to avoid. We shall therefore make a
simplifying assumption that a single type of propositions $\Omega$ is sufficient for all
our purposes.

In fact, the construction of the Dedekind reals is quite resilient to logical
manipulations. There are several ways in which we can make sense of using a single type
$\Omega$:
%
\begin{enumerate}

\item We could identify $\Omega$ with the ambiguous $\prop$ and track all the universes
  that appear in definitions and constructions.

\item We could assume the propositional resizing axiom,
  \index{propositional!resizing}%
  as in \autoref{subsec:prop-subsets}, which essentially collapses the $\prop_{\UU_i}$'s to the
  lowest level\index{universe level}, which we call $\Omega$.

\item A classical mathematician who is not interested in the intricacies of type-theoretic
  universes or computation may simply assume the law of excluded middle~\eqref{eq:lem} for
  mere propositions so that $\Omega \jdeq \bool$.
  \index{excluded middle}
  This not only eradicates questions about
  levels\index{universe level} of $\prop$, but also turns everything we do into the standard classical\index{mathematics!classical}
  construction of real numbers.

\item On the other end of the spectrum one might ask for a minimal requirement that makes
  the constructions work. The condition that a mere predicate be a Dedekind cut is
  expressible using only conjunctions, disjunctions, and existential quantifiers\index{quantifier!existential} over~\Q, which
  is a countable set. Thus we could take $\Omega$ to be the initial \emph{$\sigma$-frame},
  \index{initial!sigma-frame@$\sigma$-frame}%
  \index{sigma-frame@$\sigma$-frame!initial|defstyle}%
  i.e., a lattice\index{lattice} with countable joins\index{join!in a lattice} in which binary meets distribute over countable
  joins. (The initial $\sigma$-frame cannot be the two-point lattice $\bool$ because
  $\bool$ is not closed under countable joins, unless we assume excluded middle.) This
  would lead to a construction of~$\Omega$ as a higher inductive-inductive type, but one
  experiment of this kind in \autoref{sec:cauchy-reals} is enough.
\end{enumerate}

In all of the above cases $\Omega$ is a set.
%
Without further ado, we translate the informal definition into type theory.
Throughout this chapter, we use the
logical notation from \autoref{defn:logical-notation}.

\begin{defn} \label{defn:dedekind-reals}
  A \define{Dedekind cut}
  \indexsee{Dedekind!cut}{cut, Dedekind}%
  \indexdef{cut!Dedekind}%
  is a pair $(L, U)$ of mere predicates $L : \Q \to \Omega$ and $U
  : \Q \to \Omega$ which is:
  %
  \begin{enumerate}
  \item \emph{inhabited:} $\exis{q : \Q} L(q)$ and $\exis{r : Q} U(r)$,
  \item \emph{rounded:} for all $q, r : \Q$,
    \index{rounded!Dedekind cut}
    %
    \begin{align*}
      L(q) &\Leftrightarrow \exis{r : \Q} (q < r) \land L(r)
      \qquad\text{and}\\
      U(r) &\Leftrightarrow \exis{q : \Q} (q < r) \land U(q),
    \end{align*}
  \item \emph{disjoint:} $\lnot (L(q) \land U(q))$ for all $q : \Q$,
  \item \emph{located:} $(q < r) \Rightarrow L(q) \lor U(r)$ for all $q, r : \Q$.
  \index{locatedness}%
  \end{enumerate}
  %
  We let $\dcut(L, U)$ denote the conjunction of these conditions. The type of
  \define{Dedekind reals} is
  \indexsee{Dedekind!real numbers}{real numbers, De\-de\-kind}%
  \indexdef{real numbers!Dedekind}%
  %
  \begin{equation*}
    \RD \defeq \setof{ (L, U) : (\Q \to \Omega) \times (\Q \to \Omega) | \dcut(L,U)}.
  \end{equation*}
\end{defn}

It is apparent that $\dcut(L, U)$ is a mere proposition, and since $\Q \to \Omega$ is a
set the Dedekind reals form a set too. See
\autoref{ex:RD-extended-reals,ex:RD-lower-cuts,ex:RD-interval-arithmetic} for variants of
Dedekind cuts which lead to extended reals, lower and upper reals, and the interval
domain.

There is an embedding $\Q \to \RD$ which associates with each rational $q : \Q$ the cut
$(L_q, U_q)$ where
%
\begin{equation*}
  L_q(r) \defeq (r < q)
  \qquad\text{and}\qquad
  U_q(r) \defeq (q < r).
\end{equation*}
%
We shall simply write $q$ for the cut $(L_q, U_q)$ associated with a rational number.

\subsection{The algebraic structure of Dedekind reals}
\label{sec:algebr-struct-dedek}

The construction of the algebraic and order-theoretic structure of Dedekind reals proceeds
as usual in intuitionistic logic. Rather than dwelling on details we point out the
differences between the classical\index{mathematics!classical} and intuitionistic setup. Writing $L_x$ and $U_x$ for
the lower and upper cut of a real number $x : \RD$, we define addition as%
%
\indexdef{addition!of Dedekind reals}%
\begin{align*}
  L_{x + y}(q) &\defeq \exis{r, s : \Q} L_x(r) \land L_y(s) \land q = r + s, \\
  U_{x + y}(q) &\defeq \exis{r, s : \Q} U_x(r) \land U_y(s) \land q = r + s,
\end{align*}
%
and the additive inverse by
%
\begin{align*}
  L_{-x}(q) &\defeq \exis{r : \Q} U_x(r) \land q = - r, \\
  U_{-x}(q) &\defeq \exis{r : \Q} L_x(r) \land q = - r.
\end{align*}
%
With these operations $(\RD, 0, {+}, {-})$ is an abelian\index{group!abelian} group. Multiplication is a bit
more cumbersome:
%
\indexdef{multiplication!of Dedekind reals}%
\begin{align*}
  L_{x \cdot y}(q) &\defeq
  \begin{aligned}[t]
    \exis{a, b, c, d : \Q} & L_x(a) \land U_x(b) \land L_y(c) \land U_y(d) \land {}\\
                           & \qquad q < \min (a \cdot c, a \cdot d, b \cdot c, b \cdot d),
  \end{aligned} \\
  U_{x \cdot y}(q) &\defeq
  \begin{aligned}[t]
    \exis{a, b, c, d : \Q} & L_x(a) \land U_x(b) \land L_y(c) \land U_y(d) \land {}\\
                           & \qquad \max (a \cdot c, a \cdot d, b \cdot c, b \cdot d) < q.
  \end{aligned}
\end{align*}
%
\index{interval!arithmetic}%
These formulas are related to multiplication of intervals in interval arithmetic, where
intervals $[a,b]$ and $[c,d]$ with rational endpoints multiply to the interval
%
\begin{equation*}
  [a,b] \cdot [c,d] =
  [\min(a c, a d, b c, b d), \max(a c, a d, b c, b d)].
\end{equation*}
%
For instance, the formula for the lower cut can be read as saying that $q < x \cdot y$
when there are intervals $[a,b]$ and $[c,d]$ containing $x$ and $y$, respectively, such
that $q$ is to the left of $[a,b] \cdot [c,d]$. It is generally useful to think of an
interval $[a,b]$ such that $L_x(a)$ and $U_x(b)$ as an approximation of~$x$, see
\autoref{ex:RD-interval-arithmetic}.

We now have a commutative ring\index{ring} with unit
\index{unit!of a ring}%
$(\RD, 0, 1, {+}, {-}, {\cdot})$. To treat
multiplicative inverses, we must first introduce order. Define $\leq$ and $<$ as
%
\begin{align*}
  (x \leq y) &\ \defeq \ \fall{q : \Q} L_x(q) \Rightarrow L_y(q), \\
  (x < y)    &\ \defeq \ \exis{q : \Q} U_x(q) \land L_y(q).
\end{align*}

\begin{lem} \label{dedekind-in-cut-as-le}
  For all $x : \RD$ and $q : \Q$, $L_x(q) \Leftrightarrow (q < x)$ and $U_x(q)
  \Leftrightarrow (x < q)$.
\end{lem}

\begin{proof}
  If $L_x(q)$ then by roundedness there merely is $r > q$ such that $L_x(r)$, and since
  $U_q(r)$ it follows that $q < x$. Conversely, if $q < x$ then there is $r : \Q$ such
  that $U_q(r)$ and $L_x(r)$, hence $L_x(q)$ because $L_x$ is a lower set. The other half
  of the proof is symmetric.
\end{proof}

\index{partial order}%
\index{transitivity!of . for reals@of $<$ for reals}
\index{transitivity!of . for reals@of $\leq$ for reals}
\index{relation!irreflexive}
\index{irreflexivity!of . for reals@of $<$ for reals}
The relation $\leq$ is a partial order, and $<$ is transitive and irreflexive. Linearity
\index{order!linear}%
\index{linear order}%
%
\begin{equation*}
  (x < y) \lor (y \leq x)
\end{equation*}
%
is valid if we assume excluded middle, but without it we get weak linearity
%
\index{order!weakly linear}
\index{weakly linear order}
\begin{equation} \label{eq:RD-linear-order}
  (x < y) \Rightarrow (x < z) \lor (z < y).
\end{equation}
%
At first sight it might not be clear what~\eqref{eq:RD-linear-order} has to do with
linear order. But if we take $x \jdeq u - \epsilon$ and $y \jdeq u + \epsilon$ for
$\epsilon > 0$, then we get
%
\begin{equation*}
  (u - \epsilon < z) \lor (z < u + \epsilon).
\end{equation*}
%
This is linearity ``up to a small numerical error'', i.e., since it is unreasonable to
expect that we can actually compute with infinite precision, we should not be surprised
that we can decide~$<$ only up to whatever finite precision we have computed.

To see that~\eqref{eq:RD-linear-order} holds, suppose $x < y$. Then there merely exists $q : \Q$ such that $U_x(q)$ and
$L_y(q)$. By roundedness there merely exist $r, s : \Q$ such that $r < q < s$, $U_x(r)$
and $L_y(s)$. Then, by locatedness $L_z(r)$ or $U_z(s)$. In the first case we get $x < z$
and in the second $z < y$.

Classically, multiplicative inverses exist for all numbers which are different from zero.
However, without excluded middle, a stronger condition is required. Say that $x, y : \RD$
are \define{apart}
\indexdef{apartness}%
from each other, written $x \apart y$, when $(x < y) \lor (y < x)$:
%
\symlabel{apart}
\begin{equation*}
  (x \apart y) \defeq (x < y) \lor (y < x).
\end{equation*}
%
If $x \apart y$, then $\lnot (x = y)$.
The converse is true if we assume excluded middle, but is not provable constructively.
\index{mathematics!constructive}%
Indeed, if $\lnot (x = y)$ implies $x\apart y$, then a little bit of excluded middle follows; see \autoref{ex:reals-apart-neq-MP}.

\begin{thm} \label{RD-inverse-apart-0}
  A real is invertible if, and only if, it is apart from $0$.
\end{thm}

\begin{rmk}
  We observe that a real is invertible if, and only if, it is merely
  invertible.  Indeed, the same is true in any ring,\index{ring} since a ring is a set, and
  multiplicative inverses are unique if they exist.  See the discussion
  following \autoref{cor:UC}.
\end{rmk}

\begin{proof}
  Suppose $x \cdot y = 1$. Then there merely exist $a, b, c, d : \Q$ such that
  $a < x < b$, $c < y < d$ and $0 < \min (a c, a d, b c, b d)$. From $0 < a c$ and $0 < b c$ it follows
  that $a$, $b$, and $c$ are either all positive or all negative.
  Hence either $0 < a < x$ or $x < b < 0$, so that $x \apart 0$.

  Conversely, if $x \apart 0$ then
  %
  \begin{align*}
    L_{x^{-1}}(q) &\defeq
    \exis{r : \Q} U_x(r) \land ((0 < r \land q r < 1) \lor (r < 0 \land 1 < q r))
    \\
    U_{x^{-1}}(q) &\defeq
    \exis{r : \Q} L_x(r) \land ((0 < r \land q r > 1) \lor (r < 0 \land 1 > q r))
  \end{align*}
  %
  defines the desired inverse. Indeed, $L_{x^{-1}}$ and $U_{x^{-1}}$ are inhabited because
  $x \apart 0$.
\end{proof}

\index{ordered field!archimedean}%
\index{dense}%
\indexsee{order-dense}{dense}%
The archimedean principle can be stated in several ways. We find it most illuminating in the
form which says that $\Q$ is dense in $\RD$.

\begin{thm}[Archimedean principle for $\RD$] \label{RD-archimedean}
  %
  For all $x, y : \RD$ if $x < y$ then there merely exists $q : \Q$ such that
  $x < q < y$.
\end{thm}

\begin{proof}
  By definition of $<$.
\end{proof}

Before tackling completeness of Dedekind reals, let us state precisely what algebraic
structure they possess. In the following definition we are not aiming at a minimal
axiomatization, but rather at a useful amount of structure and properties.

\begin{defn} \label{ordered-field} An \define{ordered field}
  \indexdef{ordered field}%
  \indexsee{field!ordered}{ordered field}%
  is a set $F$ together with
  constants $0$, $1$, operations $+$, $-$, $\cdot$, $\min$, $\max$, and mere relations
  $\leq$, $<$, $\apart$ such that:
  %
  \begin{enumerate}
  \item $(F, 0, 1, {+}, {-}, {\cdot})$ is a commutative ring with unit;
    \index{unit!of a ring}%
    \index{ring}%
  \item $x : F$ is invertible if, and only if, $x \apart 0$;
  \item $(F, {\leq}, {\min}, {\max})$ is a lattice;
  \item the strict order $<$ is transitive, irreflexive,
    \index{relation!irreflexive}
    \index{irreflexivity!of . in a field@of $<$ in a field}%
    and weakly linear ($x < y \Rightarrow x < z \lor z < y$);\index{transitivity!of . in a field@of $<$ in a field}
    \index{order!weakly linear}
    \index{weakly linear order}
    \index{strict!order}%
    \index{order!strict}%
  \item apartness $\apart$ is irreflexive, symmetric and cotransitive ($x \apart y \Rightarrow x \apart z \lor y \apart z$);
    \index{relation!irreflexive}
    \index{irreflexivity!of apartness}%
    \indexdef{relation!cotransitive}%
    \index{cotransitivity of apartness}%
  \item for all $x, y, z : F$:
    %
    \begin{align*}
      x \leq y &\Leftrightarrow \lnot (y < x), &
      x < y \leq z &\Rightarrow x < z, \\
      x \apart y &\Leftrightarrow (x < y) \lor (y < x), &
      x \leq y < z &\Rightarrow x < z, \\
      x \leq y &\Leftrightarrow x + z \leq y + z, &
      x \leq y \land 0 \leq z &\Rightarrow x z \leq y z, \\
      x < y &\Leftrightarrow x + z < y + z, &
      0 < z \Rightarrow (x < y &\Leftrightarrow x z < y z), \\
      0 < x + y &\Rightarrow 0 < x \lor 0 < y, &
      0 &< 1.
    \end{align*}
  \end{enumerate}
  %
  Every such field has a canonical embedding $\Q \to F$. An ordered field is
  \define{archimedean}
  \indexdef{ordered field!archimedean}%
  \indexsee{archimedean property}{ordered field, archi\-mede\-an}%
  when for all $x, y : F$, if $x < y$ then there merely exists $q :
  \Q$ such that $x < q < y$.
\end{defn}

\begin{thm} \label{RD-archimedean-ordered-field}
  The Dedekind reals form an ordered archimedean field.
\end{thm}

\begin{proof}
  We omit the proof in the hope that what we have demonstrated so far makes the theorem
  plausible.
\end{proof}

\subsection{Dedekind reals are Cauchy complete}
\label{sec:RD-cauchy-complete}

Recall that $x : \N \to \Q$ is a \emph{Cauchy sequence}\indexdef{Cauchy!sequence} when it satisfies
%
\begin{equation} \label{eq:cauchy-sequence}
  \prd{\epsilon : \Qp} \sm{n : \N} \prd{m, k \geq n} |x_m - x_k| < \epsilon.
\end{equation}
%
Note that we did \emph{not} truncate the inner existential because we actually want to
compute rates of convergence---an approximation without an error estimate carries little
useful information. By \autoref{thm:ttac}, \eqref{eq:cauchy-sequence} yields a function $M
: \Qp \to \N$, called the \emph{modulus of convergence}\indexdef{modulus!of convergence}, such that $m, k \geq M(\epsilon)$
implies $|x_m - x_k| < \epsilon$. From this we get $|x_{M(\delta/2)} - x_{M(\epsilon/2)}|<
\delta + \epsilon$ for all $\epsilon : \Qp$. In fact, the map $(\epsilon \mapsto
x_{M(\epsilon/2)}) : \Qp \to \Q$ carries the same information about the limit as the
original Cauchy condition~\eqref{eq:cauchy-sequence}. We shall work with these
approximation functions rather than with Cauchy sequences.

\begin{defn} \label{defn:cauchy-approximation}
  A \define{Cauchy approximation}
  \indexdef{Cauchy!approximation}%
  is a map $x : \Qp \to \RD$ which satisfies
  %
  \begin{equation}
    \label{eq:cauchy-approx}
    \fall{\delta, \epsilon :\Qp} |x_\delta - x_\epsilon| < \delta + \epsilon.
  \end{equation}
  %
  The \define{limit}
  \index{limit!of a Cauchy approximation}%
  of a Cauchy approximation $x : \Qp \to \RD$ is a number $\ell : \RD$ such
  that
  %
  \begin{equation*}
    \fall{\epsilon, \theta : \Qp} |x_\epsilon - \ell| < \epsilon + \theta.
  \end{equation*}
\end{defn}

\begin{thm} \label{RD-cauchy-complete}
  Every Cauchy approximation in $\RD$ has a limit.
\end{thm}

\begin{proof}
  Note that we are showing existence, not mere existence, of the limit.
  Given a Cauchy approximation $x : \Qp \to \RD$, define
  %
  \begin{align*}
    L_y(q) &\defeq \exis{\epsilon, \theta : \Qp} L_{x_\epsilon}(q + \epsilon + \theta),\\
    U_y(q) &\defeq \exis{\epsilon, \theta : \Qp} U_{x_\epsilon}(q - \epsilon - \theta).
  \end{align*}
  %
  It is clear that $L_y$ and $U_y$ are inhabited, rounded, and disjoint. To establish
  locatedness, consider any $q, r : \Q$ such that $q < r$. There is $\epsilon : \Qp$ such
  that $5 \epsilon < r - q$. Since $q + 2 \epsilon < r - 2 \epsilon$ merely
  $L_{x_\epsilon}(q + 2 \epsilon)$ or $U_{x_\epsilon}(r - 2 \epsilon)$. In the first case
  we have $L_y(q)$ and in the second $U_y(r)$.

  To show that $y$ is the limit of $x$, consider any $\epsilon, \theta : \Qp$. Because
  $\Q$ is dense in $\RD$ there merely exist $q, r : \Q$ such that
  %
  \begin{narrowmultline*}
    x_\epsilon - \epsilon - \theta/2 < q < x_\epsilon - \epsilon - \theta/4
    < x_\epsilon < \\
    x_\epsilon + \epsilon + \theta/4 < r < x_\epsilon + \epsilon + \theta/2,
  \end{narrowmultline*}
  %
  and thus $q < y < r$. Now either $y < x_\epsilon + \theta/2$ or $x_\epsilon - \theta/2 < y$.
  In the first case we have
  %
  \begin{equation*}
    x_\epsilon - \epsilon - \theta/2 < q < y < x_\epsilon + \theta/2,
  \end{equation*}
  %
  and in the second
  %
  \begin{equation*}
    x_\epsilon - \theta/2 < y < r < x_\epsilon + \epsilon + \theta/2.
  \end{equation*}
  %
  In either case it follows that $|y - x_\epsilon| < \epsilon + \theta$.
\end{proof}

For sake of completeness we record the classic formulation as well.

\begin{cor}
  Suppose $x : \N \to \RD$ satisfies the Cauchy condition~\eqref{eq:cauchy-sequence}. Then
  there exists $y : \RD$ such that
  %
  \begin{equation*}
    \prd{\epsilon : \Qp} \sm{n : \N} \prd{m \geq n} |x_m - y| < \epsilon.
  \end{equation*}
\end{cor}

\begin{proof}
  By \autoref{thm:ttac} there is $M : \Qp \to \N$ such that $\bar{x}(\epsilon) \defeq
  x_{M(\epsilon/2)}$ is a Cauchy approximation. Let $y$ be its limit, which exists by
  \autoref{RD-cauchy-complete}. Given any $\epsilon : \Qp$, let $n \defeq M(\epsilon/4)$
  and observe that, for any $m \geq n$,
  %
  \begin{narrowmultline*}
    |x_m - y| \leq |x_m - x_n| + |x_n - y| =
    |x_m - x_n| + |\bar{x}(\epsilon/2) - y| < \narrowbreak
    \epsilon/4 + \epsilon/2 + \epsilon/4 = \epsilon.\qedhere
  \end{narrowmultline*}
\end{proof}

\subsection{Dedekind reals are Dedekind complete}
\label{sec:RD-dedekind-complete}

We obtained $\RD$ as the type of Dedekind cuts on $\Q$. But we could have instead started
with any archimedean ordered field $F$ and constructed Dedekind cuts\index{cut!Dedekind} on $F$. These would
again form an archimedean ordered field $\bar{F}$, the \define{Dedekind completion of $F$},%
\index{completion!Dedekind}%
\indexsee{Dedekind!completion}{completion, Dedekind}
with $F$ contained as a subfield. What happens if we apply this construction to
$\RD$, do we get even more real numbers? The answer is negative. In fact, we shall prove a
stronger result: $\RD$ is final.

Say that an ordered field~$F$ is \define{admissible for $\Omega$}
\indexsee{admissible!ordered field}{ordered field, admissible}%
\indexdef{ordered field!admissible}%
when the strict order
$<$ on~$F$ is a map ${<} : F \to F \to \Omega$.

\begin{thm} \label{RD-final-field}
  Every archimedean ordered field which is admissible for $\Omega$ is a subfield of~$\RD$.
\end{thm}

\begin{proof}
  Let $F$ be an archimedean ordered field. For every $x : F$ define $L, U : \Q \to
  \Omega$ by
  %
  \begin{equation*}
    L_x(q) \defeq (q < x)
    \qquad\text{and}\qquad
    U_x(q) \defeq (x < q).
  \end{equation*}
  %
  (We have just used the assumption that $F$ is admissible for $\Omega$.)
  Then $(L_x, U_x)$ is a Dedekind cut.\index{cut!Dedekind} Indeed, the cuts are inhabited and rounded because
  $F$ is archimedean and $<$ is transitive, disjoint because $<$ is irreflexive, and
  located because $<$ is a weak linear order. Let $e : F \to \RD$ be the map $e(x) \defeq (L_x,
  U_x)$.

  We claim that $e$ is a field embedding which preserves and reflects the order. First of
  all, notice that $e(q) = q$ for a rational number $q$. Next we have the equivalences,
  for all $x, y : F$,
  %
  \begin{narrowmultline*}
    x < y \Leftrightarrow
    (\exis{q : \Q} x < q < y) \Leftrightarrow \narrowbreak
    (\exis{q : \Q} U_x(q) \land L_y(q)) \Leftrightarrow
    e(x) < e(y),
  \end{narrowmultline*}
  %
  so $e$ indeed preserves and reflects the order. That $e(x + y) = e(x) + e(y)$ holds
  because, for all $q : \Q$,
  %
  \begin{equation*}
    q < x + y \Leftrightarrow
    \exis{r, s : \Q} r < x \land s < y \land q = r + s.
  \end{equation*}
  %
  The implication from right to left is obvious. For the other direction, if $q < x +
  y$ then there merely exists $r : \Q$ such that $q - y < r < x$, and by taking $s \defeq
  q - r$ we get the desired $r$ and $s$. We leave preservation of multiplication by $e$ as
  an exercise.
\end{proof}

To establish that the Dedekind cuts on $\RD$ do not give us anything new, we need just one
more lemma.

\begin{lem} \label{lem:cuts-preserve-admissibility}
  If $F$ is admissible for $\Omega$ then so is its Dedekind completion.
  \index{completion!Dedekind}%
\end{lem}

\begin{proof}
  Let $\bar{F}$ be the Dedekind completion of $F$. The strict order on $\bar{F}$ is
  defined by
  %
  \begin{equation*}
    ((L,U) < (L',U')) \defeq \exis{q : \Q} U(q) \land L'(q).
  \end{equation*}
  %
  Since $U(q)$ and $L'(q)$ are elements of $\Omega$, the lemma holds as long as $\Omega$
  is closed under conjunctions and countable existentials, which we assumed from the outset.
\end{proof}


\begin{cor} \label{RD-dedekind-complete}
  %
  \indexdef{complete!ordered field, Dedekind}%
  \indexdef{Dedekind!completeness}%
  The Dedekind reals are Dedekind complete: for every real-valued Dedekind cut $(L, U)$
  there is a unique $x : \RD$ such that $L(y) = (y < x)$ and $U(y) = (x < y)$ for all $y :
  \RD$.
\end{cor}

\begin{proof}
  By \autoref{lem:cuts-preserve-admissibility} the Dedekind completion $\barRD$ of $\RD$
  is admissible for $\Omega$, so by \autoref{RD-final-field} we have an embedding $\barRD
  \to \RD$, as well as an embedding $\RD \to \barRD$. But these embeddings must be
  isomorphisms, because their compositions are order-preserving field homomorphisms\index{homomorphism!field} which
  fix the dense subfield~$\Q$, which means that they are the identity. The corollary now
  follows immediately from the fact that $\barRD \to \RD$ is an isomorphism.
\end{proof}

\index{real numbers!Dedekind|)}%

\section{Cauchy reals}
\label{sec:cauchy-reals}

\index{real numbers!Cauchy|(}%
\index{completion!Cauchy|(}%
\indexsee{Cauchy!completion}{completion, Cauchy}%
The Cauchy reals are, by intent, the completion of \Q under limits of Cauchy sequences.\index{Cauchy!sequence}
In the classical construction of the Cauchy reals, we consider the set $\mathcal{C}$ of all Cauchy sequences in \Q and then form a suitable quotient $\mathcal{C}/{\approx}$.
Then, to show that $\mathcal{C}/{\approx}$ is Cauchy complete, we consider a Cauchy sequence $x : \N \to \mathcal{C}/{\approx}$, lift it to a sequence of sequences $\bar{x} : \N \to \mathcal{C}$, and construct the limit of $x$ using $\bar{x}$. However, the lifting of~$x$ to $\bar{x}$ uses
the axiom of countable choice (the instance of~\eqref{eq:ac} where $X=\N$) or the law of excluded middle, which we may wish to avoid.
\indexdef{axiom!of choice!countable}%
Every construction of reals whose last step is a quotient suffers from this deficiency.
There are three common ways out of the conundrum in constructive mathematics:
\index{mathematics!constructive}%
%
\index{bargaining}%
\begin{enumerate}
\item Pretend that the reals are a setoid $(\mathcal{C}, {\approx})$, i.e., the type of
  Cauchy sequences $\mathcal{C}$ with a coincidence\index{coincidence, of Cauchy approximations} relation attached to it by
  administrative decree. A sequence of reals then simply \emph{is} a sequence of Cauchy
  sequences representing them.
\item Give in to temptation and accept the axiom of countable choice. After all, the axiom
  is valid in most models of constructive mathematics based on a computational viewpoint,
  such as realizability models.
\item Declare the Cauchy reals unworthy and construct the Dedekind reals instead.
  Such a verdict is perfectly valid in certain contexts, such as in sheaf-theoretic models of constructive mathematics.
  However, as we saw in \autoref{sec:dedekind-reals}, the constructive Dedekind reals have their own problems.
\end{enumerate}

Using higher inductive types, however, there is a fourth solution, which we believe to be preferable to any of the above, and interesting even to a classical mathematician.
The idea is that the Cauchy real numbers should be the \emph{free complete metric space}\index{free!complete metric space} generated by~\Q.
In general, the construction of a free gadget of any sort requires applying the gadget operations repeatedly many times to the generators.
For instance, the elements of the free group on a set $X$ are not just binary products and inverses of elements of $X$, but words obtained by iterating the product and inverse constructions.
Thus, we might naturally expect the same to be true for Cauchy completion, with the relevant ``operation'' being ``take the limit of a Cauchy sequence''.
(In this case, the iteration would have to take place transfinitely, since even after infinitely many steps there will be new Cauchy sequences to take the limit of.)

The argument referred to above shows that if excluded middle or countable choice hold, then Cauchy completion is very special: when building the completion of a space, it suffices to stop applying the operation after \emph{one step}.
This may be regarded as analogous to the fact that free monoids and free groups can be given explicit descriptions in terms of (reduced) words.
However, we saw in \autoref{sec:free-algebras} that higher inductive types allow us to construct free gadgets \emph{directly}, whether or not there is also an explicit description available.
In this section we show that the same is true for the Cauchy reals (a similar technique would construct the Cauchy completion of any metric space; see \autoref{ex:metric-completion}).
Specifically, higher inductive types allow us to \emph{simultaneously} add limits of Cauchy sequences and quotient by the coincidence relation, so that we can avoid the problem of lifting a sequence of reals to a sequence of representatives.
\index{completion!Cauchy|)}%


\subsection{Construction of Cauchy reals}
\label{sec:constr-cauchy-reals}

The construction of the Cauchy reals $\RC$ as a higher inductive type is a bit more subtle than that of the free algebraic structures considered in \autoref{sec:free-algebras}.
We intend to include a ``take the limit'' constructor whose input is a Cauchy sequence of reals, but the notion of ``Cauchy sequence of reals'' depends on having some way to measure the ``distance'' between real numbers.
In general, of course, the distance between two real numbers will be another real number, leading to a potentially problematic circularity.

However, what we actually need for the notion of Cauchy sequence of reals is not the general notion of ``distance'', but a way to say that ``the distance\index{distance} between two real numbers is less than $\epsilon$'' for any $\epsilon:\Qp$.
This can be represented by a family of binary relations, which we will denote $\mathord{\close\epsilon} : \RC\to\RC\to \prop$.
The intended meaning of $x \close\epsilon y$ is $|x - y| < \epsilon$, but since we do not have notions of subtraction, absolute value, or inequality available yet (we are only just defining $\RC$, after all), we will have to define these relations $\close\epsilon$ at the same time as we define $\RC$ itself.
And since $\close\epsilon$ is a type family indexed by two copies of $\RC$, we cannot do this with an ordinary mutual (higher) inductive definition; instead we have to use a \emph{higher inductive-inductive definition}.
\index{inductive-inductive type!higher}

Recall from \autoref{sec:generalizations} that the ordinary notion of inductive-inductive definition allows us to define a type and a type family indexed by it by simultaneous induction.
Of course, the ``higher'' version of this allows both the type and the family to have path constructors as well as point constructors.
We will not attempt to formulate any general theory of higher inductive-inductive definitions, but hopefully the description we will give of $\RC$ and $\close\epsilon$ will make the idea transparent.

\begin{rmk}
  We might also consider a \emph{higher inductive-recursive definition}, in which $\close\epsilon$ is defined using the \emph{recursion} principle of $\RC$, simultaneously with the \emph{inductive} definition of $\RC$.
  We choose the inductive-inductive route instead for two reasons.
  Firstly, higher inductive-re\-cur\-sive definitions seem to be more difficult to justify in homotopical semantics.
  Secondly, and more importantly, the inductive-inductive definition yields a more powerful induction principle, which we will need in order to develop even the basic theory of Cauchy reals.
\end{rmk}

Finally, as we did for the discussion of Cauchy completeness of the Dedekind reals in \autoref{sec:RD-cauchy-complete}, we will work with \emph{Cauchy approximations} (\autoref{defn:cauchy-approximation}) instead of Cauchy sequences.
Of course, our Cauchy approximations will now consist of Cauchy reals, rather than Dedekind reals or rational numbers.

\begin{defn}\label{defn:cauchy-reals}
  Let $\RC$ and the relation $\closesym:\Qp \times \RC \times \RC \to \type$ be the following higher inductive-inductive type family.
  The type $\RC$ of \define{Cauchy reals}
  \indexdef{real numbers!Cauchy}%
  \indexsee{Cauchy!real numbers}{real numbers, Cau\-chy}%
  is generated by the following constructors:
  \begin{itemize}
  \item \emph{rational points:}
    for any $q : \Q$ there is a real $\rcrat(q)$.
    \index{rational numbers!as Cauchy real numbers}%
  \item \emph{limit points}:
    for any $x : \Qp \to \RC$ such that
    %
    \begin{equation}
      \label{eq:RC-cauchy}
      \fall{\delta, \epsilon : \Qp} x_\delta \close{\delta + \epsilon} x_\epsilon
    \end{equation}
    %
    there is a point $\rclim(x) : \RC$. We call $x$ a \define{Cauchy approximation}.
    \indexdef{Cauchy!approximation}%
    \index{limit!of a Cauchy approximation}%
    %
  \item \emph{paths:}
    for $u, v : \RC$ such that
    %
    \begin{equation}
      \label{eq:RC-path}
      \fall{\epsilon : \Qp} u \close\epsilon v
    \end{equation}
    %
    then there is a path $\rceq(u, v) : \id[\RC]{u}{v}$.
  \end{itemize}
  Simultaneously, the type family $\closesym:\RC\to\RC\to\Qp \to\type$ is generated by the following constructors.
  Here $q$ and $r$ denote rational numbers; $\delta$, $\epsilon$, and $\eta$ denote positive rationals; $u$ and $v$ denote Cauchy reals; and $x$ and $y$ denote Cauchy approximations:
  \begin{itemize}
  \item for any $q,r,\epsilon$, if $-\epsilon < q - r < \epsilon$, then $\rcrat(q) \close\epsilon \rcrat(r)$,
  \item for any $q,y,\epsilon,\delta$, if $\rcrat(q) \close{\epsilon - \delta} y_\delta$, then $\rcrat(q) \close{\epsilon} \rclim(y)$,
  \item for any $x,r,\epsilon,\delta$, if $x_\delta \close{\epsilon - \delta} \rcrat(r)$, then $\rclim(x) \close\epsilon \rcrat(r)$,
  \item for any $x,y,\epsilon,\delta,\eta$, if $x_\delta \close{\epsilon - \delta - \eta} y_\eta$, then $\rclim(x) \close\epsilon \rclim(y)$,
  \item for any $u,v,\epsilon$, if $\xi,\zeta : u \close{\epsilon} v$, then $\xi=\zeta$ (propositional truncation).
  \end{itemize}
\end{defn}

\mentalpause

The first constructor of $\RC$ says that any rational number can be regarded as a real number.
The second says that from any Cauchy approximation to a real number, we can obtain a new real number called its ``limit''.
And the third expresses the idea that if two Cauchy approximations coincide, then their limits are equal.

The first four constructors of $\closesym$ specify when two rational numbers are close, when a rational is close to a limit, and when two limits are close.
In the case of two rational numbers, this is just the usual notion of $\epsilon$-closeness for rational numbers, whereas the other cases can be derived by noting that each approximant $x_\delta$ is supposed to be within $\delta$ of the limit $\rclim(x)$.

We remind ourselves of proof-relevance: a real number obtained from $\rclim$ is represented not
just by a Cauchy approximation $x$, but also a proof $p$ of~\eqref{eq:RC-cauchy}, so we
should technically have written $\rclim(x,p)$ instead of just $\rclim(x)$.
A similar observation also applies to $\rceq$ and~\eqref{eq:RC-path}, but we shall write just
$\rceq : u = v$ instead of $\rceq(u, v, p) : u = v$. These abuses of notation are
mitigated by the fact that we are omitting mere propositions and information that is
readily guessed.
Likewise, the last constructor of $\mathord{\close\epsilon}$ justifies our leaving the other four nameless.

We are immediately able to populate $\RC$ with many real numbers. For suppose $x : \N \to
\Q$ is a traditional Cauchy sequence\index{Cauchy!sequence} of rational numbers, and let $M : \Qp \to \N$ be its
modulus of convergence. Then $\rcrat \circ x \circ M : \Qp \to \RC$ is a Cauchy
approximation, using the first constructor of $\closesym$ to produce the necessary witness.
Thus, $\rclim(\rcrat \circ x \circ m)$ is a real number. Various famous
real numbers $\sqrt{2}$, $\pi$, $e$, \dots{} are all limits of such Cauchy sequences of
rationals.

\subsection{Induction and recursion on Cauchy reals}
\label{sec:induct-recurs-cauchy}

In order to do anything useful with $\RC$, of course, we need to give its induction principle.
As is the case whenever we inductively define two or more objects at once, the basic induction principle for $\RC$ and $\closesym$ requires a simultaneous induction over both at once.
Thus, we should expect it to say that assuming two type families over $\RC$ and $\closesym$, respectively, together with data corresponding to each constructor, there exist sections of both of these families.
However, since $\closesym$ is indexed on two copies of $\RC$, the precise dependencies of these families is a bit subtle.
The induction principle will apply to any pair of type families:
\begin{align*}
A&:\RC\to\type\\
B&:\prd{x,y:\RC} A(x) \to A(y) \to \prd{\epsilon:\Qp} (x\close\epsilon y) \to \type.
\end{align*}
The type of $A$ is obvious, but the type of $B$ requires a little thought.
Since $B$ must depend on $\closesym$, but $\closesym$ in turn depends on two copies of $\RC$ and one copy of $\Qp$, it is fairly obvious that $B$ must also depend on the variables $x,y:\RC$ and $\epsilon:\Qp$ as well as an element of $(x\close\epsilon y)$.
What is slightly less obvious is that $B$ must also depend on $A(x)$ and $A(y)$.

This may be more evident if we consider the non-dependent case (the recursion principle), where $A$ is a simple type (rather than a type family).
In this case we would expect $B$ not to depend on $x,y:\RC$ or $x\close\epsilon y$.
But the recursion principle (along with its associated uniqueness principle) is supposed to say that $\RC$ with $\close\epsilon$ is an ``initial object'' in some category, so in this case the dependency structure of $A$ and $B$ should mirror that of $\RC$ and $\close\epsilon$: that is, we should have $B:A\to A\to \Qp \to \type$.
Combining this observation with the fact that, in the dependent case, $B$ must also depend on $x,y:\RC$ and $x\close\epsilon y$, leads inevitably to the type given above for $B$.

\symlabel{RC-recursion}
It is helpful to think of $B$ as an $\epsilon$-indexed family of relations between the types $A(x)$ and $A(y)$.
With this in mind, we may write $B(x,y,a,b,\epsilon,\xi)$ as $(x,a) \bsim_\epsilon^\xi (y,b)$.
Since $\xi:x\close\epsilon y$ is unique when it exists, we generally omit it from the notation and write $(x,a) \bsim_\epsilon (y,b)$; this is harmless as long as we keep in mind that this relation is only defined when $x\close\epsilon y$.
We may also sometimes simplify further and write $a\bsim_\epsilon b$, with $x$ and $y$ inferred from the types of $a$ and $b$, but sometimes it will be necessary to include them for clarity.

\index{induction principle!for Cauchy reals}%
Now, given a type family $A:\RC\to\type$ and a family of relations $\bsim$ as above, the hypotheses of the induction principle consist of the following data, one for each constructor of $\RC$ or $\closesym$:
\begin{itemize}
\item For any $q : \Q$, an element $f_q:A(\rcrat(q))$.
\item For any Cauchy approximation $x$, and any $a:\prd{\epsilon:\Qp} A(x_\epsilon)$ such that
  \begin{equation}
    \fall{\delta, \epsilon : \Qp}
    (x_\delta,a_\delta) \bsim_{\delta+\epsilon} (x_\epsilon,a_\epsilon),
    \label{eq:depCauchyappx}
  \end{equation}
  an element $f_{x,a}:A(\rclim(x))$.
  We call such $a$ a \define{dependent Cauchy approximation}
  \indexdef{Cauchy!approximation!dependent}%
  \indexsee{approximation, Cauchy}{Cauchy approximation}%
  \indexdef{dependent!Cauchy approximation}%
  over $x$.
\item For $u, v : \RC$ such that $h:\fall{\epsilon : \Qp} u \close\epsilon v$, and all $a:A(u)$ and $b:A(v)$ such that
  $\fall{\epsilon:\Qp} (u,a) \bsim_\epsilon (v,b)$,
  a dependent path $\dpath{A}{\rceq(u,v)}{a}{b}$.
\item For $q,r:\Q$ and $\epsilon:\Qp$, if $-\epsilon < q - r < \epsilon$, we have
  \narrowequation{(\rcrat(q),f_q) \bsim_\epsilon (\rcrat(r),f_r).}
\item For $q:\Q$ and $\delta,\epsilon:\Qp$ and $y$ a Cauchy approximation, and $b$ a dependent Cauchy approximation over $y$, if $\rcrat(q) \close{\epsilon - \delta} y_\delta$, then
  \[(\rcrat(q),f_q) \bsim_{\epsilon-\delta} (y_\delta,b_\delta)
  \;\Rightarrow\;
  (\rcrat(q),f_q) \bsim_\epsilon (\rclim(y),f_{y,b}).\]
\item Similarly, for $r:\Q$ and $\delta,\epsilon:\Qp$ and $x$ a Cauchy approximation, and $a$ a dependent Cauchy approximation over $x$, if $x_\delta \close{\epsilon - \delta} \rcrat(r)$, then
  \[(x_\delta,a_\delta) \bsim_{\epsilon-\delta} (\rcrat(r),f_r)
  \;\Rightarrow\;
  (\rclim(x),f_{x,a}) \bsim_\epsilon (\rcrat(q),f_r).
  \]
\item For $\epsilon,\delta,\eta:\Qp$ and $x,y$ Cauchy approximations, and $a$ and $b$ dependent Cauchy approximations over $x$ and $y$ respectively, if we have $x_\delta \close{\epsilon - \delta - \eta} y_\eta$, then
  \[ (x_\delta,a_\delta) \bsim_{\epsilon - \delta - \eta} (y_\eta,b_\eta)
  \;\Rightarrow\;
  (\rclim(x),f_{x,a}) \bsim_\epsilon (\rclim(y),f_{y,b}).\]
\item For $\epsilon:\Qp$ and $x,y:\RC$ and $\xi,\zeta:x\close{\epsilon} y$, and $a:A(x)$ and $b:A(y)$, any two elements of $(x,a) \bsim_\epsilon^\xi (y,b)$ and $(x,a) \bsim_\epsilon^\zeta (y,b)$ are dependently equal over $\xi=\zeta$.
  Note that as usual, this is equivalent to asking that $\bsim$ takes values in mere propositions.
\end{itemize}
Under these hypotheses, we deduce functions
\begin{align*}
  f&:\prd{x:\RC} A(x)\\
  g&:\prd{x,y:\RC}{\epsilon:\Qp}{\xi:x\close{\epsilon} y}
  (x,f(x)) \bsim_\epsilon^\xi (y,f(y))
\end{align*}
which compute as expected:
\begin{align}
  f(\rcrat(q)) &\defeq f_q, \label{eq:rcsimind1}\\
  f(\rclim(x)) &\defeq f_{x,(f,g)[x]}. \label{eq:rcsimind2}
\end{align}
Here $(f,g)[x]$ denotes the result of applying $f$ and $g$ to a Cauchy approximation $x$ to obtain a dependent Cauchy approximation over $x$.
That is, we define $(f,g)[x]_\epsilon \defeq f(x_\epsilon) : A(x_\epsilon)$, and then for any $\epsilon,\delta:\Qp$ we have $g(x_\epsilon,x_\delta,\epsilon+\delta,\xi)$ to witness the fact that $(f,g)[x]$ is a dependent Cauchy approximation, where $\xi: x_\epsilon \close{\epsilon+\delta} x_\delta$ arises from the assumption that $x$ is a Cauchy approximation.

We will never use this notation again, so don't worry about remembering it.
Generally we use the pattern-matching convention, where $f$ is defined by equations such as~\eqref{eq:rcsimind1} and~\eqref{eq:rcsimind2} in which the right-hand side of~\eqref{eq:rcsimind2} may involve the symbols $f(x_\epsilon)$ and an assumption that they form a dependent Cauchy approximation.

However, this induction principle is admittedly still quite a mouthful.
To help make sense of it, we observe that it contains as special cases two separate induction principles for~$\RC$ and for~$\closesym$.
Firstly, suppose given only a type family $A:\RC\to\type$, and define $\bsim$ to be constant at \unit.
Then much of the required data becomes trivial, and we are left with:
\begin{itemize}
\item for any $q : \Q$, an element $f_q:A(\rcrat(q))$,
\item for any Cauchy approximation $x$, and any $a:\prd{\epsilon:\Qp} A(x_\epsilon)$, an element $f_{x,a}:A(\rclim(x))$,
\item for $u, v : \RC$ and $h:\fall{\epsilon : \Qp} u \close\epsilon v$, and $a:A(u)$ and $b:A(v)$, we have $\dpath{A}{\rceq(u,v)}{a}{b}$.
\end{itemize}
Given these data, the induction principle yields a function $f:\prd{x:\RC} A(x)$ such that
\begin{align*}
  f(\rcrat(q)) &\defeq f_q,\\
  f(\rclim(x)) &\defeq f_{x,f(x)}.
\end{align*}
We call this principle \define{$\RC$-induction}; it says essentially that if we take $\close\epsilon$ as given, then $\RC$ is inductively generated by its constructors.

In particular, if $A$ is a mere property, the third hypothesis in $\RC$-induction is trivial.
Thus, we may prove mere properties of real numbers by simply proving them for rationals and for limits of Cauchy approximations.
Here is an example.

\begin{lem}
  For any $u:\RC$ and $\epsilon:\Qp$, we have $u\close\epsilon u$.
\end{lem}
\begin{proof}
  Define $A(u) \defeq \fall{\epsilon:\Qp} (u\close\epsilon u)$.
  Since this is a mere proposition (by the last constructor of $\closesym$), by $\RC$-induction, it suffices to prove it when $u$ is $\rcrat(q)$ and when $u$ is $\rclim(x)$.
  In the first case, we obviously have $|q-q|<\epsilon$ for any $\epsilon$, hence $\rcrat(q) \close\epsilon \rcrat(q)$ by the first constructor of $\closesym$.
  %
  And in the second case, we may assume inductively that $x_\delta \close\epsilon x_\delta$ for all $\delta,\epsilon:\Qp$.
  Then in particular, we have $x_{\epsilon/3} \close{\epsilon/3} x_{\epsilon/3}$, whence $\rclim(x) \close{\epsilon} \rclim(x)$ by the fourth constructor of $\closesym$.
\end{proof}

\begin{thm}\label{thm:Cauchy-reals-are-a-set}
  $\RC$ is a set.
\end{thm}
\begin{proof}
  We have just shown that the mere relation
  \narrowequation{P(u,v) \defeq \fall{\epsilon:\Qp} (u\close\epsilon v)}
  is reflexive.
  Since it implies identity, by the path constructor of $\RC$, the result follows from \autoref{thm:h-set-refrel-in-paths-sets}.
\end{proof}

We can also show that although $\RC$ may not be a quotient of the set of Cauchy sequences of \emph{rationals}, it is nevertheless a quotient of the set of Cauchy sequences of \emph{reals}.
(Of course, this is not a valid \emph{definition} of $\RC$, but it is a useful property.)
We define the type of Cauchy approximations to be
%
\symlabel{cauchy-approximations}%
\index{Cauchy!approximation!type of}%
\begin{equation*}
  \CAP \defeq
  \setof{ x : \Qp \to \RC |
    \fall{\epsilon, \delta : \Qp} x_\delta \close{\delta + \epsilon} x_\epsilon
  }.
\end{equation*}
The second constructor of $\RC$ gives a function $\rclim:\CAP\to\RC$.

\begin{lem} \label{RC-lim-onto}
  Every real merely is a limit point: $\fall{u : \RC} \exis{x : \CAP} u = \rclim(x)$.
  In other words, $\rclim:\CAP\to\RC$ is surjective.
\end{lem}
\begin{proof}
  By $\RC$-induction, we may divide into cases on $u$.
  Of course, if $u$ is a limit $\rclim(x)$, the statement is trivial.
  So suppose $u$ is a rational point $\rcrat(q)$; we claim $u$ is equal to $\rclim(\lam{\epsilon} \rcrat(q))$.
  By the path constructor of $\RC$, it suffices to show $\rcrat(q) \close\epsilon \rclim(\lam{\epsilon} \rcrat(q))$ for all $\epsilon:\Qp$.
  And by the second constructor of $\closesym$, for this it suffices to find $\delta:\Qp$ such that $\rcrat(q)\close{\epsilon-\delta} \rcrat(q)$.
  But by the first constructor of $\closesym$, we may take any $\delta:\Qp$ with $\delta<\epsilon$.
\end{proof}

%

\begin{lem} \label{RC-lim-factor}
  If $A$ is a set and $f : \CAP \to A$ respects coincidence\index{coincidence!of Cauchy approximations} of Cauchy approximations, in the sense that
  %
  \begin{equation*}
    \fall{x, y : \CAP} \rclim(x) = \rclim(y) \Rightarrow f(x) = f(y),
  \end{equation*}
  %
  then $f$ factors uniquely through $\rclim : \CAP \to \RC$.
\end{lem}
\begin{proof}
  Since $\rclim$ is surjective, by \autoref{lem:images_are_coequalizers}, $\RC$ is the quotient of $\CAP$ by the kernel pair\index{kernel!pair} of $\rclim$.
  But this is exactly the statement of the lemma.
\end{proof}

For the second special case of the induction principle, suppose instead that we take $A$ to be constant at $\unit$.
In this case, $\bsim$ is simply an $\epsilon$-indexed family of relations on $\epsilon$-close pairs of real numbers, so we may write $u\bsim_\epsilon v$ instead of $(u,\ttt)\bsim_\epsilon (v,\ttt)$.
Then the required data reduces to the following, where $q, r$ denote rational numbers, $\epsilon, \delta, \eta$ positive rational numbers, and $x, y$ Cauchy approximations:
\begin{itemize}
\item if $-\epsilon < q - r < \epsilon$, then
  $\rcrat(q) \bsim_\epsilon \rcrat(r)$,
\item if $\rcrat(q) \close{\epsilon - \delta} y_\delta$ and
  $\rcrat(q)\bsim_{\epsilon-\delta} y_\delta$,
  then $\rcrat(q) \bsim_\epsilon \rclim(y)$,
\item if $x_\delta \close{\epsilon - \delta} \rcrat(r)$ and
  $x_\delta \bsim_{\epsilon-\delta} \rcrat(r)$,
  then $\rclim(y) \bsim_\epsilon \rcrat(q)$,
\item if $x_\delta \close{\epsilon - \delta - \eta} y_\eta$ and
  $x_\delta\bsim_{\epsilon - \delta - \eta} y_\eta$,
  then $\rclim(x) \bsim_\epsilon \rclim(y)$.
\end{itemize}
The resulting conclusion is $\fall{u,v:\RC}{\epsilon:\Qp} (u\close\epsilon v) \to (u \bsim_\epsilon v)$.
We call this principle \define{$\closesym$-induction}; it says essentially that if we take $\RC$ as given, then $\close\epsilon$ is inductively generated (as a family of types) by \emph{its} constructors.
For example, we can use this to show that $\closesym$ is symmetric.

\begin{lem}\label{thm:RCsim-symmetric}
  For any $u,v:\RC$ and $\epsilon:\Qp$, we have $(u\close\epsilon v) = (v\close\epsilon u)$.
\end{lem}
\begin{proof}
  Since both are mere propositions, by symmetry it suffices to show one implication.
  Thus, let $(u\bsim_\epsilon v) \defeq (v\close\epsilon u)$.
  By $\closesym$-induction, we may reduce to the case that $u\close\epsilon v$ is derived from one of the four interesting constructors of $\closesym$.
  In the first case when $u$ and $v$ are both rational, the result is trivial (we can apply the first constructor again).
  In the other three cases, the inductive hypothesis (together with commutativity of addition in $\Q$) yields exactly the input to another of the constructors of $\closesym$ (the second and third constructors switch, while the fourth stays put).
\end{proof}

The general induction principle, which we may call \define{$(\RC,\closesym)$-induction}, is therefore a sort of joint $\RC$-induction and $\closesym$-induction.
Consider, for instance, its non-dependent version, which we call \define{$(\RC,\closesym)$-recursion}, which is the one that we will have the most use for.
\index{recursion principle!for Cauchy reals}%
Ordinary $\RC$-recursion tells us that to define a function $f : \RC \to A$ it suffices to:
\begin{enumerate}
\item for every $q : \Q$ construct $f(\rcrat(q)) : A$,
\item for every Cauchy approximation $x : \Qp \to \RC$, construct $f(x) : A$,
  assuming that $f(x_\epsilon)$ has already been defined for all $\epsilon : \Qp$,
\item prove $f(u) = f(v)$ for all $u, v : \RC$ satisfying $\fall{\epsilon:\Qp} u\close\epsilon v$.\label{item:rcrec3}
\end{enumerate}
However, it is generally quite difficult to show~\ref{item:rcrec3} without knowing something about how $f$ acts on $\epsilon$-close Cauchy reals.
The enhanced principle of $(\RC,\closesym)$-recursion remedies this deficiency, allowing us to specify an \emph{arbitrary} ``way in which $f$ acts on $\epsilon$-close Cauchy reals'', which we can then prove to be the case by a simultaneous induction with the definition of $f$.
This is the family of relations $\bsim$.
Since $A$ is independent of $\RC$, we may assume for simplicity that $\bsim$ depends only on $A$ and $\Qp$, and thus there is no ambiguity in writing $a\bsim_\epsilon b$ instead of $(u,a) \bsim_\epsilon (v,b)$.
In this case, defining a function $f:\RC\to A$ by $(\RC,\closesym)$-recursion requires the following cases (which we now write using the pattern-matching convention).
\begin{itemize}
\item For every $q : \Q$, construct $f(\rcrat(q)) : A$.
\item For every Cauchy approximation $x : \Qp \to \RC$, construct $f(x) : A$, assuming inductively that $f(x_\epsilon)$ has already been defined for all $\epsilon : \Qp$ and form a ``Cauchy approximation with respect to $\bsim$'', i.e.\ that $\fall{\epsilon,\delta:\Qp} (f(x_\epsilon) \bsim_{\epsilon+\delta} f(x_\delta))$.
\item Prove that the relations $\bsim$ are \emph{separated}, i.e.\ that, for any $a,b:A$,
  \indexdef{relation!separated family of}%
  \indexdef{separated family of relations}%
\narrowequation{(\fall{\epsilon:\Qp} a\bsim_\epsilon b) \Rightarrow (a=b).}
\item Prove that if $-\epsilon< q-r <\epsilon$ for $q,r:\Q$, then $f(\rcrat(q))\bsim_\epsilon f(\rcrat(r))$.
\item For any $q:\Q$ and any Cauchy approximation $y$, prove that
\narrowequation{f(\rcrat(q)) \bsim_\epsilon f(\rclim(y)),} assuming inductively that $\rcrat(q)\close{\epsilon-\delta} y_\delta$ and $f(\rcrat(q)) \bsim_{\epsilon-\delta} f(y_\delta)$ for some $\delta:\Qp$, and that $\eta \mapsto f(x_\eta)$ is a Cauchy approximation with respect to $\bsim$.
\item For any Cauchy approximation $x$ and any $r:\Q$, prove that
\narrowequation{f(\rclim(x)) \bsim_\epsilon f(\rcrat(r)),}
assuming inductively that $x_\delta \close{\epsilon-\delta} \rcrat(r)$ and $f(x_\delta) \bsim_{\epsilon-\delta} f(\rcrat(r))$ for some $\delta:\Qp$, and that $\eta\mapsto f(x_\eta)$ is a Cauchy approximation with respect to $\bsim$.
\item For any Cauchy approximations $x,y$, prove that
\narrowequation{f(\rclim(x)) \bsim_\epsilon f(\rclim(y)),}
assuming inductively that $x_\delta \close{\epsilon-\delta-\eta} y_\eta$ and $f(x_\delta) \bsim_{\epsilon-\delta-\eta} f(y_\eta)$ for some $\delta,\eta:\Qp$, and that $\theta\mapsto f(x_\theta)$ and $\theta\mapsto f(y_\theta)$ are Cauchy approximations with respect to $\bsim$.
\end{itemize}
Note that in the last four proofs, we are free to use the specific definitions of $f(\rcrat(q))$ and $f(\rclim(x))$ given in the first two data.
However, the proof of separatedness must apply to \emph{any} two elements of $A$, without any relation to $f$: it is a sort of ``admissibility'' condition on the family of relations $\bsim$.
Thus, we often verify it first, immediately after defining $\bsim$, before going on to define $f(\rcrat(q))$ and $f(\rclim(x))$.

Under the above hypotheses, $(\RC,\closesym)$-recursion yields a function $f:\RC\to A$ such that $f(\rcrat(q))$ and $f(\rclim(x))$ are judgmentally equal to the definitions given for them in the first two clauses.
Moreover, we may also conclude
\begin{equation}
  \fall{u,v:\RC}{\epsilon:\Qp} (u\close\epsilon v) \to (f(u) \bsim_\epsilon f(v)).\label{eq:RC-sim-recursion-extra}
\end{equation}

As a paradigmatic example, $(\RC,\closesym)$-recursion allows us to extend functions defined on $\Q$ to all of $\RC$, as long as they are sufficiently continuous.
\index{function!continuous}%

\begin{defn}\label{defn:lipschitz}
  A function $f:\Q\to\RC$ is \define{Lipschitz}
  \indexdef{function!Lipschitz}%
  \indexdef{Lipschitz!function}%
  \indexdef{Lipschitz!constant}%
  \indexdef{constant!Lipschitz}%
  if there exists $L:\Qp$ (the \define{Lipschitz constant}) such that
  \[ |q - r|<\epsilon \Rightarrow (f(q) \close{L\epsilon} f(r)) \]
  for all $\epsilon:\Qp$ and $q,r:\Q$.
  %
  Similarly, $g:\RC\to\RC$ is \define{Lipschitz} if there exists $L:\Qp$ such that
  \[ (u\close\epsilon v) \Rightarrow (g(u) \close{L\epsilon} g(v)) \]
  for all $\epsilon:\Qp$ and $u,v:\RC$..
\end{defn}

In particular, note that by the first constructor of $\closesym$, if $f:\Q\to\Q$ is Lipschitz in the obvious sense, then so is the composite $\Q\xrightarrow{f} \Q \to \RC$.

\begin{lem}\label{RC-extend-Q-Lipschitz}
  Suppose $f : \Q \to \RC$ is Lipschitz with constant $L : \Qp$.
  Then there exists a Lipschitz map $\bar{f} : \RC \to \RC$, also with constant $L$, such that $\bar{f}(\rcrat(q)) \jdeq f(q)$ for all $q:\Q$.
\end{lem}

\begin{proof}
  % Uniqueness follows directly from \autoref{RC-continuous-eq}.
  We define $\bar{f}$ by $(\RC,\closesym)$-recursion, with codomain $A\defeq \RC$.
  We define the relation $\mathord{\bsim}: \RC \to \RC \to \Qp \to \prop$ to be
  \begin{align*}
    (u \bsim_\epsilon v) &\defeq (u \close{L\epsilon} v).
  \end{align*}
  For $q : \Q$, we define
  %
  \begin{equation*}
    \bar{f}(\rcrat(q)) \defeq \rcrat(f(q)).
  \end{equation*}
  %
  For a Cauchy approximation $x : \Qp \to \RC$, we define
  %
  \begin{equation*}
    \bar{f}(\rclim(x)) \defeq \rclim(\lamu{\epsilon : \Qp} \bar{f}(x_{\epsilon/L})).
  \end{equation*}
  %
  For this to make sense, we must verify that $y \defeq \lamu{\epsilon : \Qp} \bar{f}(x_{\epsilon/L})$ is a Cauchy approximation.
  However, the inductive hypothesis for this step is that for any $\delta,\epsilon:\Qp$ we have $\bar{f}(x_\delta) \bsim_{\delta+\epsilon} \bar{f}(x_\epsilon)$, i.e.\ $\bar{f}(x_\delta) \close{L\delta+L\epsilon} \bar{f}(x_\epsilon)$.
  Thus we have
  \[y_\delta \jdeq f(x_{\delta/L}) \close{\delta + \epsilon} f(x_{\epsilon/L})   \jdeq y_\epsilon. \]

  For proving separatedness, we simply observe that $\fall{\epsilon:\Qp} a\bsim_\epsilon b$ means $\fall{\epsilon:\Qp} a\close{L\epsilon} b$, which implies $\fall{\epsilon:\Qp}a\close\epsilon b$ and thus $a=b$.

  To complete the $(\RC,\closesym)$-recursion, it remains to verify the four conditions on $\bsim$.
  This basically amounts to proving that $\bar f$ is Lipschitz for all the four constructors of $\closesym$.
  \begin{enumerate}
  \item When $u$ is $\rcrat(q)$ and $v$ is $\rcrat(r)$ with $-\epsilon < |q-r| <\epsilon$, the assumption that $f$ is Lipschitz yields $f(q) \close{L\epsilon} f(r)$, hence $\bar{f}(\rcrat(q)) \bsim_\epsilon \bar{f}(\rcrat(r))$ by definition.
  \item When $u$ is $\rclim(x)$ and $v$ is $\rcrat(q)$ with $x_\eta \close{\epsilon - \eta} \rcrat(q)$, then the
      inductive hypothesis is $\bar{f}(x_\eta) \close{L \epsilon - L \eta} \rcrat(f(q))$, which proves
      \narrowequation{\bar{f}(\rclim(x)) \close{L \epsilon} \bar{f}(\rcrat(q))}
      by the third constructor of $\closesym$.
  \item The symmetric case when $u$ is rational and $v$ is a limit is essentially identical.
  \item When $u$ is $\rclim(x)$ and $v$ is $\rclim(y)$, with $\delta, \eta : \Qp$ such that $x_\delta \close{\epsilon - \delta - \eta} y_\eta$,
      the inductive hypothesis is $\bar{f}(x_\delta) \close{L \epsilon - L \delta - L \eta} \bar{f}(y_\eta)$, which proves $\bar{f}(\rclim(x)) \close{L
        \epsilon} \bar{f}(\rclim(y))$ by the fourth constructor of $\closesym$.
  \end{enumerate}
  This completes the $(\RC,\closesym)$-recursion, and hence the construction of $\bar f$.
  The desired equality $\bar f(\rcrat(q))\jdeq f(q)$ is exactly the first computation rule for $(\RC,\closesym)$-recursion, and the additional condition~\eqref{eq:RC-sim-recursion-extra} says exactly that $\bar f$ is Lipschitz with constant $L$.
\end{proof}

At this point we have gone about as far as we can without a better characterization of $\closesym$.
We have specified, in the constructors of $\closesym$, the conditions under which we want Cauchy reals of the two different forms to be $\epsilon$-close.
However, how do we know that in the resulting inductive-inductive type family, these are the \emph{only} witnesses to this fact?
We have seen that inductive type families (such as identity types, see \autoref{sec:identity-systems}) and higher inductive types have a tendency to contain ``more than was put into them'', so this is not an idle question.

In order to characterize $\closesym$ more precisely, we will define a family of relations $\approx_\epsilon$ on $\RC$ \emph{recursively}, so that they will compute on constructors, and prove that this family is equivalent to $\close\epsilon$.

\begin{thm}\label{defn:RC-approx}
  There is a family of mere relations $\mathord\approx:\RC\to\RC\to\Qp\to\prop$ such that
  \begin{align}
    (\rcrat(q) \approx_\epsilon \rcrat(r))  &\defeq
    (-\epsilon < q - r < \epsilon)\label{eq:RCappx1}\\
    (\rcrat(q) \approx_\epsilon \rclim(y)) &\defeq
    \exis{\delta : \Qp} \rcrat(q) \approx_{\epsilon - \delta} y_\delta\label{eq:RCappx2}\\
    (\rclim(x) \approx_\epsilon \rcrat(r)) &\defeq
    \exis{\delta : \Qp} x_\delta \approx_{\epsilon - \delta} \rcrat(r)\label{eq:RCappx3}\\
    (\rclim(x) \approx_\epsilon \rclim(y)) &\defeq
    \exis{\delta, \eta : \Qp} x_\delta \approx_{\epsilon - \delta - \eta} y_\eta.\label{eq:RCappx4}
  \end{align}
  Moreover, we have
  \begin{gather}
    (u \approx_\epsilon v) \Leftrightarrow \exis{\theta:\Qp} (u \approx_{\epsilon-\theta} v) \label{RC-sim-rounded}\\
    (u \approx_\epsilon v) \to (v\close\delta w) \to (u\approx_{\epsilon+\delta} w)\label{eq:RC-sim-rtri}\\
    (u \close\epsilon v) \to (v\approx_\delta w) \to (u\approx_{\epsilon+\delta} w)\label{eq:RC-sim-ltri}.
  \end{gather}
\end{thm}

The additional conditions~\eqref{RC-sim-rounded}--\eqref{eq:RC-sim-ltri} turn out to be required in order to make the inductive definition go through.
Condition~\eqref{RC-sim-rounded} is called being \define{rounded}.
\indexsee{relation!rounded}{rounded relation}%
\indexdef{rounded!relation}%
Reading it from right to left gives \define{monotonicity} of $\approx$,
\index{monotonicity}%
\index{relation!monotonic}%
%
\begin{equation*}
  (\delta < \epsilon) \land (u \approx_\delta v) \Rightarrow (u \approx_\epsilon v)
\end{equation*}
%
while reading it left to right to \define{openness} of $\approx$,
\index{open!relation}%
\index{relation!open}%
%
\begin{equation*}
  (u \approx_\epsilon v) \Rightarrow \exis{\epsilon : \Qp} (\delta < \epsilon) \land (u \approx_\delta v).
\end{equation*}
%
Conditions~\eqref{eq:RC-sim-rtri} and~\eqref{eq:RC-sim-ltri} are forms of the triangle inequality, which say that $\approx$ is a ``module'' over $\closesym$ on both sides.

\begin{proof}
  We will define $\mathord\approx:\RC\to\RC\to\Qp\to\prop$ by double $(\RC,\closesym)$-recursion.
  First we will apply $(\RC,\closesym)$-recursion with codomain the subset of $\RC\to\Qp\to\prop$ consisting of those families of predicates which are rounded and satisfy the one appropriate form of the triangle inequality.
  Thinking of these predicates as half of a binary relation, we will write them as $(u,\epsilon) \mapsto (\hapx_\epsilon u)$, with the symbol $\hapname$ referring to the whole relation.
  Now we can write $A$ precisely as
  \begin{multline*}
    A \defeq\; \Bigg\{ \hapname :\RC\to\Qp\to\prop \;\bigg|\; \\
    \Big(\fall{u:\RC}{\epsilon:\Qp}
    \big((\hapx_\epsilon u) \Leftrightarrow \exis{\theta:\Qp} (\hapx_{\epsilon-\theta} u)\big)\Big)  \\
    \land \Big(\fall{u,v:\RC}{\eta,\epsilon:\Qp} (u\close\epsilon v) \to\\
    \big((\hapx_\eta u) \to (\hapx_{\eta+\epsilon} v) \big) \land \big((\hapx_\eta v) \to (\hapx_{\eta+\epsilon} u) \big)\Big)\Bigg\}
  \end{multline*}
  As usual with subsets, we will use the same notation for an inhabitant of $A$ and its first component $\hapname$.
  As the family of relations required for $(\RC,\closesym)$-recursion, we consider the following, which will ensure the other form of the triangle inequality:
  \begin{narrowmultline*}
    (\hapname \bsim_\epsilon \hapbname ) \defeq \narrowbreak
    \fall{u:\RC}{\eta:\Qp} ((\hapx_\eta u) \to (\hapxb_{\epsilon+\eta} u))
    \land \narrowbreak
    ((\hapxb_\eta u) \to (\hapx_{\epsilon+\eta} u)).
  \end{narrowmultline*}
  We observe that these relations are separated.
  For assuming
  \narrowequation{\fall{\epsilon:\Qp} (\hapname \bsim_\epsilon \hapbname),}
  to show $\hapname = \hapbname$ it suffices to show $(\hapx_\epsilon u) \Leftrightarrow (\hapxb_\epsilon u)$ for all $u:\RC$.
  But $\hapx_\epsilon u$ implies $\hapx_{\epsilon-\theta} u$ for some $\theta$, by roundedness, which together with $\hapname \bsim_\epsilon \hapbname$ implies $\hapxb_\epsilon u$; and the converse is identical.

  Now the first two data the recursion principle requires are the following.
  \begin{itemize}
  \item For any $q:\Q$, we must give an element of $A$, which we denote $(\rcrat(q)\approx_{(\blank)} \blank)$.
  \item For any Cauchy approximation $x$, if we assume defined a function $\Qp \to A$, which we will denote by $\epsilon \mapsto (x_\epsilon \approx_{(\blank)} \blank)$, with the property that
    % \[ \fall{u,v:\RC}{\delta,\epsilon,\eta:\Qp} (x_\delta \approx_\eta u) \to (u\close{\delta+\epsilon} v) \to (x_\epsilon \approx_{\eta+\delta+\epsilon} v) \]
    \begin{equation}
      \fall{u:\RC}{\delta,\epsilon,\eta:\Qp} (x_\delta \approx_\eta u) \to (x_\epsilon \approx_{\eta+\delta+\epsilon} u),\label{eq:appxrec2}
    \end{equation}
    we must give an element of $A$, which we write as $(\rclim(x)\approx_{(\blank)} \blank)$.
  \end{itemize}
  In both cases, we give the required definition by using a nested $(\RC,\closesym)$-recursion, with codomain the subset of $\Qp\to\prop$ consisting of rounded families of mere propositions.
  Thinking of these propositions as zero halves of a binary relation, we will write them as $\epsilon \mapsto (\tap{\epsilon})$, with the symbol $\tapname$ referring to the whole family.
  Now we can write the codomain of these inner recursions precisely as
  \begin{narrowmultline*}
    C \defeq
    \bigg\{ \tapname :\Qp\to\prop \;\;\Big|\;\; \narrowbreak
    \fall{\epsilon:\Qp} \Big((\tap\epsilon) \Leftrightarrow \exis{\theta:\Qp} (\tap{\epsilon-\theta})\Big)\bigg\}
  \end{narrowmultline*}
  We take the required family of relations to be the remnant of the triangle inequality:
  \begin{narrowmultline*}
    (\tapname \bbsim_\epsilon \tapbname) \defeq
    \fall{\eta:\Qp} ((\tap\eta) \to (\tapb{\epsilon+\eta})) \land
    \narrowbreak
    ((\tapb\eta) \to (\tap{\epsilon+\eta})).
  \end{narrowmultline*}
  These relations are separated by the same argument as for $\bsim$, using roundedness of all elements of $C$.

  Note that if such an inner recursion succeeds, it will yield a family of predicates $\hapname : \RC\to\Qp\to \prop$ which are rounded
(since their image in $\Qp\to\prop$ lies in $C$) and satisfy
  \[ \fall{u,v:\RC}{\epsilon:\Qp} (u\close\epsilon v) \to \big((\hapx_{(\blank)} u) \bbsim_\epsilon (\hapx_{(\blank)} u)\big). \]
  Expanding out the definition of $\bbsim$, this yields precisely the third condition for $\hapname$ to belong to $A$; thus it is exactly what we need.

  It is at this point that we can give the definitions~\eqref{eq:RCappx1}--\eqref{eq:RCappx4}, as the first two clauses of each of the two inner recursions, corresponding to rational points and limits.
  In each case, we must verify that the relation is rounded and hence lies in $C$.
  In the rational-rational case~\eqref{eq:RCappx1} this is clear, while in the other cases it follows from an inductive hypothesis.
  (In~\eqref{eq:RCappx2} the relevant inductive hypothesis is that $(\rcrat(q) \approx_{(\blank)} y_\delta) : C$, while in~\eqref{eq:RCappx3} and~\eqref{eq:RCappx4} it is that $(x_\delta \approx_{(\blank)} \blank) : A$.)

  The remaining data of the sub-recursions consist of showing that \eqref{eq:RCappx1}--\eqref{eq:RCappx4} satisfy the triangle inequality on the right with respect to the constructors of $\closesym$.
  There are eight cases --- four in each sub-recursion --- corresponding to the eight possible ways that $u$, $v$, and $w$ in~\eqref{eq:RC-sim-rtri} can be chosen to be rational points or limits.
  First we consider the cases when $u$ is $\rcrat(q)$.
  \begin{enumerate}
  \item Assuming $\rcrat(q)\approx_\phi \rcrat(r)$ and $-\epsilon<|r-s|<\epsilon$, we must show $\rcrat(q)\approx_{\phi+\epsilon} \rcrat(s)$.
    But by definition of $\approx$, this reduces to the triangle inequality for rational numbers.
  \item We assume $\phi,\epsilon,\delta:\Qp$ such that $\rcrat(q)\approx_\phi \rcrat(r)$ and $\rcrat(r) \close{\epsilon-\delta} y_\delta$, and inductively that
    \begin{equation}
      \fall{\psi:\Qp}(\rcrat(q) \approx_{\psi} \rcrat(r)) \to (\rcrat(q) \approx_{\psi+\epsilon-\delta} y_\delta).\label{eq:RCappx-rtri-rrl1}
    \end{equation}
    We assume also that $\psi,\delta\mapsto (\rcrat(q) \approx_{\psi} y_\delta)$ is a Cauchy approximation with respect to $\bbsim$, i.e.\
    \begin{equation}
      \fall{\psi,\xi,\zeta:\Qp} (\rcrat(q) \approx_{\psi} y_\xi) \to (\rcrat(q) \approx_{\psi+\xi+\zeta} y_\zeta),\label{eq:RCappx-rtri-rrl2}
    \end{equation}
    although we do not need this assumption in this case.
    Indeed, \eqref{eq:RCappx-rtri-rrl1} with $\psi\defeq \phi$ yields immediately $\rcrat(q) \approx_{\phi+\epsilon-\delta} y_\delta$, and hence $\rcrat(q) \approx_{\phi+\epsilon} \rclim(y)$ by definition of $\approx$.
  \item We assume $\phi,\epsilon,\delta:\Qp$ such that $\rcrat(q)\approx_\phi \rclim(y)$ and $y_\delta \close{\epsilon-\delta} \rcrat(r)$, and inductively that
    \begin{gather}
      \fall{\psi:\Qp}(\rcrat(q) \approx_{\psi} y_\delta) \to (\rcrat(q) \approx_{\psi+\epsilon-\delta} \rcrat(r)).\label{eq:RCappx-rtri-rlr1}\\
      \fall{\psi,\xi,\zeta:\Qp} (\rcrat(q) \approx_{\psi} y_\xi) \to (\rcrat(q) \approx_{\psi+\xi+\zeta} y_\zeta).\label{eq:RCappx-rtri-rlr2}
    \end{gather}
    By definition, $\rcrat(q)\approx_\phi \rclim(y)$ means that we have $\theta:\Qp$ with $\rcrat(q) \approx_{\phi-\theta} y_\theta$.
    By assumption~\eqref{eq:RCappx-rtri-rlr2}, therefore, we have also $\rcrat(q) \approx_{\phi+\delta} y_\delta$, and then by~\eqref{eq:RCappx-rtri-rlr1} it follows that $\rcrat(q) \approx_{\phi+\epsilon} \rcrat(r)$, as desired.
  \item We assume $\phi,\epsilon,\delta,\eta:\Qp$ such that $\rcrat(q)\approx_\phi \rclim(y)$ and $y_\delta \close{\epsilon-\delta-\eta} z_\eta$, and inductively that
    \begin{gather}
      \fall{\psi:\Qp}(\rcrat(q) \approx_{\psi} y_\delta) \to (\rcrat(q) \approx_{\psi+\epsilon-\delta-\eta} z_\eta), \label{eq:RCappx-rtri-rll1}\\
      \fall{\psi,\xi,\zeta:\Qp} (\rcrat(q) \approx_{\psi} y_\xi) \to (\rcrat(q) \approx_{\psi+\xi+\zeta} y_\zeta), \label{eq:RCappx-rtri-rll2}\\
      \fall{\psi,\xi,\zeta:\Qp} (\rcrat(q) \approx_{\psi} z_\xi) \to (\rcrat(q) \approx_{\psi+\xi+\zeta} z_\zeta). \label{eq:RCappx-rtri-rll3}
    \end{gather}
    Again, $\rcrat(q)\approx_\phi \rclim(y)$ means we have $\xi:\Qp$ with $\rcrat(q) \approx_{\phi-\xi} y_\xi$, while~\eqref{eq:RCappx-rtri-rll2} then implies $\rcrat(q) \approx_{\phi+\delta} y_\delta$ and~\eqref{eq:RCappx-rtri-rll1} implies $\rcrat(q) \approx_{\phi+\epsilon-\eta} z_\eta$.
    But by definition of $\approx$, this implies $\rcrat(q) \approx_{\phi+\epsilon} \rclim(z)$ as desired.
  \end{enumerate}
  Now we move on to the cases when $u$ is $\rclim(x)$, with $x$ a Cauchy approximation.
  In this case, the ambient inductive hypothesis of the definition of $(\rclim(x) \approx_{(\blank)} {\blank}) : A$ is that we have ${(x_\delta \approx_{(\blank)} {\blank})}: A$, so that in addition to being rounded they satisfy the triangle inequality on the right.
  \begin{enumerate}\setcounter{enumi}{4}
  \item Assuming $\rclim(x)\approx_\phi \rcrat(r)$ and $-\epsilon<|r-s|<\epsilon$, we must show $\rclim(x)\approx_{\phi+\epsilon} \rcrat(s)$.
    By definition of $\approx$, the former means $x_\delta \approx_{\phi-\delta} \rcrat(r)$, so that above triangle inequality implies $x_\delta \approx_{\epsilon+\phi-\delta} \rcrat(s)$, hence $\rclim(x)\approx_{\phi+\epsilon} \rcrat(s)$ as desired.
  \item We assume $\phi,\epsilon,\delta:\Qp$ such that $\rclim(x)\approx_\phi \rcrat(r)$ and $\rcrat(r) \close{\epsilon-\delta} y_\delta$, and two unneeded inductive hypotheses.
    %
    By definition, we have $\eta:\Qp$ such that $x_\eta \approx_{\phi-\eta} \rcrat(r)$, so the inductive triangle inequality gives $x_\eta \approx_{\phi+\epsilon-\eta-\delta} y_\delta$.
    The definition of $\approx$ then immediately yields $\rclim(x) \approx_{\phi+\epsilon} \rclim(y)$.
  \item We assume $\phi,\epsilon,\delta:\Qp$ such that $\rclim(x)\approx_\phi \rclim(y)$ and $y_\delta \close{\epsilon-\delta} \rcrat(r)$, and two unneeded inductive hypotheses.
    By definition we have $\xi,\theta:\Qp$ such that $x_\xi \approx_{\phi-\xi-\theta} y_\theta$.
    Since $y$ is a Cauchy approximation, we have $y_\theta \close{\theta+\delta} y_\delta$, so the inductive triangle inequality gives $x_\xi \approx_{\phi+\delta-\xi} y_\delta$ and then $x_\xi \close{\phi+\epsilon-\xi} \rcrat(r)$.
    The definition of $\approx$ then gives $\rclim(x) \approx_{\phi+\epsilon}\rcrat(r)$, as desired.
  \item Finally, we assume $\phi,\epsilon,\delta,\eta:\Qp$ such that $\rclim(x)\approx_\phi \rclim(y)$ and $y_\delta \close{\epsilon-\delta-\eta} z_\eta$.
    Then as before we have $\xi,\theta:\Qp$ with $x_\xi \approx_{\phi-\xi-\theta} y_\theta$, and two applications of the triangle inequality suffices as before.
  \end{enumerate}

  This completes the two inner recursions, and thus the definitions of the families of relations $(\rcrat(q)\approx_{(\blank)}\blank)$ and $(\rclim(x)\approx_{(\blank)}\blank)$.
  Since all are elements of $A$, they are rounded and satisfy the triangle inequality on the right with respect to $\closesym$.
% , and satisfy~\eqref{eq:appxrec2}.
  What remains is to verify the conditions relating to $\bsim$, which is to say that these relations satisfy the triangle inequality on the \emph{left} with respect to the constructors of $\closesym$.
  The four cases correspond to the four choices of rational or limit points for $u$ and $v$ in~\eqref{eq:RC-sim-ltri}, and since they are all mere propositions, we may apply $\RC$-induction and assume that $w$ is also either rational or a limit.
  This yields another eight cases, whose proofs are essentially identical to those just given; so we will not subject the reader to them.
\end{proof}

We can now prove:

\begin{thm}\label{thm:RC-sim-characterization}
  For any $u,v:\RC$ and $\epsilon:\Qp$ we have $(u\close\epsilon v) = (u\approx_\epsilon v)$.
\end{thm}
\begin{proof}
  Since both are mere propositions, it suffices to prove bidirectional implication.
  For the left-to-right direction, we use $\closesym$-induction applied to $C(u,v,\epsilon)\defeq (u\approx_\epsilon v)$.
  Thus, it suffices to consider the four constructors of $\closesym$.
  In each case, $u$ and $v$ are specialized to either rational points or limits, so that the definition of $\approx$ evaluates, and the inductive hypothesis always applies.

  For the right-to-left direction, we use $\RC$-induction to assume that $u$ and $v$ are rational points or limits, allowing $\approx$ to evaluate.
  But now the definitions of $\approx$, and the inductive hypotheses, supply exactly the data required for the relevant constructors of $\closesym$.
\end{proof}

\index{encode-decode method}%
Stretching a point, one might call $\approx$ a fibration of ``codes'' for $\closesym$, with the two directions of the above proof being \encode and \decode respectively.
By the definition of $\approx$, from \autoref{thm:RC-sim-characterization} we get equivalences
\begin{align*}
  (\rcrat(q) \close\epsilon \rcrat(r))  &=
  (-\epsilon < q - r < \epsilon)\\
  (\rcrat(q) \close\epsilon \rclim(y)) &=
  \exis{\delta : \Qp} \rcrat(q) \close{\epsilon - \delta} y_\delta\\
  (\rclim(x) \close\epsilon \rcrat(r)) &=
  \exis{\delta : \Qp} x_\delta \close{\epsilon - \delta} \rcrat(r)\\
  (\rclim(x) \close\epsilon \rclim(y)) &=
  \exis{\delta, \eta : \Qp} x_\delta \close{\epsilon - \delta - \eta} y_\eta.
\end{align*}
Our proof also provides the following additional information.

\begin{cor}
  \index{triangle!inequality for R@inequality for $\RC$}%
  \indexsee{inequality!triangle}{triangle inequality}%
  $\closesym$ is rounded\index{rounded!relation} and satisfies the triangle inequality:
    \begin{gather}
      \eqvspaced{
        (u \close\epsilon v)
      }{
        \exis{\theta : \Qp} u \close{\epsilon - \theta} v
      }\\
      (u\close\epsilon v) \to (v\close\delta w) \to (u\close{\epsilon+\delta} w). \label{item:RC-sim-triangle}
    \end{gather}
\end{cor}
% \begin{proof}
%   The construction of $\approx$ showed simultaneously that it is rounded, and satisfies ``triangle inequalities'' such as
%   \[ (u\approx_\epsilon v) \to (v\close\delta w) \to (u\approx_{\epsilon+\delta} w). \]
%   Thus, both properties follow from \autoref{thm:RC-sim-characterization}.
% \end{proof}

With the triangle inequality in hand, we can show that ``limits'' of Cauchy approximations actually behave like limits.

\begin{lem}\label{thm:RC-sim-lim}
  For any $u:\RC$, Cauchy approximation $y$, and $\epsilon,\delta:\Qp$, if $u\close\epsilon y_\delta$ then $u\close{\epsilon+\delta} \rclim(y)$.
\end{lem}
\begin{proof}
  We use $\RC$-induction on $u$.
  If $u$ is $\rcrat(q)$, then this is exactly the second constructor of $\closesym$.
  Now suppose $u$ is $\rclim(x)$, and that each $x_\eta$ has the property that for any $y,\epsilon,\delta$, if $x_\eta\close\epsilon y_\delta$ then $x_\eta \close{\epsilon+\delta} \rclim(y)$.
  In particular, taking $y\defeq x$ and $\delta\defeq\eta$ in this assumption, we conclude that $x_\eta \close{\eta+\theta} \rclim(x)$ for any $\eta,\theta:\Qp$.

  Now let $y,\epsilon,\delta$ be arbitrary and assume $\rclim(x) \close\epsilon y_\delta$.
  By roundedness, there is a $\theta$ such that $\rclim(x) \close{\epsilon-\theta} y_\delta$.
  Then by the above observation, for any $\eta$ we have $x_\eta \close{\eta+\theta/2} \rclim(x)$, and hence $x_\eta \close{\epsilon+\eta-\theta/2} y_\delta$ by the triangle inequality.
  Hence, the fourth constructor of $\closesym$ yields $\rclim(x) \close{\epsilon+2\eta+\delta-\theta/2} \rclim(y)$.
  Thus, if we choose $\eta \defeq \theta/4$, the result follows.
\end{proof}

\begin{lem}\label{thm:RC-sim-lim-term}
  For any Cauchy approximation $y$ and any $\delta,\eta:\Qp$ we have $y_\delta \close{\delta+\eta} \rclim(y)$.
\end{lem}
\begin{proof}
  Take $u\defeq y_\delta$ and $\epsilon\defeq \eta$ in the previous lemma.
\end{proof}

\begin{rmk}
  We might have expected to have $y_\delta \close{\delta} \rclim(y)$, but this fails in examples.
  For instance, consider $x$ defined by $x_\epsilon \defeq \epsilon$.
  Its limit is clearly $0$, but we do not have $|\epsilon - 0 |<\epsilon$, only $\le$.
\end{rmk}

As an application, \autoref{thm:RC-sim-lim-term} enables us to show that the extensions of Lipschitz functions from \autoref{RC-extend-Q-Lipschitz} are unique.

\begin{lem}\label{RC-continuous-eq}
  \index{function!continuous}%
  Let $f,g:\RC\to\RC$ be continuous, in the sense that
  \[ \fall{u:\RC}{\epsilon:\Qp}\exis{\delta:\Qp}\fall{v:\RC} (u\close\delta v) \to (f(u) \close\epsilon f(v)) \]
  and analogously for $g$.
  If $f(\rcrat(q))=g(\rcrat(q))$ for all $q:\Q$, then $f=g$.
\end{lem}
\begin{proof}
  We prove $f(u)=g(u)$ for all $u$ by $\RC$-induction.
  The rational case is just the hypothesis.
  Thus, suppose $f(x_\delta)=g(x_\delta)$ for all $\delta$.
  We will show that $f(\rclim(x))\close\epsilon g(\rclim(x))$ for all $\epsilon$, so that the path constructor of $\RC$ applies.

  Since $f$ and $g$ are continuous, there exist $\theta,\eta$ such that for all $v$, we have
  \begin{align*}
    (\rclim(x)\close\theta v) &\to (f(\rclim(x)) \close{\epsilon/2} f(v))\\
    (\rclim(x)\close\eta v) &\to (g(\rclim(x)) \close{\epsilon/2} g(v)).
  \end{align*}
  Choosing $\delta < \min(\theta,\eta)$, by \autoref{thm:RC-sim-lim-term} we have both $\rclim(x)\close\theta y_\delta$ and $\rclim(x)\close\eta y_\delta$.
  Hence
  \[ f(\rclim(x)) \close{\epsilon/2} f(y_\delta) = g(y_\delta) \close{\epsilon/2} g(\rclim(x))\]
  and thus $f(\rclim(x))\close\epsilon g(\rclim(x))$ by the triangle inequality.
\end{proof}

\subsection{The algebraic structure of Cauchy reals}
\label{sec:algebr-struct-cauchy}

We first define the additive structure $(\RC, 0, {+}, {-})$. Clearly, the additive unit element
$0$ is just $\rcrat(0)$, while the additive inverse ${-} : \RC \to \RC$ is obtained as the
extension of the additive inverse ${-} : \Q \to \Q$, using \autoref{RC-extend-Q-Lipschitz}
with Lipschitz constant~$1$. We have to work a bit harder for addition.

\begin{lem} \label{RC-binary-nonexpanding-extension}
  Suppose $f : \Q \times \Q \to \Q$ satisfies, for all $q, r, s : \Q$,
  %
  \begin{equation*}
    |f(q, s) - f(r, s)| \leq |q - r|
    \qquad\text{and}\qquad
    |f(q, r) - f(q, s)| \leq |r - s|.
  \end{equation*}
  %
  Then there is a function $\bar{f} : \RC \times \RC \to \RC$ such that
  $\bar{f}(\rcrat(q), \rcrat(r)) = f(q,r)$ for all $q, r : \Q$. Furthermore,
  for all $u, v, w : \RC$ and $q : \Qp$,
  %
  \begin{equation*}
    u \close\epsilon v \Rightarrow \bar{f}(u,w) \close\epsilon \bar{f}(v,w)
    \quad\text{and}\quad
    v \close\epsilon w \Rightarrow \bar{f}(u,v) \close\epsilon \bar{f}(u,w).
  \end{equation*}
\end{lem}

\begin{proof}
  We use $(\RC, {\closesym})$-recursion to construct the curried form of $\bar{f}$ as a map
  $\RC \to A$ where $A$ is the space of non-expanding\index{function!non-expanding}\index{non-expanding function} real-valued
  functions:
  %
  \begin{equation*}
    A \defeq
    \setof{ h : \RC \to \RC |
      \fall{\epsilon : \Qp} \fall{u, v : \RC}
      u \close\epsilon v \Rightarrow h(u) \close\epsilon h(v)
    }.
  \end{equation*}
  %
  We shall also need a suitable $\bsim_\epsilon$ on $A$, which we define as
  %
  \begin{equation*}
    (h \bsim_\epsilon k) \defeq \fall{u : \RC} h(u) \close\epsilon k(u).
  \end{equation*}
  %
  Clearly, if $\fall{\epsilon : \Qp} h \bsim_\epsilon k$ then $h(u) = k(u)$ for all $u :
  \RC$, so $\bsim$ is separated.

  For the base case we define $\bar{f}(\rcrat(q)) : A$, where $q : \Q$, as the
  extension of the Lipschitz map $\lam{r} f(q,r)$ from $\Q \to \Q$ to $\RC \to \RC$, as
  constructed in \autoref{RC-extend-Q-Lipschitz} with Lipschitz constant~$1$. Next, for a
  Cauchy approximation $x$, we define $\bar{f}(\rclim(x)) : \RC \to \RC$ as
  %
  \begin{equation*}
    \bar{f}(\rclim(x))(v) \defeq \rclim (\lam{\epsilon} \bar{f}(x_\epsilon)(v)).
  \end{equation*}
  %
  For this to be a valid definition, $\lam{\epsilon} \bar{f}(x_\epsilon)(v)$ should be a
  Cauchy approximation, so consider any $\delta, \epsilon : \Q$. Then by assumption
  $\bar{f}(x_\delta) \bsim_{\delta + \epsilon} \bar{f}(x_\epsilon)$, hence
  $\bar{f}(x_\delta)(v) \close{\delta + \epsilon} \bar{f}(x_\epsilon)(v)$. Furthermore,
  $\bar{f}(\rclim(x))$ is non-expanding because $\bar{f}(x_\epsilon)$ is such by induction
  hypothesis. Indeed, if $u \close\epsilon v$ then, for all $\epsilon : \Q$,
  %
  \begin{equation*}
    \bar{f}(x_{\epsilon/3})(u) \close{\epsilon/3} \bar{f}(x_{\epsilon/3})(v),
  \end{equation*}
  %
  therefore $\bar{f}(\rclim(x))(u) \close\epsilon \bar{f}(\rclim(x))(v)$ by the fourth constructor of $\closesym$.

  We still have to check four more conditions, let us illustrate just one. Suppose
  $\epsilon : \Qp$ and for some $\delta : \Qp$ we have $\rcrat(q) \close{\epsilon - \delta}
  y_\delta$ and $\bar{f}(\rcrat(q)) \bsim_{\epsilon - \delta} \bar{f}(y_\delta)$. To show
  $\bar{f}(\rcrat(q)) \bsim_\epsilon \bar{f}(\rclim(y))$, consider any $v : \RC$ and observe that
  %
  \begin{equation*}
    \bar{f}(\rcrat(q))(v) \close{\epsilon - \delta} \bar{f}(y_\delta)(v).
  \end{equation*}
  %
  Therefore, by the second constructor of $\closesym$, we have
  \narrowequation{\bar{f}(\rcrat(q))(v) \close\epsilon \bar{f}(\rclim(y))(v)}
  as required.
\end{proof}

We may apply \autoref{RC-binary-nonexpanding-extension} to any bivariate rational function
which is non-expanding separately in each variable. Addition is such a function, therefore
we get ${+} : \RC \times \RC \to \RC$.
\indexdef{addition!of Cauchy reals}%
Furthermore, the extension is unique as long as we
require it to be non-expanding in each variable, and just as in the univariate case,
identities on rationals extend to identities on reals. Since composition of non-expanding
maps is again non-expanding, we may conclude that addition satisfies the usual properties,
such as commutativity and associativity.
\index{associativity!of addition!of Cauchy reals}%
Therefore, $(\RC, 0, {+}, {-})$ is a commutative
group.

We may also apply \autoref{RC-binary-nonexpanding-extension} to the functions $\min : \Q \times
\Q \to \Q$ and $\max : \Q \times \Q \to \Q$, which turns $\RC$ into a lattice. The partial
order $\leq$ on $\RC$ is defined in terms of $\max$ as
%
\symlabel{leq-RC}
\index{order!non-strict}%
\index{non-strict order}%
\begin{equation*}
  (u \leq v) \defeq (\max(u, v) = v).
\end{equation*}
%
The relation $\leq$ is a partial order because it is such on $\Q$, and the axioms of a
partial order are expressible as equations in terms of $\min$ and $\max$, so they transfer
to $\RC$.

\index{absolute value}%
Another function which extends to $\RC$ by the same method is the absolute value $|{\blank}|$.
Again, it has the expected properties because they transfer from $\Q$ to $\RC$.

\symlabel{lt-RC}
From $\leq$ we get the strict order $<$ by
\index{strict!order}%
\index{order!strict}%
%
\begin{equation*}
  (u < v) \defeq \exis{q, r : \Q} (u \leq \rcrat(q)) \land (q < r) \land (\rcrat(r) \leq v).
\end{equation*}
%
That is, $u < v$ holds when there merely exists a pair of rational numbers $q < r$ such that $x \leq
\rcrat(q)$ and $\rcrat(r) \leq v$. It is not hard to check that $<$ is irreflexive and
transitive, and has other properties that are expected for an ordered field.
The archimedean principle follows directly from the definition of~$<$.

\index{ordered field!archimedean}%
\begin{thm}[Archimedean principle for $\RC$] \label{RC-archimedean}
  %
  For every $u, v : \RC$ such that $u < v$ there merely exists $q : \Q$ such that $u < q < v$.
\end{thm}

\begin{proof}
  From $u < v$ we merely get $r, s : \Q$ such that $u \leq r < s \leq v$, and we may take $q
  \defeq (r + s) / 2$.
\end{proof}

We now have enough structure on $\RC$ to express $u \close\epsilon v$ with standard concepts.

\begin{lem}\label{thm:RC-le-grow}
  If $q:\Q$ and $u:\RC$ satisfy $u\le \rcrat(q)$, then for any $v:\RC$ and $\epsilon:\Qp$, if $u\close\epsilon v$ then $v\le \rcrat(q+\epsilon)$.
\end{lem}
\begin{proof}
  Note that the function $\max(\rcrat(q),\blank):\RC\to\RC$ is Lipschitz with constant $1$.
  First consider the case when $u=\rcrat(r)$ is rational.
  For this we use induction on $v$.
  If $v$ is rational, then the statement is obvious.
  If $v$ is $\rclim(y)$, we assume inductively that for any $\epsilon,\delta$, if $\rcrat(r)\close\epsilon y_\delta$ then $y_\delta \le \rcrat(q+\epsilon)$, i.e.\ $\max(\rcrat(q+\epsilon),y_\delta)=\rcrat(q+\epsilon)$.

  Now assuming $\epsilon$ and $\rcrat(r)\close\epsilon \rclim(y)$, we have $\theta$ such that $\rcrat(r)\close{\epsilon-\theta} \rclim(y)$, hence $\rcrat(r)\close\epsilon y_\delta$ whenever $\delta<\theta$.
  Thus, the inductive hypothesis gives $\max(\rcrat(q+\epsilon),y_\delta)=\rcrat(q+\epsilon)$ for such $\delta$.
  But by definition,
  \[\max(\rcrat(q+\epsilon),\rclim(y)) \jdeq \rclim(\lam{\delta} \max(\rcrat(q+\epsilon),y_\delta)).\]
  Since the limit of an eventually constant Cauchy approximation is that constant, we have
  \[\max(\rcrat(q+\epsilon),\rclim(y)) = \rcrat(q+\epsilon),\] hence $\rclim(y)\le \rcrat(q+\epsilon)$.

  Now consider a general $u:\RC$.
  Since $u\le \rcrat(q)$ means $\max(\rcrat(q),u)=\rcrat(q)$, the assumption $u\close\epsilon v$ and the Lipschitz property of $\max(\rcrat(q),-)$ imply $\max(\rcrat(q),v) \close\epsilon \rcrat(q)$.
  Thus, since $\rcrat(q)\le \rcrat(q)$, the first case implies $\max(\rcrat(q),v) \le \rcrat(q+\epsilon)$, and hence $v\le \rcrat(q+\epsilon)$ by transitivity of $\le$.
\end{proof}

\begin{lem}\label{thm:RC-lt-open}
  Suppose $q:\Q$ and $u:\RC$ satisfy $u<\rcrat(q)$.  Then:
  \begin{enumerate}
  \item For any $v:\RC$ and $\epsilon:\Qp$, if $u\close\epsilon v$ then $v< \rcrat(q+\epsilon)$.\label{item:RCltopen1}
  \item There exists $\epsilon:\Qp$ such that for any $v:\RC$, if $u\close\epsilon v$ we have $v<\rcrat(q)$.\label{item:RCltopen2}
  \end{enumerate}
\end{lem}
\begin{proof}
  By definition, $u<\rcrat(q)$ means there is $r:\Q$ with $r<q$ and $u\le \rcrat(r)$.
  Then by \autoref{thm:RC-le-grow}, for any $\epsilon$, if $u\close\epsilon v$ then $v\le \rcrat(r+\epsilon)$.
  Conclusion~\ref{item:RCltopen1} follows immediately since $r+\epsilon<q+\epsilon$, while for~\ref{item:RCltopen2} we can take any $\epsilon <q-r$.
\end{proof}

We are now able to show that the auxiliary relation $\closesym$ is what we think it is.

\begin{thm} \label{RC-sim-eqv-le}
  \index{distance}%
  $\eqv{(u \close\epsilon v)}{(|u - v| < \rcrat(\epsilon))}$
  for all $u, v : \RC$ and $\epsilon : \Qp$.
\end{thm}
\begin{proof}
  The Lipschitz properties of subtraction and absolute value imply that if $u\close\epsilon v$, then $|u-v| \close\epsilon |u-u| = 0$.
  Thus, for the left-to-right direction, it will suffice to show that if $u\close\epsilon 0$, then $|u|<\rcrat(\epsilon)$.
  We proceed by $\RC$-induction on $u$.

  If $u$ is rational, the statement follows immediately since absolute value and order extend the standard ones on $\Qp$.
  If $u$ is $\rclim(x)$, then by roundedness we have $\theta:\Qp$ with $\rclim(x)\close{\epsilon-\theta} 0$.
  By the triangle inequality, therefore, we have $x_{\theta/3} \close{\epsilon-2\theta/3} 0$, so the inductive hypothesis yields $|x_{\theta/3}|<\rcrat(\epsilon-2\theta/3)$.
  But $x_{\theta/3} \close{2\theta/3} \rclim(x)$, hence $|x_{\theta/3}| \close{2\theta/3} |\rclim(x)|$ by the Lipschitz property, so \autoref{thm:RC-lt-open}\ref{item:RCltopen1} implies $|\rclim(x)|<\rcrat(\epsilon)$.

  In the other direction, we use $\RC$-induction on $u$ and $v$.
  If both are rational, this is the first constructor of $\closesym$.

  If $u$ is $\rcrat(q)$ and $v$ is $\rclim(y)$, we assume inductively that for any $\epsilon,\delta$, if $|\rcrat(q)-y_\delta|<\rcrat(\epsilon)$ then $\rcrat(q) \close{\epsilon} y_\delta$.
  Fix an $\epsilon$ such that $|\rcrat(q) - \rclim(y)|<\rcrat(\epsilon)$.
  Since $\Q$ is order-dense in $\RC$, there exists $\theta<\epsilon$ with $|\rcrat(q) - \rclim(y)|<\rcrat(\theta)$.
  Now for any $\delta,\eta$ we have $\rclim(y)\close{2\delta} y_\delta$, hence by the Lipschitz property
  \[ |\rcrat(q) - \rclim(y)| \close{\delta+\eta} |\rcrat(q) - y_\delta|. \]
  Thus, by \autoref{thm:RC-lt-open}\ref{item:RCltopen1}, we have $|\rcrat(q) - y_\delta| < \rcrat(\theta+2\delta)$.
  So by the inductive hypothesis, $\rcrat(q) \close{\theta+2\delta} y_\delta$, and thus $\rcrat(q)\close{\theta+4\delta} \rclim(y)$ by the triangle inequality.
  Thus, it suffices to choose $\delta \defeq (\epsilon-\theta)/4$.

  The remaining two cases are entirely analogous.
\end{proof}

\indexdef{multiplication!of Cauchy reals}%
Next, we would like to equip $\RC$ with multiplicative structure. For each $q : \Q$ the
map $r \mapsto q \cdot r$ is Lipschitz with constant\footnote{We defined Lipschitz
  constants as \emph{positive} rational numbers.} $|q| + 1$, and so we can extend it to
multiplication by $q$ on the real numbers. Therefore $\RC$ is a vector space\index{vector!space} over $\Q$.
In general, we can define multiplication of real numbers as
%
\begin{equation}
  u \cdot v \defeq
  {\textstyle \frac{1}{2}} \cdot ((u + v)^2 - u^2 - v^2),\label{mult-from-square}
\end{equation}
%
so we just need squaring\index{squaring function} $u \mapsto u^2$ as a map $\RC \to \RC$. Squaring is not a
Lipschitz map, but it is Lipschitz on every bounded domain, which allows us to patch it
together. Define the open and closed intervals
%
\indexdef{interval!open and closed}%
\indexdef{open!interval}%
\indexdef{closed!interval}%
\begin{equation*}
  [u,v] \defeq \setof{ x : \RC | u \leq x \leq v }
  \qquad\text{and}\qquad
  (u,v) \defeq \setof{ x : \RC | u < x < v }.
\end{equation*}
%
Although technically an element of $[u,v]$ or $(u,v)$ is a Cauchy real number together with a proof, since the latter inhabits a mere proposition it is uninteresting.
Thus, as is common with subset types, we generally write simply $x:[u,v]$ whenever $x:\RC$ is such that $u\leq x \leq v$, and similarly.

\begin{thm} \label{RC-squaring}
  %
  There exists a unique function ${(\blank)}^2 : \RC \to \RC$ which extends squaring $q \mapsto
  q^2$ of rational numbers and satisfies
  %
  \begin{equation*}
    \fall{n : \N}
    \fall{u, v : [-n, n]}
    |u^2 - v^2| \leq 2 \cdot n \cdot |u - v|.
  \end{equation*}
\end{thm}

\begin{proof}
  We first observe that for every $u : \RC$ there merely exists $n : \N$ such that $-n
  \leq u \leq n$, see \autoref{ex:traditional-archimedean}, so the map
  %
  \begin{equation*}
    e : \Parens{\sm{n : \N} [-n, n]} \to \RC
    \qquad\text{defined by}\qquad
    e(n, x) \defeq x
  \end{equation*}
  %
  is surjective. Next, for each $n : \N$, the squaring map
  %
  \begin{equation*}
    s_n : \setof{ q : \Q | -n \leq q \leq n } \to \Q
    \qquad\text{defined by}\qquad
    s_n(q) \defeq q^2
  \end{equation*}
  %
  is Lipschitz with constant $2 n$, so we can use \autoref{RC-extend-Q-Lipschitz} to
  extend it to a map $\bar{s}_n : [-n, n] \to \RC$ with Lipschitz constant $2 n$, see
  \autoref{RC-Lipschitz-on-interval} for details. The maps $\bar{s}_n$ are compatible: if
  $m < n$ for some $m, n : \N$ then $s_n$ restricted to $[-m, m]$ must agree with $s_m$
  because both are Lipschitz, and therefore continuous in the sense
  of~\autoref{RC-continuous-eq}. Therefore, by \autoref{lem:images_are_coequalizers} the map
  %
  \begin{equation*}
    \Parens{\sm{n : \N} [-n, n]} \to \RC,
    \qquad\text{given by}\qquad
    (n, x) \mapsto s_n(x)
  \end{equation*}
  %
  factors uniquely through $\RC$ to give us the desired function.
\end{proof}

At this point we have the ring structure of the reals and the archimedean order. To
establish $\RC$ as an archimedean ordered field, we still need inverses.

\begin{thm}
  \index{apartness}%
  A Cauchy real is invertible if, and only if, it is apart from zero.
\end{thm}

\begin{proof}
  First, suppose $u : \RC$ has an inverse $v : \RC$ By the archimedean principle there is $q :
  \Q$ such that $|v| < q$. Then $1 = |u v| < |u| \cdot v < |u| \cdot q$ and hence $|u| >
  1/q$, which is to say that $u \apart 0$.

  For the converse we construct the inverse map
  %
  \begin{equation*}
    ({\blank})^{-1} : \setof{ u : \RC | u \apart 0 } \to \RC
  \end{equation*}
  %
  by patching together functions, similarly to the construction of squaring in
  \autoref{RC-squaring}. We only outline the main steps. For every $q : \Q$ let
  %
  \begin{equation*}
    [q, \infty) \defeq \setof{u : \RC | q \leq u}
    \qquad\text{and}\qquad
    (-\infty, q] \defeq \setof{u : \RC | u \leq -q}.
  \end{equation*}
  %
  Then, as $q$ ranges over $\Qp$, the types $(-\infty, q]$ and $[q, \infty)$ jointly cover
  $\setof{u : \RC | u \apart 0}$. On each such $[q, \infty)$ and $(-\infty, q]$ the
  inverse function is obtained by an application of \autoref{RC-extend-Q-Lipschitz}
  with Lipschitz constant $1/q^2$. Finally, \autoref{lem:images_are_coequalizers}
  guarantees that the inverse function factors uniquely through $\setof{ u : \RC | u
    \apart 0 }$.
\end{proof}

We summarize the algebraic structure of $\RC$ with a theorem.

\begin{thm} \label{RC-archimedean-ordered-field}
  The Cauchy reals form an archimedean ordered field.
\end{thm}

\subsection{Cauchy reals are Cauchy complete}
\label{sec:cauchy-reals-cauchy-complete}

We constructed $\RC$ by closing $\Q$ under limits of Cauchy approximations, so it better
be the case that $\RC$ is Cauchy complete. Thanks to \autoref{RC-sim-eqv-le} there is no
difference between a Cauchy approximation $x : \Qp \to \RC$ as defined in the construction
of $\RC$, and a Cauchy approximation in the sense of \autoref{defn:cauchy-approximation}
(adapted to $\RC$).

Thus, given a Cauchy approximation $x : \Qp \to \RC$ it is quite natural to expect that
$\rclim(x)$ is its limit, where the notion of limit is defined as in
\autoref{defn:cauchy-approximation}. But this is so by \autoref{RC-sim-eqv-le} and
\autoref{thm:RC-sim-lim-term}. We have proved:

\begin{thm}
  Every Cauchy approximation in $\RC$ has a limit.
\end{thm}

An archimedean ordered field in which every Cauchy approximation has a limit is called
\define{Cauchy complete}.
\indexdef{Cauchy!completeness}%
\indexdef{complete!ordered field, Cauchy}%
\index{ordered field}%
The Cauchy reals are the least such field.

\begin{thm} \label{RC-initial-Cauchy-complete}
  The Cauchy reals embed into every Cauchy complete ar\-chi\-me\-de\-an ordered field.
\end{thm}

\begin{proof}
  \index{limit!of a Cauchy approximation}%
  Suppose $F$ is a Cauchy complete archimedean ordered field. Because limits are unique,
  there is an operator $\lim$ which takes Cauchy approximations in $F$ to their limits. We
  define the embedding $e : \RC \to F$ by $(\RC, {\closesym})$-recursion as
  %
  \begin{equation*}
    e(\rcrat(q)) \defeq q
    \qquad\text{and}\qquad
    e(\rclim(x)) \defeq \lim (e \circ x).
  \end{equation*}
  %
  A suitable $\bsim$ on $F$ is
  %
  \begin{equation*}
    (a \bsim_\epsilon b) \defeq |a - b| < \epsilon.
  \end{equation*}
  %
  This is a separated relation because $F$ is archimedean. The rest of the clauses for
  $(\RC, {\closesym})$-recursion are easily checked. One would also have to check that $e$ is
  an embedding of ordered fields which fixes the rationals.
\end{proof}

\index{real numbers!Cauchy|)}%

\section{Comparison of Cauchy and Dedekind reals}
\label{sec:comp-cauchy-dedek}

\index{real numbers!Dedekind|(}%
\index{real numbers!Cauchy|(}%
\index{depression|(}

Let us also say something about the relationship between the Cauchy and Dedekind reals. By
\autoref{RC-archimedean-ordered-field}, $\RC$ is an archimedean ordered field. It is also
admissible\index{ordered field!admissible} for $\Omega$, as can be easily checked. (In case $\Omega$ is the initial
$\sigma$-frame
\index{initial!sigma-frame@$\sigma$-frame}%
\index{sigma-frame@$\sigma$-frame!initial}%
it takes a simple induction, while in other cases it is immediate.)
Therefore, by \autoref{RD-final-field} there is an embedding of ordered fields
%
\begin{equation*}
  \RC \to \RD
\end{equation*}
%
which fixes the rational numbers.
(We could also obtain this from \autoref{RC-initial-Cauchy-complete,RD-cauchy-complete}.)
In general we do not expect $\RC$ and $\RD$ to coincide
without further assumptions.

\begin{lem} \label{lem:untruncated-linearity-reals-coincide}
  %
  If for every $x : \RD$ there merely exists
  %
  \begin{equation}
    \label{eq:untruncated-linearity}
    c : \prd{q, r : \Q} (q < r) \to (q < x) + (x < r)
  \end{equation}
  %
  then the Cauchy and Dedekind reals coincide.
\end{lem}

\begin{proof}
  Note that the type in~\eqref{eq:untruncated-linearity} is an untruncated variant
  of~\eqref{eq:RD-linear-order}, which states that~$<$ is a weak linear order.
  We already know that $\RC$ embeds into $\RD$, so it suffices to show that every Dedekind
  real merely is the limit of a Cauchy sequence\index{Cauchy!sequence} of rational numbers.

  Consider any $x : \RD$. By assumption there merely exists $c$ as in the statement of the
  lemma, and by inhabitation of cuts\index{cut!Dedekind} there merely exist $a, b : \Q$ such that $a < x < b$.
  We construct a sequence\index{sequence} $f : \N \to \setof{ \pairr{q, r} \in \Q \times \Q | q < r }$ by
  recursion:
  %
  \begin{enumerate}
  \item Set $f(0) \defeq \pairr{a, b}$.
  \item Suppose $f(n)$ is already defined as $\pairr{q_n, r_n}$ such that $q_n < r_n$.
    Define $s \defeq (2 q_n + r_n)/3$ and $t \defeq (q_n + 2 r_n)/3$. Then $c(s,t)$
    decides between $s < x$ and $x < t$. If it decides $s < x$ then we set $f(n+1) \defeq
    \pairr{s, r_n}$, otherwise $f(n+1) \defeq \pairr{q_n, t}$.
  \end{enumerate}
  %
  Let us write $\pairr{q_n, r_n}$ for the $n$-th term of the sequence~$f$. Then it is easy
  to see that $q_n < x < r_n$ and $|q_n - r_n| \leq (2/3)^n \cdot |q_0 - r_0|$ for all $n
  : \N$. Therefore $q_0, q_1, \ldots$ and $r_0, r_1, \ldots$ are both Cauchy sequences
  converging to the Dedekind cut~$x$. We have shown that for every $x : \RD$ there merely
  exists a Cauchy sequence converging to $x$.
\end{proof}

The lemma implies that either countable choice or excluded middle suffice for coincidence
of $\RC$ and $\RD$.

\begin{cor} \label{when-reals-coincide}
  \index{axiom!of choice!countable}%
  \index{excluded middle}%
  If excluded middle or countable choice holds then $\RC$ and $\RD$ are equivalent.
\end{cor}

\begin{proof}
  If excluded middle holds then $(x < y) \to (x < z) + (z < y)$ can be proved: either $x <
  z$ or $\lnot (x < z)$. In the former case we are done, while in the latter we get $z <
  y$ because $z \leq x < y$. Therefore, we get~\eqref{eq:untruncated-linearity} so that we
  can apply \autoref{lem:untruncated-linearity-reals-coincide}.

  Suppose countable choice holds. The set $S = \setof{ \pairr{q, r} \in \Q \times \Q | q <
    r }$ is equivalent to $\N$, so we may apply countable choice to the statement that $x$
  is located,
  %
  \begin{equation*}
    \fall{\pairr{q, r} : S} (q < x) \lor (x < r).
  \end{equation*}
  %
  Note that $(q < x) \lor (x < r)$ is expressible as an existential statement $\exis{b :
    \bool} (b = \bfalse \to q < x) \land (b = \btrue \to x < r)$. The (curried form) of
  the choice function is then precisely~\eqref{eq:untruncated-linearity} so that
  \autoref{lem:untruncated-linearity-reals-coincide} is applicable again.
\end{proof}

\index{real numbers!Dedekind|)}%
\index{real numbers!Cauchy|)}%
\index{real numbers!agree}%

\index{depression|)}

\section{Compactness of the interval}
\label{sec:compactness-interval}

\index{mathematics!classical|(}%
\index{mathematics!constructive|(}%

We already pointed out that our constructions of reals are entirely compatible with
classical logic. Thus, by assuming the law of excluded middle~\eqref{eq:lem} and the axiom
of choice~\eqref{eq:ac} we could develop classical analysis,\index{classical!analysis}\index{analysis!classical} which would essentially
amount to copying any standard book on analysis.

\index{analysis!constructive}%
\index{constructive!analysis}%
Nevertheless, anyone interested in computation, for example a numerical analyst, ought to
be curious about developing analysis in a computationally meaningful setting. That
analysis in a constructive setting is even possible was demonstrated by~\cite{Bishop1967}.
As a sample of the differences and similarities between classical and constructive
analysis we shall briefly discuss just one topic---compactness of the closed interval
$[0,1]$ and a couple of theorems surrounding the concept.

Compactness is no exception to the common phenomenon in constructive mathematics that
classically equivalent notions bifurcate. The three most frequently used notions of
compactness are:
%
\indexdef{compactness}%
\begin{enumerate}
\item \define{metrically compact:} ``Cauchy complete and totally bounded'',
  \indexdef{metrically compact}%
  \indexdef{compactness!metric}%
\item \define{Bolzano--Weierstra\ss{} compact:} ``every sequence has a convergent subsequence'',
  \index{compactness!Bolzano--Weierstrass@Bolzano--Weierstra\ss{}}%
  \indexsee{Bolzano--Weierstrass@Bolzano--Weierstra\ss{}}{compactness}%
  \index{sequence}%
\item \define{Heine-Borel compact:} ``every open cover has a finite subcover''.
  \index{compactness!Heine-Borel}%
  \indexsee{Heine-Borel}{compactness}%
\end{enumerate}
%
These are all equivalent in classical mathematics.
Let us see how they fare in homotopy type theory. We can use either the Dedekind or the
Cauchy reals, so we shall denote the reals just as~$\R$. We first recall several basic
definitions.

\indexsee{space!metric}{metric space}
\index{metric space|(}%

\begin{defn} \label{defn:metric-space}
  A \define{metric space}
  \indexdef{metric space}%
  $(M, d)$ is a set $M$ with a map $d : M \times M \to \R$
  satisfying, for all $x, y, z : M$,
  %
  \begin{align*}
    d(x,y) &\geq 0, &
    d(x,y) &= d(y,x), \\
    d(x,y) &= 0 \Leftrightarrow x = y, &
    d(x,z) &\leq d(x,y) + d(y,z).
  \end{align*}
  %
\end{defn}

\begin{defn} \label{defn:complete-metric-space}
  A \define{Cauchy approximation}
  \index{Cauchy!approximation}%
  in $M$ is a sequence $x : \Qp \to M$ satisfying
  %
  \begin{equation*}
    \fall{\delta, \epsilon} d(x_\delta, x_\epsilon) < \delta + \epsilon.
  \end{equation*}
  %
  \index{limit!of a Cauchy approximation}%
  The \define{limit} of a Cauchy approximation $x : \Qp \to M$ is a point $\ell : M$
  satisfying
  %
  \begin{equation*}
    \fall{\epsilon, \theta : \Qp} d(x_\epsilon, \ell) < \epsilon + \theta.
  \end{equation*}
  %
  \indexdef{metric space!complete}%
  \indexdef{complete!metric space}%
  A \define{complete metric space} is one in which every Cauchy approximation has a limit.
\end{defn}

\begin{defn} \label{defn:total-bounded-metric-space}
  For a positive rational $\epsilon$, an \define{$\epsilon$-net}
  \indexdef{epsilon-net@$\epsilon$-net}%
  in a metric space $(M,
  d)$ is an element of
  %
  \begin{equation*}
    \sm{n : \N}{x_1, \ldots, x_n : M}
    \fall{y : M} \exis{k \leq n} d(x_k, y) < \epsilon.
  \end{equation*}
  %
  In words, this is a finite sequence of points $x_1, \ldots, x_n$ such that every point
  in $M$ merely is within $\epsilon$ of some~$x_k$.

  A metric space $(M, d)$ is \define{totally bounded}
  \indexdef{totally bounded metric space}%
  \indexdef{metric space!totally bounded}%
  when it has $\epsilon$-nets of all
  sizes:
  %
  \begin{equation*}
    \prd{\epsilon : \Qp}
    \sm{n : \N}{x_1, \ldots, x_n : M}
    \fall{y : M} \exis{k \leq n} d(x_k, y) < \epsilon.
  \end{equation*}
\end{defn}

\begin{rmk}
  In the definition of total boundedness we used sloppy notation $\sm{n : \N}{x_1, \ldots, x_n : M}$. Formally, we should have written $\sm{x : \lst{M}}$ instead,
  where $\lst{M}$ is the inductive type of finite lists\index{type!of lists} from \autoref{sec:bool-nat}.
  However, that would make the rest of the statement a bit more cumbersome to express.
\end{rmk}

Note that in the definition of total boundedness we require pure existence of an
$\epsilon$-net, not mere existence. This way we obtain a function which assigns to each
$\epsilon : \Qp$ a specific $\epsilon$-net. Such a function might be called a ``modulus of
total boundedness''. In general, when porting classical metric notions to homotopy type
theory, we should use propositional truncation sparingly, typically so that we avoid
asking for a non-constant map from $\R$ to $\Q$ or $\N$. For instance, here is the
``correct'' definition of uniform continuity.

\begin{defn} \label{defn:uniformly-continuous}
  A map $f : M \to \R$ on a metric space is \define{uniformly continuous}
  \indexdef{function!uniformly continuous}%
  \indexdef{uniformly continuous function}%
  when
  %
  \begin{equation*}
    \prd{\epsilon : \Qp}
    \sm{\delta : \Qp}
    \fall{x, y : M}
    d(x,y) < \delta \Rightarrow |f(x) - f(y)| < \epsilon.
  \end{equation*}
  %
  In particular, a uniformly continuous map has a modulus of uniform continuity\indexdef{modulus!of uniform continuity},
  which is a function that assigns to each $\epsilon$ a corresponding $\delta$.
\end{defn}

Let us show that $[0,1]$ is compact in the first sense.

\begin{thm} \label{analysis-interval-ctb}
  \index{compactness!metric}%
  \index{interval!open and closed}%
  The closed interval $[0,1]$ is complete and totally bounded.
\end{thm}

\begin{proof}
  Given $\epsilon : \Qp$, there is $n : \N$ such that $2/k < \epsilon$, so we may take the
  $\epsilon$-net $x_i = i/k$ for $i = 0, \ldots, k-1$. This is an $\epsilon$-net because,
  for every $y : [0,1]$ there merely exists $i$ such that $0 \leq i < k$ and $(i -
  1)/k < y < (i+1)/k$, and so $|y - x_i| < 2/k < \epsilon$.

  For completeness of $[0,1]$, consider a Cauchy approximation $x : \Qp \to
  [0,1]$ and let $\ell$ be its limit in $\R$. Since $\max$ and $\min$ are Lipschitz maps,
  the retraction $r : \R \to [0,1]$ defined by $r(x) \defeq \max(0, \min(1, x))$ commutes
  with limits of Cauchy approximations, therefore
  %
  \begin{equation*}
    r(\ell) =
    r (\lim x) =
    \lim (r \circ x) =
    r (\lim x) =
    \ell,
  \end{equation*}
  %
  which means that $0 \leq \ell \leq 1$, as required.
\end{proof}

We thus have at least one good notion of compactness in homotopy type theory.
Unfortunately, it is limited to metric spaces because total boundedness is a metric
notion. We shall consider the other two notions shortly, but first we prove that a
uniformly continuous map on a totally bounded space has a \define{supremum},
\indexsee{least upper bound}{supremum}%
i.e.\ an upper bound which is less than or equal to all other upper bounds.

\begin{thm} \label{ctb-uniformly-continuous-sup}
  %
  \indexdef{supremum!of uniformly continuous function}%
  A uniformly continuous map $f : M \to \R$ on a totally bounded metric space
  $(M, d)$ has a supremum $m : \R$. For every $\epsilon : \Qp$ there exists $u : M$ such
  that $|m - f(u)| < \epsilon$.
\end{thm}

\begin{proof}
  Let $h : \Qp \to \Qp$ be the modulus of uniform continuity of~$f$.
  We define an approximation $x : \Qp \to \R$ as follows: for any $\epsilon : \Q$ total
  boundedness of $M$ gives a $h(\epsilon)$-net $y_0, \ldots, y_n$. Define
  %
  \begin{equation*}
    x_\epsilon \defeq \max (f(y_0), \ldots, f(y_n)).
  \end{equation*}
  %
  We claim that $x$ is a Cauchy approximation. Consider any $\epsilon, \eta : \Q$, so that
  %
  \begin{equation*}
    x_\epsilon \jdeq \max (f(y_0), \ldots, f(y_n))
    \quad\text{and}\quad
    x_\eta \jdeq \max (f(z_0), \ldots, f(z_m))
  \end{equation*}
  %
  for some $h(\epsilon)$-net $y_0, \ldots, y_n$ and $h(\eta)$-net $z_0, \ldots, z_m$.
  Every $z_i$ is merely $h(\epsilon)$-close to some $y_j$, therefore $|f(z_i) - f(y_j)| <
  \epsilon$, from which we may conclude that
  %
  \begin{equation*}
    f(z_i) < \epsilon + f(y_j) \leq \epsilon + x_\epsilon,
  \end{equation*}
  %
  therefore $x_\eta < \epsilon + x_\epsilon$. Symmetrically we obtain $x_\eta < \eta +
  x_\eta$, therefore $|x_\eta - x_\epsilon| < \eta + \epsilon$.

  We claim that $m \defeq \lim x$ is the supremum of~$f$. To prove that $f(x) \leq m$ for
  all $x : M$ it suffices to show $\lnot (m < f(x))$. So suppose to the contrary that $m <
  f(x)$. There is $\epsilon : \Qp$ such that $m + \epsilon < f(x)$. But now merely for
  some $y_i$ participating in the definition of $x_\epsilon$ we get $|f(x) - f(y_i) <
  \epsilon$, therefore $m < f(x) - \epsilon < f(y_i) \leq m$, a contradiction.

  We finish the proof by showing that $m$ satisfies the second part of the theorem, because
  it is then automatically a least upper bound. Given any $\epsilon : \Qp$, on one hand
  $|m - f(x_{\epsilon/2})| < 3 \epsilon/4$, and on the other $|f(x_{\epsilon/2}) - f(y_i)| <
  \epsilon/4$ merely for some $y_i$ participating in the definition of $x_{\epsilon/2}$,
  therefore by taking $u \defeq y_i$ we obtain $|m - f(u)| < \epsilon$ by triangle
  inequality.
\end{proof}

Now, if in \autoref{ctb-uniformly-continuous-sup} we also knew that $M$ were complete, we
could hope to weaken the assumption of uniform continuity to continuity, and strengthen
the conclusion to existence of a point at which the supremum is attained. The usual proofs
of these improvements rely on the facts that in a complete totally bounded space
%
\begin{enumerate}
\item continuity implies uniform continuity, and
\item every sequence has a convergent subsequence.
\end{enumerate}
%
The first statement follows easily from Heine-Borel compactness, and the second is just
Bolzano--Weierstra\ss{} compactness.
\index{compactness!Bolzano--Weierstrass@Bolzano--Weierstra\ss{}}%
Unfortunately, these are both somewhat problematic. Let
us first show that Bolzano--Weierstra\ss{} compactness implies an instance of excluded middle
known as the \define{limited principle of omniscience}:
\indexsee{axiom!limited principle of omniscience}{limited principle of omniscience}%
\indexdef{limited principle of omniscience}%
for every $\alpha : \N \to \bool$,
%
\begin{equation} \label{eq:lpo}
  \Parens{\sm{n : \N} \alpha(n) = \btrue} +
  \Parens{\prd{n : \N} \alpha(n) = \bfalse}.
\end{equation}
%
Computationally speaking, we would not expect this principle to hold, because it asks us to decide
whether infinitely many values of a function are~$\bfalse$.

\begin{thm} \label{analysis-bw-lpo}
  %
  Bolzano--Weierstra\ss{} compactness of $[0,1]$ implies the limited principle of omniscience.
  \index{compactness!Bolzano--Weierstrass@Bolzano--Weierstra\ss{}}%
\end{thm}

\begin{proof}
  Given any $\alpha : \N \to \bool$, define the sequence\index{sequence} $x : \N \to [0,1]$ by
  %
  \begin{equation*}
    x_n \defeq
    \begin{cases}
      0 & \text{if $\alpha(k) = \bfalse$ for all $k < n$,}\\
      1 & \text{if $\alpha(k) = \btrue$ for some $k < n$}.
    \end{cases}
  \end{equation*}
  %
  If the Bolzano--Weierstra\ss{} property holds, there exists a strictly increasing $f : \N \to
  \N$ such that $x \circ f$ is a Cauchy sequence\index{Cauchy!sequence}. For a sufficiently large $n :
  \N$ the $n$-th term $x_{f(n)}$ is within $1/6$ of its limit. Either $x_{f(n)} < 2/3$ or
  $x_{f(n)} > 1/3$. If $x_{f(n)} < 2/3$ then~$x_n$ converges to $0$ and so $\prd{n : \N}
  \alpha(n) = \bfalse$. If $x_{f(n)} > 1/3$ then $x_{f(n)} = 1$, therefore $\sm{n : \N}
  \alpha(n) = \btrue$.
\end{proof}

While we might not mourn Bolzano--Weierstra\ss{} compactness too much, it seems harder to live
without Heine--Borel compactness, as attested by the fact that both classical mathematics
and Brouwer's Intuitionism accepted it. As we do not want to wade too deeply into general
topology, we shall work with basic open sets. In the case of $\R$ these are the open
intervals with rational endpoints. A family of such intervals, indexed by a type~$I$,
would be a map
%
\begin{equation*}
  \mathcal{F} : I \to \setof{(q, r) : \Q \times \Q | q < r},
\end{equation*}
%
with the idea that a pair of rationals $(q, r)$ with $q < r$ determines the type $\setof{ x : \R | q < x < r}$. It is slightly more convenient to allow degenerate intervals as well, so we take a
\define{family of basic intervals}
\indexdef{family!of basic intervals}%
\indexdef{interval!family of basic}%
to be a map
%
\begin{equation*}
  \mathcal{F} : I \to \Q \times \Q.
\end{equation*}
%
To be quite precise, a family is a dependent pair $(I, \mathcal{F})$, not just
$\mathcal{F}$. A \define{finite family of basic intervals} is one indexed by $\setof{ m :
  \N | m < n}$ for some $n : \N$. We usually present it by a finite list $[(q_0, r_0), \ldots,
(q_{n-1}, r_{n-1})]$. Finally, a \define{finite subfamily}\indexdef{subfamily, finite, of intervals} of $(I, \mathcal{F})$ is given
by a list of indices $[i_1, \ldots, i_n]$ which then determine the finite family
$[\mathcal{F}(i_1), \ldots, \mathcal{F}(i_n)]$.

As long as we are aware of the distinction between a pair $(q, r)$ and the corresponding
interval $\setof{ x : \R | q < x < r}$, we may safely use the same notation $(q, r)$ for
both. Intersections\indexdef{intersection!of intervals} and inclusions\indexdef{inclusion!of intervals}\indexdef{containment!of intervals} of intervals are expressible in terms of their
endpoints:
%
\symlabel{interval-intersection}
\symlabel{interval-subset}
\begin{align*}
  (q, r) \cap (s, t) &\ \defeq\  (\max(q, s), \min(r, t)),\\
  (q, r) \subseteq (s, t) &\ \defeq\ (q < r \Rightarrow s \leq q < r \leq t).
\end{align*}
%
We say that $\intfam{i}{I}{(q_i, r_i)}$ \define{(pointwise) covers $[a,b]$}
\indexdef{interval!pointwise cover}%
\indexdef{cover!pointwise}%
\indexdef{pointwise!cover}%
when
%
\begin{equation} \label{eq:cover-pointwise-truncated}
  \fall{x : [a,b]} \exis{i : I} q_i < x < r_i.
\end{equation}
%
The \define{Heine-Borel compactness for $[0,1]$}
\indexdef{compactness!Heine-Borel}%
states that every covering family of $[0,1]$
merely has a finite subfamily which still covers $[0,1]$.

\index{depression}
\begin{thm} \label{classical-Heine-Borel}
  \index{excluded middle}%
  If excluded middle holds then $[0,1]$ is Heine-Borel compact.
\end{thm}

\begin{proof}
  Assume for the purpose of reaching a contradiction that a family $\intfam{i}{I}{(a_i,
    b_i)}$ covers $[0,1]$ but no finite subfamily does. We construct a sequence of closed
  intervals $[q_n, r_n]$ which are nested, their sizes shrink to~$0$, and none of them is covered
  by a finite subfamily of $\intfam{i}{I}{(a_i, b_i)}$.

  We set $[q_0, r_0] \defeq [0,1]$. Assuming $[q_n, r_n]$ has been constructed, let $s
  \defeq (2 q_n + r_n)/3$ and $t \defeq (q_n + 2 r_n)/3$. Both $[q_n, t]$ and $[s, r_n]$
  are covered by $\intfam{i}{I}{(a_i, b_i)}$, but they cannot both have a finite subcover,
  or else so would $[q_n, r_n]$. Either $[q_n, t]$ has a finite subcover or it does not.
  If it does we set $[q_{n+1}, r_{n+1}] \defeq [s, r_n]$, otherwise we set $[q_{n+1},
  r_{n+1}] \defeq [q_n, t]$.

  The sequences $q_0, q_1, \ldots$ and $r_0, r_1, \ldots$ are both Cauchy and they
  converge to a point $x : [0,1]$ which is contained in every $[q_n, r_n]$.
  There merely exists $i : I$ such that $a_i < x < b_i$. Because the sizes of the
  intervals $[q_n, r_n]$ shrink to zero, there is $n : \N$ such that $a_i < q_n \leq x
  \leq r_n < b_i$, but this means that $[q_n, r_n]$ is covered by a single interval $(a_i,
  b_i)$, while at the same time it has no finite subcover. A contradiction.
\end{proof}

Without excluded middle, or a pinch of Brouwerian Intuitionism, we seem to be stuck.
Nevertheless, Heine-Borel compactness of $[0,1]$ \emph{can} be recovered in a constructive
setting, in a fashion that is still compatible with classical mathematics! For this to be
done, we need to revisit the notion of cover. The trouble with
\eqref{eq:cover-pointwise-truncated} is that the truncated existential allows a space to
be covered in any haphazard way, and so computationally speaking, we stand no chance of
merely extracting a finite subcover. By removing the truncation we get
%
\begin{equation} \label{eq:cover-pointwise}
  \prd{x : [0,1]} \sm{i : I} q_i < x < r_i,
\end{equation}
%
which might help, were it not too demanding of covers. With this definition we
could not even show that $(0,3)$ and $(2,5)$ cover $[1,4]$ because that would amount
to exhibiting a non-constant map $[1,4] \to \bool$, see
\autoref{ex:reals-non-constant-into-Z}.  Here we can take a lesson from ``pointfree topology''
\index{pointfree topology}%
\index{topology!pointfree}%
(i.e.\ locale theory):
\index{locale}%
the notion of cover ought to be expressed in terms of open sets, without
reference to points. Such a ``holistic'' view of space will then allow us to analyze the
notion of cover, and we shall be able to recover Heine-Borel compactness.  Locale
theory uses power sets,
\index{power set}%
which we could obtain by assuming propositional resizing;
\index{propositional!resizing}%
but instead we can steal ideas from the predicative cousin of locale theory,
\index{mathematics!predicative}%
which is called ``formal topology''.
\index{formal!topology}%

\index{acceptance|(}

Suppose that we have a family $\pairr{I, \mathcal{F}}$ and an interval $(a, b)$. How might
we express the fact that $(a,b)$ is covered by the family, without referring to points?
Here is one: if $(a, b)$ equals some $\mathcal{F}(i)$ then it is covered by the family.
And another one: if $(a,b)$ is covered by some other family $(J, \mathcal{G})$, and in
turn each $\mathcal{G}(j)$ is covered by $\pairr{I, \mathcal{F}}$, then $(a,b)$ is covered
$\pairr{I, \mathcal{F}}$. Notice that we are listing \emph{rules} which can be used to
\emph{deduce} that $\pairr{I, \mathcal{F}}$ covers $(a,b)$. We should find sufficiently
good rules and turn them into an inductive definition.

\begin{defn} \label{defn:inductive-cover}
  %
  The \define{inductive cover $\cover$}
  \indexdef{inductive!cover}%
  \indexdef{cover!inductive}%
  is a mere relation
  %
  \begin{equation*}
    {\cover} : (\Q \times \Q) \to \Parens{\sm{I : \type} (I \to \Q \times \Q)} \to \prop
  \end{equation*}
  %
  defined inductively by the following rules, where $q, r, s, t$ are rational numbers and
  $\pairr{I, \mathcal{F}}$, $\pairr{J, \mathcal{G}}$ are families of basic intervals:
  %
  \begin{enumerate}

  \item \emph{reflexivity:}
    \index{reflexivity!of inductive cover}%
    $\mathcal{F}(i) \cover \pairr{I, \mathcal{F}}$ for all $i : I$,

  \item \emph{transitivity:}
    \index{transitivity!of inductive cover}%
    if $(q, r) \cover \pairr{J, \mathcal{G}}$ and $\fall{j : J} \mathcal{G}(j) \cover \pairr{I,\mathcal{F}}$
    then $(q, r) \cover \pairr{I, \mathcal{F}}$,

  \item \emph{monotonicity:}
    \index{monotonicity!of inductive cover}%
    if $(q, r) \subseteq (s, t)$ and $(s,t) \cover \pairr{I, \mathcal{F}}$ then $(q, r) \cover
    \pairr{I, \mathcal{F}}$,

  \item \emph{localization:}
    \index{localization of inductive cover}%
    if $(q, r) \cover (I, \mathcal{F})$ then $(q, r) \cap (s, t) \cover
    \intfam{i}{I}{(\mathcal{F}(i) \cap (s, t))}$.

  \item \label{defn:inductive-cover-interval-1}
    if $q < s < t < r$ then $(q, r) \cover [(q, t), (r, s)]$,

  \item \label{defn:inductive-cover-interval-2}
    $(q, r) \cover \intfam{u}{\setof{ (s,t) : \Q \times \Q | q < s < t < r}}{u}$.
  \end{enumerate}
\end{defn}

The definition should be read as a higher-inductive type in which the listed rules are
point constructors, and the type is $(-1)$-truncated. The first four clauses are of a
general nature and should be intuitively clear. The last two clauses are specific to the
real line: one says that an interval may be covered by two intervals if they overlap,
while the other one says that an interval may be covered from within. Incidentally, if $r
\leq q$ then $(q, r)$ is covered by the empty family by the last clause.

Inductive covers enjoy the Heine-Borel property, the proof of which requires a lemma.

\begin{lem} \label{reals-formal-topology-locally-compact}
  Suppose $q < s < t < r$ and $(q, r) \cover \pairr{I, \mathcal{F}}$. Then there merely
  exists a finite subfamily of $\pairr{I, \mathcal{F}}$ which inductively covers $(s, t)$.
\end{lem}

\begin{proof}
  We prove the statement by induction on $(q, r) \cover \pairr{I, \mathcal{F}}$. There are
  six cases:
  %
  \begin{enumerate}

  \item Reflexivity: if $(q, r) = \mathcal{F}(i)$ then by monotonicity $(s, t)$ is covered
    by the finite subfamily $[\mathcal{F}(i)]$.

  \item Transitivity:
    suppose $(q, r) \cover \pairr{J, \mathcal{G}}$ and $\fall{j : J} \mathcal{G}(j) \cover
    \pairr{I, \mathcal{F}}$. By the inductive hypothesis there merely exists
    $[\mathcal{G}(j_1), \ldots, \mathcal{G}(j_n)]$ which covers $(s, t)$.
    Again by the inductive hypothesis, each of $\mathcal{G}(j_k)$ is covered by a finite
    subfamily of $\pairr{I, \mathcal{F}}$, and we can collect these into a finite
    subfamily which covers $(s, t)$.

  \item Monotonicity:
    if $(q, r) \subseteq (u, v)$ and $(u, v) \cover \pairr{I, \mathcal{F}}$ then we may
    apply the inductive hypothesis to $(u, v) \cover \pairr{I, \mathcal{F}}$ because $u <
    s < t < v$.

  \item Localization:
    suppose $(q', r') \cover \pairr{I, \mathcal{F}}$ and $(q, r) = (q', r') \cap (a, b)$.
    Because $q' < s < t < r'$, by the inductive hypothesis there is a finite subcover
    $[\mathcal{F}(i_1), \ldots, \mathcal{F}(i_n)]$ of $(s, t)$. We also know that $a < s <
    t < b$, therefore $(s, t) = (s, t) \cap (a, b)$ is covered by
    $[\mathcal{F}(i_1) \cap (a,b), \ldots, \mathcal{F}(i_n) \cap (a,b)]$, which is a
    finite subfamily of $\intfam{i}{I}{(\mathcal{F}(i) \cap (a, b))}$.

  \item If $(q, r) \cover [(q, v), (u, r)]$ for some $q < u < v < r$ then by monotonicity
    $(s, t) \cover [(q, v), (u, r)]$.

  \item Finally, $(s, t) \cover \intfam{z}{\setof{ (u,v):\Q \times \Q | q < u < v < r}}{z}$ by
    reflexivity. \qedhere
  \end{enumerate}
\end{proof}

Say that \define{$\pairr{I, \mathcal{F}}$ inductively covers
  $[a, b]$} when there merely exists $\epsilon : \Qp$ such that $(a - \epsilon, b +
\epsilon) \cover \pairr{I, \mathcal{F}}$.

\begin{cor} \label{interval-Heine-Borel}
  \index{compactness!Heine-Borel}%
  \index{interval!open and closed}%
  A closed interval is Heine-Borel compact for inductive covers.
\end{cor}

\begin{proof}
  Suppose $[a, b]$ is inductively covered by $\pairr{I, \mathcal{F}}$, so there merely is
  $\epsilon : \Qp$ such that $(a - \epsilon, b + \epsilon) \cover \pairr{I, \mathcal{F}}$.
  By \autoref{reals-formal-topology-locally-compact} there is a finite subcover of
  $(a - \epsilon/2, b + \epsilon/2)$, which is therefore a finite subcover of $[a, b]$.
\end{proof}

Experience from formal topology\index{topology!formal} shows that the rules for inductive covers are sufficient
for a constructive development of pointfree topology. But we can also provide our own
evidence that they are a reasonable notion.

\begin{thm} \label{inductive-cover-classical}
  \mbox{}
  %
  \begin{enumerate}
  \item An inductive cover is also a pointwise cover.
  \item Assuming excluded middle, a pointwise cover is also an inductive cover.
  \end{enumerate}
\end{thm}

\begin{proof}
  \mbox{}
  %
  \begin{enumerate}

  \item
    Consider a family of basic intervals $\pairr{I, \mathcal{F}}$, where we write $(q_i,
    r_i) \defeq \mathcal{F}(i)$, an interval $(a,b)$ inductively covered by $\pairr{I,
      \mathcal{F}}$, and $x$ such that $a < x < b$.
    %
    We prove by induction on $(a,b) \cover \pairr{I, \mathcal{F}}$ that there merely
    exists $i : I$ such that $q_i < x < r_i$. Most cases are pretty obvious, so we show
    just two. If $(a,b) \cover \pairr{I, \mathcal{F}}$ by reflexivity, then there merely
    is some $i : I$ such that $(a,b) = (q_i, r_i)$ and so $q_i < x < r_i$. If $(a,b)
    \cover \pairr{I, \mathcal{F}}$ by transitivity via $\intfam{j}{J}{(s_j, t_j)}$ then by
    the inductive hypothesis there merely is $j : J$ such that $s_j < x < t_j$, and then since
    $(s_j, t_j) \cover \pairr{I, \mathcal{F}}$ again by the inductive hypothesis there merely
    exists $i : I$ such that $q_i < x < r_i$. Other cases are just as exciting.

  \item Suppose $\intfam{i}{I}{(q_i, r_i)}$ pointwise covers $(a, b)$. By
    \autoref{defn:inductive-cover-interval-2} of \autoref{defn:inductive-cover} it
    suffices to show that $\intfam{i}{I}{(q_i, r_i)}$ inductively covers $(c, d)$ whenever
    $a < c < d < b$, so consider such $c$ and $d$. By \autoref{classical-Heine-Borel}
    there is a finite subfamily $[i_1, \ldots, i_n]$ which already pointwise covers $[c,
    d]$, and hence $(c,d)$. Let $\epsilon : \Qp$ be a Lebesgue number
    \index{Lebesgue number}
    for $(q_{i_1}, r_{i_1}), \ldots, (q_{i_n}, r_{i_n})$ as in
    \autoref{ex:finite-cover-lebesgue-number}. There is a positive $k : \N$ such that $2 (d - c)/k
    < \min(1, \epsilon)$. For $0 \leq i \leq k$ let
    %
    \begin{equation*}
      c_k \defeq ((k - i) c + i d) / k.
    \end{equation*}
    %
    The intervals $(c_0, c_2)$, $(c_1, c_3)$, \dots, $(c_{k-2}, c_k)$ inductively cover
    $(c,d)$ by repeated use of transitivity and~\autoref{defn:inductive-cover-interval-1}
    in \autoref{defn:inductive-cover}. Because their widths are below $\epsilon$ each of
    them is contained in some $(q_i, r_i)$, and we may use transitivity and monotonicity to
    conclude that $\intfam{i}{I}{(q_i, r_i)}$ inductively cover $(c, d)$. \qedhere
  \end{enumerate}
\end{proof}

The upshot of the previous theorem is that, as far as classical mathematics is concerned,
there is no difference between a pointwise and an inductive cover. In particular, since it
is consistent to assume excluded middle in homotopy type theory, we cannot exhibit an
inductive cover which fails to be a pointwise cover. Or to put it in a different way, the
difference between pointwise and inductive covers is not what they cover but in the
\emph{proofs} that they cover.

We could write another book by going on like this, but let us stop here and hope that we
have provided ample justification for the claim that analysis can be developed in homotopy
type theory. The curious reader should consult \autoref{ex:mean-value-theorem} for
constructive versions of the mean value theorem.

\index{acceptance|)}

\index{mathematics!classical|)}%
\index{mathematics!constructive|)}%

\section{The surreal numbers}
\label{sec:surreals}

\index{surreal numbers|(}%

In this section we consider another example of a higher inductive-in\-duc\-tive type, which draws together many of our threads: Conway's field \NO of \emph{surreal numbers}~\cite{conway:onag}.
The surreal numbers are the natural common generalization of the (Dedekind) real numbers (\autoref{sec:dedekind-reals}) and the ordinal numbers (\autoref{sec:ordinals}).
Conway, working in classical\index{mathematics!classical} mathematics with excluded middle and Choice, defines a surreal number to be a pair of \emph{sets} of surreal numbers, written $\surr L R$, such that every element of $L$ is strictly less than every element of $R$.
This obviously looks like an inductive definition, but there are three issues with regarding it as such.

Firstly, the definition requires the relation of (strict) inequality between surreals, so that relation must be defined simultaneously with the type \NO of surreals.
(Conway avoids this issue by first defining \emph{games}\index{game!Conway}, which are like surreals but omit the compatibility condition on $L$ and $R$.)
As with the relation $\closesym$ for the Cauchy reals, this simultaneous definition could \emph{a priori} be either inductive-inductive or inductive-recursive.
We will choose to make it inductive-inductive, for the same reasons we made that choice for $\closesym$.

Moreover, we will define strict inequality $<$ and non-strict inequality $\le$ for surreals separately (and mutually inductively).
Conway defines $<$ in terms of $\le$, in a way which is sensible classically but not constructively.
\index{mathematics!constructive}%
Furthermore, a negative definition of $<$ would make it unacceptable as a hypothesis of the constructor of a higher inductive type (see \autoref{sec:strictly-positive}).

Secondly, Conway says that $L$ and $R$ in $\surr L R$ should be ``sets of surreal numbers'', but the naive meaning of this as a predicate $\NO\to\prop$ is not positive, hence cannot be used as input to an inductive constructor.
However, this would not be a good type-theoretic translation of what Conway means anyway, because in set theory the surreal numbers form a proper class, whereas the sets $L$ and $R$ are true (small) sets, not arbitrary subclasses of \NO.
In type theory, this means that \NO will be defined relative to a universe \UU, but will itself belong to the next higher universe $\UU'$, like the sets \ord and \card of ordinals and cardinals, the cumulative hierarchy $V$, or even the Dedekind reals in the absence of propositional resizing.
\index{propositional!resizing}%
We will then require the ``sets'' $L$ and $R$ of surreals to be \UU-small, and so it is natural to represent them by \emph{families} of surreals indexed by some \UU-small type.
(This is all exactly the same as what we did with the cumulative hierarchy in \autoref{sec:cumulative-hierarchy}.)
That is, the constructor of surreals will have type
\[ \prd{\LL,\RR:\UU} (\LL\to\NO) \to (\RR\to \NO) \to (\text{some condition}) \to \NO \]
which is indeed strictly positive.\index{strict!positivity}

Finally, after giving the mutual definitions of \NO and its ordering, Conway declares two surreal numbers $x$ and $y$ to be \emph{equal} if $x\le y$ and $y\le x$.
This is naturally read as passing to a quotient of the set of ``pre-surreals'' by an equivalence relation.
%(In set-theoretic foundations, one has to us an additional trick to deal with large equivalence classes.)
However, in the absence of the axiom of choice, such a quotient presents the same problem as the quotient in the usual construction of Cauchy reals: it will no longer be the case that a pair of families \emph{of surreals} yield a new surreal $\surr L R$, since we cannot necessarily ``lift'' $L$ and $R$ to families of pre-surreals.
Of course, we can solve this problem in the same way we did for Cauchy reals, by using a \emph{higher} inductive-inductive definition.

\begin{defn}\label{defn:surreals}
  The type \NO of \define{surreal numbers},
  \indexdef{surreal numbers}%
  \indexsee{number!surreal}{surreal numbers}%
  along with the relations $\mathord<:\NO\to\NO\to\type$ and $\mathord\le:\NO\to\NO\to\type$, are defined higher inductive-inductively as follows.
  The type \NO has the following constructors.
  \begin{itemize}
  \item For any $\LL,\RR:\UU$ and functions $\LL\to \NO$ and $\RR\to \NO$, whose values we write as $x^L$ and $x^R$ for $L:\LL$ and $R:\RR$ respectively, if $\fall{L:\LL}{R:\RR} x^L<x^R$, then there is a surreal number $x$.
  \item For any $x,y:\NO$ such that $x\le y$ and $y\le x$, we have $\noeq(x,y):x=y$.
  \end{itemize}
  We will refer to the inputs of the first constructor as a \define{cut}.
  \indexdef{cut!of surreal numbers}%
  If $x$ is the surreal number constructed from a cut, then the notation $x^L$ will implicitly assume $L:\LL$, and similarly $x^R$ will assume $R:\RR$.
  In this way we can usually avoid naming the indexing types $\LL$ and $\RR$, which is convenient when there are many different cuts under discussion.
  Following Conway, we call $x^L$ a \emph{left option}\indexdef{option of a surreal number} of $x$ and $x^R$ a \emph{right option}.

  The path constructor implies that different cuts can define the same surreal number.
  Thus, it does not make sense to speak of the left or right options of an arbitrary surreal number $x$, unless we also know that $x$ is defined by a particular cut.
  Thus in what follows we will say, for instance, ``given a cut defining a surreal number $x$'' in contrast to ``given a surreal number $x$''.

  The relation $\le$ has the following constructors.
  \index{non-strict order}%
  \index{order!non-strict}%
  \begin{itemize}
  \item Given cuts defining two surreal numbers $x$ and $y$, if $x^L<y$ for all $L$, and $x<y^R$ for all $R$, then $x\le y$.
  \item Propositional truncation:
    for any $x,y:\NO$, if $p,q:x\le y$, then $p=q$.
  \end{itemize}
  And the relation $<$ has the following constructors.
  \index{strict!order}%
  \index{order!strict}%
  \begin{itemize}
    % Don't technically need x in the first one and y in the second one to be defined by cuts?
  \item Given cuts defining two surreal numbers $x$ and $y$, if there is an $L$ such that $x\le y^L$, then $x<y$.
  \item Given cuts defining two surreal numbers $x$ and $y$, if there is an $R$ such that $x^R\le y$, then $x<y$.
  \item Propositional truncation: for any $x,y:\NO$, if $p,q:x<y$, then $p=q$.
  \end{itemize}
\end{defn}

\noindent
We compare this with Conway's definitions:
\begin{itemize}\footnotesize
\item[-] If $L,R$ are any two sets of numbers, and no member of $L$ is $\ge$ any member of $R$, then there is a number $\surr L R$.
  All numbers are constructed in this way.
\item[-] $x\ge y$ iff (no $x^R\le y$ and $x\le$ no $y^L$).
\item[-] $x=y$ iff ($x \ge y$ and $y\ge x$).
\item[-] $x>y$ iff ($x\ge y$ and $y\not\ge x$).
\end{itemize}
The inclusion of $x\ge y$ in the definition of $x>y$ is unnecessary if all objects are [surreal] numbers rather than ``games''\index{game!Conway}.
Thus, Conway's $<$ is just the negation of his $\ge$, so that his condition for $\surr L R$ to be a surreal is the same as ours.
Negating Conway's $\le$ and canceling double negations, we arrive at our definition of $<$, and we can then reformulate his $\le$ in terms of $<$ without negations.

We can immediately populate $\NO$ with many surreal numbers.
Like Conway, we write
\symlabel{surreal-cut}
\[\surr{x,y,z,\dots}{u,v,w,\dots}\]
for the surreal number defined by a cut where $\LL\to\NO$ and $\RR\to\NO$ are families described by $x,y,z,\dots$ and $u,v,w,\dots$.
Of course, if $\LL$ or $\RR$ are $\emptyt$, we leave the corresponding part of the notation empty.
There is an unfortunate clash with the standard notation $\setof{x:A | P(x)}$ for subsets, but we will not use the latter in this section.
\begin{itemize}
\item We define $\iota_{\nat}:\nat\to\NO$ recursively by
  \begin{align*}
    \iota_{\nat}(0) &\defeq \surr{}{},\\
    \iota_\nat(\suc(n)) &\defeq \surr{\iota_\nat(n)}{}.
  \end{align*}
  That is, $\iota_\nat(0)$ is defined by the cut consisting of $\emptyt\to\NO$ and $\emptyt\to\NO$.
  Similarly, $\iota_\nat(\suc(n))$ is defined by $\unit\to\NO$ (picking out $\iota_\nat(n)$) and $\emptyt\to\NO$.
\item Similarly, we define $\iota_{\Z}:\Z\to\NO$ using the sign-case recursion principle (\autoref{thm:sign-induction}):
  \begin{align*}
    \iota_{\Z}(0) &\defeq \surr{}{},\\
    \iota_\Z(n+1) &\defeq \surr{\iota_\Z(n)}{} & &\text{$n\ge 0$,}\\
    \iota_\Z(n-1) &\defeq \surr{}{\iota_\Z(n)} & &\text{$n\le 0$.}
  \end{align*}
\item By a \define{dyadic rational}
  \indexdef{rational numbers!dyadic}%
  \indexsee{dyadic rational}{rational numbers, dyadic}%
  we mean a pair $(a,n)$ where $a:\Z$ and $n:\nat$, and such that if $n>0$ then $a$ is odd.
  We will write it as $a/2^n$, and identify it with the corresponding rational number.
  If $\Q_D$ denotes the set of dyadic rationals, we define $\iota_{\Q_D}:\Q_D\to\NO$ by induction on $n$:
  \begin{align*}
    \iota_{\Q_D}(a/2^0) &\defeq \iota_\Z(a),\\
    \iota_{\Q_D}(a/2^n) &\defeq \surr{a/2^n - 1/2^n}{a/2^n + 1/2^n},
    \quad \text{for $n>0$.}
  \end{align*}
  Here we use the fact that if $n>0$ and $a$ is odd, then $a/2^n \pm 1/2^n$ is a dyadic rational with a smaller denominator than $a/2^n$.
\item We define $\iota_{\RD}:\RD\to\NO$, where $\RD$ is (any version of) the Dedekind reals from \autoref{sec:dedekind-reals}, by
  \begin{align*}
    \iota_{\RD}(x) &\defeq
    \surr{q\in\Q_D \text{ such that } q<x}{q\in\Q_D \text{ such that } x<q}.
  \end{align*}
  Unlike in the previous cases, it is not obvious that this extends $\iota_{\Q_D}$ when we regard dyadic rationals as Dedekind reals.
  This follows from the simplicity theorem (\autoref{thm:NO-simplicity}).
\item Recall the type \ord of \emph{ordinals}\index{ordinal} from \autoref{sec:ordinals}, which is well-ordered by the relation $<$, where $A<B$ means that $A = \ordsl B b$ for some $b:B$.
  We define $\iota_{\ord}:\ord\to\NO$ by well-founded recursion (\autoref{thm:wfrec}) on $\ord$:
  \begin{equation*}
    \iota_{\ord}(A) \defeq
    \surr{\iota_\ord(\ordsl A a) \text{ for all } a:A}{}.
  \end{equation*}
  It will also follow from the simplicity theorem that $\iota_\ord$ restricted to finite ordinals agrees with $\iota_\nat$.
\item A few more interesting examples taken from Conway:
  \begin{align*}
    \omega &\defeq \surr{0,1,2,3,\dots}{} \qquad\text{(also an ordinal)}\\
    -\omega &\defeq \surr{}{\dots,-3,-2,-1,0}\\
    1/\omega &\defeq \textstyle\surr{0}{1,\frac12,\frac14,\frac18,\dots}\\
    \omega-1 &\defeq \surr{0,1,2,3,\dots}{\omega}\\
    \omega/2 &\defeq \surr{0,1,2,3,\dots}{\dots,\omega-2,\omega-1,\omega}.
  \end{align*}
\end{itemize}

In identifying surreal numbers presented by different cuts, the following simple observation is useful.

\begin{thm}[Conway's simplicity theorem]\label{thm:NO-simplicity}
  \index{simplicity theorem}%
  \index{theorem!Conway's simplicity}%
  Suppose $x$ and $z$ are surreal numbers defined by cuts, and that the following hold.
  \begin{itemize}
  \item $x^L < z < x^R$ for all $L$ and $R$.
  \item For every left option $z^L$ of $z$, there exists a left option $x^{L'}$ with $z^L\le x^{L'}$.
  \item For every right option $z^R$ of $z$, there exists a right option $x^{R'}$ with $x^{R'}\le z^R$.
  \end{itemize}
  Then $x=z$.
\end{thm}
\begin{proof}
  Applying the path constructor of $\NO$, we must show $x\le z$ and $z\le x$.
  The first entails showing $x^L<z$ for all $L$, which we assumed, and $x<z^R$ for all $R$.
  But by assumption, for any $z^R$ there is an $x^{R'}$ with $x^{R'}\le z^R$ hence $x<z^R$ as desired.
  Thus $x\le z$; the proof of $z\le x$ is symmetric.
\end{proof}

\index{induction principle!for surreal numbers}
In order to say much more about surreal numbers, however, we need their induction principle.
The mutual induction principle for $(\NO,\le,<)$ applies to three families of types:
\begin{align*}
  A &: \NO\to\type\\
  B &: \prd{x,y:\NO}{a:A(x)}{b:A(y)} (x\le y) \to \type\\
  C &: \prd{x,y:\NO}{a:A(x)}{b:A(y)} (x<y) \to \type.
\end{align*}
As with the induction principle for Cauchy reals, it is helpful to think of $B$ and $C$ as families of relations between the types $A(x)$ and $A(y)$.
\symlabel{NO-recursion}
Thus we write $B(x,y,a,b,\xi)$ as $(x,a) \ble^\xi (y,b)$ and $C(x,y,a,b,\xi)$ as $(x,a) \blt^\xi (y,b)$.
Similarly, we usually omit the $\xi$ since it inhabits a mere proposition and so is uninteresting, and we may often omit $x$ and $y$ as well, writing simply $a\ble b$ or $a\blt b$.
With these notations, the hypotheses of the induction principle are the following.
\begin{itemize}
\item For any cut defining a surreal number $x$, together with
  \begin{enumerate}
  \item for each $L$, an element $a^L:A(x^L)$, and
  \item for each $R$, an element $a^R:A(x^R)$, such that
  \item for all $L$ and $R$ we have $(x^L,a^L) \blt (x^R,a^R)$
  \end{enumerate}
  there is a specified element $f_a:A(x)$.
  We call such data a \define{dependent cut}
  \indexdef{cut!of surreal numbers!dependent}%
  \indexdef{dependent!cut}%
  over the cut defining~$x$.
\item For any $x,y:\NO$ with $a:A(x)$ and $b:A(y)$, if $x\le y$ and $y\le x$ and also $(x,a) \ble (y,b)$
  and $(y,b) \ble (x,a)$,
  then $\dpath{A}{\noeq}{a}{b}$.
\item Given cuts defining two surreal numbers $x$ and $y$, and dependent cuts $a$ over $x$ and $b$ over $y$, such that for all $L$ we have $x^L<y$ and $(x^L,a^L)\blt (y,f_b)$,
  and for all $R$ we have $x<y^R$ and $(x,f_a) \blt (y^R,b^R)$,
  then $(x,f_a) \ble (y,f_b)$.
\item $\ble$ takes values in mere propositions.
\item Given cuts defining two surreal numbers $x$ and $y$, dependent cuts $a$ over $x$ and $b$ over $y$, and an $L_0$ such that $x\le y^{L_0}$ and $(x,f_a) \ble (y^{L_0},b^{L_0})$,
  we have $(x,f_a) \blt (y,f_b)$.
\item Given cuts defining two surreal numbers $x$ and $y$, dependent cuts $a$ over $x$ and $b$ over $y$, and an ${R_0}$ such that $x^{R_0}\le y$ together with $(x^{R_0},a^{R_0}),\ble (y,f_b)$,
  we have $(x,f_a) \blt (y,f_b)$.
\item $\blt$ takes values in mere propositions.
\end{itemize}
Under these hypotheses we deduce a function $f:\prd{x:\NO} A(x)$ such that
\begin{align}
  f(x) &\;\jdeq\; f_{f[x]} \label{eq:noind1}\\
  (x\le y) &\;\Rightarrow\; (x,f(x)) \ble (y,f(y)) \notag\\
  (x< y) &\;\Rightarrow\; (x,f(x)) \blt (y,f(y)). \notag
\end{align}
In the computation rule~\eqref{eq:noind1} for the point constructor, $x$ is a surreal number defined by a cut, and $f[x]$ denotes the dependent cut over $x$ defined by applying $f$ (and using the fact that $f$ takes $<$ to $\blt$).
As usual, we will generally use pattern-matching notation, where the definition of $f$ on a cut $\surr{x^L}{x^R}$ may use the symbols $f(x^L)$ and $f(x^R)$ and the assumption that they form a dependent cut.

As with the Cauchy reals, we have special cases resulting from trivializing some of $A$, $\ble$, and~$\blt$.
Taking $\ble$ and $\blt$ to be constant at \unit, we have \define{\NO-induction}, which for simplicity we state only for mere properties:
\begin{itemize}
\item Given $P:\NO\to\prop$, if $P(x)$ holds whenever $x$ is a surreal number defined by a cut such that $P(x^L)$ and $P(x^R)$ hold for all
$L$ and $R$, then $P(x)$ holds for all $x:\NO$.
\end{itemize}
This should be compared with Conway's remark:
\begin{quote}\footnotesize
  In general when we wish to establish a proposition $P(x)$ for all numbers $x$, we will prove it inductively by deducing $P(x)$ from the truth of all the propositions $P(x^L)$ and $P(x^R)$.
  We regard the phrase ``all numbers are constructed in this way'' as justifying the legitimacy of this procedure.
\end{quote}
With $\NO$-induction, we can prove

\begin{thm}[Conway's Theorem 0]\label{thm:NO-refl-opt}\
  \index{theorem!Conway's 0}%
  \begin{enumerate}
  \item For any $x:\NO$, we have $x\le x$.\label{item:NO-le-refl}
  \item For any $x:\NO$ defined by a cut, we have $x^L <x$ and $x<x^R$ for all $L$ and $R$.\label{item:NO-lt-opt}
  \end{enumerate}
\end{thm}
\begin{proof}
  Note first that if $x\le x$, then whenever $x$ occurs as a left option of some cut $y$, we have $x<y$ by the first constructor of $<$, and similarly whenever $x$ occurs as a right option of a cut $y$, we have $y<x$ by the second constructor of $<$.
  In particular,~\ref{item:NO-le-refl}$\Rightarrow$\ref{item:NO-lt-opt}.

  We prove~\ref{item:NO-le-refl} by $\NO$-induction on $x$.
  Thus, assume $x$ is defined by a cut such that $x^L\le x^L$ and $x^R \le x^R$ for all $L$ and $R$.
  But by our observation above, these assumptions imply $x^L<x$ and $x<x^R$ for all $L$ and $R$, yielding $x\le x$ by the constructor of $\le$.
\end{proof}

\begin{cor}\label{thm:NO-set}
  \NO is a 0-type.
%  (As with $V$, it might be confusing to say that it is a ``set''.)
\end{cor}
\begin{proof}
  The mere relation $R(x,y)\defeq (x\le y) \land (y\le x)$ implies identity by the path constructor of $\NO$, and contains the diagonal by \autoref{thm:NO-refl-opt}\ref{item:NO-le-refl}.
  Thus, \autoref{thm:h-set-refrel-in-paths-sets} applies.
\end{proof}

By contrast, Conway's Theorem 1 (transitivity of $\le$) is somewhat harder to establish with our definition; see \autoref{thm:NO-unstrict-transitive}.

% Of course, we also have:

% \begin{lem}
%   Every surreal number is merely defined by a cut.
% \end{lem}
% \begin{proof}
%   Obvious by $\NO$-induction.
% \end{proof}

We will also need the joint recursion principle, \define{$(\NO,\le,<)$-recursion}, which it is convenient to state as follows.
Suppose $A$ is a type equipped with relations $\mathord\ble:A\to A\to\prop$ and $\mathord\blt:A\to A\to\prop$.
Then we can define $f:\NO\to A$ by doing the following.
\begin{enumerate}
\item For any $x$ defined by a cut, assuming $f(x^L)$ and $f(x^R)$ to be defined such that $f(x^L)\blt f(x^R)$ for all $L$ and $R$, we must define $f(x)$.  (We call this the \emph{primary clause} of the recursion.)\label{item:NO-rec-primary}
\item Prove that $\ble$ is \emph{antisymmetric}\index{relation!antisymmetric}: if $a\ble b$ and $b\ble a$, then $a=b$.
\item For $x,y$ defined by cuts such that $x^L<y$ for all $L$ and $x<y^R$ for all $R$, and assuming inductively that $f(x^L)\blt f(y)$ for all $L$, $f(x)\blt f(y^R)$ for all $R$, and also that $f(x^L)\blt f(x^R)$ and $f(y^L)\blt f(y^R)$ for all $L$ and $R$, we must prove $f(x)\ble f(y)$.
\item For $x,y$ defined by cuts and an $L_0$ such that $x\le y^{L_0}$, and assuming inductively that $f(x)\ble f(y^{L_0})$, and also that $f(x^L)\blt f(x^R)$ and $f(y^L)\blt f(y^R)$ for all $L$ and $R$, we must prove $f(x)\blt f(y)$.
\item For $x,y$ defined by cuts and an $R_0$ such that $x^{R_0}\le y$, and assuming inductively that $f(x^{R_0})\ble f(y)$, and also that $f(x^L)\blt f(x^R)$ and $f(y^L)\blt f(y^R)$ for all $L$ and $R$, we must prove $f(x)\blt f(y)$.\label{item:NO-rec-last}
\end{enumerate}
The last three clauses can be more concisely described by saying we must prove that $f$ (as defined in the first clause) takes $\le$ to $\ble$ and $<$ to $\blt$.
We will refer to these properties by saying that \emph{$f$ preserves inequalities}.
Moreover, in proving that $f$ preserves inequalities, we may assume the particular instance of $\le$ or $<$ to be obtained from one of its constructors, and we may also use inductive hypotheses that $f$ preserves all inequalities appearing in the input to that constructor.

If we succeed at~\ref{item:NO-rec-primary}--\ref{item:NO-rec-last} above, then we obtain $f:\NO\to A$, which computes on cuts as specified by~\ref{item:NO-rec-primary}, and which preserves all inequalities:
%
\begin{narrowmultline*}
  \fall{x,y:\NO}\Big((x\le y) \to (f(x)\ble f(y))\Big) \land
  \narrowbreak
  \Big((x< y) \to (f(x)\blt f(y))\Big).
\end{narrowmultline*}
%
Like $(\RC,\closesym)$-recursion for the Cauchy reals, this recursion principle is essential for defining functions on $\NO$, since we cannot first define a function on ``pre-surreals'' and only later prove that it respects the notion of equality.

\begin{eg}
  Let us define the \emph{negation} function $\NO\to\NO$.
  We apply the joint recursion principle with $A\defeq\NO$, with $(x\ble y)\defeq (y\le x)$, and $(x\blt y)\defeq (y< x)$.
  Clearly this $\ble$ is antisymmetric.

  For the main clause in the definition, we assume $x$ defined by a cut, with $-x^L$ and $-x^R$ defined such that $-x^L \blt -x^R$ for all $L$ and $R$.
  By definition, this means $-x^R< -x^L$ for all $L$ and $R$, so we can define $-x$ by the cut $\surr{-x^R}{-x^L}$.
  This notation, which follows Conway, refers to the cut whose left options are indexed by the type $\RR$ indexing the right options of $x$, and whose right options are indexed by the type $\LL$ indexing the left options of $x$, with the corresponding families $\RR\to\NO$ and $\LL\to\NO$ defined by composing those for $x$ with negation.

  We now have to verify that $f$ preserves inequalities.
  \begin{itemize}
  \item For $x\le y$, we may assume $x^L<y$ for all $L$ and $x < y^R$ for all $R$, and show $-y\le -x$.
    But inductively, we may assume $-y <-x^L$ and $-y^R<-x$, which gives the desired result, by definition of $-y$, $-x$, and the constructor of $\le$.
  \item For $x<y$, in the first case when it arises from some $x\le y^{L_0}$, we may inductively assume $-y^{L_0} \le -x$, in which case $-y<-x$ follows by the constructor of $<$.
  \item Similarly, if $x<y$ arises from $x^{R_0}\le y$, the inductive hypothesis is $-y \le -x^R$, yielding $-y<-x$ again.
  \end{itemize}
\end{eg}

To do much more than this, however, we will need to characterize the relations $\le$ and $<$ more explicitly, as we did for the Cauchy reals in \autoref{thm:RC-sim-characterization}.
Also as there, we will have to simultaneously prove a couple of essential properties of these relations, in order for the induction to go through.

\begin{thm}\label{defn:No-codes}
  There are relations $\mathord\preceq:\NO\to\NO\to\prop$ and $\mathord\prec:\NO\to\NO\to\prop$ such that if $x$ and $y$ are surreals defined by cuts, then
  \begin{align*}
    (x\preceq y) &\defeq
    \big(\fall{L} x^L\prec y\big) \land \big(\fall{R} x\prec y^R\big)\\
    (x\prec y) &\defeq
    \big(\exis{L} x\preceq y^L\big) \lor \big(\exis{R} x^R \preceq y\big).
  \end{align*}
  Moreover, we have
  \begin{equation}\label{eq:NO-codes-unstrict}
    (x\prec y) \to (x\preceq y)
  \end{equation}
  and all the reasonable transitivity properties making $\prec$ and $\preceq$ into a ``bimodule''\index{bimodule} over $\le$ and $<$:
  \begin{equation}\label{eq:NO-codes-transitivity}
    \begin{array}{c@{\hspace{1cm}}c}
      (x \le y) \to (y\preceq z) \to (x\preceq z) &
      (x \preceq y) \to (y\le z) \to (x\preceq z) \\
      (x \le y) \to (y\prec z) \to (x\prec z) &
      (x \preceq y) \to (y< z) \to (x\prec z) \\
      (x < y) \to (y\preceq z) \to (x\prec z) &
      (x \prec y) \to (y\le z) \to (x\prec z).
  \end{array}
  \end{equation}
\end{thm}

\begin{proof}
  We define $\preceq$ and $\prec$ by double $(\NO,\le,<)$-induction on $x,y$.
  The first induction is a simple recursion, whose codomain is the subset $A$ of $(\NO\to\prop)\times (\NO\to\prop)$ consisting of pairs of predicates of which one implies the other and which satisfy ``transitivity on the right'', i.e.~\eqref{eq:NO-codes-unstrict} and the right column of~\eqref{eq:NO-codes-transitivity} with $(x\preceq \blank)$ and $(x\prec \blank)$ replaced by the two given predicates.
  As in the proof of \autoref{defn:RC-approx}, we regard these predicates as half of binary relations, writing them as $y\mapsto (\hle y)$ and $y\mapsto (\hlt y)$, with $\hlname$ denoting the pair of relations.
  % The precise definition of $A$ is
  % \begin{align*}
  %   A\defeq \bigg\{ \hlname : (\NO\to\prop)\times (\NO\to\prop) \;\bigg|\;\\
  %   \begin{split}
  %     \fall{y,z:\NO}
  %     &\Big( (\hle y) \to (y\le z) \to (\hle z) \Big)\\
  %     \land\; &\Big( (\hle y) \to (y< z) \to (\hlt z) \Big)\\
  %     \land\; &\Big( (\hlt y) \to (y\le z) \to (\hlt z) \Big)\\
  %     \land\; &\Big( (\hlt y) \to (y< z) \to (\hlt z) \Big) \bigg\}
  %   \end{split}
  % \end{align*}
  We equip $A$ with the following two relations:
  \begin{align*}
    (\hlname \ble \hlbname) &\defeq
    \fall{y:\NO} \Big( (\hleb y) \to (\hle y) \Big) \land
    \Big( (\hltb y) \to (\hlt y) \Big),\\
    (\hlname \blt \hlbname) &\defeq
    \fall{y:\NO} \Big( (\hleb y) \to (\hlt y) \Big).
    %\land \Big( (\hltb y) \to (\hlt y) \Big)
  \end{align*}
  Note that $\ble$ is antisymmetric, since if $\hlname \ble \hlbname$ and $\hlbname \ble \hlname$, then $(\hleb y) \Leftrightarrow (\hle y)$ and $(\hltb y) \Leftrightarrow (\hlt y)$ for all $y$, hence $\hlname=\hlbname$ by univalence for mere propositions and function extensionality.
  Moreover, to say that a function $\NO\to A$ preserves inequalities is exactly to say that, when regarded as a pair of binary relations on $\NO$, it satisfies ``transitivity on the left'' (the left column of~\eqref{eq:NO-codes-transitivity}).

  Now for the primary clause of the recursion, we assume given $x$ defined by a cut, and relations $(x^L \prec \blank)$, $(x^R \prec \blank)$, $(x^L \preceq \blank)$, and $(x^R \preceq \blank)$ for all $L$ and $R$, of which the strict ones imply the non-strict ones, which satisfy transitivity on the right, and such that
  \begin{equation}\label{eq:NO-prec-outer-IH}
    \fall{L,R}{y:\NO}\Big( (x^R\preceq y) \to (x^L \prec y) \Big).
    % \land\Big( (x^R \prec y) \to (x^L \prec y) \Big)
  \end{equation}
  We now have to define $(x\prec y)$ and $(x\preceq y)$ for all $y$.
  Here in contrast to \autoref{defn:RC-approx}, rather than a nested recursion, we use a nested induction, in order to be able to inductively use transitivity on the left with respect to the inequalities $x^L<x$ and $x<x^R$.
  Define $A':\NO\to\type$ by taking $A'(y)$ to be the subset $A'$ of $\prop\times\prop$ consisting of two mere propositions, denoted $\tle y$ and $\tlt y$ (with $\tlname:A'(y)$), such that
  \begin{gather}
    (\tlt y) \to (\tle y)\\
    \fall{L} (\tle y)\to (x^L\prec y) \label{eq:NO-prec-IHL}\\
    \fall{R} (x^R \preceq y) \to (\tlt y) \label{eq:NO-prec-IHR}.
  \end{gather}
  Using notation analogous to $\ble$ and $\blt$, we equip $A'$ with the two relations defined for $\tlname:A'(y)$ and $\tlbname:A'(z)$ by
  \begin{align*}
    (\tlname \bble \tlbname) &\defeq
    \Big((\tle y) \to (\tleb z)\Big) \land \Big((\tlt y) \to (\tltb z)\Big)\\
    (\tlname \bblt \tlbname) &\defeq
    \Big((\tle y) \to (\tltb z)\Big). % \land \Big(\tlt \to \tltb\Big).
  \end{align*}
  % (These are the type families $B$ and $C$ in the general induction principle.)
  Again, $\bble$ is evidently antisymmetric in the appropriate sense.
  Moreover, a function $\prd{y:\NO} A'(y)$ which preserves inequalities is precisely a pair of predicates of which one implies the other, which satisfy transitivity on the right, and transitivity on the left with respect to the inequalities $x^L<x$ and $x<x^R$.
  Thus, this inner induction will provide what we need to complete the primary clause of the outer recursion.

  For the primary clause of the inner induction, we assume also given $y$ defined by a cut, and properties $(x\prec y^L)$, $(x\prec y^R)$, $(x\preceq y^L)$, and $(x\preceq y^R)$ for all $L$ and $R$, with the strict ones implying the non-strict ones, transitivity on the left with respect to $x^L<x$ and $x<x^R$, and on the right with respect to $y^L<y^R$.
  % \begin{equation}
  %   \fall{L,R}\Big((x \preceq y^L) \to (x \prec y^R)\Big) % \land \Big((x \prec y^L) \to (x\prec y^R)\Big).
  %   \label{eq:NO-prec-inner-IH}
  % \end{equation}
  We can now give the definitions specified in the theorem statement:
  \begin{align}
    (x\preceq y) &\defeq
    (\fall{L} x^L\prec y) \land (\fall{R} x\prec y^R), \label{eq:NO-preceq-def}\\
    (x\prec y) &\defeq
    (\exis{L} x\preceq y^L) \lor (\exis{R} x^R \preceq y).\label{eq:NO-prec-def}
  \end{align}
  For this to define an element of $A'(y)$, we must show first that $(x\prec y) \to (x\preceq y)$.
  The assumption $x\prec y$ has two cases.
  On one hand, if there is $L_0$ with $x\preceq y^{L_0}$, then by transitivity on the right with respect to $y^{L_0}<y^R$, we have $x\prec y^R$ for all $R$.
  Moreover, by transitivity on the left with respect to $x^L<x$, we have $x^L \prec y^{L_0}$ for any $L$, hence $x^L\prec y$ by transitivity on the right.
  Thus, $x\preceq y$.

  On the other hand, if there is $R_0$ with $x^{R_0}\preceq y$, then by transitivity on the left with respect to $x^L<x^{R_0}$ we have $x^L \prec y$ for all $L$.
  And by transitivity on the left and right with respect to $x<x^{R_0}$ and $y<y^R$, we have $x\prec y^R$ for any $R$.
  Thus, $x\preceq y$.

  We also need to show that these definitions are transitive on the left with respect to $x^L<x$ and $x<x^R$.
  But if $x\preceq y$, then $x^L\prec y$ for all $L$ by definition; while if $x^R\preceq y$, then $x\prec y$ also by definition.

  Thus,~\eqref{eq:NO-preceq-def} and~\eqref{eq:NO-prec-def} do define an element of $A'(y)$.
  We now have to verify that this definition preserves inequalities, as a dependent function into $A'$, i.e.\ that these relations are transitive on the right.
  Remember that in each case, we may assume inductively that they are transitive on the right with respect to all inequalities arising in the inequality constructor.
  \begin{itemize}
  \item Suppose $x\preceq y$ and $y\le z$, the latter arising from $y^L<z$ and $y<z^R$ for all $L$ and $R$.
    Then the inductive hypothesis (of the inner recursion) applied to $y<z^R$ yields $x\prec z^R$ for any $R$.
    Moreover, by definition $x\preceq y$ implies that $x^L \prec y$ for any $L$, so by the inductive hypothesis of the outer recursion we have $x^L \prec z$.
    Thus, $x\preceq z$.
  \item Suppose $x\preceq y$ and $y<z$.
    First, suppose $y<z$ arises from $y\le z^{L_0}$.
    Then the inner inductive hypothesis applied to $y\le z^{L_0}$ yields $x \preceq z^{L_0}$, hence $x\prec z$.

    Second, suppose $y<z$ arises from $y^{R_0}\le z$.
    Then by definition, $x\preceq y$ implies $x\prec y^{R_0}$, and then the inner inductive hypothesis for $y^{R_0}\le z$ yields $x\prec z$.
  \item Suppose $x\prec y$ and $y\le z$, the latter arising from $y^L<z$ and $y<z^R$ for all $L$ and $R$.
    By definition, $x\prec y$ implies there merely exists $R_0$ with $x^{R_0}\preceq y$ or $L_0$ with $x\preceq y^{L_0}$.
    If $x^{R_0}\preceq y$, then the outer inductive hypothesis yields $x^{R_0}\preceq z$, hence $x\prec z$.
    If $x\preceq y^{L_0}$, then the inner inductive hypothesis for $y^{L_0}<z$ (which holds by the constructor of $y\le z$) yields $x\prec z$.
  % \item Suppose $x\prec y$ and $y<z$.
  %   First, suppose $y<z$ arises from $y\le z^{L_0}$.
  %   Then the inner inductive hypothesis for $y\le z^{L_0}$ yields $x\prec z^{L_0}$, hence $x\preceq z^{L_0}$; thus $x\prec z$.

  %   Second, suppose $y<z$ arises from $y^{R_0}\le z$.
  %   Then by definition, $x\prec y$ implies there merely exists $R_1$ with $x^{R_1}\preceq y$ or $L_1$ with $x\preceq y^{L_1}$.
  %   If $x^{R_1}\preceq y$, then the outer inductive hypothesis implies $x^{R_1}\prec z$, hence $x^{R_1}\preceq z$, and thus $x\prec z$.
  %   And if $x\preceq y^{L_1}$, then the inner inductive hypothesis applied to $y^{L_1}<y^{R_0}$ (which comes from $y$ being defined as a cut) and $y^{R_0}\le z$ yields $x\prec z$.
  \end{itemize}
  This completes the inner induction.
  Thus, for any $x$ defined by a cut, we have $(x\prec \blank)$ and $(x\preceq \blank)$ defined by~\eqref{eq:NO-preceq-def} and~\eqref{eq:NO-prec-def}, and transitive on the right.

  To complete the outer recursion, we need to verify these definitions are transitive on the left.
  After a $\NO$-induction on $z$, we end up with three cases that are essentially identical to those just described above for transitivity on the right.
  Hence, we omit them.
\end{proof}

\begin{thm}\label{thm:NO-encode-decode}
  For any $x,y:\NO$ we have $(x<y)=(x\prec y)$ and $(x\le y)=(x\preceq y)$.
\end{thm}
\begin{proof}
  From left to right, we use $(\NO,\le,<)$-induction where $A(x)\defeq\unit$, with $\preceq$ and $\prec$ supplying the relations $\ble$ and $\blt$.
  In all the constructor cases, $x$ and $y$ are defined by cuts, so the definitions of $\preceq$ and $\prec$ evaluate, and the inductive hypotheses apply.

  From right to left, we use $\NO$-induction to assume that $x$ and $y$ are defined by cuts.
  But now the definitions of $\preceq$ and $\prec$, and the inductive hypotheses, supply exactly the data required for the relevant constructors of $\le$ and $<$.
  % From right to left, we first prove by $\NO$-induction on $x$ that for any $y:\NO$ we have $(x\prec y) \to (x<y)$ and $(x\preceq y) \to (x\le y)$.
  % Thus, we assume this to be true for all $x^L$ and $x^R$ in a cut, and show it for the resulting $x:\NO$.
  % Next, we prove by $\NO$-induction on $y$ that $(x\prec y) \to (x<y)$ and $(x\preceq y) \to (x\le y)$, hence we assume it to be true for all $y^L$ and $y^R$ in a cut, and show it for the resulting $y:\NO$.
  % Now since $x$ and $y$ are both defined by cuts, $x\preceq y$ means that $x^L\prec y$ and $x\prec y^R$ for all $L$ and $R$.
  % By the inductive hypotheses, this gives $x^L<y$ and $x<y^R$, hence $x\le y$ by the constructor of $\le$.
  % Similarly, $x\prec y$ yields merely an $R_0$ with $x^{R_0}\preceq y$ or an $L_0$ with $x\preceq y^{L_0}$.
  % Hence merely $x^{R_0}\le y$ or $x\le y^{L_0}$ by the inductive hypothesis, so $x<y$ by a constructor.
\end{proof}

\begin{cor}\label{thm:NO-unstrict-transitive}
  The relations $\le$ and $<$ on $\NO$ satisfy
  \[ \fall{x,y:\NO} (x<y) \to (x\le y) \]
  and are transitive:
  \index{transitivity!of . for surreals@of $<$ for surreals}
  \index{transitivity!of . for surreals@of $\leq$ for surreals}
  \begin{gather*}
    (x\le y) \to (y\le z) \to (x\le z)\\
    (x\le y) \to (y< z) \to (x< z)\\
    (x< y) \to (y\le z) \to (x< z).
  \end{gather*}
\end{cor}

As with the Cauchy reals, the joint $(\NO,\le,<)$-recursion principle remains essential when defining all operations on $\NO$.

\begin{eg}
\index{addition!of surreal numbers}%
We define $\mathord+:\NO\to\NO\to\NO$ by a double recursion.
For the outer recursion, we take the codomain to be the subset of $\NO\to\NO$ consisting of functions $g$ such that $(x<y) \to (g(x)<g(x))$ and $(x\le y) \to (g(x)\le g(y))$ for all $x,y$.
For such $g,h$ we define $(g\ble h)\defeq \fall{x:\NO} g(x)\le h(x)$ and $(g\blt h)\defeq \fall{x:\NO} g(x)< h(x)$.
Clearly $\ble$ is antisymmetric.

For the primary clause of the recursion, we suppose $x$ defined by a cut, and we define $(x+\blank)$ by an inner recursion on $\NO$ with codomain $\NO$, with relations $\bble$ and $\bblt$ coinciding with $\le$ and $<$.
For the primary clause of the inner recursion, we suppose also $y$ defined by a cut, and give Conway's definition:
\[ x+y \defeq \surr{x^L+y, x+y^L}{x^R+y,x+y^R}. \]
In other words, the left options of $x+y$ are all numbers of the form $x^L+y$ for some left option $x^L$, or $x+y^L$ for some left option $y^L$.
Now we verify that this definition preserves inequality:
\begin{itemize}
\item If $y\le z$ arises from knowing that $y^L<z$ and $y<z^R$ for all $L$ and $R$, then the inner inductive hypothesis gives $x+y^L<x+z$ and $x+y < x+z^R$, while the outer inductive hypotheses give $x^L+y < x^L+z$ and $x^R+ y < x^R+z$.
  And since each $x^L+z$ is by definition a left option of $x+z$, we have $x^L+z < x+z$, and similarly $x+y < x^R+y$.
  Thus, using transitivity, $x^L+y < x+z$ and $x+y < x^R+z$, and so we may conclude $x+y \le x+z$ by the constructor of $\le$.
\item If $y<z$ arises from an $L_0$ with $y\le z^{L_0}$, then inductively $x+y \le x+z^{L_0}$, hence $x+y<x+z$ since $x+z^{L_0}$ is a right option of $x+z$.
\item Similarly, if $y<z$ arises from $y^{R_0}\le z$, then $x+y<x+z$ since $x+y^{R_0}\le x+z$.
\end{itemize}
This completes the inner recursion.
For the outer recursion, we have to verify that $+$ preserves inequality on the left as well.
After an $\NO$-induction, this proceeds in exactly the same way.
\end{eg}

\index{acceptance|(}%
\index{mathematics!formalized}%
In the Appendix to Part Zero of~\cite{conway:onag}, Conway discusses how the surreal numbers may be formalized in ZFC set theory: by iterating along the ordinals and passing to sets of representatives of lowest rank for each equivalence class, or by representing numbers with ``sign-expansions''.
He then remarks that
\begin{quote}\footnotesize
  The curiously complicated nature of these constructions tells us more about the nature of formalizations within ZF than about our system of numbers\dots
\end{quote}
and goes on to advocate for a general theory of ``permissible kinds of construction'' which should include
\begin{enumerate}\footnotesize
\item Objects may be created from earlier objects in any reasonably constructive fashion.\label{item:conway1}
\item Equality among the created objects can be any desired equivalence relation.\label{item:conway2}
\end{enumerate}
\noindent
Condition~\ref{item:conway1} can be naturally read as justifying general principles of \emph{inductive definition}, such as those presented in \autoref{sec:strictly-positive,sec:generalizations}.
In particular, the condition of strict positivity for constructors can be regarded as a formalization of what it means to be ``reasonably constructive''.
Condition~\ref{item:conway2} then suggests we should extend this to \emph{higher} inductive definitions of all sorts, in which we can impose path constructors making objects equal in any reasonable way.
For instance, in the next paragraph Conway says:
\begin{quote}\footnotesize
  \dots we could also, for instance, freely create a new object $(x,y)$ and call it the ordered pair of $x$ and $y$.
  We could also create an ordered pair $[x,y]$ different from $(x,y)$ but co-existing with it\dots
  If instead we wanted to make $(x,y)$ into an unordered pair, we could define equality by means of the equivalence relation $(x,y)=(z,t)$ if and only if $x=z,y=t$ \emph{or} $x=t,y=z$.
\end{quote}
The freedom to introduce new objects with new names, generated by certain forms of constructors, is precisely what we have in the theory of inductive definitions.
Just as with our two copies of the natural numbers $\nat$ and $\nat'$ in \autoref{sec:appetizer-univalence}, if we wrote down an identical definition to the cartesian product type $A\times B$, we would obtain a distinct product type $A\times' B$ whose canonical elements we could freely write as $[x,y]$.
And we could make one of these a type of unordered pairs by adding a suitable path constructor. % (and perhaps 0-truncating).

To be sure, Conway's point was not to complain about ZF in particular, but to argue against all foundational theories at once:
\begin{quote}\footnotesize
  \dots this proposal is not of any particular theory as an alternative to ZF\dots{}
  What is proposed is instead that we give ourselves the freedom to create arbitrary mathematical theories of these kinds, but prove a metatheorem which ensures once and for all that any such theory could be formalized in terms of any of the standard foundational theories.
\end{quote}
One might respond that, in fact, univalent foundations is not one of the ``standard foundational theories'' which Conway had in mind, but rather the \emph{metatheory} in which we may express our ability to create new theories, and about which we may prove Conway's metatheorem.
For instance, the surreal numbers are one of the ``mathematical theories'' Conway has in mind, and we have seen that they can be constructed and justified inside univalent foundations.
Similarly, Conway remarked earlier that
\begin{quote}\footnotesize
  \dots set theory would be such a theory, sets being constructed from earlier ones by processes corresponding to the usual axioms, and the equality relation being that of having the same members.
\end{quote}
This description closely matches the higher-inductive construction of the cumulative hierarchy of set theory in \autoref{sec:cumulative-hierarchy}.
Conway's metatheorem would then correspond to the fact we have referred to several times that we can construct a model of univalent foundations inside ZFC (which is outside the scope of this book).

However, univalent foundations is so rich and powerful in its own right that it would be foolish to relegate it to only a metatheory in which to construct set-like theories.
We have seen that even at the level of sets (0-types), the higher inductive types in univalent foundations yield direct constructions of objects by their universal properties (\autoref{sec:free-algebras}), such as a constructive theory of Cauchy completion (\autoref{sec:cauchy-reals}).
But most importantly, the potential to model homotopy theory and category theory directly in the foundational system (\autoref{cha:homotopy,cha:category-theory}) gives univalent foundations an advantage which no set-theoretic foundation can match.
\index{acceptance|)}%

\index{surreal numbers|)}%

\sectionNotes

Defining algebraic operations on Dedekind reals, especially multiplication, is both somewhat tricky and tedious.
There are several ways to get arithmetic going: each has its own advantages, but they all seem to require some technical work.
For instance, Richman~\cite{Richman:reals} defines multiplication on the Dedekind reals first on the positive cuts and then extends it algebraically to all Dedekind cuts, while Conway~\cite{conway:onag} has observed that the definition of multiplication for surreal numbers works well for Dedekind reals.

Our treatment of the Dedekind reals borrows many ideas from~\cite{BauerTaylor09} where the Dedekind reals are constructed in the context of Abstract Stone Duality.
\index{Abstract Stone Duality}%
This is a (restricted) form of simply typed $\lambda$-calculus with a distinguished object $\Sigma$ which classifies open sets, and by duality also the closed ones. In~\cite{BauerTaylor09} you can also find detailed proofs of the basic properties of arithmetical operations.

The fact that $\RC$ is the least Cauchy complete archimedean ordered field, as was proved in \autoref{RC-initial-Cauchy-complete}, indicates that our Cauchy reals probably coincide with the Escard{\'o}-Simpson reals~\cite{EscardoSimpson:01}.
\index{real numbers!Escardo-Simpson@Escard\'o-Simpson}%
It would be interesting to check\index{open!problem} whether this is really the case. The notion of Escard{\'o}-Simpson reals, or more precisely the corresponding closed interval, is interesting because it can be stated in any category with finite products.

In constructive set theory augmented by the ``regular extension axiom'', one may also try to define Cauchy completion by closing under limits of Cauchy sequences with a transfinite iteration.
It would also be interesting to check whether this construction agrees with ours.

It is constructive folklore that coincidence of Cauchy and Dedekind reals requires dependent choice but it is less well known that countable choice suffices. Recall that \define{dependent choice}
\indexdef{axiom!of choice!dependent}%
\index{axiom!of choice!countable}%
\index{total!relation}%
states that for a total relation $R$ on $A$, by which we mean $\fall{x : A} \exis{y : A} R(x,y)$, and for any $a : A$ there merely exists $f : \N \to A$ such that $f(0) = a$ and $R(f(n), f(n+1))$ for all $n : \N$. Our \autoref{when-reals-coincide} uses the typical trick for converting an application of dependent choice to one using countable choice. Namely, we use countable choice once to make in advance all the choices that could come up, and then use the choice function to avoid the dependent choices.

The intricate relationship between various notions of compactness in a constructive
setting is discussed in \cite{bridges2002compactness}. Palmgren~\cite{Palmgren:FT} has a
good comparison between pointwise analysis and
pointfree topology.

The surreal numbers were defined by~\cite{conway:onag}, using a sort of inductive definition but without justifying it explicitly in terms of any foundational system.
For this reason, some later authors have tended to use sign-expansions or other more explicit presentations which can be coded more obviously into set theory.
The idea of representing them in type theory was first considered by Hancock, while
Setzer and Forsberg~\cite{forsbergfinite} noted that the surreals and their inequality relations $<$ and $\le$ naturally form an inductive-inductive definition.
The \emph{higher} inductive-inductive version presented here, which builds in the correct notion of equality for surreals, is new.


\sectionExercises

\begin{ex}\label{ex:alt-dedekind-reals}
 Give an alternative definition of the Dedekind reals by first defining the square and then use \autoref{mult-from-square}.
 Check that one obtains a commutative ring.
\end{ex}

\begin{ex} \label{ex:RD-extended-reals}
  %
  Suppose we remove the boundedness condition in \autoref{defn:dedekind-reals}.
  Then we obtain the \define{extended reals}
  \indexdef{real numbers!extended}%
  \indexdef{extended real numbers}%
  which contain $-\infty \defeq
  (\emptyt, \Q)$ and $\infty \defeq (\Q, \emptyt)$. Which definitions of arithmetical
  operations on cuts still make sense for extended reals? What algebraic structure do we
  get?
\end{ex}

\begin{ex} \label{ex:RD-lower-cuts}
  %
  By considering one-sided cuts we obtain \define{lower} and \define{upper} Dedekind reals,
  \indexdef{real numbers!Dedekind!upper}%
  \indexdef{real numbers!Dedekind!lower}%
  \indexdef{lower Dedekind reals}%
  \indexdef{upper Dedekind reals}%
  \index{cut!Dedekind}%
  respectively. For example, a lower real is given by a predicate $L : \Q \to \Omega$
  which is
  %
  \begin{enumerate}
  \item \emph{inhabited:} $\exis{q : \Q} L(q)$ and
  \item \emph{rounded:} $L(q) = \exis{r : \Q} q < r \land L(r)$.
    \index{rounded!Dedekind cut}
  \end{enumerate}
  %
  (We could also require $\exis{r : \Q} \lnot L(r)$ to exclude the cut $\infty \defeq
  \Q$.) Which arithmetical operations can you define on the lower reals? In particular,
  what happens with the additive inverse?
\end{ex}

\begin{ex} \label{ex:RD-interval-arithmetic}
  %
  \index{interval!arithmetic}%
  Suppose we remove the locatedness condition in \autoref{defn:dedekind-reals}.
  Then we obtain the \define{interval domain}
  \indexdef{interval!domain}%
  $\mathbb{I}$ because cuts are allowed
  to have ``gaps'', which are just intervals. Define the partial order $\sqsubseteq$ on
  $\mathbb{I}$ by
  %
  \begin{narrowmultline*}
    ((L, U) \sqsubseteq (L', U'))
    \defeq \narrowbreak
    (\fall{q : \Q} L(q) \Rightarrow L'(q)) \land
    (\fall{q : \Q} U(q) \Rightarrow U'(q)).
  \end{narrowmultline*}
  %
  What are the maximal elements of $\mathbb{I}$ with respect to $\mathbb{I}$? Define the
  ``endpoint'' operations which assign to an element of the interval domain its lower and
  upper endpoints. Are the endpoints reals, lower reals, or upper reals (see
  \autoref{ex:RD-lower-cuts})? Which definitions of arithmetical operations on cuts still
  make sense for the interval domain?
\end{ex}

\begin{ex} \label{ex:RD-lt-vs-le}
  Show that, for all $x, y : \RD$,
  %
  \begin{equation*}
    \lnot (x < y) \Rightarrow y \leq x
  \end{equation*}
  %
  and
  %
  \begin{equation*}
    \eqv{(x \leq y)}{\Parens{\prd{\epsilon : \Qp} x < y + \epsilon}}.
  \end{equation*}
  %
  Does $\lnot (x \leq y)$ imply $y < x$?
\end{ex}

\begin{ex} \label{ex:reals-non-constant-into-Z}
  \mbox{}
  %
  \begin{enumerate}
  \item
    Assuming excluded middle, construct a non-constant map $\RD \to \Z$.
  \item
    Suppose $f : \RD \to \Z$ is a map such that $f(0) = 0$ and $f(x) \neq 0$ for all $x >
    0$. Derive from this the limited principle of omniscience~\eqref{eq:lpo}.
\index{limited principle of omniscience}%
  \end{enumerate}
\end{ex}

\begin{ex} \label{ex:traditional-archimedean}
  \index{ordered field!archimedean}%
  Show that in an ordered field $F$, density of $\Q$ and the traditional archimedean axiom
  are equivalent:
  %
  \begin{equation*}
    (\fall{x, y : F} x < y \Rightarrow \exis{q : \Q} x < q < y)
    \Leftrightarrow
    (\fall{x : F} \exis{k : \Z} x < k).
  \end{equation*}
\end{ex}

\begin{ex} \label{RC-Lipschitz-on-interval} Suppose $a, b : \Q$ and $f : \setof{ q : \Q |
    a \leq q \leq b } \to \RC$ is Lipschitz with constant~$L$. Show that there exists a unique
  extension $\bar{f} : [a,b] \to \RC$ of $f$ which is Lipschitz with
  constant~$L$. Hint: rather than redoing \autoref{RC-extend-Q-Lipschitz} for closed
  intervals, observe that there is a retraction $r : \RC \to [-n,n]$ and apply
  \autoref{RC-extend-Q-Lipschitz} to $f \circ r$.
\end{ex}

\begin{ex} \label{ex:metric-completion}
  \index{completion!of a metric space}%
  Generalize the construction of $\RC$ to construct the Cauchy completion of any metric space. First, think about which notion of real numbers is most natural as the codomain for the distance\index{distance} function of a metric space. Does it matter? Next, work out the details of two constructions:
  %
  \begin{enumerate}
  \item Follow the construction of Cauchy reals to define the completion of a metric space as an inductive-inductive type closed under limits of Cauchy sequences.\index{Cauchy!sequence}
  \item Use the following construction due to Lawvere~\cite{lawvere:metric-spaces}\index{Lawvere} and Richman~\cite{Richman00thefundamental}, where the completion of a metric space $(M, d)$ is given as the type of \define{locations}.
    \indexdef{location}%
    A location is a function $f : M \to \R$ such that
    %
    \begin{enumerate}
    \item $f(x) \geq |f(y) - d(x,y)|$ for all $x, y : M$, and
    \item $\inf_{x \in M} f(x) = 0$, by which we mean $\fall{\epsilon : \Qp} \exis{x : M} |f(x)| < \epsilon$ and $\fall{x : M} f(x) \geq 0$.
    \end{enumerate}
    %
    The idea is that $f$ looks like it is measuring the distance from a point.
  \end{enumerate}
  %
  \index{universal!property!of metric completion}%
  Finally, prove the following universal property of metric completions: a locally uniformly continuous map from a metric space to a Cauchy complete metric space extends uniquely to a locally uniformly continuous map on the completion. (We say that a map is \define{locally uniformly continuous}
  \indexdef{function!locally uniformly continuous}%
  \indexdef{locally uniformly continuous map}%
  if it is uniformly continuous on open balls.)
\end{ex}

\index{metric space|)}%

\begin{ex} \label{ex:reals-apart-neq-MP}
  \define{Markov's principle}
  \indexdef{axiom!Markov's principle}%
  \indexdef{Markov's principle}%
  says that for all $f : \nat \to \bool$,
  %
  \begin{equation*}
    (\lnot \lnot \exis{n : \nat} f(n) = \btrue)
    \Rightarrow
    \exis{n : \nat} f(n) = \btrue.
  \end{equation*}
  %
  This is a particular instance of the law of double negation~\eqref{eq:ldn}. Show that
  $\fall{x, y: \RD} x \neq y \Rightarrow x \apart y$ implies Markov's principle. Does the
  converse hold as well?
\end{ex}

\begin{ex} \label{ex:reals-apart-zero-divisors}
  \index{apartness}%
  Verify that the following ``no zero divisors'' property holds for the real numbers:
  $x y \apart 0 \Leftrightarrow x \apart 0 \land y \apart 0$.
\end{ex}

\begin{ex} \label{ex:finite-cover-lebesgue-number}
  %
  Suppose $(q_1, r_1), \ldots, (q_n, r_n)$ pointwise cover $(a, b)$. Then there is
  $\epsilon : \Qp$ such that whenever $a < x < y < b$ and $|x - y| < \epsilon$
  then there merely exists $i$ such that $q_i < x < r_i$ and $q_i < y < r_i$. Such an
  $\epsilon$ is called a \define{Lebesgue number}
  \indexdef{Lebesgue number}%
  for the given cover.
\end{ex}

\begin{ex} \label{ex:mean-value-theorem}
  %
  Prove the following approximate version of the intermediate value theorem:
  %
  \begin{quote}
    \emph{
      If $f : [0,1] \to \R$ is uniformly continuous and $f(0) < 0 < f(1)$ then
      for every $\epsilon : \Qp$ there merely exists $x : [0,1]$ such that $|f(x)| <
      \epsilon$.
    }
  \end{quote}
  %
  Hint: do not try to use the bisection method because it leads to the axiom of choice.
  Instead, approximate $f$ with a piecewise linear map. How do you construct a piecewise
  linear map?
\end{ex}

\begin{ex}\label{ex:knuth-surreal-check}
  Check whether everything in~\cite{knuth74:_surreal_number} can be done using the higher
  inductive-inductive surreals of \autoref{sec:surreals}.
\end{ex}

\index{real numbers|)}%

%%% Local Variables:
%%% mode: latex
%%% TeX-master: "hott-online"
%%% End:


%%%% Appendix

%\cleartooddpage[\thispagestyle{empty}] % Needed for correct TOC
\phantomsection % Needed for correct TOC also
\part*{Appendix}

% We use magic to get the appendix look like Bibliography and Index

\appendix

%\renewcommand{\chaptermark}[1]{\markboth{\textsc{Appendix. \thechapter. #1}}{}}
%\renewcommand{\sectionmark}[1]{\markright{\textsc{\thesection\ #1}}}

%\chapter{The Rules of Type Theory}
\label{cha:rules}
\bgroup % restrict the scope of our macros to this section

\newcommand{\ctx}{\ \mathsf{ctx}}
\newcommand{\emptyctx}{\ensuremath{\cdot}}

\newcommand{\production}{\vcentcolon\vcentcolon=}

\newcommand{\mkbox}[1]{\ensuremath{#1}}

\newcommand{\app}{\mathsf{app}}

\newcommand{\gothic}{\mathfrak}
\newcommand{\gP}{{\gothic p}}
\newcommand{\gM}{{\gothic M}}
\newcommand{\gN}{{\gothic N}}
\newcommand{\rats}{\mathbb{Q}}
\newcommand{\ints}{\mathbb{Z}}

\newcommand{\lbr}{\lbrack\!\lbrack}
\newcommand{\rbr}{\rbrack\!\rbrack}
\newcommand{\sem}[2] {\lbr #1 \rbr_{#2}}  % interpretation of the terms
\newcommand{\APP}[2] {{\sf app}(#1,#2)}  % interpretation of the terms
\newcommand{\nats}{\mathbb{N}}
\newcommand{\Con}{{\sf Con}}
\newcommand{\Elem}{{\sf Elem}}
\newcommand{\myId}{1}
\newcommand{\mypp}{{\sf p}}
\newcommand{\qq}{{\sf q}}
\newcommand{\mySp}{{\sf Sp}}
\newcommand{\conv}{\sim}
\newcommand{\LIM}{{\sf lim}}
\newcommand{\nn}{{\sf n}}
\newcommand{\Fam}{{\sf Fam}}

A major advantage of type theory is that it 
simultaneously incorporates the rules of mathematics with the rules of logic,
thus providing a completely self-contained system for formulating theorems
and their proofs, and enabling their computer verification.  In this appendix we
offer two presentations of the formal system and its syntax.  The first
presentation (in \autoref{syntax-informally}) is written informally and
emphasizes an underlying untyped type theory as basic support for the syntax
and for aspects of computation in the system.  The second (in
\autoref{syntax-more-formally}) is more formal and hence more self-contained,
and is closer to standard practice in the literature of type theory.

In \autoref{cha:typetheory}, we presented the two basic {\em judgments} of type
theory. The first, $a:A$, asserts that a term $a$ has type $A$.  The second,
$a\jdeq b:A$, states that the two terms $a$ and $b$ are {\em judgmentally equal}
at type $A$. These judgments are inductively defined by a set of {\em inference
rules}, to be presented below; a judgment holds exactly when it can be concluded
by a closed derivation composed of these inference rules.

To give a proof $a$ of a proposition $A$ is to derive $a:A$; in the book, we
usually give informal arguments which describe the construction of $a$, but
formally, one must give a precise term $a$ and a full derivation that $a:A$. It
is possible to algorithmically determine the type, if any, of a term. It is also
possible to algorithmically establish equality between terms; an algorithm is
included in \autoref{syntax-informally}.

However, the main difference between the book's type theory and formal type
theory is that formally, judgments are formulated in an ambient {\em
context}, or list of assumptions, of the form
\[
  \Gamma =  x_1:A_1, x_2:A_2,\dots,x_n:A_n
\]
Here each pair $x_i:A_i$ in the list $\Gamma$ consists of a variable $x_i$ and a
type $A_i$ containing only the variables $x_1,\dots,x_{i-1}$ freely, signifying
that each variable $x_i$ is assumed to have type $A_i$ in the context $\Gamma$. 
(We discuss the importance of the context in more depth in
\autoref{syntax-more-formally}.)

Thus, the typing judgment
\[
  \Gamma \vdash a:A
\]
denotes that $a$ has type $A$ in the context $\Gamma$; in other words, the
variables declared in $\Gamma$ may appear freely in $a$ and $A$. When $\Gamma$
is empty, we may write simply
\[
  \vdash a:A
\]
or
\[
  \emptyctx \vdash a:A
\]
where $\emptyctx$ denotes the empty context. The same applies for the equality
judgment
\[
  \Gamma \vdash a\jdeq b:A
\]

Such judgments are justified only for contexts that are well formed.  We
introduce a third primitive judgment
\[
  \vdash \Gamma \ctx
\]
to express that notion.  Intuitively, this requires that each $A_i$ is a valid
type in the context $x_1:A_1, x_2:A_2,\dots,x_{i-1}:A_{i-1}$.

We write $B[a/x]$ for the substitution of a term $a$ for free occurrences of
the variable $x$ in the term $B$, with possible renaming for avoiding capture
of variables, as discussed in \autoref{sec:function-types}.  We use
$$B[a_1/x_1,\dots,a_n/x_n]$$ as an abbreviation for the repeated substitution
$$B[a_1/x_1]\dots[a_n/x_n].$$

To {\em bind} a variable $x$ to an expression $B$ means to incorporate both of
them into a larger expression or {\em abstraction}, whose purpose is to
annnotate $B$ with information about which variable is available next for
substitution.  Various notations are used for binding, such as $x \mapsto B$,
$\lam x B$, and $x.B$, depending on the situation.  We may write $C[a]$ for the
substitution of a term $a$ for the variable in the abstracted expression, i.e.,
we may define $(x.B)[a]$ to be $B[a/x]$.  As discussed in
\autoref{sec:function-types}, changing the name of a bound variable everywhere
within an expression is considered not to change the expression.  

One may also regard each variable $x_i$ of a sentence
\[
  x_1:A_1, x_2:A_2,\dots,x_n:A_n \vdash a : A
\]
to be bound within the sentence, with its {\em scope} incorporating the
expressions $A_{i+1}$, \dots, $A_n$, $a$, and $A$.

\section{The inference rules, informally}\label{syntax-informally}

\subsection{The Raw Syntax}

The objects and types of our type theory may be written as terms using
the following syntax, which is an extension of $\lambda$-calculus with {\em
  variables} $x, x',\dots$, {\em primitive} constants $c,c',\dots$, {\em
  defined} constants $f,f',\dots$, and term forming operations
\[
  t \production x \mid \lam{x} t \mid t(t') \mid c \mid f
\]

The notation used here means that a term $t$ is either a variable $x$, or it
has the form $\lam{x} t$ where $x$ is a variable and $t$ is a term, or it has
the form $t(t')$ where $t$ and $t'$ are terms, or it is a primitive constant
$c$, or it is a defined constant $f$.  The syntactic markers '$\lambda$', '(',
')', and '.' are punctuation for guiding the human eye.

We use $t(t_1,\dots,t_n)$ as an abbreviation for the repeated application
$t(t_1)(t_2)\dots (t_n)$.  We may also use {\em infix} notation, writing $t_1\;
\star\; t_2$ for $\star(t_1,t_2)$ when $\star$ is a primitive or defined
constant.

Each defined constant will have zero, one or more {\em defining equations}.
There will be two kinds of defined constant.  An {\em explicit} defined
constant $f$ will have a single defining equation
  \[ f(x_1,\dots,x_n)\defeq t,\]
where $t$ does not involve $f$.  

As an example, we introduce the explicit defined constant $\circ$, for
composition of functions, with defining equation
  \[ \circ (x,y)(z) \defeq x(y(z)),\]
and we use infix notation $x\circ y$ for $\circ(x,y)$.

The second kind of defined constant will be used in connection with a form of type having some primitive constants that are used to introduce elements into types of that form.  With each such primitive constant $c$ there will be a defining equation of the form
\[
  f(x_1,\dots,x_n,c(y_1,\dots,y_m)) \defeq t,
\]
where now $f$ may occur in $t$, but now only in such a way that, in the context
where $f$ is introduced it will be a totally defined typed function.  The
paradigm examples of such defined functions are the functions defined by
primitive recursion on the natural numbers.  We may call this kind of
definition of a function a {\em total recursive definition}.  In computer
science and logic this kind of definition of a function on a recursive data
type has been called a {\em definition by structural recursion}.

We introduce the notion of {\em convertibility} $t \conv t'$ between terms $t$
and $t'$ as the equivalence relation generated by all instances of the
elementary reduction
\[
  (\lam{x} t)(u) \conv t[u/x].
\]
and by all instances of defining equations, such as the one for $x \circ y$
presented above, and by those to be introduced below.

The equality judgments $t \jdeq u : A$ are derived by the following single rule.
\begin{itemize}
\item if $t:A$, $u:A$, and $t \conv u$, then $t \jdeq u : A$
\end{itemize}
Equality is an equivalence relation.

%% \begin{itemize}
%% \item if $t:A$, then $t \jdeq t : A$
%% \item if $t \jdeq u : A$, then $u \jdeq t : A$
%% \item if $t \jdeq u : A$ and $u \jdeq v : A$, then $t \jdeq v : A$
%% \end{itemize}

The following conversion rule allows us to replace a type by one equal to it in
a typing judgment.
\begin{itemize}
\item if $a:A$ and $A \jdeq B$ then $a:B$
\end{itemize}

\subsection{Type Universes}

We introduce a hierarchy of {\em universes} denoted by primitive constants
$\UU_n$, for each $n=0,1,\ldots$.  They satisfy the following rules.  The first
two say that the universes form a sequential hierarchy of types, and the third expresses
the idea that an object of a universe can serve as a type and stand to the
right of a colon in judgments.

\begin{itemize}
\item $\UU_m : \UU_n$ for $m < n$
\item if $A:\UU_m$ and $m \le n$, then $A:\UU_n$.
\item if $\Gamma \vdash A : \UU_n$, and $x$ is a new variable, then $\vdash \Gamma, x:A \ctx$
\end{itemize}

In the body of the book, an equality judgment $A \jdeq B : \UU_n$ between types
$A$ and $B$ is usually abbreviated to $A \jdeq B$.  This is an instance of
typical ambiguity, and the choice of $n$ doesn't affect the validity of the judgment.

\subsection{Dependent function types (\texorpdfstring{$\Pi$}{Π}-types)}

We introduce a primitive constant $c_\Pi$.  An expression of the form
$c_\Pi(A,\lam{a} B)$ will be written as $\tprd{a:A}B$.  Judgments concerning
such expressions and expressions of the form $\lam{x} b$ are introduced by the following rules.

\begin{itemize}
\item if $\Gamma \vdash A:\UU_n$ and $\Gamma,a:A \vdash B:\UU_n$, then $\Gamma \vdash \tprd{a:A}B : \UU_n$
\item if $\Gamma, a:A \vdash b:B$ then $\Gamma \vdash (\lam{a} b) : (\tprd{a:A} B)$
\item if $g:\tprd{a:A} B$ and $t:A$ then $g(t):B[t/a]$
\end{itemize}

If $a$ does not occur freely in $B$, we abbreviate $\tprd{a:A} B$ as $A
\rightarrow B$ and derive the following rule.

\begin{itemize}
\item if $g:A \rightarrow B$ and $t:A$ then $g(t):B$
\end{itemize}

\subsection{Natural numbers}

The type of natural numbers is obtained by introducing primitive constants
$\N$, $0$, and $\suc$ with the following rules.
\begin{itemize}
  \item $\N : \UU_0$,
  \item $0:\N$,
  \item $\suc:\N\rightarrow \N$.
\end{itemize}

Furthermore, we can define functions by primitive recursion.  If we have
$C : \N \rightarrow \UU_k $ we can introduce a defined constant $f:\tprd{n:\N}C(n)$ whenever we have
  \begin{align*}
    d & : C(0) \\
    e & : \tprd{x:\N}(C(x)\rightarrow C(\suc (x)))
  \end{align*}
with the defining equations
  \begin{align*}
    f(0) & \defeq d \\
    f(\suc (x)) & \defeq e(x,f(x))
  \end{align*}
 
As usual $C,d,e$ may have been obtained in an implicit context $\Gamma$ in which variables $x_1,\ldots,x_n$ are declared.  Then there will be the extra implicit parameters $x_1,\ldots,x_n$, so that the fully explicit primitive recursion schema is
  \begin{align*}
    f(x_1,\dots,x_n,0) & \defeq d(x_1,\dots,x_n) \\
    f(x_1,\dots,x_n,\suc (x)) & \defeq e(x_1,\dots,x_n,x,f(x_1,\dots,x_n,x))
  \end{align*}

\subsection{The finite types}

We introduce primitive constants $\ttt$, $\bfalse$, $\btrue$, $\emptyt$,
$\unit$, $\bool$ satisfying the following rules.

\begin{itemize}
\item $\emptyt : \UU_0$, $\unit : \UU_0$, $\bool : \UU_0$,
\item $\ttt:\unit$, $\bfalse:\bool$, $\btrue:\bool$.
\end{itemize}

Given $C : \emptyt \rightarrow \UU_n$ we can introduce a defined constant $f:\tprd{x:\emptyt} C(x)$, with no defining equations.

Given $C : \unit \rightarrow \UU_n$ and $c : C(\ttt)$ we can introduce a defined constant $f:\tprd{x:\unit} C(x)$, with defining equation $f(\ttt) \defeq c$.

Given $C : \bool \rightarrow \UU_n$, $c : C(\bfalse)$, and $c' : C(\btrue)$, we can introduce a defined constant $f:\tprd{x:\bool} C(x)$, with defining equations
$f(\bfalse)\defeq c$ and $f(\btrue)\defeq c'$.

\subsection{Dependent pair types (\texorpdfstring{$\Sigma$}{Σ}-types)}

We introduce primitive constants $c_\Sigma$ and $c_{\mathsf{pair}}$.  An
expression of the form $c_\Sigma(A,\lam{a} B)$ will be written as $\sm{a:A}B$,
and an expression of the form $c_{\mathsf{pair}}(a,b)$ will be written as $\tup
a b$.  We write $A\times B$ instead of $\sm{x:A} B$ if $x$ is not free in $B$.

Judgments concerning such expressions are introduced by the following
rules.

\begin{itemize}
\item if $A:\UU_n$ and $B: A \rightarrow \UU_n$, then $\sm{a:A}B(a) : \UU_n$
\item if, in addition, $x:A$ and $y:B(x)$, then $\tup x y:\sm{a:A}B(a)$
\end{itemize}

If we have $A$ and $B$ as above, $C : \sm{a:A}B(a) \rightarrow \UU_m$, and
\[
  d:\tprd{x:A}{y:B(x)} C(\tup x y)
\]
we can introduce a defined constant 
\[
  f:\tprd{p:\sm{a:A}B(a)} C(p)
\]
with the defining equation
\[
  f(\tup x y)\defeq d(x,y).
\]

\subsection{Coproduct types}
We introduce primitive constants $c_+$, $c_\inlsym$, and $c_\inrsym$.
We will write $A+B$ instead of $c_+(A,B)$, $\inl(a)$ instead of
$c_\inlsym(a)$, and $\inr(a)$ instead of $c_\inrsym(a)$.

\begin{itemize}
\item if $A,B : \UU_n$ then $A + B : \UU_n$
\item moreover, $\inl: A \rightarrow A+B$ and $\inr: B \rightarrow A+B$
\end{itemize}

If we have $A$ and $B$ as above, $C : A+B \rightarrow \UU_m$, $c:\tprd{a:A} C(\inl(a))$, and $c':\tprd{b:B} C(\inr(b))$,
then we can introduce a defined constant $f:\tprd{x:A+B}C(x)$ with the defining equations
\begin{align*}
  f(\inl(a)) & \defeq c(a) \\
  f(\inr(b)) & \defeq c'(b)
\end{align*}

\subsection{$W$-Types}

For $W$-types we introduce primitive constants $c_\wtypesym$ and $c_\suppsym$.
An expression of the form $c_\wtypesym(A,\lam{a} B)$ will be written as
$\wtype{a:A}B$, and an expression of the form $c_\suppsym(x,u)$ will be written
as $\supp(x,u)$

\begin{itemize}
\item if $A:\UU_n$ and $B: A \rightarrow \UU_n$, then $\wtype{a:A}B : \UU_n$
\item if moreover, $a:A$ and $u:B(a)\rightarrow \wtype{a:A}B$ then $\supp(a,u):\wtype{a:A}B$.
\end{itemize}
 
Here also we can define functions by total recursion.  If we have $A$ and $B$
as above and $C : \wtype{a:A}B \rightarrow \UU_m$, then we can introduce a defined constant
$f:\tprd{z:\wtype{a:A}B} C(z)$ whenever we have
\[
  d:\tprd{x:A}{u:B(x) \rightarrow \wtype{a:A}B}((\tprd{y:B}C(u(y))) \rightarrow C(\supp(x,u))
\]
with the defining equation
\[
  f(\supp(x,u)) \defeq d(x,u,f\circ u)
\]

\subsection{Identity types}

We introduce primitive constants $c_\idsym$ and $c_\reflsym$.  We will write
$\id[A] a b$ for $c_\idsym(A,a,b)$ and $\refl a$ for $c_\reflsym(A,a)$, when
$a:A$ is understood.

\begin{itemize}
\item if $A : \UU_n$, $a:A$, and $b:A$ then $\id[A] a b : \UU_n$
\item if, moreover, $a:A$, then $\refl a :\id[A] a a $.
\end{itemize}

If $\Gamma, y:A, z:\id[A] a y \vdash C : \UU_m$ and $d:C[a/y,\refl{a}/z]$ then we can introduce a defined constant 
\[
  f:\tprd{y:A}{z:\id[A] a y} C
\]
with defining equation
\[
  f(a,\refl{a})\defeq d.
\]

\subsection{Algorithmic and semantic issues}

\message{Thierry and Peter can now start re-writing this section.  The
  following text is old.}

\subsubsection*{Conversion and reduction}

Together with $\beta$-conversion
\[
  (\lam{x} t)(u) \defeq t[u/x]
\]
and thinking of $\defeq$ as a rewriting rule (unfolding definitions),
this forms a rewriting system which has the confluence (or Church-Rosser) property: we can
define $t \conv u$ to mean that $t$ and $u$ can be reduced to the same term by using
$\beta$-reduction and recursion.


\subsubsection*{Some Syntactical Properties of the Type Theory}
 This system has the following syntactical properties.

\begin{thm}\label{red}
If $A : \UU$ and $A \conv A'$ then $A' : \UU$.
If $t:A$ and $t \conv t'$ then $t':A$.
\end{thm}

\begin{thm}\label{SN}
 If $A : \UU$ then $A$ is strongly normalizable.
If $t:A$ then $A$ and $t$ are strongly normalizable. % note: ``strongly normalizable'' is undefined
\end{thm}

We say that a term is {\em in normal form} if it cannot be reduced.  A closed
normal type has to be a primitive type, i.e., to be of the form $c(\vec{v})$
for some primitive constant $c$ (where $\vec{v}$ may be omitted if empty, for
instance, as with $\N$).  More generally we have the following explicit
description of terms in normal form.

\begin{lem}\label{normal}
The terms in normal form can be described by the following syntax
\begin{align*}
 v & \production  k \mid \lam{x} v \mid c(\vec{v}) \mid f(\vec{v}) \\
 k &\production x \mid k(v) \mid f(\vec{v})(k)
\end{align*}
where $f(\vec{v})$ represents a partial application of the defined function $f$.
In particular, a type in normal form is of the form $k$ or $c(\vec{v})$.
\end{lem}

\begin{thm}
If $A$ is in normal form then the 
judgment $A : \UU$ is decidable. If $A : \UU$ and $t$ is in normal form then the judgment
$t:A$ is decidable.
\end{thm}


 A corollary is the {\em consistency} property: there is no proof of $\emptyt$ in the empty
context. Indeed if we have $t:\emptyt$ then by Theorems \ref{red} and \ref{SN} the term $t$ will reduce
to a term in normal form $t'$ that satisfies $t':\emptyt$, but this is not possible by a 
purely combinatorial argument using Lemma \ref{normal}. Similarly, we have the following
{\em canonicity} property: if $t:N$ in the empty context, then $t$ has to reduce to a
normal form $\suc^k(0)$ for some numeral $k$. Finally, it means that, if we restrict to terms
in normal form, the typing relation is {\em decidable}, and identifying type-checking with
{\em proof-checking}, we can indeed ``recognize a proof of an assertion when we see one''.

\egroup

\section{The inference rules, more formally}\label{syntax-more-formally}

\bgroup % restrict the following macros to this section

%% Basic syntax of type theory:
% judgements
\renewcommand{\G}{\Gamma}
\newcommand{\ctx}{\ensuremath{\mathsf{ctx}}}
\newcommand{\emptyctx}{\cdot}
\newcommand{\wfctx}[1]{\vdash #1\ \ctx}
\newcommand{\oftp}[3]{#1 \vdash #2 : #3}
\newcommand{\jdeqtp}[4]{#1 \vdash #2 \jdeq #3 : #4}
\newcommand{\judg}[2]{#1 \vdash #2}
\newcommand{\tmtp}[2]{#1 \mathord{:} #2}
% rules
\newcommand{\form}{\textsc{form}}
\newcommand{\intro}{\textsc{intro}}
\newcommand{\elim}{\textsc{elim}}
\newcommand{\comp}{\textsc{comp}}
\newcommand{\Weak}{\mathsf{Wkg}}
\newcommand{\Vble}{\mathsf{Vble}}
\newcommand{\Exch}{\mathsf{Exch}}
\newcommand{\Subst}{\mathsf{Subst}}

\let\syn\mathsf

\textbf{TODO:} 
\begin{enumerate}
\item explain $\to$ vs $\vdash$ for type families (in chapter 1)
\item explain where binding occurs
\item two ways of handling univalence
\item emphasize f/i/e/c pattern, and congruence
\item admissibility of structural rules; don't grow context down
\item explain the diff wrt MLTT (new axioms, choice of presentation)
\item explain arguments with binders in elimination rules
\item contexts: note that unlike chapter 1, all type formers are independent
\end{enumerate}

%Alternatively, we may regard $B$ as an {\em abstraction}, which signifies that
%a variable, $x$, say, is bound within $B$ and can be renamed without changing
%the identity of $B$.  In that case, we'll let $B[a]$ denote the result of
%substituting $a$ for the (outermost) variable bound in $B$ and removing the
%abstraction that recorded its presence and its name.  

Our presentation of the structural rules is based largely on
\cite{hofmann:syntax-and-semantics}, which also includes a full construction of
the syntax.  
%Our selection of logical rules, and in particular our treatment of
%the universe, follows \cite{martin-lof:bibliopolis}.

We take as basic the judgment forms
\begin{mathpar}
\wfctx\G
\and
\oftp\G{a}{A}
\and
\jdeqtp\G{a}{a'}{A}
\end{mathpar}

\subsection{Contexts}

\begin{mathpar}
  \inferrule*[right=\ctx-\textsc{emp}]
  {\ }
  {\wfctx\emptyctx}
\and
  \inferrule*[right=\ctx-\textsc{ext}]
  {\wfctx\G \\ \oftp\G{A}{\UU_i}}
  {\wfctx{(\G,\tmtp xA)}}
\end{mathpar}

\subsection{Structural Rules}

The structural rules of the type theory are (where $\mathcal{J}$ may be any the
conclusion of any of the judgment forms):

\begin{mathpar}
  \inferrule*[right=$\Vble$]
  {\wfctx{(\G,\tmtp xA,\Delta)}}
  {\oftp{\G,\tmtp xA,\Delta}{x}{A}}
\and
  \inferrule*[right=$\Subst$]
  {\oftp\G{a}{A} \\ \judg{\G,\tmtp xA,\Delta}{\mathcal{J}}}
  {\judg{\G,\Delta[a/x]}{\mathcal{J}[a/x]}}
\and
  \inferrule*[right=$\Weak$]
  {\oftp\G{A}{\UU_i} \\ \judg{\G,\Delta}{\mathcal{J}}}
  {\judg{\G,\tmtp xA,\Delta}{\mathcal{J}}}
\end{mathpar}

Definitional equality (also known as syntactic or judgmental equality):
\begin{mathparpagebreakable}
  \inferrule*{\oftp\G{a}{A}}{\jdeqtp\G{a}{a}{A}}
\and
  \inferrule*{\jdeqtp\G{a}{b}{A}}{\jdeqtp\G{b}{a}{A}}
\and
  \inferrule*{\jdeqtp\G{a}{b}{A} \\ \jdeqtp\G{b}{c}{A}}{\jdeqtp\G{a}{c}{A}}
\and
  \inferrule*{\oftp\G{a}{A} \\ \jdeqtp\G{A}{B}{\UU_i}}{\oftp\G{a}{B}}
\and
  \inferrule*{\jdeqtp\G{a}{b}{A} \\ \jdeqtp\G{A}{B}{\UU_i}}{\jdeqtp\G{a}{b}{B}}
\end{mathparpagebreakable}

Additionally, in the logical rules below, we assume rules stating that each constructor preserves definitional equality in each of its arguments; for instance, along with the $\Pi$-\intro\ rule, we assume the rule
\[
  \inferrule*[right=$\Pi$-\intro-eq]
  {\jdeqtp\G{A}{A'}{\UU_i} \\
   \jdeqtp{\G,\tmtp xA}{B}{B'}{\UU_i} \\
   \jdeqtp{\G,\tmtp xA}{b}{b'}{B}}
  {\jdeqtp\G{\lam{x:A} b}{\lam{x:A'} b'}{\tprd{x:A} B}}
\]

\subsection{Universes and families}

In the rules below, $i$ is a natural number.

\begin{mathpar}
\inferrule*[right=\UU-\textsc{intro}]
  {\ }
  {\oftp\G{\UU_i}{\UU_{i+1}}}
\and
\inferrule*[right=\UU-\textsc{cumul}]
  {\oftp\G{A}{\UU_i}}
  {\oftp\G{A}{\UU_{i+1}}}
\end{mathpar}

\subsection{Dependent function types (\texorpdfstring{$\Pi$}{Π}-types)}

\begin{mathparpagebreakable}
  \inferrule*[right=$\Pi$-\form]
  {\oftp{\G,\tmtp xA}{B}{\UU_i}}
  {\oftp\G{\tprd{x:A}B}{\UU_i}}
\and
  \inferrule*[right=$\Pi$-\intro]
  {\oftp{\G,\tmtp xA}{B}{\UU_i} \\ \oftp{\G,\tmtp xA}{b}{B}}
  {\oftp\G{\lam{x:A} b}{\tprd{x:A} B}}
\and
  \inferrule*[right=$\Pi$-\elim]
  {\oftp\G{f}{\tprd{x:A} B} \\ \oftp\G{a}{A}}
  {\oftp\G{f(a)}{B[a/x]}}
\and
  \inferrule*[right=$\Pi$-\comp]
  {\oftp{\G,\tmtp xA}{B}{\UU_i} \\ \oftp{\G,\tmtp xA}{b}{B} \\ \oftp\G{a}{A}}
  {\jdeqtp\G{(\lam{x:A} b)(a)}{b[a/x]}{B[a/x]}}
\end{mathparpagebreakable}

As a special case of this, when $B$ does not depend on $x$, we obtain the
ordinary function type $A\to B := \tprd{x:A} B$. \\

\subsection{Dependent pair types (\texorpdfstring{$\Sigma$}{Σ}-types)}

\begin{mathparpagebreakable}
  \inferrule*[right=$\Sigma$-\form]
  {\oftp{\Gamma,\tmtp xA}{B}{\UU_i}}
  {\oftp\G{\tsm{x:A} B}{\UU_i}}
\and
  \inferrule*[right=$\Sigma$-\intro]
  {\oftp{\G,\tmtp xA}{B}{\UU_i} \\
   \oftp\G{a}{A} \\ \oftp\G{b}{B[a/x]}}
  {\oftp\G{\tup ab}{\tsm{x:A} B}}
\and
  \inferrule*[right=$\Sigma$-\elim]
  {\oftp{\G,\tmtp y{\tsm{x:A} B}}{C}{\UU_i} \\
   \oftp{\G,\tmtp aA,\tmtp b{B[a/x]}}{g}{C[\tup ab/y]} \\
   \oftp\G{p}{\tsm{x:A} B}}
  {\oftp\G{\ind{\tsm{x:A} B}(y.C,a.b.g,p)}{C[\fst(p)/x,\snd(p)/y]}}
\and
  \inferrule*[right=$\Sigma$-\comp]
  {\oftp{\G,\tmtp xA}{B}{\UU_i} \\
   \oftp\G{a'}{A} \\ \oftp\G{b'}{B[a'/x]} \\
   \oftp{\G,\tmtp y{\tsm{x:A} B}}{C}{\UU_i} \\
   \oftp{\G,\tmtp aA,\tmtp b{B[a/x]}}{g}{C[\tup ab/y]}}
  {\oftp\G{\ind{\tsm{x:A} B}(y.C,a.b.g,\tup{a'}{b'})}{C[a'/a,b'/b]}}
\end{mathparpagebreakable}

Again, the special case where $B$ does not depend on $x$ is of particular
interest: this gives the cartesian product $A \times B := \tsm{x:A} B$. \\

\subsection{Identity types}

\begin{mathparpagebreakable}
  \inferrule*[right=$\idsym$-\form]
  {\oftp\G{A}{\UU_i} \\ \oftp\G{a}{A} \\ \oftp\G{b}{A}}
  {\oftp\G{\id[A]{a}{b}}{\UU_i}}
\and
  \inferrule*[right=$\idsym$-\intro]
  {\oftp\G{A}{\UU_i} \\ \oftp\G{a}{A}}
  {\oftp\G{\refl a}{\id[A]aa}}
\and
  \inferrule*[right=$\idsym$-\elim]
  {\oftp{\G,\tmtp xA,\tmtp yA,\tmtp p{\id[A]xy}}{C}{\UU_i} \\
   \oftp{\G,\tmtp zA}{c}{C[z/x,z/y,\refl z/p]} \\
   \oftp\G{a}{A} \\ \oftp\G{b}{A} \\ \oftp\G{p'}{\id[A]ab}}
  {\oftp\G{\ind{\idsym_A}(x.y.p.C,z.c,a,b,p)}{C[a/x,b/y,p'/p]}}
\and
  \inferrule*[right=$\idsym$-\comp]
  {\oftp{\G,\tmtp xA,\tmtp yA,\tmtp p{\id[A]xy}}{C}{\UU_i} \\
   \oftp{\G,\tmtp zA}{c}{C[z/x,z/y,\refl z/p]} \\
   \oftp\G{a}{A}}
  {\jdeqtp\G{\ind{\idsym_A}(x.y.p.C,z.c,a,a,\refl a)}{c[a/z]}{C[a/x,a/y,\refl a/p]}}
\end{mathparpagebreakable}

%\subsection{$\wtypesym$-types}
%\textbf{TODO}
%
%\begin{mathparpagebreakable}
%  \inferrule*[right=$\wtypesym$-\form]{\Gamma,\ x \oftype A \vdash B(x)\UU_i}{\oftp\G \wtypesym_{x \oftype A} B(x)\UU_i}
%\and
%  \inferrule*[right=$\wtypesym$-\intro]{\Gamma,\ x \oftype A \vdash
%  B(x)\UU_i}{\Gamma, \ x \oftype A, \ y \oftype [B(x), \wtypesym_{u \oftype A}
%  B(u)] \vdash \synsup(x, y) : \wtypesym_{u \oftype A} B(u)}
%\\
%  \mathclap{\inferrule*[right=$\wtypesym$-\elim]
%    {\Gamma, \ w \oftype \wtypesym_{x \oftype A} B(x) \vdash C(w) \UU_i \\
%     \Gamma,\ x \oftype A,\ y \oftype [B(x), \wtypesym_{u \oftype A} B(u)],\ z \oftype \Pi_{u \oftype B(x)} C(\mathsf{app}(y, u)) \hspace{1.55cm} \\
%     \hspace{6cm} \vdash d(x, y, z) : C(\synsup(x, y))}
%  {\Gamma,\ w \oftype \wtypesym_{x \oftype A} B(x) \vdash \textsf{wrec}_{d} (w) : C(w)}}
%\\
%  \mathclap{\inferrule*[right=$\wtypesym$-\comp]
%  {\Gamma, \ w \oftype \wtypesym_{x \oftype A} B(x) \vdash C(w) \UU_i \\
%   \Gamma,\ x \oftype A,\ y \oftype [B(x), \wtypesym_{u \oftype A} B(u)],\ z \oftype \Pi_{u \oftype B(x)} C(\mathsf{app}(y, u)) \hspace{2cm} \\
%    \hspace{6.45cm} \vdash d(x, y, z) : C(\synsup(x, y))}
%  {\Gamma, \ x \oftype A, \ y \oftype [B(x), \wtypesym_{u \oftype A} B(u)]
%  \vdash \textsf{wrec}_{d} (\synsup(x, y)) \hspace{2.8cm} \\
%    \hspace{2.4cm} = d(x, y, \lambda u \oftype B(x). \mathsf{wrec}_d(\mathsf{app}(y, u))): C(\synsup(x, y))}}
%\end{mathparpagebreakable}

\subsection{The empty type $\emptyt$}

\begin{mathparpagebreakable}
  \inferrule*[right=$\emptyt$-\form]
  {\ }
  {\oftp\G\emptyt{\UU_i}}
\and
  \text{(No $\emptyt$-\intro.)}
\and
  \inferrule*[right=$\emptyt$-\elim]
  {\oftp{\G,\tmtp x\emptyt}{C}{\UU_i} \\ \oftp\G{z}{\emptyt}}
  {\oftp\G{\ind{\emptyt}(x.C,z)}{C[z/x]}}
\and
  \text{(No $\emptyt$-\comp.)}
\end{mathparpagebreakable}

\subsection{The unit type $\unit$}

\begin{mathparpagebreakable}
  \inferrule*[right=$\unit$-\form]
  {\ }
  {\oftp\G\unit{\UU_i}}
\and
  \inferrule*[right=$\unit$-\intro]
  {\ }
  {\oftp\G{\ttt}{\unit}}
\and
  \inferrule*[right=$\unit$-\elim]
  {\oftp{\G,\tmtp x\unit}{C}{\UU_i} \\
   \oftp{\G,\tmtp x\unit}{c}{C} \\
   \oftp\G{y}{\unit}}
  {\oftp\G{\ind{\unit}(x.C,x.c,y)}{C[y/x]}}
\and
  \inferrule*[right=$\unit$-\comp]
  {\oftp{\G,\tmtp x\unit}{C}{\UU_i} \\
   \oftp{\G,\tmtp x\unit}{c}{C}}
  {\jdeqtp\G{\ind{\unit}(x.C,x.c,\ttt)}{c[\ttt/x]}{C[\ttt/x]}}
\end{mathparpagebreakable}

\textbf{TODO:} natural numbers

\subsection{Coproduct types}

\begin{mathparpagebreakable}
  \inferrule*[right=$+$-\form]
  {\oftp\G{A}{\UU_i} \\ \oftp\G{B}{\UU_i}}
  {\oftp\G{A+B}{\UU_i}}
\and
  \inferrule*[right=$+$-\intro${}_1$]
  {\oftp\G{A}{\UU_i} \\ \oftp\G{B}{\UU_i} \\ \oftp\G{a}{A}}
  {\oftp\G{\inl(a)}{A+B}}
\and
  \inferrule*[right=$+$-\intro${}_2$]
  {\oftp\G{A}{\UU_i} \\ \oftp\G{B}{\UU_i} \\ \oftp\G{b}{B}}
  {\oftp\G{\inr(b)}{A+B}}
\and
  \inferrule*[right=$+$-\elim]
  {\oftp{\G,\tmtp x{(A+B)}}{C}{\UU_i} \\
   \oftp{\G,\tmtp aA}{c}{C[\inl(a)/x]} \\
   \oftp{\G,\tmtp bB}{d}{C[\inr(b)/x]} \\
   \oftp\G{z}{A+B}}
  {\oftp\G{\ind{A+B}(x.C,a.c,b.d,z)}{C[z/x]}}
\and
  \inferrule*[right=$+$-\comp${}_1$]
  {\oftp{\G,\tmtp x{(A+B)}}{C}{\UU_i} \\
   \oftp{\G,\tmtp aA}{c}{C[\inl(a)/x]} \\
   \oftp{\G,\tmtp bB}{d}{C[\inr(b)/x]} \\
   \oftp\G{a'}{A}}
  {\jdeqtp\G{\ind{A+B}(x.C,a.c,b.d,\inl(a'))}{c[a'/a]}{C[\inl(a')/x]}}
\and
  \inferrule*[right=$+$-\comp${}_2$]
  {\oftp{\G,\tmtp x{(A+B)}}{C}{\UU_i} \\
   \oftp{\G,\tmtp aA}{c}{C[\inl(a)/x]} \\
   \oftp{\G,\tmtp bB}{d}{C[\inr(b)/x]} \\
   \oftp\G{b'}{B}}
  {\jdeqtp\G{\ind{A+B}(x.C,a.c,b.d,\inr(b'))}{d[b'/b]}{C[\inr(b')/x]}}
\end{mathparpagebreakable}

\subsection{Further rules} \label{subsec:optional-rules}

In this section we present the \emph{$\eta$-rule} for $\Pi$-types
%and the \emph{functional extensionality} rule(s). Our formulation of the latter
%is taken from \cite{garner:depprod}; see also \cite{hofmann:thesis}.

\begin{mathparpagebreakable}
  \inferrule*[right=$\Pi$-$\eta$]
  {\oftp\G{f}{\tprd{x:A} B}}
  {\jdeqtp\G{f}{(\lam{x:A}f(x))}{\tprd{x:A} B}}
\end{mathparpagebreakable}

\textbf{TODO:} function extensionality rules

\egroup

\section{Notes}\label{subsec:general-remarks}

  %This presentation is strongly inspired by two  Martin-L\"of 1972 and 1973.

  The system of rules with introduction (primitive constants) and elimination
  and computation rules (defined constant) is inspired by Gentzen natural
  deduction. The possibility of strengthening the elimination rule for
  existential quantification was indicated in \cite{Howard-1969}. The
  strengthening of the axioms for disjunction appears in \cite{Martin-Lof-1972},
  and for absurdity elimination and identity type in \cite{Martin-Lof-1973}. The
  $W$-types were introduced in \cite{Martin-Lof-1979}. They generalize a notion
  of trees introduced by \cite{Tait-1968}.
  %inspired from unpublished work of Spector.

  The generalized form of primitive recursion for natural numbers and ordinals
  appear in \cite{Hilbert-1925}.  This motivated G\"odel's system $T$,
  \cite{Goedel-T-1958}, which was analyzed by \cite{Tait-1966}, who used,
  following \cite{Goedel-1958}, the terminology ``definitional equality'' for
  conversion: two terms are {\em definitionally equal} if they reduce to a
  common term by means of a sequence of applications of the reduction
  rules. This terminology was also used by de Bruijn \cite{deBruijn-1973} in his
  presentation of {\em Automath}.

  Streicher \cite[Theorem 4.13]{Streicher-1991}, explains how to give the
  semantics in contextual category of terms in normal form using a simple syntax
  similar to the one we have presented.

%%% Local Variables: 
%%% mode: latex
%%% TeX-master: "main"
%%% End: 



% Joke
\nocite{Angiuli13}
\nocite{BauerAcceptanceVideo}

%%%% Bibliography
\bibliographystyle{halpha}
\phantomsection % black magic to get TOC to point to correct page
\addcontentsline{toc}{part}{\bibname}
\markboth{}{\textsc{Bibliography}}
{\renewcommand{\markboth}[2]{} % Prevent bibliography from resetting the header to something silly
%\OPTbibliographyfont
\bibliography{references}}

%\cleartooddpage[\thispagestyle{empty}]

%%%% Index of symbols

%\phantomsection % black magic to get TOC to point to correct page
\markboth{}{\textsc{Index of symbols}}
\renewcommand{\markboth}[2]{}
\addcontentsline{toc}{part}{Index of symbols}
\chapter*{Index of symbols}

% Shorthand for \pageref, we have lots of these.
\newcommand{\pg}[1]{p.~\pageref{#1}}

% The entries in this table are sorted "alphabetically" whatever that means.

\begin{supertabular}{p{0.2\textwidth}@{\hspace*{2.5em}}p{0.65\textwidth}}
  $x \defeq a$ & definition, \pg{defn:defeq}
  \\
  $a \jdeq b$  & judgmental equality, \pg{defn:judgmental-equality}
  \\
  $a =_A b$, $a = b$  & identity type, \pg{sec:identity-types}
  \\
  $\idtypevar{A}(a,b)$ & identity type, \pg{sec:identity-types}
  \\
  $\exis{x:A} B(x)$ & logical notation for mere existential, \pg{defn:logical-notation}
  \\
  $\fall{x:A} B(x)$ & logical notation for dependent function type, \pg{defn:logical-notation}
  \\
  $\im(f)$ & image of map $f$, \pg{defn:modal-image}
  \\
  $\im_n(f)$ & $n$-image of map $f$, \pg{defn:modal-image}
  \\
  $\ind{\emptyt}$ & induction for ${\emptyt}$, \pg{defn:induction-emptyt},
  \\
  $\ind{\unit}$ & induction for ${\unit}$, \pg{defn:induction-unit},
  \\
  $\ind{\bool}$ & induction for ${\bool}$, \pg{defn:induction-bool},
  \\
  $\ind{\nat}$ & induction for ${\nat}$, \pg{defn:induction-nat}, and
  \\
  $\ind{=_A}^{ML}$ & Martin-L\"of induction for $=_A$, \pg{defn:induction-ML-id},
  \\
  $\ind{=_A}^{PM}$ & Paulin-Mohring induction for $=_A$, \pg{defn:induction-PM-id},
  \\
  $\ind{A \times B}$ & induction for ${A \times B}$, \pg{defn:induction-times},
  \\
  $\ind{\sm{x:A} B(x)}$ & induction for ${\sm{x:A} B}$, \pg{defn:induction-sm},
  \\
  $\ind{A + B}$ & induction for ${A + B}$, \pg{defn:induction-plus},
  \\
  $\ind{\wtype{x:A} B(x)}$ & induction for ${\wtype{x:A} B}$, \pg{defn:induction-wtype}
  \\
  $\lam{x} b(x)$ & $\lambda$-abstraction, \pg{eq:lambda-abstraction}
  \\
  $\pairr{a,b}$ & (dependent) pair, \pg{sec:finite-product-types} and \pg{defn:dependent-pair}
  \\
  ${defn:pairpath}$ & constructor for $=_{A \times B}$, \pg{defn:pairpath}
  \\
  $\prd{x:A} B(x)$ & dependent function type, \pg{sec:pi-types}
  \\
  $\proj1(t)$ & the first projection from a pair, \pg{defn:proj} and \pg{defn:dependent-proj1}
  \\
  $\proj2(t)$ & the second projection from a pair, \pg{defn:proj} and \pg{defn:dependent-proj1}
  \\
  $\rec{\emptyt}$ & recursor for ${\emptyt}$, \pg{defn:recursor-emptyt},
  \\
  $\rec{\unit}$ & recursor for ${\unit}$, \pg{defn:recursor-unit},
  \\
  $\rec{\bool}$ & recursor for ${\bool}$, \pg{defn:recursor-bool},
  \\
  $\rec{\nat}$ & recursor for ${\nat}$, \pg{defn:recursor-nat}, and
  \\
  $\rec{A \times B}$ & recursor for ${A \times B}$, \pg{defn:recursor-times},
  \\
  $\rec{\sm{x:A} B(x)}$ & recursor for ${\sm{x:A} B}$, \pg{defn:recursor-sm},
  \\
  $\rec{A + B}$ & recursor for ${A + B}$, \pg{defn:recursor-plus},
  \\
  $\rec{\wtype{x:A} B(x)}$ & recursor for ${\wtype{x:A} B}$, \pg{defn:recursor-wtype}
  \\
  $\setof{x : A | P(x)}$ & subset type, \pg{defn:setof}
  \\
  $\sm{x:A} B(x)$ & dependent pair type, \pg{sec:sigma-types}
  \\
  $\supp(a, f)$ & constructor for $W$-type, \pg{defn:supp}
  \\
  $\wtype{x:A} B(x)$ & $W$-type (inductive type), \pg{sec:w-types}
  \\
  & (To be continued. In \texttt{macros.tex} XXX marks how far we got.)
  \\
  % Keep a trailing \\ at the end or suffer a LaTeX error
\end{supertabular}


%%% Local Variables: 
%%% mode: latex
%%% TeX-master: "main"
%%% End: 


%\cleartooddpage[\thispagestyle{empty}]

%%%%% Index of terms

%% Global cross-references for index
\indexsee{principle}{axiom}
\indexsee{number!real}{real numbers}
\indexsee{abelian group}{group, abelian}
\indexsee{sequence!Cauchy}{Cauchy sequence}
\indexsee{adjunction}{adjoint functor}
\indexsee{higher topos}{$(\infty,1)$-topos}
\indexsee{topos!higher}{$(\infty,1)$-topos}
\indexsee{source!of a function}{domain}
\indexsee{target!of a function}{codomain}
\indexsee{type!truncation of}{truncation}
\indexsee{propositional!truncation}{truncation}
\indexsee{proof-relevant mathematics}{mathematics, proof-relevant}
\indexsee{classical!mathematics}{mathematics, classical}
\indexsee{classical!logic}{logic}
\indexsee{constructive!mathematics}{mathematics, constructive}
\indexsee{constructive!logic}{logic}
\indexsee{intuitionistic logic}{logic}
\indexsee{definition!inductive}{type, inductive}
\indexsee{inductive!definition}{type, inductive}
\indexsee{bounded!totally}{totally bounded}
\indexsee{sum!of numbers}{addition}
\indexsee{continuous map}{function, continuous}
\indexsee{function!continuity of@``continuity'' of}{``continuity''}
\indexsee{function!functoriality of@``functoriality'' of}{``functoriality''}
\indexsee{codes}{encode-decode method}
\indexsee{inequality}{order}
\indexsee{Coq@\Coq}{proof assistant}
\indexsee{Agda@\Agda}{proof assistant}
\indexsee{NuPRL@\NuPRL}{proof assistant}
\indexsee{generator!of an inductive type}{constructor}
\indexsee{groupoid!.infinity-@$\infty$-}{$\infty$-groupoid}
\indexsee{higher groupoid}{$\infty$-groupoid}
\indexsee{hierarchy!of n-types@of $n$-types}{$n$-type}
\indexsee{homotopy!n-type@$n$-type}{$n$-type}
\indexsee{homotopy!theory, classical}{classical homotopy theory}
\indexsee{homotopy!fiber}{fiber}
\indexsee{homotopy!limit}{limit of types}
\indexsee{homotopy!colimit}{colimit of types}
\indexsee{implementation}{proof assistant}
\indexsee{notation, abuse of}{abuse of notation}
\indexsee{language, abuse of}{abuse of language}
\indexsee{operator!induction}{induction principle}
\indexsee{operator!modal}{modality}
\indexsee{commutative!group}{group, abelian}
\indexsee{countable axiom of choice}{axiom of choice, countable}

% tell the index to get itself into the table of contents
\phantomsection % black magic to get TOC to point to correct page
\addcontentsline{toc}{part}{Index}
\markboth{}{\textsc{Index}}
\renewcommand{\markboth}[2]{}
{%\OPTindexfont
%\setlength{\columnsep}{\OPTindexcolumnsep}
%\printindex
}

% The back cover
%\ifOPTcover
\cleardoublepage
\pagestyle{empty}
\cleartoevenpage

%%%%%%%%%%%%%%%%%%%% Back cover %%%%%%%%%%%%%%%%%%%%
\ThisLRCornerWallPaper{0.7}{cover-lores-back}
\pagecolor{covercolor}
\input{blurb.tex}
\else
\fi

%%% Local Variables: 
%%% mode: latex
%%% TeX-master: "hott-online"
%%% End: 


\end{document}

%%% Local Variables: 
%%% mode: latex
%%% TeX-master: "hott-online"
%%% End: 

\input{version.tex}

\usepackage{longtable}

\title{Errata for the HoTT Book, first edition%
%% VERSION MARKER
}

\begin{document}
\maketitle

For the benefit of all readers, the available PDF and printed copies of the book are being updated on a rolling basis with minor corrections and clarifications as we receive them. Every copy has a version marker that can be found on the title page and is of the form "first-edition-XX-gYYYYYYY", where XX is a natural number and YYYYYYY is the git commit hash that uniquely identifies the exact version. Higher values of XX indicate more recent copies.

Below is a list of corrections and clarifications that have been made
%% BEGIN STARTPOINT
so far
%% END STARTPOINT
(except for trivial formatting and spacing changes), along with the version marker in which they were first made.
This list is current as of \today\ and version marker ``\OPTversion''.

While the page numbering may differ between copies with different version markers (and indeed, already differs between the letter/A4 and printed/ebook copies with the same version marker), we promise that the numbering of chapters, sections, theorems, and equations will remain constant, and no new mathematical content will be added, unless and until there is a second edition.

\noindent
\begin{longtable}{llp{10.5cm}}
  \textbf{Location} & \textbf{Fixed in} & \textbf{Change} \\ \hline \endhead
%% BEGIN ERRATA
  %
  % Chapter 1
  %
  \autoref{sec:types-vs-sets}
  & 182-gb29ea2f
  & Change notation $a\jdeq_A b$ to $a\jdeq b : A$, to match that used in \autoref{cha:rules}.
  (Neither are used anywhere else in the book.)\\
  %
  \autoref{sec:types-vs-sets}
  & 154-g42698c2
  & Clarify that algorithmic decidability of judgmental equality is only meta-theoretic.\\
  %
  \autoref{sec:types-vs-sets}
  & 154-gac9b226
  & Mention notation $a=b=c=d$ to mean ``$a=b$ and $b=c$ and $c=d$, hence $a=d$'', possibly including judgmental equalities.\\
  %
  \autoref{sec:universes}
  & 42-g4bc5cc2
  & Cumulativity means some elements do not have unique types, the index $i$ on $\UU_i$ is not an internal natural number, and typical ambiguity must be justified by reinserting indices.\\
  %
  \autoref{sec:universes,sec:pi-types}
  & 42-ga34b313
  & Explain that we can't define $\Fin$ and $\fmax$ yet where we first mention them.\\
  %
  \autoref{sec:pi-types}
  & 165-g0ad2aba
  & Add $\mathsf{swap}$ as another example of a polymorphic function, and discuss the use of subscripts and implicit arguments to dependent functions.\\
  %
  \autoref{rmk:introducing-new-concepts}
  & 80-g8f95fa5
  & In the discussion of formation rules, the dependent function type example should be $\prd{x:A} B(x)$.\\
  %
  \autoref{sec:finite-product-types}
  & 51-g67e86db
  & Better explanation of recursion on product types, why it is justified, and how it relates to the uniqueness principle.\\
  %
  \autoref{sec:sigma-types}
  & 2-gbe277a8
  & In the types of $g$ and $\ind{\sm{x:A}B(x)}$, there is a $\prd{a:A}{b:B(x)}$ in which $x$ should be $a$.\\
  %
  \autoref{sec:sigma-types}
  & 27-gd0bfa0d
  & At two places in the definition of $\ac$, $R(a,\fst(g(x)))$ should be $R(x,\fst(g(x)))$.\\
  %
  \autoref{sec:sigma-types}
  & 125-g7fdadbf
  & When substituting $\lam{x} \fst(g(x))$ for $f$ while verifying that $\ac$ is well-typed, the left side of the judgmental equality should be $\tprd{x:A} R(x,\fst(g(x)))$, not $\tprd{x:A} R(x,\fst(f(x)))$.\\
  %
  \autoref{sec:coproduct-types}
  & 30-g264d934
  & In two displayed equations, $f(\inl(b))$ should be $f(\inr(b))$.\\
  %
  Theorem \ref{thm:allbool-trueorfalse} 
  & 391-g1ce619a
  & This should not be called a ``Theorem'', since we have not yet introduced what that means.
  Instead it should say ``We construct an element of\dots''.\\
  %
  \autoref{sec:type-booleans}
  & 125-g433f87e
  & In the definition of binary products in terms of $\bool$, the definitions of $\fst(p)$ and $\snd(p)$ should be switched to match the order of arguments to $\rec\bool$ and $\ind\bool$.\\
  % 
  \autoref{sec:pat}
  & 111-g1e868fa
  & When translating English to type theory, ``unnamed variables'' are unnamed in English but must be named in type theory.\\
  %
  \autoref{sec:identity-types}
  & 154-g4ef49f7
  & Emphasize that path induction, like all other induction principles, defines a \emph{specified} function.\\
  %
  \autoref{sec:identity-types}
  & 244-gd58529d
  & In proof that path induction implies based path induction, $D(x,y,p)$ should be written $\prd{C : \prd{z:A} (\id[A]{x}{z}) \to \UU} \left( \cdots \right)$ so the type of $C$ matches the premise of based path induction. \\
  %
  \autoref{rmk:the-only-path-is-refl}
  & 563-g3286941
  & The facts that any $(x,y,p): \sm{x,y:A}(\id{x}{y})$ is equal to $(x,x,\refl{x})$, and that any $(y,p):\sm{y:A}(\id[A]{a}{y})$ is equal to $(a,\refl{a})$, can be proven by path induction and based path induction respectively.\\
  %
  \autoref{ex:iterator}
  & 78-gcce4dc0
  & The second defining equation of $\ite$ should have right-hand side $c_s(\ite(C,c_0,c_s,n))$.\\
  %
  \autoref{ex:iterator}
  & 293-g4663bfe
  & The defining equations of the recursor derived from the iterator only hold propositionally, and require the induction principle to prove.\\
  % 
  \autoref{ex:prod-via-bool}
  & 229-ged891f3
  & This exercise requires function extensionality (\autoref{sec:compute-pi}).\\
  %
  \autoref{ex:nat-semiring}
  & 450-g7f38c9a
  & This exercise requires symmetry and transitivity of equality, \autoref{lem:opp,lem:concat}.\\
  %
  \autoref{ex:ackermann}
  & 110-gfe4641b
  & To match the usual Ackermann--P\'eter function, the second displayed equation should be $\ack(\suc(m),0) \jdeq \ack(m,1)$.\\
  %
  % Chapter 2
  %
  \autoref{cha:basics}
  & 239-gaf3d682
  & In the chapter introduction, clarify that topological homotopies between paths must be endpoint-preserving.\\
  %
  \autoref{lem:opp}
  & 166-g37b78ef
  & Add remarks before and after the proof about how a theorem's statement and proof should be interpreted as exhibiting an element of some type.\\
  %
  \autoref{lem:concat}
  & 374-g0bc0908
  & In the penultimate display in the first proof, $d(x,z,q)$ should be simply $d$.\\
  %
  \autoref{sec:equality}
  & 435-gee0b28a
  & In the third paragraph after \autoref{lem:concat}, $p\ct\refl{x}\jdeq p$ should be $p\ct\refl{y}\jdeq p$.\\
  % 
  \autoref{sec:equality}
  & 165-g18642ca
  & Mention that the notation $a=b=c=d$, and its displayed variant, indicate concatenation of paths.\\
  %
  \autoref{sec:equality}
  & 253-gdd47c75
  & \autoref{thm:omg}\ref{item:omg4} justifies writing $p\ct q \ct r$ and so on.\\
  %
  \autoref{thm:EckmannHilton}
  & 253-gdd47c75
  & The induction defining $\alpha\rightwhisker r$ has defining equation $\alpha \rightwhisker \refl{b} \jdeq \opp{\mathsf{ru}_p} \ct \alpha \ct \mathsf{ru}_q$, with $\mathsf{ru}_p$ the right unit law.
  For $\alpha\hct\beta = \alpha\ct\beta$ to be well-typed, we assume $p\jdeq q \jdeq r \jdeq s\jdeq \refl{a}$ and use $\mathsf{ru}_{\refl{a}} = \refl{\refl{a}}$ and its dual.
  Proving $\alpha\hct\beta = \alpha\hct'\beta$ requires induction not only on $\alpha$ and $\beta$ but then on the two remaining 1-paths.
  After the proof, remark that we trust the reader to construct such operations from now on.\\
  %
  \autoref{def:loopspace}
  & 233-gc3fb777
  & The three displays should be $\defeq$'s rather than $=$'s.\\
  %
  \autoref{sec:functors}
  & 336-g8ff8a7f
  & In the type of $\apfunc{f}$ towards the end of the first proof of \autoref{lem:map}, $g(x)$ should be $f(y)$.\\
  %
  \autoref{sec:fibrations}
  & 154-g4ef49f7
  & Emphasize that unlike fibrations in classical homotopy theory, type families come with a \emph{specified} path-lifting function.\\
  %
  \autoref{sec:fibrations}
  & 343-g6efd724
  & The functions \autoref{eq:ap-to-apd} and \autoref{eq:apd-to-ap} are obtained by concatenating with $\transconst Bp{f(x)}$ and its inverse, respectively.\\
  %
  \autoref{cor:hom-fg}
  & 253-gdd47c75
  & Canceling $H(x)$ may be done by whiskering with $\opp{(H(x))}$.\\
  %
  \autoref{sec:compute-cartprod}
  & 74-g9896e32
  & In the type of $\pairpath$ (just after the proof of \autoref{thm:path-prod}), the second factor in the domain should be $\id{\proj{2}(x)}{\proj{2}(y)}$.\\
  %
  \autoref{thm:trans-prod}
  & 349-gc7fd9d8
  & The path is in $A(w)\times B(w)$, not $A(y)\times B(y)$.\\
  %
  \autoref{thm:trans-prod}
  & 76-ga42354c
  & The third displayed judgmental equality in the proof should be $\transfib{B}{p}{\proj{2}x} \jdeq \proj2x$.\\
  %
  \autoref{thm:path-sigma}
  & 507-g8f10eda
  & In the proof, the equation $f(g(\refl{},\refl{}))=\refl{}$ should be $f (g(\refl{w_1},\refl{w_2})) = (\refl{w_1},\refl{w_2})$.\\
  %
  \autoref{sec:compute-pi}
  & 269-g3880fe2
  & The paragraph preceding the definition of $\transfib{\Pi_A(B)}{p}{f}$ (before \autoref{eq:transport-arrow-families}) misstated the (already given) type of $p$.\\
  %
  \autoref{axiom:univalence}
  & 408-geee0345
  & The text prior to the display should read ``For any $A,B:\type$, the function~\eqref{eq:uidtoeqv} is an equivalence; hence we have''\\
  %
  \autoref{thm:paths-respects-equiv}
  & 310-gd5fa240
  & The second half of the proof is more involved than the first.
  It follows abstractly using the 2-out-of-6 property (\autoref{ex:2-out-of-6}), or more concretely by concatenating with $\opp{\alpha_{f(a)}} \ct {\alpha_{f(a)}}$ on each side and then repeatedly using naturality and functoriality.\\
  %
  \autoref{sec:compute-paths}
  & 236-g32be999
  & The second display after the proof of \autoref{thm:paths-respects-equiv} should be $\prd{x:A} (\id[f(x)=g(x)] {\happly(p)(x)}{\happly(q)(x)})$.\\
  %
  \autoref{thm:transport-path2}
  & 364-g3c47534
  & The right-hand side of the displayed equality should be $\opp{(\apdfunc{f}(p))} \ct \apfunc{(\transfibf{B}{p})}(q) \ct \apdfunc{g}(p)$.\\
  %
  \autoref{sec:compute-coprod}
  & 101-g645f763
  & In \autoref{thm:path-coprod} and the preceding paragraph, in the equivalence $\eqv{(\inl(a)=x)}{\code(x)}$, the variable $a$ should be $a_0$. \\
  %
  \autoref{sec:compute-coprod}
  & 370-g114db82
  & In the two displays after the proof of \autoref{thm:path-coprod}, the terms should be $\encode(\inl(a), {\blank})$ and $\encode(\inr(b), {\blank})$.\\
  %
  \autoref{sec:equality-semigroups}
  & 261-g4ccda0a
  & In the first displayed pair of equations, the type of $p_2$ should be $\transfib{\semigroupstrsym}{p_1}{(m,a)} = {(m',a')}$.\\
  %
  \autoref{sec:equality-semigroups}
  & 402-g2297ecb
  & The right hand side of the last displayed equation should be $m'(e(x_1),e(x_2))$.\\
  %
  \autoref{sec:universal-properties}
  & 305-g64685f1
  & In the discussion of universal properties for product types and $\Sigma$-types surrounding \autoref{eq:sigma-lump}, the phrases ``left-to-right'' and ``right-to-left'' should be switched.\\
  %
  \autoref{cha:basics} Notes
  & 379-ga57eab2
  & It should be mentioned that Hofmann and Streicher (1998) proposed an axiom similar to univalence, which is correct (and equivalent to univalence) for a universe of 1-types.\\
  % 
  % Chapter 3
  %
  \autoref{subsec:prop-subsets}
  & 86-g39feab1
  & The definition of subset containment should say $\prd{x:A}(P(x)\rightarrow Q(x))$, not $\fall{x:A}(P(x)\Rightarrow Q(x))$, as the latter notation has not been introduced yet.\\
  %
  \autoref{thm:retract-contr}
  & 95-gce0131f
  & In the proof, $p$ should be $r$ to match the preceding definition of retraction.\\
  %
  % Chapter 4
  %
  \autoref{lem:qinv-autohtpy}
  & 87-g693e9b9
  & At the end of the proof, \autoref{thm:contr-paths} should be cited as the reason why $\sm{g:A\to A} (g = \idfunc[A])$ is contractible.\\
  %
  \autoref{thm:equiv-iso-adj}
  & 275-g8ea9f71
  & In the proof, the path concatenations in the definitions of $\epsilon'$ and $\tau$ were written in reverse order.\\
  %
  \autoref{lem:coh-hprop}
  & 296-ge3dc076
  & In the proof, $\id[\hfib{f}{fx}]{(fgx,\epsilon(fx))}{(x,\refl{fx})}$ should be $\id[\hfib{f}{fx}]{(gfx,\epsilon(fx))}{(x,\refl{fx})}$.\\
  %
  \autoref{thm:equiv-biinv-isequiv}
  & 272-gfd47093
  & At the end of the proof, the equivalence follows from the fact that $\ishae(f)$, not $\iscontr(f)$, is a mere proposition. \\
  %
  \autoref{thm:lequiv-contr-hae}
  & 299-g85b729b
  & In the proof, $\lcoh{f}{g}{\epsilon}$ should be $\rcoh{f}{g}{\epsilon}$, and the final displayed equation should have $\proj{2}$ applied to both occurrences of $P(fx)$.\\
  %
  \autoref{lem:func_retract_to_fiber_retract}
  & 265-g64000fb
  & The path concatenations in the definitions of $\varphi_b$ and $\psi_b$ (and subsequent equations) are reversed, and each $f(a)$ in the next two displayed equations should be $g(a)$.\\
  %
  \autoref{fibwise-fiber-total-fiber-equiv}
  & 275-g84ab032
  & The first equivalence in the proof is not by~\eqref{eq:sigma-lump} but by \autoref{ex:sigma-assoc}.\\
  %
  \autoref{fibwise-fiber-total-fiber-equiv}
  & 202-g775a3f0
  & The last equivalence in the proof is not by~\eqref{eq:path-lump} but by \autoref{thm:omit-contr,thm:contr-paths,ex:sigma-assoc}.\\
  %
  \autoref{thm:nobject-classifier-appetizer}
  & 205-gf9fe386
  & In the proof, $e\cdot \proj1$ should be $\trans{(\ua(e))}{\proj1}$.  Also, explain its computation better.\\
  %
  \autoref{sec:univalence-implies-funext}
  & 114-gaba76c8
  & The point of \autoref{UA-eqv-hom-eqv} is that it follows from univalence without assuming function extensionality separately.\\
  %
  \autoref{contrfamtotalpostcompequiv}
  & 484-g2ce1249
  & In the statement, ``precomposition'' should be ``post-composition''.\\
  %
  \autoref{ex:symmetric-equiv}
  & 358-g9543064
  & The text should be ``Show that for any $A,B:\UU$, the following type is equivalent to $\eqv A B$.  Can you extract from this a definition of a type satisfying the three desiderata of $\isequiv(f)$?''\\
  %
  % Chapter 5
  %
  \autoref{sec:w-types}
  & 125-g433f87e
  & In the definition of $\natw$, use $\bfalse$ for $0$ and $\btrue$ for $\suc$, to match the ordering of $\bfalse$ and $\btrue$ in \autoref{sec:type-booleans}.\\
  %
  \autoref{sec:w-types}
  & 551-g82b74bf
  & The definitions of $\natw$ and $\lst A$ as $\w$-types should be $\wtype{b:\bool} \rec\bool(\bbU,\emptyt,\unit,b)$ and $\wtype{x: \unit + A} \rec{\unit + A}(\bbU,  \emptyt,  \lamu{a:A} \unit, x)$.\\
  %
  \autoref{sec:w-types}
  & 218-g42219cb
  & In the description of the constructor $\supp$, its second argument is more clearly written as $f : B(a) \to \wtype{x:A} B(x)$.\\
  %
  \autoref{sec:w-types}
  & 525-gb1957b8
  & In the computation rule, the recursive call to $\rec{}$ is missing an argument. 
  It should read $\rec{\wtype{x:A} B(x)}(E,e,\supp(a,f)) \jdeq e(a,f,\big(\lamu{b:B(a)} \rec{\wtype{x:A} B(x)}(E,e,f(b))\big))$.\\
  %
  \autoref{sec:w-types}
  & 570-g6ec04c3
  & In the verification that $\dbl$ computes as expected, $e_t$ should be $e_0$ and $e_f$ should be $e_1$.\\
  %
  \autoref{sec:initial-alg}
  & 554-g9b2a34b
  & The definition of the type of $\w$-homomorphisms (just before \autoref{thm:w-hinit}) should read $\whom_{A,B}((C, s_C),(D,s_D)) \defeq \sm{f : C \to D} \prd{a:A}{h:B(a)\to C} \id{f(s_C(a,h))}{s_D(a, f\circ h)}$.\\
  % 
  \autoref{thm:identity-systems}
  & 139-gd5c5d01
  & In the proof of \ref{item:identity-systems4}$\Rightarrow$\ref{item:identity-systems1}, the type of $D'$ should be $(\sm{b:A} R(b)) \to \type$.\\
  %
  % Chapter 6
  %
  \autoref{sec:dependent-paths}
  & 54-gd4a47c2
  & Soon after \autoref{rmk:defid}, the phrase ``An element $b:P(\base)$ in the fiber over the constructor $\base:\nat$'' should say $\base:\Sn^1$.\\
  %
  \autoref{thm:uniqueness-for-functions-on-S1}
  & 423-gf763ae1
  & \autoref{thm:transport-path,thm:dpath-path} are needed to put $q$ in the form required by the induction principle.\\
  %
  \autoref{thm:interval-funext}
  & 417-g4aa6a15
  & Added \autoref{ex:funext-from-interval}: the function constructed in \autoref{thm:interval-funext} is actually an inverse to $\happly$, so that the full function extensionality axiom follows from an interval type.\\
  %
  \autoref{sec:circle}
  & 327-g7cbe31c
  & In the first sentence after the proof of \autoref{thm:apd2}, ``$P:\Sn^2\to P$'' should be ``$P:\Sn^2\to\type$''.\\
  %
  \autoref{sec:cell-complexes}
  & 289-gdefeb8c
  & In the induction principle for the torus, the types of $p'$ and $q'$ should be $\dpath P p {b'} {b'}$ and $\dpath P q b b$ respectively.\\
  %
  \autoref{sec:hubs-spokes}
  & 289-gdefeb8c
  & In the induction principle for the torus, the types of $p'$ and $q'$ should be $\dpath P p {b'} {b'}$ and $\dpath P q b b$ respectively.\\
  %
  \autoref{sec:hittruncations}
  & 468-g5472874
  & The induction principle for $\brck{A}$ should conclude $f(\bproj a)\jdeq g(a)$, not $f(\bproj a)\jdeq a$.  And in the hypotheses of the induction principle for $\trunc0 A$ and in the proof of \autoref{thm:trunc0-ind}, $v:\dpath{B}{u(x,y,p,q)}{p}{q}$ should instead be $v:\dpath{B}{u(x,y,p,q)}{r}{s}$.\\
  %
  \autoref{lem:quotient-when-canonical-representatives}
  & 514-g18ade45
  & Instead of ``is the set-quotient of $A$ by $\eqr$'', the statement should say ``satisfies the universal property of the set-quotient of $A$ by~$\eqr$, and hence is equivalent to it''.
  In the proof, the second displayed equation should be $e'(g, s) (x,p) \defeq g(x)$.
  The fourth displayed equation should be $e(e'(g, s)) \jdeq e(g \circ \proj{1}) \jdeq (g \circ \proj{1} \circ q, {\nameless})$, the fifth should be $g(\proj{1}(q(x))) \jdeq g(r(x)) = g(x)$, and the proof should conclude with ``$g$ respects $\eqr$ by the assumption $s$''.\\
  %
  \autoref{thm:sign-induction}
  & 535-g0a9abfe
  & The ``computation rules'' satisfied by $f$ are only propositional equalities.
  Also, the proof requires transport across a few unmentioned equivalences.\\
  %
  \autoref{thm:looptothe}
  & 535-g0a9abfe
  & The defining clauses should use $\defid$ rather than $\defeq$ (see the erratum for \autoref{thm:sign-induction}).
  Also, the first clause should say $\refl{a}$ rather than $\refl{\base}$.\\
  %
  \autoref{thm:flattening-cp}
  & 457-g411ec6d
  & The right-hand side of the displayed equation in the proof should be $(\cc(g(b)),D(b)(y))$.\\
  %
  \autoref{sec:flattening}
  & 519-gc99a54c
  & $f$ denotes a map $B\to A$ in this section and should not be re-used for functions defined by induction on $\sm{w:W} P(w)$; we may use $k$ instead.
  Thus $f$ should be $k$ in the last sentence of \autoref{thm:flattening-rect}; the first sentence of its proof; the last sentence of the next paragraph; the last sentence of \autoref{thm:flattening-rectnd}; the first, second, and last sentences of its proof; throughout the statement and proof of \autoref{thm:ap-sigma-rect-path-pair}; the statement of \autoref{thm:flattening-rectnd-beta-ppt}; and the second sentence of its proof.\\
  %
  \autoref{thm:flattening-rect}
  & 537-gdf3b51d
  & In the display after the definition of $q$, the transport in the first line should be with respect to $x\mapsto Q(\cct'(g(b),x))$, and in the second line the subscript of $\apfunc{}$ should be $x\mapsto \cct'(g(b),x)$.\\
  %
  \autoref{thm:ap-sigma-rect-path-pair}
  & 501-ge895f81
  & Both occurrences of $P$ in the statement should be $Y$, and both occurrences of $Q$ in the proof should be $Z$.\\
  %
  % Chapter 7
  %
  \autoref{thm:h-level-retracts}
  & 180-gb672a4d
  & In the last displayed equation of the proof, $q$ should be $r$.\\
  %
  \autoref{thm:isaprop-isofhlevel}
  & 101-g713f48c
  & The base case in the proof is just \autoref{thm:isprop-iscontr}.\\
  %
  \autoref{sec:truncations}
  & 480-gdc84050
  & The third paragraph is wrong: in contrast to \autoref{rmk:spokes-no-hub}, it \emph{would} actually work to define $\trunc nA$ omitting the hub point.\\
  %
  \autoref{thm:path-truncation}
  & 412-gb9582fc
  & In the proof, \encode and \decode should be switched.\\
  %
  \autoref{lem:connected-map-equiv-truncation}
  & 367-g1c8c07e
  & In the proof that the first composite is the identity, all occurrences of $y$ should be $f(x)$.\\
  %
  \autoref{ex:s2-colim-unit}
  & 101-ga366be2
  & ``entires'' should be ``entirely''.\\
  %
  % Chapter 8
  %
  \autoref{lem:s1-encode-decode}
  & 535-g0a9abfe
  & The proof by induction on $n:\Z$ is justified by \autoref{thm:sign-induction}, not \autoref{thm:looptothe}.\\
  %
  \autoref{thm:iscontr-s1cover}
  & 535-g0a9abfe
  & The clauses defining $q_z$ should use $\defid$ rather than $\defeq$ (see the erratum for \autoref{thm:sign-induction}).\\
  %
  \autoref{thm:hopf-fibration}
  & 256-g9e6fcb8
  & The phrase ``whose fibers are $\Sn^1$'' should be ``whose fiber over the basepoint is $\Sn ^1$''.
  The same change should be made in \autoref{ex:HopfJr,ex:SuperHopf}.\\
  %
  \autoref{thm:conn-trunc-variable-ind}
  & 396-g868335b
  & In the proof, the function $k$ should have type $\prd{a:A} P(f(a))$.
  It should also be named $\ell$, to avoid confusion with the integer $k$.\\
  %
  \autoref{thm:freudcode}
  & 87-g3f977b2
  & In the second displayed equation in the proof, $\merid(x_1)$ should be $\opp{\merid(x_1)}$.\\
  %
  \autoref{thm:wedge-connectivity}
  & 399-g8897c94
  & In the last sentence of the proof, ``$(n-1)$-connected'' should be ``$(n-1)$-truncated''.\\
  %
  \autoref{thm:freudlemma}
  & 88-g0c0be67
  & The type of $m$ should be $a_1=a_2$, the second display should begin with $C(a_1,\transfib{B}{\opp m}{b})$, and the proof should say ``we may assume $a_2$ is $a_1$ and $m$ is $\refl{a_1}$''.\\
  %
  \autoref{sec:freudenthal}
  & 165-gd5584c6
  & In~\eqref{eq:freudcompute1}, $r''$ should be $r'$, the end point of $r$ should be $\transfib{B}{\opp{\merid(x_0)}}{q}$, and obtaining $r'$ requires also identifying this with $q \ct \opp{\merid(x_0)}$.
  Similarly, in~\eqref{eq:freudcompute2}, the end point of $r$ should be $\transfib{B}{\opp{\merid(x_1)}}{q}$.\\
  %
  \autoref{sec:freudenthal}
  & 474-g5289470
  & $\pi_3(\Sn^2)=\Z$ should be stated as \autoref{thm:pi3s2}, following from \autoref{cor:pis2-hopf,thm:pinsn}.\\
  %
  % Chapter 9
  %
  \autoref{thm:rezk-completion}
  & 313-g8ee79db
  & In the second proof, the third constructor of $\widehat A_0$ is unneeded; it follows from the fourth constructor and path induction.
  In the fifth constructor, $j(g)\ct j(f)$ should be $j(f)\ct j(g)$, and similarly throughout the proof.
  Finally, for consistency, the 1-truncation constructor should be included explicitly (this was intended to be implied by "higher inductive 1-type").\\
  %
  \autoref{cha:category-theory} Notes
  & 379-ga57eab2
  & It should be mentioned that Hofmann and Streicher (1998) also considered this definition of category.\\
  %
  % Chapter 10
  %
  \autoref{thm:ordord}
  & 140-g55de417
  & The second sentence of the proof should say ``By well-founded induction on $A$, suppose $\ordsl A b$ is accessible for all $b<a$''.\\
  %
  \autoref{thm:ordunion}
  & 140-gd7f8960
  & The statement should say $X:\UU$ rather than $X:\UU_\UU$.\\
  %
  \autoref{thm:wellorder}
  & 140-gcca0bcf
  & The penultimate sentence of the proof should say ``if $a<b$ and $b<c$'' rather than ``if $a<b$ and $a<c$''.\\
  %
  % Chapter 11
  %
  \autoref{dedekind-in-cut-as-le}
  & 165-gb002a64
  & The statement should say ``For all $x : \RD$ and $q : \Q$, $L_x(q) \Leftrightarrow (q < x)$ and $U_x(q)
  \Leftrightarrow (x < q)$''.\\
  %
  \autoref{RD-inverse-apart-0}
  & 165-g179b359
  & In the proof, the sentence beginning ``From $0<ac$ it follows'' should be replaced by ``From $0 < a c$ and $0 < b c$ it follows
  that $a$, $b$, and $c$ are either all positive or all negative.
  Hence either $0 < a < x$ or $x < b < 0$, so that $x \apart 0$''.\\
  %
  \autoref{lem:untruncated-linearity-reals-coincide}
  & 87-g82b27c3
  & \eqref{eq:untruncated-linearity} should be $c:\prd{q, r : \Q} (q < r) \to (q < x) + (x < r)$, and therefore the use of $c$ in the proof should be $c(s,t)$ rather than $c(x,s,t)$.\\
  %
  \autoref{ex:mean-value-theorem}
  & 222-g3453cf1
  & This is the intermediate value theorem, not the mean value theorem.\\
  %
  % Appendix A
  %
  \autoref{cha:rules}
  & 165-g76db618
  & After the introduction of the judgment ``$\wfctx{\Gamma}$'' in the Preliminaries, the sentence beginning ``Therefore, if $\oftp\Gamma aA$, \dots'' should say instead ``In particular, therefore, if $\oftp\Gamma aA$, \dots''.\\
  %
  \autoref{subsec:contexts}
  & 64-g7c2312e
  & Clarify the distinction between typing judgments and context well-formedness judgments, and
  remove the $\vdash$ from the notation for the latter.\\
  %
  \autoref{sec:more-formal-sigma}
  & 26-gcd691e8
  & In $\Sigma$-\comp\ and the following paragraph, $y.C$ should be $z.C$, and ``we bind \dots $y$ in $C$'' should likewise say $z$.\\
  %
  \autoref{sec:more-formal-unit}
  & 338-g4e1c688
  & The $c$ argument in the eliminator for $\unit$ (in the $\unit$-\elim\ and $\unit$-\comp\ rules) should not bind a variable of type $\unit$.\\
%% END ERRATA
\end{longtable}

\end{document}
